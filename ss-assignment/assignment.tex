\PassOptionsToPackage{table}{xcolor}
\documentclass[openright,twoside,10pt]{memoir}
\synctex=1

% DOCUMENT LAYOUT
\usepackage{xltxtra} % includes fixltx2e, metalogo, xunicode, fontspec
\setromanfont[Ligatures=TeX,Numbers={Lining},AutoFakeBold=1.0]{EB Garamond}
\setmonofont[Scale=0.8]{Consolas}
\setsansfont[Scale=0.9]{Optima}
\usepackage[protrusion=true,factor=2000]{microtype} 


% TABLE OF CONTENTS
\setsecnumdepth{subsection}
\setlength{\beforechapskip}{0pt}
\setlength{\afterchapskip}{10pt} % vspace under chapters
\setlength{\belowcaptionskip}{-8pt} % vspace mellem caption og tekst
\setlength{\cftparskip}{0pt}
\setlength{\cftbeforechapterskip}{6pt}
\setlength{\cftbeforepartskip}{14pt}

%math
\usepackage{amsmath}
\usepackage[group-separator={,}]{siunitx}

% LANGUAGE
\usepackage{polyglossia} %XeTeX babel
\setdefaultlanguage[variant=uk]{english}

% For \smallsetminus
\usepackage{ amssymb }

%TIKZ
\usepackage{fancybox}
\usepackage{tikz}
\usetikzlibrary{calc,trees,positioning,arrows,chains,shapes.geometric,%
    decorations.pathreplacing,decorations.pathmorphing,shapes,%
    matrix,shapes.symbols}
        
\tikzset{
>=stealth',
  punktchain/.style={
    rectangle, rounded corners, draw=black, very thick, 
    text width=6em, minimum height=1.5em, 
    text centered, on chain},
  combtree/.style={
    rectangle, rounded corners, draw=black, very thick, 
    text width=5em, minimum height=1.5em, 
    text centered, on chain},
  line/.style={draw, thick, <-},
  element/.style={tape, top color=white, 
  	bottom color=blue!50!black!60!, minimum width=8em,
    draw=blue!40!black!90, very thick, text width=10em, 
    minimum height=3.5em, text centered, on chain},
  every join/.style={->, thick,shorten >=1pt},
  decoration={brace},
  tuborg/.style={decorate, very thick},
  tubnode/.style={midway, right=2pt},
}
\tikzstyle{square}=[rectangle, draw, rounded corners=1mm, solid]

% OTHER
\usepackage{tabularx} % tabularx
\usepackage{epstopdf} % to include eps files
\usepackage{enumitem} % for [noitemsep] in enumerate/itemize lists
\graphicspath{ {images/} }
\usepackage{color}
\usepackage{multicol}
\usepackage{array}
\usepackage{listings}	%for code examples
%\usepackage{microtype}
\usepackage{caption}
\usepackage{hyperref}

%---- CODE LISTING STYLING
% Taken from: http://stackoverflow.com/questions/741985/latex-source-code-listing-like-in-professional-books
\lstset{
  basicstyle=\footnotesize\ttfamily, 
  numbers=left,              
  numberstyle=\tiny,        
  numbersep=5pt,
  tabsize=2,
  extendedchars=true,
  breaklines=true,
  keywordstyle=\color{red},
  frame=b,         
  stringstyle=\color{cyan}\ttfamily,
  showspaces=false,
  showtabs=false,
  xleftmargin=17pt,
  framexleftmargin=17pt,
  framexrightmargin=5pt,
  framexbottommargin=4pt,
         backgroundcolor=\color{lightgray},
  showstringspaces=false
}
\DeclareCaptionFont{white}{\color{white}}
\DeclareCaptionFormat{listing}{\colorbox[cmyk]{0.43, 0.35, 0.35,0.01}{\parbox{\textwidth}{\hspace{15pt}#1#2#3}}}
\captionsetup[lstlisting]{format=listing,labelfont=white,textfont=white, singlelinecheck=false, margin=0pt, font={bf,footnotesize}}
%----

% grammar stuff

% PARAGRAPH SPACING
\parskip=5pt plus 2pt minus 1pt
\parindent=0pt

% WIDER MARGINS
\let\oldmarginpar\marginpar
\renewcommand\marginpar[1]{\-\oldmarginpar[\raggedleft\footnotesize #1]%
{\raggedright\footnotesize\color{red} #1}}
% Margins
\setstocksize{11in}{8.5in}
\settrimmedsize{11in}{8.5in}{*}
\settrims{0in}{0in}
\settypeblocksize{9.0in}{6in}{*}
\setlrmargins{1.25in}{*}{*}
\setulmargins{1.0in}{*}{*}
\setheadfoot{14pt}{26pt}
\setheaderspaces{*}{13pt}{*}

% Fra settings.tex
\setlrmarginsandblock{3.5cm}{2.5cm}{*}
\setulmarginsandblock{2.5cm}{3.0cm}{*}

\checkandfixthelayout

% CHANGE BULLETS IN ITEMIZE
\renewcommand{\labelitemi}{$\bullet$}
\renewcommand{\labelitemii}{$\circ$}
\renewcommand{\labelitemiii}{$\cdot$}

%colors
\definecolor{shcomment}{rgb}{0.12, 0.38, 0.18 }
\definecolor{shkeyword}{rgb}{0.37, 0.08, 0.25}  % #5F1441
\definecolor{shstring}{rgb}{0.06, 0.10, 0.98} % #101AF9

%table colors
\usepackage{colortbl}
\definecolor{lightgray}{rgb}{0.95, 0.95, 0.95}

% macros
\newcommand{\productname}{{\fontspec{OptimusPrinceps}Junta}}
\newcommand{\CS}{C$^\sharp$}

\newcommand{\secref}[1]{section \ref{#1}}
\newcommand{\chapref}[1]{chapter \ref{#1}}
\newcommand{\figref}[1]{figure \ref{#1}}
\newcommand{\lstref}[1]{listing \ref{#1}}
\newcommand{\apref}[1]{appendix \ref{#1}}
\newcommand{\tableref}[1]{table \ref{#1}}
\newcommand{\itemref}[1]{element \ref{#1}}
\newcommand{\pseudoref}[1]{algorithm \ref{#1}}
\newcommand{\capt}[1]{\caption{\emph{#1}}}
\newcommand{\csref}[1]{code sample (\ref{cs:#1})}

%code-like refs
\newcommand{\tokenref}[1]{{\textbf{#1}}}
\newcommand{\classref}[1]{\textbf{{#1}}}
\newcommand{\methodref}[1]{\textbf{{#1}}}
\newcommand{\varref}[1]{\textbf{{#1}}}
\newcommand{\typeref}[1]{\textbf{{#1}}}

\newcommand{\todo}[1]{\colorbox{yellow}{\color{red}\textbf{TODO:} \hspace{1ex} #1}}

\newcommand{\codesample}[1]{
  \vspace{-0.6cm}
  \begin{center}
  \begin{tabularx}{0.9\textwidth}{m{9cm} X m{2cm} }
  \parbox[t]{8cm}{
    \vspace{-0.1cm}
    \input{codesamples/#1}}
    & &
    \parbox{2cm}{\hfill
      \begin{equation}\label{cs:#1}
      \end{equation}
    } 
  \end{tabularx}
  \end{center}
  \vspace{-0.4cm}
}

%big-step-semantic
\newcommand{\infrule}[2]
           {\parbox{4.5cm}{$$ \frac{#1}{#2}\hspace{.5cm}$$}}
	
\newcommand{\ra}{\rightarrow}
\newcommand{\lag}{\langle}
\newcommand{\rag}{\rangle}


%itemize
\newenvironment{dlist}{
\begin{itemize}[noitemsep]
}{
\end{itemize}
}

%enumerate
\newenvironment{nlist}{
\begin{enumerate}[noitemsep]
}{
\end{enumerate}
}

\newcommand{\literal}[1]{\texttt{\color[rgb]{0.400 0.000 0.933}#1}}
\newcommand{\type}[1]{\texttt{\color[rgb]{0.000 0.200 0.400}#1}}
\newcommand{\identifier}[1]{\texttt{\color[rgb]{0.000 0.200 0.400}#1}}
\newcommand{\variable}[1]{\texttt{\color[rgb]{0.694 0.537 0.384}\$#1}}
\newcommand{\constant}[1]{\texttt{\color[rgb]{0.012 0.408 0.733}#1}}
\newcommand{\function}[1]{\texttt{\color[rgb]{0.012 0.408 0.733}#1}}
\newcommand{\keyword}[1]{\texttt{\color[rgb]{0.000 0.529 0.000}#1}}

\newcommand{\operator}[4][]{\texttt{\type{#1} \textbf{\color{nicered}#2} \type{#3} $\rightarrow$ \type{#4}}}

\newcommand{\constdef}[3]{\texttt{\constant{#1}#2 : \type{#3}}}
\newcommand{\farg}[2]{\variable{#1} : \type{#2}}
\newcommand{\typedef}[3]{\texttt{\type{#1} \variable{#2} : \type{#3}}}

\newcommand{\opstar}{$*$}

\newcommand{\fig}[3][scale=1.0]{\begin{figure}[ht]
  \center
  \includegraphics[#1]{pictures/#2}
  \capt{#3}
  \label{fig:#2}
\end{figure}}

\newenvironment{ebnf} {
  \begin{center}
    \begin{tabular}{>{\hfill}p{2.5cm} c p{9cm}}
}{
  \end{tabular}
\end{center}
}

%rules for ebnf
\newcommand{\grule}[2]{$#1$ & $\rightarrow$ & \parbox[t]{9cm}{$#2$} \\}
\newcommand{\gnl}{$\\$}
\newcommand{\galt}[1]{& | & $#1$ \\}
\newcommand{\grange}{\cdots}
\newcommand{\gcomment}[1]{\textrm{#1}}
\newcommand{\gtsq}{\texttt{"'"}}
\newcommand{\gtbs}{\texttt{"\textbackslash"}}
\newcommand{\gtdq}{\texttt{'"'}}
\newcommand{\gor}{ \: | \: }
\newcommand{\grep}[1]{\{#1\} \: }
\newcommand{\gopt}[1]{[\: #1 \:] \: }
\newcommand{\ggrp}[1]{(\: #1 \:) \: }
\newcommand{\gter}[1]{\texttt{"#1"}}
\newcommand{\gex}{\: - \:}
\newcommand{\gcat}{ \; }

%rules for formation rules
\newcommand{\for}{ \: | \: }

% Table
\newcommand{\tab}[8][\textwidth]{
\begin{table}[ht]
  \centering
  \rowcolors{3}{white}{lightgray}
    \begin{tabularx}{#1}{ r | >{\centering\arraybackslash}X>{\centering\arraybackslash}X>{\centering\arraybackslash}X>{\centering\arraybackslash}X>{\centering\arraybackslash}X>{\centering\arraybackslash}X>{\centering\arraybackslash}X>{\centering\arraybackslash}X>{\centering\arraybackslash}X>{\centering\arraybackslash}X>{\centering\arraybackslash}X>{\centering\arraybackslash}X>{\centering\arraybackslash}X>{\centering\arraybackslash}X>{\centering\arraybackslash}X>{\centering\arraybackslash}X>{\centering\arraybackslash}X>{\centering\arraybackslash}X>{\centering\arraybackslash}X>{\centering\arraybackslash}X>{\centering\arraybackslash}X>{\centering\arraybackslash}X>{\centering\arraybackslash}X>{\centering\arraybackslash}X>{\centering\arraybackslash}X>{\centering\arraybackslash}X>{\centering\arraybackslash}X>{\centering\arraybackslash}X>{\centering\arraybackslash}X>{\centering\arraybackslash}X>{\centering\arraybackslash}X } %Errr, doesn't seem to give any errors, but fix it if possible
  \hiderowcolors
  \hline
  & \multicolumn{#3}{>{\columncolor[rgb]{1,1,1}}c}{\textbf{#5}} \\
   \textbf{#6}    & #7 \\ \hline
   \showrowcolors
	#8
	\hline
    \end{tabularx}
    \capt{#4}
    \label{table:#2}
\end{table}
}

\newcommand{\tabrow}[2]{ #1 & #2 \\ }

\usepackage{color,calc,graphicx,soul}

\definecolor{nicered}{rgb}{.647,.129,.149}
%\definecolor{nicered}{rgb}{.047,.136,.290} % aau color
\makeatletter
\newlength\dlf@normtxtw
\setlength\dlf@normtxtw{\textwidth}
\def\myhelvetfont{\def\sfdefault{mdput}}
\newsavebox{\feline@chapter}
\newcommand\feline@chapter@marker[1][4cm]{%
  \sbox\feline@chapter{%
    \resizebox{!}{#1}{\fboxsep=1pt%
      \colorbox{nicered}{\color{white} \bfseries\sffamily\thechapter}%
    }}%
  \rotatebox{90}{%
    \resizebox{%
      \heightof{\usebox{\feline@chapter}}+\depthof{\usebox{\feline@chapter}}}%
    {!}{\scshape\so\@chapapp}}\quad%
  \raisebox{\depthof{\usebox{\feline@chapter}}}{\usebox{\feline@chapter}}%
}
\newcommand\feline@chm[1][4cm]{%
  \sbox\feline@chapter{\feline@chapter@marker[#1]}%
  \makebox[0pt][l]{% aka \rlap
    \makebox[1cm][r]{\usebox\feline@chapter}%
  }}
\makechapterstyle{d402e13}{
  % Inspried by daleif1 (http://goo.gl/YKLm7)

  % Chapters
  \renewcommand\chapnamefont{\normalfont\Large\scshape\raggedleft\so}
  \renewcommand\chaptitlefont{\normalfont\huge\bfseries\scshape\color{nicered}\fontspec{OptimusPrinceps}}
  \renewcommand\chapternamenum{\fontspec{OptimusPrinceps}}
  \renewcommand\printchaptername{}
  \renewcommand\printchapternum{\null\hfill\feline@chm[2.5cm]\par}
  \renewcommand\afterchapternum{\par\vskip\midchapskip}
  \renewcommand\printchaptertitle[1]{\chaptitlefont\raggedleft ##1\par}
  
  % Sections
  \setsecheadstyle{\normalfont\Large\scshape\color{nicered}\fontspec{OptimusPrinceps}}

  % Subsections
  \setsubsecheadstyle{\normalfont\large\scshape\color{nicered}\fontspec{OptimusPrinceps}}
  
  % Parts
  \renewcommand*{\partnamefont}{\color{white}}
  \renewcommand*{\partnumfont}{\color{white}}
  \renewcommand*{\parttitlefont}{
    \hspace{100pt}
    \resizebox{!}{80pt}{\fboxsep=1pt%
      \colorbox{nicered}{\color{white} \bfseries\sffamily
      \fontspec{OptimusPrinceps} Appendix}%
    }%
    \HUGE\color{nicered}
  }
}
\makeatother


\chapterstyle{d402e13}


% Pygments syntex highlightning
\usepackage{fancyvrb}

\makeatletter
\def\PY@reset{\let\PY@it=\relax \let\PY@bf=\relax%
    \let\PY@ul=\relax \let\PY@tc=\relax%
    \let\PY@bc=\relax \let\PY@ff=\relax}
\def\PY@tok#1{\csname PY@tok@#1\endcsname}
\def\PY@toks#1+{\ifx\relax#1\empty\else%
    \PY@tok{#1}\expandafter\PY@toks\fi}
\def\PY@do#1{\PY@bc{\PY@tc{\PY@ul{%
    \PY@it{\PY@bf{\PY@ff{#1}}}}}}}
\def\PY#1#2{\PY@reset\PY@toks#1+\relax+\PY@do{#2}}

\expandafter\def\csname PY@tok@gd\endcsname{\def\PY@tc##1{\textcolor[rgb]{0.63,0.00,0.00}{##1}}}
\expandafter\def\csname PY@tok@gu\endcsname{\let\PY@bf=\textbf\def\PY@tc##1{\textcolor[rgb]{0.50,0.00,0.50}{##1}}}
\expandafter\def\csname PY@tok@gt\endcsname{\def\PY@tc##1{\textcolor[rgb]{0.00,0.27,0.87}{##1}}}
\expandafter\def\csname PY@tok@gs\endcsname{\let\PY@bf=\textbf}
\expandafter\def\csname PY@tok@gr\endcsname{\def\PY@tc##1{\textcolor[rgb]{1.00,0.00,0.00}{##1}}}
\expandafter\def\csname PY@tok@cm\endcsname{\def\PY@tc##1{\textcolor[rgb]{0.53,0.53,0.53}{##1}}}
\expandafter\def\csname PY@tok@vg\endcsname{\let\PY@bf=\textbf\def\PY@tc##1{\textcolor[rgb]{0.87,0.47,0.00}{##1}}}
\expandafter\def\csname PY@tok@m\endcsname{\let\PY@bf=\textbf\def\PY@tc##1{\textcolor[rgb]{0.40,0.00,0.93}{##1}}}
\expandafter\def\csname PY@tok@mh\endcsname{\let\PY@bf=\textbf\def\PY@tc##1{\textcolor[rgb]{0.00,0.33,0.53}{##1}}}
\expandafter\def\csname PY@tok@cs\endcsname{\let\PY@bf=\textbf\def\PY@tc##1{\textcolor[rgb]{0.80,0.00,0.00}{##1}}}
\expandafter\def\csname PY@tok@ge\endcsname{\let\PY@it=\textit}
\expandafter\def\csname PY@tok@vc\endcsname{\def\PY@tc##1{\textcolor[rgb]{0.20,0.40,0.60}{##1}}}
\expandafter\def\csname PY@tok@il\endcsname{\let\PY@bf=\textbf\def\PY@tc##1{\textcolor[rgb]{0.00,0.00,0.87}{##1}}}
\expandafter\def\csname PY@tok@go\endcsname{\def\PY@tc##1{\textcolor[rgb]{0.53,0.53,0.53}{##1}}}
\expandafter\def\csname PY@tok@cp\endcsname{\def\PY@tc##1{\textcolor[rgb]{0.33,0.47,0.60}{##1}}}
\expandafter\def\csname PY@tok@gi\endcsname{\def\PY@tc##1{\textcolor[rgb]{0.00,0.63,0.00}{##1}}}
\expandafter\def\csname PY@tok@gh\endcsname{\let\PY@bf=\textbf\def\PY@tc##1{\textcolor[rgb]{0.00,0.00,0.50}{##1}}}
\expandafter\def\csname PY@tok@ni\endcsname{\let\PY@bf=\textbf\def\PY@tc##1{\textcolor[rgb]{0.53,0.00,0.00}{##1}}}
\expandafter\def\csname PY@tok@nl\endcsname{\let\PY@bf=\textbf\def\PY@tc##1{\textcolor[rgb]{0.60,0.47,0.00}{##1}}}
\expandafter\def\csname PY@tok@nn\endcsname{\let\PY@bf=\textbf\def\PY@tc##1{\textcolor[rgb]{0.05,0.52,0.71}{##1}}}
\expandafter\def\csname PY@tok@no\endcsname{\let\PY@bf=\textbf\def\PY@tc##1{\textcolor[rgb]{0.00,0.20,0.40}{##1}}}
\expandafter\def\csname PY@tok@na\endcsname{\def\PY@tc##1{\textcolor[rgb]{0.00,0.00,0.80}{##1}}}
\expandafter\def\csname PY@tok@nb\endcsname{\def\PY@tc##1{\textcolor[rgb]{0.00,0.44,0.13}{##1}}}
\expandafter\def\csname PY@tok@nc\endcsname{\let\PY@bf=\textbf\def\PY@tc##1{\textcolor[rgb]{0.73,0.00,0.40}{##1}}}
\expandafter\def\csname PY@tok@nd\endcsname{\let\PY@bf=\textbf\def\PY@tc##1{\textcolor[rgb]{0.33,0.33,0.33}{##1}}}
\expandafter\def\csname PY@tok@ne\endcsname{\let\PY@bf=\textbf\def\PY@tc##1{\textcolor[rgb]{1.00,0.00,0.00}{##1}}}
\expandafter\def\csname PY@tok@nf\endcsname{\let\PY@bf=\textbf\def\PY@tc##1{\textcolor[rgb]{0.00,0.40,0.73}{##1}}}
\expandafter\def\csname PY@tok@si\endcsname{\def\PY@bc##1{\setlength{\fboxsep}{0pt}\colorbox[rgb]{0.93,0.93,0.93}{\strut ##1}}}
\expandafter\def\csname PY@tok@s2\endcsname{\def\PY@bc##1{\setlength{\fboxsep}{0pt}\colorbox[rgb]{1.00,0.94,0.94}{\strut ##1}}}
\expandafter\def\csname PY@tok@vi\endcsname{\def\PY@tc##1{\textcolor[rgb]{0.20,0.20,0.73}{##1}}}
\expandafter\def\csname PY@tok@nt\endcsname{\def\PY@tc##1{\textcolor[rgb]{0.00,0.47,0.00}{##1}}}
\expandafter\def\csname PY@tok@nv\endcsname{\def\PY@tc##1{\textcolor[rgb]{0.60,0.40,0.20}{##1}}}
\expandafter\def\csname PY@tok@s1\endcsname{\def\PY@bc##1{\setlength{\fboxsep}{0pt}\colorbox[rgb]{1.00,0.94,0.94}{\strut ##1}}}
\expandafter\def\csname PY@tok@gp\endcsname{\let\PY@bf=\textbf\def\PY@tc##1{\textcolor[rgb]{0.78,0.36,0.04}{##1}}}
\expandafter\def\csname PY@tok@sh\endcsname{\def\PY@bc##1{\setlength{\fboxsep}{0pt}\colorbox[rgb]{1.00,0.94,0.94}{\strut ##1}}}
\expandafter\def\csname PY@tok@ow\endcsname{\let\PY@bf=\textbf\def\PY@tc##1{\textcolor[rgb]{0.00,0.00,0.00}{##1}}}
\expandafter\def\csname PY@tok@sx\endcsname{\def\PY@tc##1{\textcolor[rgb]{0.87,0.13,0.00}{##1}}\def\PY@bc##1{\setlength{\fboxsep}{0pt}\colorbox[rgb]{1.00,0.94,0.94}{\strut ##1}}}
\expandafter\def\csname PY@tok@bp\endcsname{\def\PY@tc##1{\textcolor[rgb]{0.00,0.44,0.13}{##1}}}
\expandafter\def\csname PY@tok@c1\endcsname{\def\PY@tc##1{\textcolor[rgb]{0.53,0.53,0.53}{##1}}}
\expandafter\def\csname PY@tok@kc\endcsname{\let\PY@bf=\textbf\def\PY@tc##1{\textcolor[rgb]{0.00,0.53,0.00}{##1}}}
\expandafter\def\csname PY@tok@c\endcsname{\def\PY@tc##1{\textcolor[rgb]{0.53,0.53,0.53}{##1}}}
\expandafter\def\csname PY@tok@mf\endcsname{\let\PY@bf=\textbf\def\PY@tc##1{\textcolor[rgb]{0.40,0.00,0.93}{##1}}}
\expandafter\def\csname PY@tok@err\endcsname{\def\PY@tc##1{\textcolor[rgb]{1.00,0.00,0.00}{##1}}\def\PY@bc##1{\setlength{\fboxsep}{0pt}\colorbox[rgb]{1.00,0.67,0.67}{\strut ##1}}}
\expandafter\def\csname PY@tok@kd\endcsname{\let\PY@bf=\textbf\def\PY@tc##1{\textcolor[rgb]{0.00,0.53,0.00}{##1}}}
\expandafter\def\csname PY@tok@ss\endcsname{\def\PY@tc##1{\textcolor[rgb]{0.67,0.40,0.00}{##1}}}
\expandafter\def\csname PY@tok@sr\endcsname{\def\PY@tc##1{\textcolor[rgb]{0.00,0.00,0.00}{##1}}\def\PY@bc##1{\setlength{\fboxsep}{0pt}\colorbox[rgb]{1.00,0.94,1.00}{\strut ##1}}}
\expandafter\def\csname PY@tok@mo\endcsname{\let\PY@bf=\textbf\def\PY@tc##1{\textcolor[rgb]{0.27,0.00,0.93}{##1}}}
\expandafter\def\csname PY@tok@mi\endcsname{\let\PY@bf=\textbf\def\PY@tc##1{\textcolor[rgb]{0.00,0.00,0.87}{##1}}}
\expandafter\def\csname PY@tok@kn\endcsname{\let\PY@bf=\textbf\def\PY@tc##1{\textcolor[rgb]{0.00,0.53,0.00}{##1}}}
\expandafter\def\csname PY@tok@o\endcsname{\def\PY@tc##1{\textcolor[rgb]{0.20,0.20,0.20}{##1}}}
\expandafter\def\csname PY@tok@kr\endcsname{\let\PY@bf=\textbf\def\PY@tc##1{\textcolor[rgb]{0.00,0.53,0.00}{##1}}}
\expandafter\def\csname PY@tok@s\endcsname{\def\PY@bc##1{\setlength{\fboxsep}{0pt}\colorbox[rgb]{1.00,0.94,0.94}{\strut ##1}}}
\expandafter\def\csname PY@tok@kp\endcsname{\let\PY@bf=\textbf\def\PY@tc##1{\textcolor[rgb]{0.00,0.20,0.53}{##1}}}
\expandafter\def\csname PY@tok@w\endcsname{\def\PY@tc##1{\textcolor[rgb]{0.73,0.73,0.73}{##1}}}
\expandafter\def\csname PY@tok@kt\endcsname{\let\PY@bf=\textbf\def\PY@tc##1{\textcolor[rgb]{0.20,0.20,0.60}{##1}}}
\expandafter\def\csname PY@tok@sc\endcsname{\def\PY@tc##1{\textcolor[rgb]{0.00,0.27,0.87}{##1}}}
\expandafter\def\csname PY@tok@sb\endcsname{\def\PY@bc##1{\setlength{\fboxsep}{0pt}\colorbox[rgb]{1.00,0.94,0.94}{\strut ##1}}}
\expandafter\def\csname PY@tok@k\endcsname{\let\PY@bf=\textbf\def\PY@tc##1{\textcolor[rgb]{0.00,0.53,0.00}{##1}}}
\expandafter\def\csname PY@tok@se\endcsname{\let\PY@bf=\textbf\def\PY@tc##1{\textcolor[rgb]{0.40,0.40,0.40}{##1}}\def\PY@bc##1{\setlength{\fboxsep}{0pt}\colorbox[rgb]{1.00,0.94,0.94}{\strut ##1}}}
\expandafter\def\csname PY@tok@sd\endcsname{\def\PY@tc##1{\textcolor[rgb]{0.87,0.27,0.13}{##1}}}

\def\PYZbs{\char`\\}
\def\PYZus{\char`\_}
\def\PYZob{\char`\{}
\def\PYZcb{\char`\}}
\def\PYZca{\char`\^}
\def\PYZam{\char`\&}
\def\PYZlt{\char`\<}
\def\PYZgt{\char`\>}
\def\PYZsh{\char`\#}
\def\PYZpc{\char`\%}
\def\PYZdl{\char`\$}
\def\PYZhy{\char`\-}
\def\PYZsq{\char`\'}
\def\PYZdq{\char`\"}
\def\PYZti{\char`\~}
% for compatibility with earlier versions
\def\PYZat{@}
\def\PYZlb{[}
\def\PYZrb{]}
\makeatother








\title{Big-step semantik for \junta{} - fri projektrelateret opgave}
\author{\href{mailto:d402f13@cs.aau.dk}{d402f13}}

\begin{document}
\maketitle

Dette er vores besvarelse på den frie projektrelaterede opgave, hvor vi har valgt
en delmængde af vores sprog, \junta{}, som er et funktionelt objektorienteret
programmeringssprog.

Tankerne bag sproget er, at man skal kunne programmere brætspil i sproget,
der kan spilles ved hjælp af en simulator. Mange funktionelle principper
ses i sproget, da der ingen sideeffekter er og ingen programtilstande
findes.

Herunder defineres semantikken for følgende konstruktioner i
sproget: Let-in-udtryk, metode-kald, member-tilgang, liste-tilgang, lambda-udtryk,
set-konstruktionen og typedefinitioner (typer minder om klasser i andre
objektorienterede sprog).

\tableofcontents

Definitionerne er på engelsk, da rapporten skrives på engelsk (og disse
ses som et udtræk af rapporten).

\chapter{Syntactic categories}
Before we can describe the behaviour of programs written in \junta{} and
their lexical structure, we must first present the syntax of programs. At this
point, we are only interested in a notion of abstract syntax because we do not
need to concern ourselves with operator precedence and so forth.

The different syntactic categories are seen in table \ref{table:syn-cat}.

\begin{table}[ht]
  \begin{center}
    \begin{tabular}{rl}
      \hline
      $n \in$ & \textbf{Integer}         \\
      $x \in$ & \textbf{Variable}        \\
      $s \in$ & \textbf{String}          \\
      $E \in$ & \textbf{Expression}      \\
      $P \in$ & \textbf{Pattern}         \\
      $L \in$ & \textbf{List}            \\
      $X \in$ & \textbf{VarList}         \\
      $Y \in$ & \textbf{Coordinate}      \\
      $Z \in$ & \textbf{Direction}       \\
      $C \in$ & \textbf{ConstantNames}   \\
      $T \in$ & \textbf{TypeNames}       \\
      $D_{G} \in$ & \textbf{GlobalDef}   \\
      $D_{M} \in$ & \textbf{MemberDef}   \\
      \hline
    \end{tabular}  
    \capt{The syntactic categories of \hspace{0.02cm} \productname{}.}
    \label{table:syn-cat}
  \end{center}
\end{table}

\chapter{Environments}
In this section we present the definitions which will be used throughout the
construction of semantics for \productname{}. In the following definitions we
use the syntactic categories presented in \tableref{table:syn-cat}. For some of
the definitions we define arbitrary members which will be used in the semantics
of the constructs of the language.

\textbf{Definition} (Type environment) \hspace{0.5cm} The set of type environments is the set of
partial functions from type names to type values:

\[
  \mathbf{EnvT} = \mathbf{TypeNames} \rightharpoonup \mathbf{TypeValues}
\]

\textbf{Definition} (Constant environment) \hspace{0.5cm} The set of constant environments is the set
of partial functions from constant names to expressions and variable lists:

\[
  \mathbf{EnvC} = \mathbf{ConstantNames} \rightharpoonup \mathbf{Expression}
  \times \mathbf{VarList}
\]

An arbitrary member is defined as $env_{C} \in \mathbf{EnvC}$.

\textbf{Definition} (Variable environment) \hspace{0.5cm} The set of variable environments is the set
of partial functions from variables to values:

\[
  \mathbf{EnvV} = \mathbf{Variable} \rightharpoonup \mathbf{Values}
\]


An arbitrary member is defined as $env_{V} \in \mathbf{EnvV}$.

\textbf{Definition} (Values) \hspace{0.5cm} The set of values can contain many
different values, and is defined as follows:

\begin{align*}
 \mathbf{Values} = \mathbf{Integers} \cup \mathbf{Strings} \cup \mathbf{Lists}
 \cup \mathbf{Patterns} \cup \mathbf{Coordinates} \cup \mathbf{Directions}\\ 
 \cup\; \mathbf{TypeValues} \cup \mathbf{FunctionValues} \cup \mathbf{ObjectValues} 
 \cup \mathbf{Booleans}
\end{align*}

\textbf{Definition} (List values) \hspace{0.5cm} The values of lists is
defined as follows:

\[
  \mathbf{ListValues} = \mathbf{Integers} \times \mathbf{Elements}
\]

A length of an arbitrary list is defined as $l \in \mathbf{Integer}$.

\textbf{Definition} (List elements) \hspace{0.5cm} The set of elements in
lists is the set of partial functions from integers to values:

\[
  \mathbf{Elements} = \mathbf{Integers} \rightharpoonup \mathbf{Values}
\]

An arbitrary member is defined as $elem \in \mathbf{Elements}$.

\textbf{Definition} (Function values) \hspace{0.5cm} The set of function values
is defined as follows:
 
\[
  \mathbf{FunctionValues} = \mathbf{VarLists} \times \mathbf{Expressions} \times
  \mathbf{EnvV} \times \mathbf{EnvC}
\]

A function value consists of a variable list, an expression and the set of variable
and constant environments.

\textbf{Definition} (Type values) \hspace{0.5cm} The set of type values is
defined as follows:

\[
  \mathbf{TypeValues} = \mathbf{TypeNames} \times \mathbf{VarLists} \times
  \mathbf{D_{M}} \times \mathbf{List} \times \mathbf{TypeValues }
\]

An arbitrary member is defined as $t \in \mathbf{TypeValues}$.

A type value consists of a type name, a variable list, the member definitions,
a list and the super type's type value. A type value is recursively defined since
$\mathbf{TypeValues}$ is defined in terms of itself. This is not a problematic
definition because these must be considered as values for two different types.
The $\mathbf{TypeValues}$ on the right-hand side is in fact the super type's set
of values whereas the $\mathbf{TypeValues}$ on the left-hand side is the current
type's set of values.  The list in the definition is in fact the super type's
parameters whereas the variable list is the current type's formal parameters.

\textbf{Definition} (Object values) \hspace{0.5cm} The set of object values (an
instantiated $\mathbf{TypeValue}$) is defined as follows:

\[
  \mathbf{ObjectValues} = \mathbf{TypeValues} \times \mathbf{EnvC} \times
  \mathbf{EnvV} \times \mathbf{ObjectValues}
\]

An object value consists of the set of type values, the set of constant
environment, the set of variable environments and the set of object values. This
is yet another recursively defined definition, because an object value can have
a parent object value, this is the case when an object value extends another
object value. 

\section{Formulation rules}
Each syntactic category is used in one or more of the formulation rules
presented in \figref{fig:form-rules}. The formulation rules define the structure
of the members of the syntactic categories. 

Not all of the constituents of the formulation rules are syntactic categories.
We for instance see different parentheses, forward slashes, different operators,
and words like $\texttt{this}$ and $\texttt{define}$. These are part of the
construction of the given formulation rule. If they are omitted from a rule then
it is not valid in \productname{}.

\begin{figure}[ht]
  \begin{center}
    \begin{tabular}[ht]{r l}
      $E\; \mathbf{::=}$ & $n$ \for $x$ \for $s$ \for $Y$ \for $Z$ \for $T$ \for
      $C$ \for $L$ \for $\texttt{-} E$ \for $\left(\; E\; \right)$ \for
      $\texttt{/}\; P\; \texttt{/}$ \for \\ 
      & $E_{1}\; L$ \for $E_{1}\texttt{.}C$ \for $\texttt{\#}\; X\;
      \texttt{=>}\; E$ \for $\texttt{if}\; E_{1}\; \texttt{then}\; E_{2}\;
      \texttt{else}\; E3$ \for $E_{1}\; \texttt{is}\; E_{2}$ \for
      $\texttt{not}\; E$ \for \\ 
      & $E_{1}\; \texttt{and}\; E_{2}$ \for
      $E_{1} \;\texttt{or}\; E_{2}$ \for $E_{1}\; \texttt{==}\; E_{2}$ \for
      $E_{1}\; \texttt{!=}\; E_{2}$ \for $E_{1}\; \texttt{<}\; E_{2}$ \for
      $E_{1}\; \texttt{>}\; E_{2}$ \for \\ 
      & $E_{1}\; \texttt{<=}\; E_{2}$ \for
      $E_{1}\; \texttt{>=}\; E_{2}$ \for $E_{1}\; \texttt{+}\; E_{2}$ \for
      $E_{1}\; \texttt{-}\; E_{2}$ \for $E_{1}\; \texttt{*}\; E_{2}$ \for
      $E_{1}\; \texttt{/}\; E_{2}$ \for \\ 
      & $E_{1}\; \texttt{\%}\; E_{2}$ \for \texttt{this} \for \texttt{super}
      \for $\texttt{let}\; x_{1}\; \texttt{=}\; E_{1},\; x_{2}\; \texttt{=}\;
      E_{2},\; \cdots,\; x_{k}\; \texttt{=}\; E_{k}\; \texttt{in}\; E_{k+1}$ \for \\ 
      & $\texttt{set}\; x_{1}\; \texttt{=}\; E_{1},\; x_{2}\; \texttt{=}\;
      E_{2},\; \cdots,\; x_{k}\; \texttt{=}\; E_{k}$ \\
      
      $L\; ::=$ & $\texttt{[} E_{1},\; \cdots, E_{k} \texttt{]}$ \\
      
      $X\; ::=$ & $\texttt{[} x_{1},\; \cdots,\; x_{k} \texttt{]}$ \for
      $\texttt{[} x_{1},\; \cdots,\; \dots x_{k} \texttt{]}$ \for $\texttt{[}
      \dots x \texttt{]}$ \\
      
      $D_{G}\; ::=$ & $\texttt{define}\; C\; \texttt{=}\; E\; D_{G}$ \for
      $\texttt{define}\; C\; X\; \texttt{=}\; E\; D_{G}$ \for $\texttt{type}\;
      T\; X\; D_{G}$ \for \\ 
      & $\texttt{type}\; T\; X\; \texttt{extends}\; T\; L\; D_{G}$ \for
      $\texttt{type}\; T\; X\; \left\{D_{M}\right\}\; D_{G}$ \for \\
      & $\texttt{type}\; T\; X\; \texttt{extends}\; T\; L\;
      \left\{D_{M}\right\} D_{G}$ \for $\varepsilon$ \\

      $D_{M} ::=$ & $\texttt{define}\; C\; \texttt{=}\; E\; D_{M}$ \for
      $\texttt{define}\; C\; X\; \texttt{=}\; E\; D_{M}$ \for $\texttt{define}\;
      \texttt{abstract}\; C\; D_{M}$ \for \\ 
      & $\texttt{define}\; \texttt{abstract}\; C\; X\; D_{M}$ \for
      $\texttt{data}\; x\; \texttt{=}\; E\; D_{M}$ \for $\varepsilon$ \\

    \end{tabular}  
    \capt{The formulation rules for the syntactic categories of \hspace{0.02cm}
    \productname{}.}
    \label{fig:form-rules} 
  \end{center}
\end{figure}

The $\mathbf{::=}$ means that the left-hand side of the rule can be any one of
the $\for$-separated right-hand sides. Furthermore, we use ``$\cdots$'' to
illustrate a repetition of some element in the rule. We have also used the
slightly different ``$\dots$'' to illustrate the three dots that precede a
variable argument (vars) (this is explained in the report - and has been omitted
from this document). It should be clear from the context which of the two is
being used.

The $\varepsilon$ represents an empty definition.

\chapter{Big-step semantics}

\section{Lists}
The semantics presented in \tableref{semantic:lists} are the transition rules
for lists.

\begin{table}[ht]
  \begin{center}
    \begin{tabular*}{\textwidth}{l l}
      \hline \\
      $[\mbox{LIST}_{\mbox{ACCESS-1}}]$ & \infrule{env_{T},
      env_{C}, env_{V} \vdash \lag E\; \rag \ra v_2  \qquad env_{T}, env_{C},
      env_{V} \vdash \lag L \rag \ra v_3}
      {env_{T}, env_{C}, env_{V} \vdash \lag E\; L \rag \ra v_1} \\
       & where $v_2 = \left(l_1, elem_1\right)$ \\
       & and $v_3 = (l_2,elem_2)$ \\
       & and $l_2 = 1$ \\
       & and $i = elem_2 =0$ \vspace{0.1cm} \\
       & and $v_1 = \left\{
	 \begin{array}{l l}
           elem_1\; i         & \quad \text{if $i \geq 0$}\\
           elem_1\; (l_1 + i) & \quad \text{if $i \geq 0$}
	 \end{array} \right.$ \\
	 
      $[\mbox{LIST}_{\mbox{ACCESS-2}}]$ & \infrule{env_{T},
      env_{C}, env_{V} \vdash \lag E \rag \ra v_2  \qquad env_{T}, env_{C},
      env_{V} \vdash \lag L \rag \ra v_3}
      {env_{T}, env_{C}, env_{V} \vdash \lag E\; L \rag \ra v_1} \\
       & where $v_2 = \left(l_1, elem_1\right)$ \\
       & and $v_3 = (l_2,elem_2)$ \\
       & and $l_2 = 2$ \vspace{0.1cm} \\
       & and $i = \left\{
	 \begin{array}{l l}
           elem_2\; 0         & \quad \text{if $elem_2\; 0 \geq 0$}\\
           l_1 + elem_2\; 0   & \quad \text{if $elem_2\; 0 < 0$}
	 \end{array} \right.$ \vspace{0.1cm} \\
       & and $j = \left\{
	 \begin{array}{l l}
           elem_2\; 1         & \quad \text{if $elem_2\; 1 \geq 0$}\\
           l_1 + elem_2\; 1   & \quad \text{if $elem_2\; 1 < 0$}
	 \end{array} \right.$ \vspace{0.1cm} \\
       & and $elem_3\; z = \left\{
	 \begin{array}{l l}
           elem_1\; i+1       & \quad \text{if $z = 0$}\\
	   \hspace{1cm} \vdots &   \\
           elem_1\; i+n-1     & \quad \text{if $z = n-1$}
	 \end{array} \right.$ \vspace{0.1cm} \\
       & and $v_1 = (n=j-i+1, elem_3)$ \\
       & \\
       \hline
    \end{tabular*}
    \capt{Transition rules for let expressions.}
    \label{semantic:lists}
  \end{center}
\end{table}

\section{Let-expressions}

The semantics presented in \tableref{semantic:let} are the transition rules for
the let expression.  This is transition rule is defined recursively to best
illustrate the functionality of the expression.

\begin{table}[ht]
  \begin{tabular*}{\textwidth}{l l}
    \hline \\
    $[\mbox{LET-1}]$ & \infrule{env_{T}, env_{C}, env_{V}[x_{1}
    \mapsto v_{1}] \vdash \lag \texttt{let}\; x_{2} = E_{2}, \cdots,\; x_{k} =
    E_{k}\; \texttt{in}\; E_{k+1} \rag \ra v_{k+1}}
    {env_{T}, env_{C}, env_{V} \vdash \lag \texttt{let}\; x_{1} = E_{1},\; x_{2}
    = E_{2}, \cdots, x_{k} = E_{k}\; \texttt{in}\; E_{k+1} \rag \ra v_{k+1}} \\
    & where $env_{T}, env_{C}, env_{V} \vdash E_{1} \ra v_{1}$\\
    & and $k \geq 2$ \\
    
    $[\mbox{LET-2}]$ & \infrule{env_{T}, env_{C}, env_{V}[x_{1}
    \mapsto v_{1}] \vdash \lag E_{2}\rag \ra v_{2}} {env_{T}, env_{C}, env_{V} \vdash
    \lag \texttt{let}\; x_{1} = E_{1}\; \texttt{in}\; E_{2} \rag \ra v_{2}} \\
    & where $env_{T}, env_{C}, env_{V} \vdash E_{1} \ra v_{1}$ \\
    & and $k < 2$ \\
    & \\
    \hline \\
  \end{tabular*}
  \capt{Transition rules for let expressions.}
  \label{semantic:let}
\end{table}

The transition rules for $[\mbox{LET-1}]$ is recursively because we must
evaluate each expression $(x_{1}=E_{1})$ before we move on to the next one. This
is a must because of the fact that the next expressions can in fact make use of
the previous expressions value. As an example take a look at the following code
sample:

\codesample{letbigstep.junta}

So, each call where there are more than one expression to be evaluated we call
the transition rule $[\mbox{LET-1}]$ where $k \geq 2$. Here the expression first
in line to be evaluated will be evaluated before a new call to one of the two
transition rules is made. When we reach a let expression with only one
expression then we call the transition rule $[\mbox{LET-2}]$ where $k < 2$.

\section{Lambda expressions}

The semantics presented in \tableref{semantic:lambda} is the transition rule for
the lambda expression. 

\begin{table}[ht]
  \begin{tabular*}{\textwidth}{l l c}
    \hline \\
    $[\mbox{LAMBDA}]$ & $env_{T}, env_{C}, env_{V} \vdash \lag \texttt{\#} \;
    X\; \texttt{=>}\; E \rag \ra v$ & \hspace{1cm} where $v = \left(X, E,
    env_{V}, env_{C}\right)$ \\
    & & \\
    \hline
  \end{tabular*}
  \capt{Transition rules for lambda expressions.}
  \label{semantic:lambda}
\end{table}

The three environments ($env_{T}, env_{C}, env_{V}$) must be known before it is
possible to execute a lambda expression. We need to know which types, constants
and different variables are given in the specific scope.

The lambda expressions evaluates to a value $v$. The side condition of the
transition rule explains that $v$ is assigned the $4$-tuple.

\section{Set expressions}

The semantics presented in \tableref{semantic:set} is the transition rules for
set expressions.

\begin{table}[ht]
  \begin{tabular*}{\textwidth}{l l}
    \hline \\
    $[\mbox{SET}]$ & \infrule{env_C, env_V, env_T \vdash \lag E_1
      \rag\ra u_1 \quad
    \ldots \quad env_C, env_V, env_T \vdash \lag E_k \rag \ra u_k}
    {env_C, env_V, env_T \vdash \lag \texttt{set}\; x_1 = E_1, \ldots, x_k =
    E_k \rag \ra v_1} \\
    & where $env_v\; \texttt{this} = \left(t, env'_c, env'_v, v_2 \right)$ \\
    & and $v_1 = \left( t, env'_c, env'_v, v_2\right)$ \\
    & and $env''_v = env'_v \left[ x_1 \mapsto u_1, \ldots, x_k \mapsto u_k
    \right]$ \\
    & \\
    \hline
  \end{tabular*}
  \capt{Transition rules for set expressions.}
  \label{semantic:set}
\end{table}

\section{Function calls}

The semantics presented in \tableref{semantic:callfun} is transition rules for
function calls.

\begin{table}[ht]
  \begin{tabular*}{\textwidth}{l l}
    \hline \\
    $[\mbox{CALL}_{\mbox{FUN}}]$ & \hspace{0.1cm} $env_C, env_V,
    env_T \vdash \lag E \rag \ra v_2$ \\
    & \hspace{0.1cm} $env_C, env_V, env_T \vdash \lag L \rag \ra v_3$
    \vspace{-0.3cm} \\
    & \infrule{env'_C, env''_V, env_T \vdash \lag E' \rag \ra v_1}{env_C, env_V,
    env_T \vdash \lag E\; L\; \rag \ra v_1} \\
    & where $v_2 = \left(X, E', env'_V, env'_C\right)$ \\
    & and $v_3 = \left(l, elem\right)$ \\
    & and $env''_V = \left[x_1 \mapsto elem\; 1, \ldots, x_n \mapsto elem\; n \right]$ \\
    & \\
    \hline
  \end{tabular*}
  \capt{Transition rules for set expressions.}
  \label{semantic:callfun}
\end{table}

The semantics presented in \tableref{semantic:memaccess} is the transition rule
for member access.

\begin{table}[ht]
  \begin{tabular*}{\textwidth}{l l}
    \hline \\
    $[\mbox{MEMBER}_{\mbox{ACCESS}}]$ & \infrule{env_C, env_V, env_T
    \vdash \lag E \rag \ra v_1}{env_C, env_V, env_T \vdash \lag E\texttt{.}C
  \rag \ra v_3} \\
     & where $v_2 = \left(t, env'_C, env'_V, v_2 \right)$ \\
     & and $env'_C\; C = v_3$ \\
     & \\
     \hline
  \end{tabular*}
  \capt{Transition rules for set expressions.}
  \label{semantic:memaccess}
\end{table}

\section{Type definitions}

The semantics presented in \tableref{semantic:typedef} are the transition rules for type definitions.
These type definitions have some optional arguments which correspond with the
written grammar for these definitions, and this is why there are four transition
rules described.

\begin{table}[ht]
    \begin{tabular*}{\textwidth}{l l}
      \hline \\
      $\left[\mbox{TYPEDEF}\right]$ & \infrule{env_{C} \vdash
      \lag D_{G}, env_{T}[T \mapsto \left(T, X, \varepsilon, \varepsilon,
      \varepsilon \right)] \rag \ra env_{T}'}
      {env_{C} \vdash \lag \texttt{type}\; T\; X\; D_{G},\; env_{T} \rag \ra
      env_{T}'} \\

      $\left[\mbox{TYPEDEF}_{\mbox{BODY}}\right]$ &
      \infrule{env_{C} \vdash \lag D_{G}, env_{T}[T \mapsto \left(T, X, D_{M},
      \varepsilon, \varepsilon \right)] \rag \ra env_{T}'}
      {env_{C} \vdash \lag \texttt{type}\; T\; X\; \left\{D_{M}\right\}\;
      D_{G},\; env_{T} \rag \ra env_{T}'} \\

      $\left[\mbox{TYPEDEF}_{\mbox{EXTEND}}\right]$ &
      \infrule{env_{C} \vdash \lag D_{G}, env_{T}[T_{1} \mapsto \left(T_{1}, X,
      \varepsilon, L, T_{2} \right)] \rag \ra env_{T}'}
      {env_{C} \vdash \lag \texttt{type}\; T_{1}\; X\; \texttt{extends}\;
      T_{2}\; L\; D_{G},\; env_{T} \rag \ra env_{T}'} \\

      $\left[\mbox{TYPEDEF}_{\mbox{EXTEND-BODY}}\right]$ &
      \infrule{env_{C} \vdash \lag D_{G}, env_{T}[T_{1} \mapsto \left(T_{1}, X,
      D_{M}, L, T_{2} \right)] \rag \ra env_{T}'}
      {env_{C} \vdash \lag \texttt{type}\; T_{1}\; X\; \texttt{extends}\;
      T_{2}\; L\; \left\{D_{M}\right\}\; D_{G},\; env_{T} \rag \ra env_{T}'} \\
      \hline \\
    \end{tabular*}
    \capt{Transition rules for type definitions.}
    \label{semantic:typedef}
\end{table}

In the premises of the rules we present a 5-tuple where $env_{T}$ is updated
according to the rule. In three of the four 5-tuples we include the symbol
$\varepsilon$, which denotes that the given position of the symbol is an empty
slot. This is again due to the fact that we have some optional arguments.

The 5-tuple is ordered as follows:

\begin{nlist}
  \item $\mathbf{T_{1}}$ - current type
  \item $\mathbf{X}$ - current type's formal parameters
  \item $\mathbf{D_{M}}$ - member definitions
  \item $\mathbf{L}$ - super type's parameters
  \item $\mathbf{T_{2}}$ - super type
\end{nlist}

Throughout the transition rules we use the 5-tuple to update the type environment.
\end{document}
