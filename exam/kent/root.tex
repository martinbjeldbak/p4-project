\section{Kontekstuelle begrænsninger}

\begin{frame}
\frametitle{Kontekstuelle begrænsninger}
\begin{center}
\begin{itemize}
\item Hvad er kontekstuelle begrænsninger?
\item Hvad kræver Junta?
\item ScopeChecker
\end{itemize}
\end{center}
\end{frame}

%***TYPER***

%DET KAN VI
\begin{frame}[fragile]
\frametitle{Typer}
\begin{lstlisting}
define foo = A[].bar
type A[]{
  define bar = 10
}
\end{lstlisting}
\begin{center}
\begin{itemize}                                  
\item Anvendte typer kan bindes til én og kun én erklæring
\item Typer har de medlemmer, der tilgås
\end{itemize}
\end{center}
\end{frame}

\begin{frame}[fragile]

\begin{figure}[ht]
\begin{center}
\begin{tikzpicture}[level/.style={sibling distance=30mm/#1}]
\node [square] {Type definition}
  child {node [square,xshift=1.5cm]  {Type}}
  child {node [square,xshift=0.5cm]  {Variable list}}
  child {node [rectangle,xshift=-0.5cm] {\textbf{Extends}} edge from parent[dashed]
  	child{node [square] {Type} edge from parent[solid]}
  	child{node [square] {List} edge from parent[solid]}
  	}
  child {node [square,xshift=-1.5cm] {\textit{Type body}} edge from parent[dashed];};
  
\end{tikzpicture}
\end{center}
\capt{The abstract syntax tree for the type definition node.}
\label{ast:typedef}
\end{figure}

\end{frame}

\begin{frame}[fragile]
\frametitle{Typer}
\begin{lstlisting}
define foo = A.bar[3, 7]
type A[]{
  define bar = 10
}
\end{lstlisting}
\begin{center}
\textcolor{red}{
\begin{itemize}                                  
\item Tjekke korrespondance mellem formelle / aktuelle parametre
\end{itemize}
}
\end{center}
\end{frame}

\begin{frame}[fragile]
\frametitle{Typer}
\begin{lstlisting}
define foo = let $a = B[]
               in $a.bar
type A[]{
  define bar = 10
}
type B[]{

}
\end{lstlisting}
\begin{center}
\textcolor{red}{
\begin{itemize}                                  
\item Inferere en variabels type
\end{itemize}
}
\end{center}
\end{frame}

\begin{frame}[fragile]
\frametitle{Typer}
\begin{lstlisting}
define someObject = A[1, 2]
type A[$a, $b]
type B[] extends A[1, 2]
\end{lstlisting}
\begin{center}
\begin{itemize}                                  
\item Instantiering foregår med korrekt antal parametre
\item Subtype kalder supertype med korrekt antal parametre
\item Supertype eksisterer!
\end{itemize}
\end{center}
\end{frame}

\begin{frame}[fragile]
\frametitle{Typer}
\begin{lstlisting}
type A[] extends B[]
type B[] extends C[]
type C[] extends A[]
\end{lstlisting}

\begin{center}                                 
\item Ingen cyklisk nedarvning
\end{center}
\end{frame}

\begin{frame}[fragile]
\begin{figure}[ht]
  \begin{center}
    \begin{tikzpicture}[level/.style={sibling distance=30mm/#1}]      
      \node [square] (a) {A};
      \node [square, yshift=-4em, xshift=-2.5em] (b) {B};
      \node [square, yshift=-4em, xshift=2.5em] (d) {D};
      \node [square, yshift=-8em, xshift=-5em] (c) {C};
      \node [square, yshift=0em, xshift=12em] (e) {E};
      \node [square, yshift=-4em, xshift=12em] (f) {F};

      \draw[<-, thick,] (a) -- (b);
      \draw[<-, thick,] (b) -- (c);
      \draw[<-, thick,] (a) -- (d);
      \draw[<-, thick,] (e) -- (f);
    \end{tikzpicture}
  \end{center}
  \capt{En pil går fra X til Y hvis X er en subtype af Y.}
  \label{fig:topological}
\end{figure}
\end{frame}

\begin{frame}[fragile]
\begin{figure}[ht]
  \begin{center}
    \begin{tikzpicture}[level/.style={sibling distance=30mm/#1}]      
      \node [square] (a) {$A_2$};
      \node [square, yshift=-4em, xshift=-2.5em] (b) {$B_1$};
      \node [square, yshift=-4em, xshift=2.5em] (d) {$D_0$};
      \node [square, yshift=-8em, xshift=-5em] (c) {$C_0$};
      \node [square, yshift=0em, xshift=12em] (e) {$E_1$};
      \node [square, yshift=-4em, xshift=12em] (f) {$F_0$};

      \draw[<-, thick,] (a) -- (b);
      \draw[<-, thick,] (b) -- (c);
      \draw[<-, thick,] (a) -- (d);
      \draw[<-, thick,] (e) -- (f);
    \end{tikzpicture}
  \end{center}
  \capt{En pil går fra X til Y hvis X er en subtype af Y.}
  \label{fig:topological}
\end{figure}
\end{frame}

\begin{frame}[fragile]
\begin{figure}[ht]
  \begin{center}
    \begin{tikzpicture}[level/.style={sibling distance=30mm/#1}]      
      \node [square] (a) {$A_1$};
      \node [square, yshift=-4em, xshift=-2.5em] (b) {$B_0$};
      \node [square, yshift=0em, xshift=12em] (e) {$E_0$};

      \draw[<-, thick,] (a) -- (b);
    \end{tikzpicture}
  \end{center}
  \capt{En pil går fra X til Y hvis X er en subtype af Y.}
  \label{fig:topological}
\end{figure}
 Topologisk sorteret sekvens: C, D, F
\end{frame}

\begin{frame}[fragile]
\begin{figure}[ht]
  \begin{center}
    \begin{tikzpicture}[level/.style={sibling distance=30mm/#1}]      
      \node [square] (a) {$A_0$};
    \end{tikzpicture}
  \end{center}
  \capt{En pil går fra X til Y hvis X er en subtype af Y.}
  \label{fig:topological}
\end{figure}
 Topologisk sorteret sekvens: C, D, F, B, E
\end{frame}

\begin{frame}[fragile] 
 Topologisk sorteret sekvens: C, D, F, B, E, A
 
 	\begin{figure}[ht]
  \begin{center}
    \begin{tikzpicture}[level/.style={sibling distance=30mm/#1}]      
      \node [square] (a) {$A_1$};
      \node [square, yshift=-4em, xshift=-2.5em] (b) {$B_1$};
      \node [square, yshift=-4em, xshift=+2.5em] (c) {$C_1$};
      
      \draw[<-, thick,] (a) -- (b);
      \draw[<-, thick,] (b) -- (c);
      \draw[<-, thick,] (c) -- (a);
    \end{tikzpicture}
  \end{center}
  \capt{Hvad med cyklisk nedarvning?}
  \label{fig:topological}
\end{figure}

 Omvendt: A, E, B, F, D, C
\end{frame}


\begin{frame}[fragile] 
\begin{table}[h]
\begin{tabular}{|l|l|l|}
 \cline{1-1} \cline{3-3} 
Figur          &  & KantFigur : Figur    \\ \cline{1-1} \cline{3-3} 
\textcolor{gray}{areal} &  &                      \\
               &  &  \textcolor{gray}{antalKanter} \\
               &  &                      \\
               &  &                      \\ \cline{1-1} \cline{3-3} 
\end{tabular}
\end{table}
\begin{table}[h]
\begin{tabular}{|l|l|l|}
 \cline{1-1} \cline{3-3} 
Firkant : KantFigur &  & AnvendtFirkant       \\ \cline{1-1} \cline{3-3} 
areal      &  &                      \\
antalKanter     &  &                      \\
\textcolor{gray}{h}          &  & h               \\
\textcolor{gray}{b}          &  & b              \\ \cline{1-1} \cline{3-3} 
\end{tabular}
\end{table}
\end{frame}

\begin{frame}[fragile] 
\begin{table}[h]
\begin{tabular}{|l|l|l|}
 \cline{1-1} \cline{3-3} 
Figur          &  & KantFigur : Figur    \\ \cline{1-1} \cline{3-3} 
\textcolor{gray}{areal} &  & \textcolor{gray}{areal}       \\
               &  & \textcolor{gray}{antalKanter} \\
               &  &                      \\
               &  &                      \\ \cline{1-1} \cline{3-3} 
\end{tabular}
\end{table}
\begin{table}[h]
\begin{tabular}{|l|l|l|}
 \cline{1-1} \cline{3-3} 
Firkant : KantFigur &  & AnvendtFirkant       \\ \cline{1-1} \cline{3-3} 
areal               &  &                      \\
antalKanter         &  &                      \\
\textcolor{gray}{h}          &  & h                    \\
\textcolor{gray}{b}          &  & b                    \\ \cline{1-1} \cline{3-3} 
\end{tabular}
\end{table}
\end{frame}

\begin{frame}[fragile] 
\begin{table}[h]
\begin{tabular}{|l|l|l|}
 \cline{1-1} \cline{3-3} 
Figur          &  & KantFigur : Figur    \\ \cline{1-1} \cline{3-3} 
\textcolor{gray}{areal} &  & \textcolor{gray}{areal}       \\
               &  & \textcolor{gray}{antalKanter} \\
               &  &                      \\
               &  &                      \\ \cline{1-1} \cline{3-3} 
\end{tabular}
\end{table}
\begin{table}[h]
\begin{tabular}{|l|l|l|}
 \cline{1-1} \cline{3-3} 
Firkant : KantFigur &  & AnvendtFirkant       \\ \cline{1-1} \cline{3-3} 
areal               &  & areal                \\
antalKanter         &  & antalKanter          \\
\textcolor{gray}{h}          &  & h                    \\
\textcolor{gray}{b}          &  & b                    \\ \cline{1-1} \cline{3-3} 
\end{tabular}
\end{table}
\end{frame}

\begin{frame}[fragile] 
\begin{table}[h]
\begin{tabular}{|l|l|l|}
 \cline{1-1} \cline{3-3} 
Figur          &  & KantFigur : Figur    \\ \cline{1-1} \cline{3-3} 
\textcolor{gray}{areal} &  & \textcolor{gray}{areal}       \\
               &  & \textcolor{gray}{antalKanter} \\
               &  &                      \\
               &  &                      \\ \cline{1-1} \cline{3-3} 
\end{tabular}
\end{table}
\begin{table}[h]
\begin{tabular}{|l|l|l|}
 \cline{1-1} \cline{3-3} 
Firkant : KantFigur &  & AnvendtFirkant       \\ \cline{1-1} \cline{3-3} 
areal               &  & areal                \\
antalKanter         &  & antalKanter          \\
\textcolor{gray}{h}          &  & h                    \\
\textcolor{gray}{b}          &  & b                    \\ \cline{1-1} \cline{3-3} 
\end{tabular}
\end{table}
\begin{center}
\texttt{AnvendtFirkant[].areal} $\rightarrow$ ok

\texttt{Firkant[].h} $\rightarrow$ \textcolor{red}{ikke tilladt}

Abstrakte typer identificeres
\end{center}
\end{frame}

%CONSTANTS
\begin{frame}[fragile]
\frametitle{Variabler}
\begin{lstlisting}
let $a = 2, $b = 3 
  in ...

add[$a, $b] = ...

#[$a, $b] => ...

type A[]{
 data $a
 data $b
 ...
}
\end{lstlisting}
\begin{center}
\begin{itemize}                                  
\item Åbner nyt scope
\item Variabler tilføjes til nuværende scope
\end{itemize}
\end{center}
\end{frame}

\begin{frame}[fragile]
\frametitle{Variabler}
\begin{lstlisting}
let $a = 2, $b = 3 in 
  let $a = 5, $f = 7 in
    $a + $b + $f
\end{lstlisting}
\begin{center}
\fig[height=5em]{../../report/pictures/scope2}{En stak af symboltabeller, tillader nestede let-in-udtryk}

Variabel erklæres, som allerede findes i aktivt scope = \textcolor{red}{ScopeError}.

\end{center}
\end{frame}

%VARIABLER


