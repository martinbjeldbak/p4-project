\documentclass{memoir}
\usepackage{fontspec}

\begin{document}

\begin{description}
  \item[Mødetidspunkter:] \hfill \\
  \begin{itemize}
  \item 9-16 man-tor (9-14 fre)
  \item Er man forhindret i at møde eller har behov for at få tidligere fri, så skal dette noteres i den fælles Google Calendar.
  \item Engang imellem er der mulighed for at arbejde hjemmefra.
  \end{itemize}
  
  \item[Forsinkelser:] \hfill \\
  \begin{itemize}
  \item Er man forhindret i at møde i grupperummet, så skal dette oplyses så hurtigt som muligt til gruppen ved skrive en mail til d402f13@cs.aau.dk
  \end{itemize}
  
  \item[Pause:] \hfill \\
  \begin{itemize}
  \item Ingen fastlagt tidsrum til pauser. De holdes, når vi føler, at der er brug for en pause.
  \end{itemize}
  
  \item[Gruppearbejde:] \hfill \\
  \begin{itemize}
  \item Der benyttes scrum til arbejdsprocessen og der holdes derfor ''daily scrum``-møder i begyndelsen af hver dag. Maks. 15 min.
  \item ''Sprint planning´´-møder holdes i begyndelsen af hver sprint. Maks. 2 timer
  \item ''Sprint review´´-møde holdes, når der afsluttes et sprint. Maks. 3 timer
  \end{itemize}
  
  \item[Dagsorden:] \hfill \\
  \begin{itemize}
  \item Til vejledermøder benytter vi en fastlagt dagsorden. Her diskuteres følgende punkter: status, arbejdsblade, hvad sker der så, evt, planlægning af næste møde, evaluering af møde.
  \end{itemize}
  
  \item[Deadlines og planlægning:] \hfill \\
  \begin{itemize}
  \item Der benyttes scrum til udviklingsmetoden
  \item Deadline for projektaflevering er 29. maj 2013
  \item Der bruges 5 sprints á 3 ugers varighed (der er 16 uger mellem 6. feb og 29. maj)
  \end{itemize}
  
  \item[Vejledermøder:] \hfill \\
  \begin{itemize}
    \item Der skal være et vejledermøde ca. en gang om ugen, hvor gruppen vil gennemgå en planlagt dagsorden med vejlederen.
  \end{itemize}
 
\end{description}

\end{document}
