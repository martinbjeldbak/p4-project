\chapter*{Preface}
\label{chap:preface}

This report describes the development of a domain-specific programming
language called \productname{} which was designed to program playable board games. The 
specific topic was to design, define, and implement a programming language for 
generic game playing. 

The project and the developed software, including all source code and
this report, can be found at the following address:

\begin{quote}
  \url{https://bitbucket.org/Acolarh/p4-project/src/}  
\end{quote}

Throughout this project we had been following three courses from which we have
gained the needed knowledge and help to develop the programming language. The
courses are; Languages and Compilers, Principles of Operating Systems and
Concurrency, and Syntax and Semantics.

For this project we have developed a scanner, parser, scope checker and
an interpreter, along with a graphical interface that makes playing
games possible. These are all written in Java, and basic understanding
for programming is needed to understand code examples. Furthermore,
we also give examples of code written in \productname{} which is
illustrated differently from the Java code. This will be explained once
it is used in the report.

This report is typeset in \XeTeX{} using the open-source EB Garamond
font for the body, and Microsoft's Consolas font for any monospaced
text, such as code listings and teletyped text. Headlines use the free
Optimus Princeps font.

\section*{Reading manual}

This report is divided into chapters wherein we have sections about specific
subjects relevant to the project.

Chapter \ref{chap:requirements} presents a long list of requirements which
\productname{} must respect. This chapter can be skipped without losing any
important information to understand the following chapters. The requirements are 
derived from the previous \chapref{chap:analysis}. The requirements are used as
a checklist to make sure that we end up with a useful product.

\section*{Special thanks to}

We wish to thank our supervisor for his patience and guidance throughout
the project. We also wish to thank Hans Hüttel for reading semantics of
\productname{} through and providing constructive criticism.

%This means that we must analyse board games and try to outline which aspects of
%board games our language must be able to describe. These are aspects like how
%many players can play? What is the board size? Turns? Possible actions? Winning
%condition?

%So what is the goal of this project? Why do we need a new programming language
%for generic game playing - is it not possible to code games in C or Java? Yes,
%it is possible to code games in already existing programming languages but the
%goal of the project is that the students gain knowledge of important underlying
%concepts in the world of programming languages. How are these concepts derived?
%How are they formally described and represented in an implementation? 

%Obviously, all software is written in some kind of a programming language and
%compiled or interpreted so it can be executed. Design, definition and
%implementation of programming languages is a central topic of Computer Science.
%By gaining a better understanding of these topics the student will be able to
%grasp the possibilities of different programming languages and programming
%paradigms and what their differences are.\cite[p. 22]{dat-stud-ordning} We will
%discuss different paradigms in \secref{sec:paradigms}. 

%The goal is that:
%\begin{quote}
%the student must learn how to design and implement a programming language and
%how this process can be supported by formal definitions of the languages syntax
%and semantics and the techniques and methods to construct a translater for the
%language.\cite[p. 22]{dat-stud-ordning}
%\end{quote}

%This report presents and documents the process and work of which we've been
%through to reach this goal.
