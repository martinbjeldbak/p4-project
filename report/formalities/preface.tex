\chapter*{Preface}
\label{chap:preface}
This project describes the development of a domain-specific programming language that can be used to describe board games. The specific topic was to design, define, and implement a programing language for generic game playing. 

%This means that we must analyse board games and try to outline which aspects of board games our language must be able to describe. These are aspects like how many players can play? What is the board size? Turns? Possible actions? Winning condition?

So what is the goal of this project? Why do we need a new programming language for generic game playing - is it not possible to code games in C or Java? Yes, it is possible to code games in already existing programming languages but the goal of the project is that the students gain knowledge of important underlying concepts in the world of programming languages. How are these concepts derived? How are they formally described and represented in an implementation? 

Obviously, all software is written in some kind of a programming language and compiled or interpreted so it can be executed. Design, definition and implementation of programming languages is a central topic of Computer Science. By gaining a better understanding of these topics the student will be able to grasp the possibilities of different programming languages and programming paradigms and what their differences are.\cite[p. 22]{dat-stud-ordning} We will discuss different paradigms in \secref{sec:paradigms}. 

The goal is that:
\begin{quote}
the student must learn how to design and implement a programming language and how this process can be supported by formal definitions of the languages syntax and semantics and the techniques and methods to construct a translater for the language.\cite[p. 22]{dat-stud-ordning}
\end{quote}

This report presents and documents the process and work of which we've been through to reach this goal.