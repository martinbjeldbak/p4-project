\chapter{Abstract Node Types}
\label{ap:ant}

In \secref{sec:ant} we presented a few examples of how we have implemented the
grammar described throughout \chapref{chap:design}. Here we present the rest of the
implementations of these abstract node types (ANTs). 

\section{Program structure}
The following sections present the structure of programs. The sections are 
structured as follows:

\begin{dlist}
  \item Program
  \item Definition
  \item Constant definition
  \item Type definition
  \item Type body and member definition
  \item Abstract definition
  \item Variable list
\end{dlist}

The ANT for program was presented in \secref{sec:ant}.

\section{Definition}
Programmers have the opportunity of defining constants and types in their
programs.  The parser knows that the following piece of code is a constant when
it meets the keyword \gter{define}, and that the following piece of code is a
type when it meets the keyword \gter{type}. These two types of definitions will
be illustrated with their productions in the following two sections. The 
production for a defintion is as follows:

\begin{ebnf}
\grule{definition}{constant\_def}
\galt{type\_def}
\end{ebnf}

We do not present a figure to illustrate the AST for this production. The
production says that a defintion can either be a defintion of a constant or of a
type. But the program node says that it is possible to have multiple
definitions, so in fact the definitions of constants and types can be
interleaved.

\section{Constant definition}
The first type of definition is that of a constant. The production for a
this definition is as follows:

\begin{ebnf}
\grule{constant\_def}{\gter{define} \gcat constant \gcat \gopt{varlist} \gcat \gter{=} \gcat \gcat expression}
\end{ebnf}

Every definition of this type begins with the \gter{define} keyword. The next
token must be a constant and this constant can be followed by a list of
variables, which is optional. Then an expression is assigned to the constant.

\begin{figure}[ht]
\begin{center}
\begin{tikzpicture}[level/.style={sibling distance=30mm/#1}]
\node [square] {Constant definition}
  child {node [square,xshift=0.7cm] {Constant}}
  child {node [square] {Variable list} edge from parent [dashed]}
  child {node [square,xshift=-0.7cm] {\textit{Expression}}};
\end{tikzpicture}
\end{center}
\capt{The abstract syntax tree for the constant definition node.}
\label{ast:constdef}
\end{figure}


Figure \ref{ast:constdef} illustrates the ANT for the constant definition. In
the grammar presented above the production also consists of a keyword and an
assignment symbol. These are not shown in the ANT because they are not part 
of an AST (but they are part of a parse tree). The optional node is illustrated 
with a dashed edge from the parent.

\section{Type definition}
\label{ap:typedef}

The second type of definition is that of a type. The following production
specifies this:

\begin{ebnf}
\grule{type\_def}{\gter{type} \gcat type \gcat varlist \gcat \gopt{\gter{extends} \gcat type \gcat list} \gopt{type\_body}}
\end{ebnf}

The parser knows that the following is a type definition by reading the
\gter{type} keyword. The production says that a type definition can inheret from
another type by the use of the \gter{extends} keyword. Furthermore, it is
possible to have a type body which is explained further in the following
section.

\input{billeder/tikz/typedef}




Figure \ref{ast:typedef} illustrates the ANT of the type
definition. This ANT consists of a type followed by a variable
list. After the variable list there are two optional choices. The first choice
is to inheret from a super type, and this is done by declaring the type of the super
type followed by a declaration of a list. The list following the super type is
not optional. We have illustrated this by adding a ``\textbf{Extends}'' node as
optional folllowed by the two child nodes which are not optional any more if the
extends keyword is present. Our implementation of these child nodes is not
implemented with an extra node as we illustrate in \figref{ast:typedef}, but it 
is simply defined in the code that if the
keyword is present then the next input must be a type followed by a list.
The second choice is to include a type body. These two choices are independent
of each other.

\section{Type body and member definition}

In this section we have coupled two productions because they are intertwined
and we have chosen not to visualise the ANT of the member definition because it
is simply a root with one child. The production of a type definition and the
definition of a member is as follows:

\begin{ebnf}
\grule{type\_body}{\gter{\{} \gcat \grep{member\_def} \gcat \gter{\}}}
\grule{member\_def}{abstract\_def}
\galt{constant\_def}
\end{ebnf}

The type body begins with a left brace and ends with a right brace which
encapsulates the entire code of the type body. Within these braces it is
possible to have zero, one, or more member definitions. The code which consists
of a member definition can be either an abstract definition or a constant
definition. As it was in the program production, it is also possible to
interleave the two definitions in this production.

\begin{figure}[ht]
\begin{center}
\begin{tikzpicture}[level/.style={sibling distance=30mm/#1}]
\node [square] {Type body}
  child {node [square,xshift=-0.2cm] (a) {\textit{Member definition}}}
  child {node [square,xshift=0.2cm] (b) {\textit{Member definition}}};
  
\path (a)--(b) node [midway] {$\cdots$};
\end{tikzpicture}
\end{center}
\capt{The abstract syntax tree for the type body node.}
\label{ast:typebody}
\end{figure}

Figure \ref{ast:typebody} illustrates the ANT of the type body. It consists of a
root and zero, one, or more member definitions.

A description of the constant definition was presented earlier in this section
and the description of the abstract definition will follow this section.

\section{Abstract definition}
The definition of an abstract constant is similar to the definition of a
``regular'' constant. There are two differences. The first difference is that
following the \gter{define} keyword another keyword must appear, namely the
\gter{abstract} keyword. The second difference is that there is no expression
assigned at the end of the production, thus being abstract. The following
production presents the production from the grammar:

\begin{ebnf}
\grule{abstract\_def}{\gter{define} \gcat \gter{abstract} \gcat constant \gcat \gopt{varlist}}
\end{ebnf}

Figure \ref{ast:abstrdef} illustrates the ANT of this production.


\begin{figure}[ht]
\begin{center}
\begin{tikzpicture}[level/.style={sibling distance=30mm/#1}]
\node [square] {Abstract definition}
  child {node [square,xshift=0.3cm] {Constant}}
  child {node [square,xshift=-0.3cm] {Variable list} edge from parent[dashed]};
\end{tikzpicture}
\end{center}
\capt{The abstract syntax tree for the abstract definition node.}
\label{ast:abstrdef}
\end{figure}


In \figref{ast:abstrdef} the only noticeable difference between the constant and 
a definition is that there is no expression as the last child.  This is due to
that the ANTs do not have keywords and such as child nodes.


\section{Variable list and variable arguments}
Within this section we have couple two productions because they are intertwined. 
The productions are presented in the following grammar:

\begin{ebnf}
\grule{varlist}{\gter{[} \gcat \gopt{variable \gcat \grep{\gter{,} \gcat variable} \gcat \gopt{\gter{,} \gcat vars} \gor vars} \gcat \gter{]}}
\grule{vars}{\gter{...} \gcat variable}
\end{ebnf}

The production specifies that a variable list begins with an occurence of a left
bracket and ends with an occurence of a right bracket. Everything between these
two brackets are optional, which means that a variable list can be empty.

The last part of the production is a bit hard to read, but it says that if the
list begins with one or more variables separated by commas then these variables
can be followed by a final comma and the variable argument (vars) which is a
list of parameters. If there are no variables then the last variable argument
can be included. So, the production consists of two parts - one part containing
zero, one, or more variables possibly followed by an occurence of variable
argument or just a single variable argument.


\begin{figure}
\begin{center}
\begin{tikzpicture}[level/.style={sibling distance=30mm/#1}]
\node [square] {Variable list}
  child {node [square, xshift=0.5cm] (a) {Variable} edge from parent[dashed];}
  child {node [square] (b) {Variable} edge from parent[dashed];}
  child {node [square, xshift=-0.5cm] {Variable arguments} edge from parent[dashed];};
\path (a)--(b) node [midway] {$\cdots$};
\end{tikzpicture}
\end{center}
\capt{The abstract syntax tree for the variable list node.}
\label{ast:variablelist}
\end{figure}


Figure \ref{ast:variablelist} illustrates this by having a root and three
children with dashed lines from the root. But this is actually a bit misleading
because of the optional parameters in the specification of the production. The
production specifies that if the programmer does not want the list to be empty
then at least one occurence of a variable must be present or a single variable
argument. After the first variable it is possible to have zero, one, or more
variables following the first one and another optional variable argument. The
easiest way to illustrate this is as it has been done in
\figref{ast:variablelist}.

This concludes the sections presenting the structure of programs.

\section{Expressions}
The following seven sections present the different expressions in
\productname{}. The sections are structured as follows:

\begin{dlist}
  \item Operations with precedence and negation
  \item Element
  \item Member access
  \item Call sequence
  \item Assignment
  \item If expression
  \item Lambda expression
  \item List
  \item No-operator
\end{dlist}

The ANT for operations and assignments were presented in \secref{sec:ant}.

\section{Element}
The following grammar presents the production for an element:

\begin{ebnf}
\grule{element}{call\_sequence \gcat \grep{member\_access}}
\end{ebnf}

An element begins with a call sequence followed by zero, one, or more member
access. This is illustrated differently in the ANT in \figref{ast:element}.


\begin{figure}[ht]
  \begin{center}\scalebox{0.7}{
\begin{tikzpicture}[level/.style={sibling distance=30mm/#1}]
\node [square] {Element}
  child {node [square, xshift=-1cm] {Call sequence}}
  child {node [square, xshift=-0.3cm] {Member access}}
  child {node [square,xshift=0.3cm] (b) {Member access} edge from parent[dashed]}
  child {node [square,xshift=1cm] (c) {Member access} edge from parent[dashed]};
  
\path (b)--(c) node [midway] {$\cdots$};
\end{tikzpicture}}
\end{center}
\end{figure}


Figure \ref{ast:element} illustrates that the ANT for an element expression will
only be constructed if there is at least one occurence of member access,
otherwise it will fall through to call sequence and the element node will not be
construted. As mentioned earlier in the section explaining operations in
\productname{}, this is done to avoid long paths in the AST with single child
nodes.

\section{Member access}

The member access begins with the \gter{.} symbol followed by a constant and
zero, one, or more lists. This is presented in the following production:

\begin{ebnf}
\grule{member\_access}{\gter{.} \gcat constant \gcat \grep{list}}
\end{ebnf}

Figure \ref{ast:memberaccess} illustrates the abstract node type for this
production. It corresponds exactly to the production.

\begin{table}[ht]
  \begin{tabular*}{\textwidth}{l l}
    \hline \\
    \hspace{3cm} $[\mbox{MEMBER}_{\mbox{ACCESS}}]$ & \infrule{env_C, env_V, env_T
    \vdash \lag e \rag \ra v_1}{env_C, env_V, env_T \vdash \lag e\texttt{.}C
  \rag \ra v_3} \\
     & where $v_2 = \left(t, env'_C, env'_V, v_2 \right)$ \\
     & and $env'_C\; C = v_3$ \\
     & \\
     \hline
  \end{tabular*}
  \capt{Transition rules for member access.}
  \label{semantic:memaccess}
\end{table}



\section{Call sequence}

A call sequence consists of an atomic value, which can be many expressions,
followed by zero, one, or more lists. The following production presents this:

\begin{ebnf}
\grule{call\_sequence}{atomic \gcat \grep{list}}
\end{ebnf}

We have illustrated the ANT in \figref{ast:callsequence}. It does not correspond
exactly to the production specified above.


\begin{figure}[ht]
\begin{center}
\begin{tikzpicture}[level/.style={sibling distance=30mm/#1}]
\node [square] {Call sequence}
  child {node [square,xshift=2.3cm] {\textit{Atomic}}}
  child {node [square,xshift=0.8cm] {List}}
  child {node [square,xshift=-0.8cm] (a) {List} edge from parent[dashed];}
  child {node [square,xshift=-2.3cm] (b) {List} edge from parent[dashed];};
  
\path (a)--(b) node [midway] {$\cdots$};
\end{tikzpicture}
\end{center}
\capt{The abstract syntax tree for the call sequence node.}
\label{ast:callsequence}
\end{figure}



Once again, \figref{ast:callsequence} illustrates that this ANT must consist of
at least one list following the atomic value if it is to be constructed.
Otherwise, it will fall through and only construct an atomic node. This is due
to the same reason as earlier, to avoid single-child nodes.

\section{If expression}
The if expression is specified in the following production:

\begin{ebnf}
\grule{if\_expr}{\gter{if} \gcat expression \gcat \gter{then} \gcat expression \gcat \gter{else} \gcat expression}
\end{ebnf}

An if expression begins with the keyword $\gter{if}$ and ends with the keyword
$\gter{else}$ followed by an expression. In between the beginning and the end the
keyword, $\gter{then}$ appears as a must in the if expression. It is for instance not
possible to omit the $\gter{else}$ statement. The ANT for this production is
illustrated in \figref{ast:ifexpr} and consists of a root with three children
which are expressions. The children are connected to the root with solid lines
which means they cannot be omitted.


\begin{figure}
\begin{center}
\begin{tikzpicture}[level/.style={sibling distance=30mm/#1}]
\node [square] {If expression}
  child {node [label={[xshift=1.2cm, yshift=0.2cm]\textbf{if}}] [ellipse,draw] {\textit{Expression}}}
  child {node [label={[xshift=-0.38cm, yshift=0.2cm]\textbf{then}}] [ellipse,draw] {\textit{Expression}}}
  child {node [label={[xshift=-2.2cm, yshift=0.2cm]\textbf{else}}] [ellipse,draw] {\textit{Expression}}};
\end{tikzpicture}
\end{center}
\capt{The abstract syntax tree for the if expression node.}
\label{ast:ifexpr}
\end{figure}


\section{Lambda expression}

The lambda expression is specified in the following production:

\begin{ebnf}
\grule{lambda\_expr}{\gter{\#} \gcat varlist \gcat \gter{=>} \gcat expression}
\end{ebnf}

Any lambda expression begins with the $\gter{\#}$ symbol which makes it clear for the
parser that this is a lambda expression. A lambda expression consists of a
variable list and an expression seperated by the $\gter{=>}$ symbol.

\begin{table}[ht]
  \begin{tabular*}{\textwidth}{l l c}
    \hline \\
    \hspace{0.5cm} $[\mbox{LAMBDA}]$ & $env_{T}, env_{C}, env_{V} \vdash \lag
    \texttt{\#} \; g\;
    \texttt{=>}\; e \rag \ra v$ & \hspace{1cm} where $v = \left(g, e, env_{V}, env_{C}\right)$ \\
    & & \\
    \hline
  \end{tabular*}
  \capt{Transition rules for lambda expressions.}
  \label{semantic:lambda}
\end{table}



Figure \ref{ast:lambdaexpr} illustrates the ANT for this expression. Because a
lambda expression is an expression it is possible to have lambda expressions
inside other lambda expressions, so it is possible to nest these expressions.

\section{List}

We also have an ANT for lists. Any list can be empty
just like a variable list but instead of containing variables a list contains
expressions. It can consist of either zero, one, or more expressions. The
specification for the abstract node type is specified in the following
production:

\begin{ebnf}
\grule{list}{\gter{[} \gcat \gopt{expression \gcat \grep{\gter{,} \gcat expression}} \gcat \gter{]}}
\end{ebnf}

A list begins and ends like a variable list with brackets with optional
expressions as the elements of the list. The ANT for this production is illustrated
in \figref{ast:list}.


\begin{figure}
\begin{center}
\begin{tikzpicture}[level/.style={sibling distance=30mm/#1}]
\node [square] {List}
  child {node [ellipse,draw] (a) {\textit{Element}} edge from parent[dashed]}
  child {node [ellipse,draw] (b) {\textit{Element}} edge from parent[dashed]};

\path (a)--(b) node [midway] {$\cdots$};
\end{tikzpicture}
\end{center}
\capt{The abstract syntax for the list node.}
\label{ast:list}
\end{figure}


The ANT in \figref{ast:list} illustrates that a list can contain zero, one, or
more expressions.

\section{Not-operator}

We have implemented a not-operator that has to do with expressions. The programming
language also conists of a not-operator which has to do with patterns. These
opreators are not the same. The operator mentioned here cannot be used with
patterns. 

This operator is not illustrated because it just derives an expression. So, it 
has one root and one child.

\section{Patterns}

The following sections are concerned with patterns and operators. A pattern
consists of a single pattern expression followed by zero, one, or more pattern
expressions.  This is specified in the following production:

\begin{ebnf}
\grule{pattern}{pattern\_expr \gcat \grep{pattern\_expr}}
\end{ebnf}

This production is very straightforward, a pattern must contain of at least one
pattern expression. The ANT for this production is illustrated in
\figref{ast:pattern}. This ANT has one solid line to the first pattern
expression followed by two dashed lines to two other pattern expressions
illustrating that these can occur zero, one, or more times.


\begin{figure}[ht]
\begin{center}
\begin{tikzpicture}[level/.style={sibling distance=40mm/#1}]
\node [square] {Pattern}
  child {node [square,xshift=0.5cm] {\textit{Pattern expression}}}
  child {node [square] (b) {\textit{Pattern expression}} edge from parent[dashed]}
  child {node [square,xshift=-0.5cm] (c) {\textit{Pattern expression}} edge from parent[dashed]};

\path (b)--(c) node [midway] {$\cdots$};
\end{tikzpicture}
\end{center}
\capt{The abstract syntax tree for the pattern node.}
\label{ast:pattern}
\end{figure}


\section{Pattern operators}

\productname{} consists of three different pattern operators. We have an
or-operator, a multiplier-operator, and a not-operator.

The or-operator derives a pattern value followed by a pattern expression that
both can be many different things. This is why they are written in an italic
font. This is illustrated in \figref{ast:pattern-or}.


\begin{figure}[ht]
\begin{center}
\begin{tikzpicture}[level/.style={sibling distance=40mm/#1}]
\node [square] {Pattern, or-operator}
  child {node [ellipse,draw,xshift=0.2cm] {\textit{Pattern value}}}
  child {node [ellipse,draw,xshift=-0.2cm] {\textit{Pattern expression}}};
\end{tikzpicture}
\end{center}
\capt{The abstract syntax tree for the pattern or-operator node.}
\label{ast:pattern-or}
\end{figure}


The multiplier-operator derives a single child namely a pattern value. This is
not illustrated because it is just a root with one child.

The not-operator also derives a single child namely a pattern check. This is not
illustrated because it is just a root with a single child.

A ``pattern value'' and a ``pattern check'' can derive different things and
would be illustrated with italic font.

This concludes the description of our implementation of the different ANTs which 
are reflected in the productions of the grammar presented throughout
\chapref{chap:design}.

