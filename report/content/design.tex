\chapter{Design}
\label{chap:design}

\section{Code example - Naughts and Crosses}

We have developed \productname{} by first brainstorming and writing a bunch of
game implementations using a ``programming language'' which felt the most
natural to us. By this we mean that we were actually using the programming
language before we had even constructed it. We began by writing programs in the
unfinished language to try to find out how it should be built and what would be
the easiest to write. The following is a result of this and it is an example of an
implementation of the board game Noughts and Crosses:

\codesample{noughtandcrosses.game}

Based on our implementations, we first define the formal grammar of
\productname{} in \secref{sec:grammar}. Afterwards, the contextual constraints
are informally described  in xxx. Furthermore, we describe the\dots in xxx. Lastly, the 
\ldots are described in xxx.

\section{Grammar}

\subsection{Notational Conventions}
We use a variant of Extended Backus-Naur Form to express the context-free grammar of
our programming language.

Each production rule assigns an expression of terminals, non-terminals and operations
to a non-terminal. E.g. in the following example the non-terminal $decimal$ is assigned
the possible terminals of $\gter{0}$ up to and including $\gter{9}$.

\begin{ebnf}
\grule{decimal}{\gter{0} \gor \gter{1} \gor \grange \gor \gter{9}}
\end{ebnf}

The following operations are used throughout this section:
\begin{center}
\begin{tabular}{r l}
  $\gopt{pattern}$ & an optional pattern \\
  $\grep{pattern}$ & zero or more repititions of pattern \\
  $\ggrp{pattern}$ & a group \\
  $pattern_1 \gor pattern_2$ & a selection \\
  $\gter{0} \gor \grange \gor \gter{9}$ & a range of terminals \\
  $pattern_1 \gex pattern_2$ & matched by $pattern_1$ but not by $pattern_2$\\
  $pattern_1 \gcat pattern_2$ & concatenation of $pattern_1$ and $pattern_2$ \\
  $\gter{test}$ & a terminal \\
  $\gtsq$ & a terminal single quotation mark \\
  $\gtdq$ & a terminal double quotation mark \\
  $\gtbs$ & a terminal backslash character \\
\end{tabular}
\end{center}


\subsection{Character Classes}
\begin{ebnf}
\grule{decimal}{\gter{0} \gor \gter{1} \gor \grange \gor \gter{9}}
\grule{lowercase}{\gter{a} \gor \gter{b} \gor \grange \gor \gter{z}}
\grule{uppercase}{\gter{A} \gor \gter{B} \gor \grange \gor \gter{Z}}
\grule{anycase}{lowercase \gor uppercase}
\grule{quotebs}{\gtdq \gor \gtbs}
\grule{unichar}{\gcomment{any unicode character}}
\grule{strchar}{unichar \gex quotebs}
\end{ebnf}
\subsection{Reserved Tokens}
\begin{ebnf}
%reserved
\grule{keyword}{\gter{game} \gor \gter{piece} \gor \gter{this} \gor \gter{width} \gor \gter{height}}
\galt{\gter{title} \gor \gter{players} \gor \gter{turnOrder} \gor \gter{board}}
\galt{\gter{grid} \gor \gter{setup} \gor \gter{wall} \gor \gter{name} \gor \gter{possibleDrops}}
\galt{\gter{possibleMoves} \gor \gter{winCondition} \gor \gter{tieCondition}}
\grule{operator}{\gter{and} \gor \gter{or} \gor \gter{not}}
\grule{pattern\_keyword}{\gter{friend} \gor \gter{foe} \gor \gter{this} \gor \gter{empty}}
\grule{pattern\_operator}{\gter{*} \gor \gter{?} \gor \gter{+} \gor \gter{!}}
\end{ebnf}
\subsection{Literals}
\begin{ebnf}
%Literals
\grule{integer}{deimal \gcat \grep{decimal}}
\grule{direction}{\gter{n} \gor \gter{s} \gor \gter{e} \gor \gter{w} \gor \gter{ne} \gor \gter{nw}}
\galt{\gter{se} \gor \gter{sw}}
\grule{coordinate}{upperccase \gcat \grep{uppercase} \gcat decimal \gcat \grep{decimal}}
\grule{string}{\gtdq \gcat \grep{strchar \gor \gtbs \gcat unichar} \gcat \gtdq}
\end{ebnf}
\subsection{Identifiers}
\begin{ebnf}
%Identifiers
\grule{function}{lowercase \gcat anycase \gcat \grep{anycase}}
\grule{identifier}{uppercase \gcat \grep{anycase}}
\grule{variable}{\gter{\$} \gcat anycase \gcat \grep{anycase}}
\end{ebnf}
\subsection{Program Structure}
\begin{ebnf}
%Program structure
\grule{program}{\grep{function\_def} \gcat game\_decl}
\grule{function\_def}{\gter{define} \gcat function \gcat \gter{[} \gcat \grep{variable} \gcat \gter{]} \gcat expression}
\grule{game\_decl}{\gter{game} \gcat declaration\_struct}
\grule{declaration\_struct}{\gter{\{} \gcat declaration \gcat \grep{declaration} \gter{\}}}
\grule{declaration}{\ggrp{keyword \gor identifier} \gcat structure}
\grule{structure}{declaration\_struct \gor expression}
\end{ebnf}
\subsection{Expressions}
\begin{ebnf}
%Expressions
\grule{expression}{function\_call}
\galt{element \gcat operator \gcat expression}
\galt{if\_expr}
\galt{lambda\_expr}
\galt{element}
\galt{\gter{not} \gcat expression}
\grule{element}{\gter{(} \gcat expression \gcat \gter{)}}
\galt{variable}
\galt{list}
\galt{pattern}
\galt{keyword}
\galt{direction}
\galt{coordinate}
\galt{integer}
\galt{string}
\galt{identifier}
\galt{function}
\grule{function\_call}{function \gcat list}
\grule{if\_expr}{\gter{if} \gcat expression \gcat \gter{then} \gcat expression \gcat \gter{else} \gcat expression}
\grule{lambda\_expr}{\gter{\#} \gcat \gter{[} \gcat \grep{variable} \gcat \gter{]} \gcat \gter{=>} \gcat expression}
\grule{list}{\gter{[} \gcat \grep{element} \gcat \gter{]}}
\end{ebnf}
\subsection{Patterns}
\begin{ebnf}
%Patterns
\grule{pattern}{\gter{/} \gcat pattern\_expr \gcat \grep{pattern\_expr} \gcat \gter{/}}
\grule{pattern\_expr}{pattern\_val \gcat \gopt{\gter{*} \gor \gter{?} \gor \gter{+}}}
\grule{pattern\_val}{direction}
\galt{variable}
\galt{pattern\_check}
\galt{\gter{!} \gcat pattern\_check}
\galt{\gter{(} \gcat pattern\_expr \gcat \grep{pattern\_expr}  \gcat \gter{)} \gcat \gopt{integer}}
\grule{pattern\_check}{\gter{friend}}
\galt{\gter{foe}}
\galt{\gter{empty}}
\galt{\gter{this}}
\galt{identifier}
\end{ebnf}

\section{Types}
\label{sec:types}

\productname{} has support for the following types: Integer, String, Direction, Coordinate,
Function, Pattern, List and Action.

There is no $null$-type or $null$-value, since all expressions must have a value. This
is also evident in the definition of the $if$-expression in \secref{sec:grammar}, in that
all $if$-expressions must have the $else$-branch.

\subsection{Integer}
The integer-type in \productname{} represents a 32-bit integer.

An integer-value can be created using an integer-literal, such as \texttt{2155} or \texttt{0}.

\subsection{String}
The string-type represents a UTF-8 encoded string.

A string-value is created using a string-literal, such as \texttt{"Hello, World!"} or \texttt{""}.

\subsection{Direction}
The direction-type represents a direction on a game board. It works like a vector
in the sense that they can be combined to compute new directions. The basic directions
are \texttt{n}, \texttt{e}, \texttt{s} and \texttt{w} (north, east, south and west).
On a 2-dimensional grid (such as for chess) north is up, east is right, south is down and 
west is left.

The directions \texttt{ne}, \texttt{nw},
\texttt{se} and \texttt{sw} are also available, although these could also be produced
by combining the basic directions (e.g. \texttt{n + e = ne}). An example of a direction combination is the
expression \texttt{n + n + e} which produces
the direction shown in \figref{fig:direction_nne}. This direction could also be produced by
\texttt{n + ne} or \texttt{ne + n}.

\fig{direction_n}{The \texttt{n}-direction.}

\fig{direction_nne}{The \texttt{n + n + e}-direction.}

\subsection{Coordinate}
This type represents a position in a grid, i.e. on the game board. It is created with
a coordinate-literal, e.g. \texttt{A1} or \texttt{AH23}. The first part (the alphabetical part)
represents the column (or x-value), i.e. \texttt{A} means column $1$ and \texttt{AH} means
column $1 \cdot 26 + 8 = 34$. The second part (the numeric part) represents the row (or y-value).

\fig{coordinates}{All coordinates on an $8 \times 8$ board.}

\subsection{Function}
Functions in \productname{} are first-class citizens, meaning that they can be used as any
other value. A function name without a list of arguments results in a reference to that
function. Function references can be passed as arguments to other functions or as a return
value.

Another way to create function references, is to use anonymous functions in the form
of lambda expressions. A lambda expression is created by combining a list of input-variables
with an expression, like so:

\codesample{lambdaexpression.garry}

In the example above, a lambda expression is assigned to to \variable{max}-variable, before
being called as a function in line 2. The \texttt{\#}-symbol is used to mark the beginning
of a lambda expression. The scope rules of lambda expressions are further described in \secref{sec:scoping}
while the declaration of named functions is described in \secref{sec:functions}.

\subsection{Boolean}
Although there are no boolean constants (``true'' and ``false'') in \productname, the type still exists, since
boolean values can be returned by some expressions. Boolean values are also required in the condition of
if-expressions.

\subsection{Pattern}
A unique feature of \productname{} is patterns.
a pattern used for matching patterns on the board.

\subsection{List}
A list is an ordered collection of values. The same value may occur more than once. 

List values may be accessed by called the list as if it was a function.
The following expression will return the element at offset 1 in the list
($5$):

\codesample{listaccess1.junta}

Ranges of elements can also be returned. For instance in the following
expression a new list is returned containing elements from offset 1 up
to and including 2 (the list $["is", "a"]$):

\codesample{listaccess2.junta}

\subsection{Identifier}
An identifier is a reference to a certain object. It always begins with an uppercase letter.
Consider for instance the \productname-implementation
of Noughts and Crosses. The implementation uses three identifiers; \identifier{Noughts},
\identifier{Crosses}, and \identifier{XOPiece}. An identifier doesn't have to be declared, and no matter
where it is used, it always refers to the same object. Two differently named identifiers can't refer
to the same object. Some uses of an identifier may add more identity to the object referred to by that
identifier. Consider for example the line:

\codesample{nncplayers.garry}

In this line, the objects referred to by \identifier{Noughts} and \identifier{Crosses}, are declared
to be players. This means that further on, when using \identifier{Noughts} and \identifier{Crosses},
they can only be used in contexts where a player-object is applicable.

The other identifier, \identifier{XOPiece}, is used as a name for the one type of piece in
Noughts and Crosses:

\codesample{nncpiece.garry}

This means that further on \identifier{XOPiece} will refer to that type of piece.


\subsection{Action}

Something about monads here...

\section{Scoping}
\label{sec:scoping}

A scope is the context in which one or more variables exist.
There are three types of scopes in \productname{}. Function scopes, lambda scopes and
``let''-scopes.

\begin{dlist}
\item What is scoping?
\item Examples of static/lexical versus dynamic scoping
\item Why do we want to use dynamic scoping
\item What does that mean for \productname?
\end{dlist}

\subsection{Function scope}
Consider a function definition such as:

\codesample{functiondef.garry}

The variables \variable{a} and \variable{b} only exist within the function \function{max}.
When calling the function:

\codesample{functioncall.garry}

A new scope will be created and the values $5$ and $23$
are assigned to \variable{a} and \variable{b}, respectively.

Named functions (such as \function{max}) always exist in the global scope.

\subsection{Lambda expression scope}

When a lambda expression is created, a reference to the scope it was created
in is saved with it. This is known as a closure, and means that a lambda
expression may access variables outside of its own scope. The accessible
variables are the variables that were available at the time of the creation
of the lambda expression.

Consider the following example:

\codesample{closuredef.garry}

A function \function{getAdder}, which takes one argument (\variable{a}) and
returns a lambda expression, is defined. Notice how \variable{a} is used within
the lambda expression. This means that when the lambda expression is created, it
must remember the value of the variables that exist in the scope, in which it is
created. The use of the \function{getAdder}-function could look like this:

\codesample{closureuse.garry}

In the first line \function{getAdder} is called with the argument, $25$. A new scope, $A$,
is created, in which the variable \variable{a} is assigned the value $25$. Then the function
expression is evaluated, which results in a new lambda expression (with a reference to scope $A$).
This is returned and assigned to \variable{adder} in line 1 of the above example.

In the second line the \variable{adder} is called as a function, which means that a new scope, $B$,
is created, in which the variable \variable{b} is assigned the value $5$. The important part is
that $B$'s parent scope is set to $A$ (which is saved with the lambda expression). The expression
(the right side of the lambda expression) is then evaluated. First the \variable{a}-variable is
encountered. The interpreter first searches the $B$-scope for \variable{a}, and when unsuccessful,
searches the parent-scope, $A$, for \variable{a}. In $A$ the variable \variable{a} holds the value
$25$, and this is returned. Then the $B$-scope is searched for the \variable{b}-variable, and the value
$5$ is returned. The two integers are added, and the final return-value of the lambda-expression
ends up being $30$.

\subsection{Let-expressions}

\productname{} only supports \emph{single assignment}. Single assignment is not assignment
in the traditional imperative sense, but rather a way of binding a value to a symbol in a
certain scope. This is done using \emph{let-expressions}. Using a let-expression creates a
new scope in which the declared variables are accessible. When leaving the scope the
variables are destroyed.

The basic format of a let-expression is:

\texttt{let VARIABLE1 = EXPRESSION1 in EXPRESSION2}

In the example, the value of \texttt{EXPRESSION1} is assigned to \texttt{VARIABLE1}, which
is available in \texttt{EXPRESSION2}. Another example could be:

\texttt{let VARIABLE1 = EXPRESSION1, VARIABLE2 = EXPRESSION2 in EXPRESSION3}

In this example, the value of \texttt{EXPRESSION1} is assigned to \texttt{VARIABLE1}, and
the value of \texttt{EXPRESSION2} is assigned to \texttt{VARIABLE2}. Both \texttt{VARIABLE1}
and \texttt{VARIABLE2} are available in \texttt{EXPRESSION3} and only in \texttt{EXPRESSION3}.

Destructive assignments are not possible \productname{}, meaning that it isn't possible to 
reassign a variable. It is however possible to hide a variable.
Consider the following expression:

\codesample{lethiding.garry}

The value of this expression is $13$. This is because within the \texttt{\variable{x} + 2}-expression
the \variable{x}-variable evaluates to $6$. But in the outer expression \variable{x} evaluates to
$5$.

Nested \emph{let}-scopes are possible. Consider for instance:

\codesample{nestedlet.garry}

In the inner scope, both \variable{x} and \variable{y} are available. This is of course equivalent
to:

\codesample{nestedlet2.garry}



\section{Operators}

\productname{} supports a number of built-in operators. In this section the operators
of \productname{} are described using the format:

\operator[LeftOperandType]{operator}{RightOperandType}{ReturnType}

A star (\opstar) means that a value of any type is applicable as an
operand.

The available types are described in \secref{sec:types}.
Operations that are not described in this section can be considered invalid.

\subsection{Operator precedence}

The operator precedence is a set of rules clarifying which operations, in an expression,
should be performed first. The reason for operator precedence in programming languages,
is to. 

\tab[\textwidth]{operatorPrecedence}{2}{The precedence of operators in \productname{}.}
         {Operator precedence}
  {Level}{Operator & Description}{
    \tabrow{1}{\texttt{f[]} & Function/constructor invocation and list access}
    \tabrow{2}{\texttt{r.m r.m[]} & Record member access and member invocation}
    \tabrow{3}{\texttt{-} & Unary negation operation}
    \tabrow{4}{\texttt{* / \%} & Multiplication, division, and modulo}
    \tabrow{5}{\texttt{+ -} & Addition and subtraction}
    \tabrow{6}{\texttt{< > <= >=} & Comparison operators}
    \tabrow{7}{\texttt{== != is} & Equality operators and type checking}
    \tabrow{8}{\texttt{and or} & Logical $and$ and $or$}
    \tabrow{9}{\texttt{not} & Logical $not$}
    \tabrow{10}{\texttt{if let \#} & if-, let-, and lambda-expressions}
}

\subsection{Boolean operators}

These operators only accept boolean operands and only return boolean values.
\begin{dlist}
  \item \operator[Boolean]{and}{Boolean}{Boolean}\\
    Returns true when both operands are true and false otherwise. 
  \item \operator[Boolean]{or}{Boolean}{Boolean}\\
    Returns true when at least one of the operands are true and false otherwise.
  \item \operator{not}{Boolean}{Boolean}\\
    Returns true if the single operand is false and false otherwise.
\end{dlist}

\subsection{Comparison operators}

These operators are used when comparing two values, they will always return
boolean values.
\begin{dlist}
  \item \operator[Integer]{<}{Integer}{Boolean}\\
    Returns true if the left operand is less than the right one.
  \item \operator[Integer]{>}{Integer}{Boolean}\\
    Returns true if the left operand is greater than the right one.
  \item \operator[Integer]{<=}{Integer}{Boolean}\\
    Returns true if the left operand is less than or equal to the right one.
  \item \operator[Integer]{>=}{Integer}{Boolean}\\
    Returns true if the left operand is greater than or equal to the right one.
  \item \operator[\opstar]{==}{\opstar}{Boolean}\\
    Returns true if the left operand is equal to the right one.
  \item \operator[\opstar]{!=}{\opstar}{Boolean}\\
    Returns true if the left operand is not equal to the right one.
  \item \operator[\opstar]{is}{Type}{Boolean}\\
    Returns true if the type of the first operand is equal to or inherits from
    the type operand.
\end{dlist}

\subsection{Integer operators}

The following operations are possible on integers:
\begin{dlist}
  \item \operator{-}{Integer}{Integer} \\
    Integer negation.
  \item \operator[Integer]{+}{Integer}{Integer} \\
    Integer addition.
  \item \operator[Integer]{-}{Integer}{Integer} \\
    Integer subtraction.
  \item \operator[Integer]{*}{Integer}{Integer} \\
    Integer multiplication.
  \item \operator[Integer]{/}{Integer}{Integer} \\
    Integer division.
  \item \operator[Integer]{\%}{Integer}{Integer} \\
    Integer modulo operation.
\end{dlist}

\subsection{String operators}

It is possible to concatenate strings:
\begin{dlist}
  \item \operator[String]{+}{String}{String} \\
    Returns the concatenation of two strings.
  \item \operator[String]{+}{\opstar}{String} \\
   \operator[\opstar]{+}{String}{String} \\
    Returns the concatenation of a string and the string-representation of another type
\end{dlist}

\subsection{List operators}

Some operators are available for list values as well:
\begin{dlist}
\item \operator[List]{+}{List}{List} \\
  Returns a list containing all elements from the first list followed
  by all elements from the second list.
\item \operator[List]{-}{List}{List} \\
  Returns a list containing all the elements from the first list that
  do not exist in the second list.
\item \operator[List]{+}{\opstar}{List} \\
  Appends any element on to the end of a list, and returns the resulting list.
\item \operator[List]{-}{\opstar} \\
  Returns a list containing the elements that do not match the right operand.
\item \operator[\opstar]{+}{List}{List} \\
  Prepends any element on to the start of a list, and returns the resulting list.
\end{dlist}

\subsection{Direction and coordinate operators}

It is possible to manipulate directions and coordinates using the following operator:
\begin{dlist}
  \item \operator[Direction]{+}{Direction}{Direction} \\
    Add a direction (vector) to another direction.
  \item \operator[Direction]{-}{Direction}{Direction} \\
    Subtract a direction from another direction.
  \item \operator[Direction]{+}{Coordinate}{Coordinate} \\
    Add a coordinate to a direction.
  \item \operator{-}{Direction}{Direction} \\
    Negate a direction.
  \item \operator[Coordinate]{-}{Coordinate}{Direction} \\
    Returns the distance between two coordinates as a direction.
  \item \operator[Coordinate]{+}{Direction}{Coordinate} \\
    Add a direction to a coordinate. 
  \item \operator[Coordinate]{-}{Direction}{Coordinate} \\
    Subtract a direction from a coordinate.
\end{dlist}
For instance adding the directions \texttt{n} and \texttt{e} produces a
direction equivalent with the direction \texttt{ne}. Adding a coordinate and direction (and visa versa) gives a coordinate. As an example, $\texttt{A2} \verb!+! \texttt{e}$ gives \texttt{B2}. More information
about the coordinate and direction types is available in \secref{sec:types}.



\section{Functions}

Functions are a big part of \productname, and are required for most calculations and operations.

In \productname functions are first-class citizens, meaning that functions are treated
as any other value in the language. Therefore one can pass functions as
arguments to other functions or returning them as values. It is also possible to create
anonymous functions (using lambda expressions), and pass these to functions.

\subsection{Standard environment}

Many functions are made available to the programmer in \productname.

\subsection{User-defined functions}

A user can define custom functions for use in the implementation of a board game.
This is done using the $define$-keyword. Functions can be declared to accept
any number of parameters or none at all. An example of a function definition could
be:

\codesample{functiondef.garry}


\section{Patterns}
\label{sec:patterns}

This section covers how to use patterns and what to use them for. The operators of a pattern looks like and behaves a little like regular expressions. This EBNF-grammar shows how a pattern is constructed:

\begin{ebnf}
\grule{pattern}{pattern\_expr \gcat \grep{pattern\_expr}}
\grule{pattern\_expr}{pattern\_val \gcat \gopt{\gter{*} \gor \gter{?} \gor \gter{+}}}
\galt{pattern\_val \gcat \gter{|} \gcat pattern\_expr}
\grule{pattern\_val}{direction}
\galt{variable}
\galt{pattern\_check}
\galt{\gter{!} \gcat pattern\_check}
\galt{\gter{(} \gcat pattern \gcat \gter{)} \gcat \gopt{integer}}
\grule{pattern\_check}{\gter{friend}}
\galt{\gter{foe}}
\galt{\gter{empty}}
\galt{\gter{this}}
\galt{type}
\end{ebnf}


A pattern is checked on a particular square, and returns either true or false. An example of a pattern is \text{/n n e empty/}. This pattern can be checked on the board seen in \figref{patternboard} on the field \textbf{A1}. The pattern says ``go one square north, go one square north, go one square east, check if square is empty''. This means that the square \textbf{B3} will be checked for emptiness. Since the square is occupied by a piece, the pattern will return false if checked on \textbf{A1}. However, the same pattern checked on \textbf{C1} will return true since the square at \textbf{D3} is empty.
\fig[scale=2]{patternboard}{A simple 4 X 4 board with 3 pieces}

With this basic introduction to a simple pattern check, the table (insert ref and convert list below to a table) describes briefly how the different pattern operators work. For each operator, an example of the use in a context are given in \secref{sec:patternexamples}. Note that the description of patterns assumes a minor understanding of regular expressions, see (ref http://www.regular-expressions.info/).

\texttt{empty} : current square contains no pieces\\
\texttt{n} : north \\
\texttt{e} : east \\
\texttt{s} : south \\
\texttt{w} : west \\
\texttt{*} : zero-to-many times\\
\texttt{+} : one-to-many times\\
\texttt{?} : zero-or-one time\\
\texttt{|} : or\\
\texttt{!} : not\\
\texttt{(} \textit{pattern} \texttt{)} : encapsulation\\
\texttt{friend} :  current square contains at least one friendly piece of the current player\\
\texttt{foe} : current square contains at least one enemy piece of the current player\\
\textit{type} : the current square is of the given type or a piece of the given type is residing on the current square.
\texttt{this} : current square contains the piece for which the check is being performed\\

\subsection{Pattern examples}
\label{sec:patternexamples}
All these examples of pattern checks are performed on the board and pieces seen in \figref{patternboard}. For each operator, two examples of a pattern check on a particular square is given. The first check returns true and the second check false.

On A1, the pattern check \texttt{/empty/} returns true because A1 is empty\\
On B2, the pattern check \texttt{/empty/} returns false because B2 is not empty\\
On A1, the pattern check \texttt{/n empty/} returns true because A2 is empty\\
On B1, the pattern check \texttt{/n empty/} returns false because B2 is not empty\\
On C3, the pattern check \texttt{/e empty/} returns true because C4 is empty\\
On C4, the pattern check \texttt{/e empty/} returns false because C5 is out of board\\
On A1, the pattern check \texttt{/n* n e empty/} returns true because moving north 2 times then north and east hits an empty square on B4\\

Notice that the *-operator causes many branches to be made. The previous pattern check, \texttt{/n* n e empty/} done on A1, has a branch checking \texttt{/n e empty/}. The branch dies because B2 is not empty. If just one branch survives, the pattern check returns true. In the example, the only branch surviving is the \texttt{n n n e empty} branch. The same rules for branching counts for the \texttt{+}, \texttt{?} and \texttt{|} operator. When a branch moves out of board it dies immediately.

On C3, the pattern check \texttt{/n* s s !empty/} returns false because neither A1 nor A2 contains a piece\\
On A1, the pattern check \texttt{/n+ e empty/} returns true only because B4 is empty\\
On B1, the pattern check \texttt{/n+ empty e !empty/} returns false because B4 is the only empty square north of B1 and C4 is empty\\
On B3, the pattern check \texttt{/s? e empty/} returns true only because C2 is empty\\
On C2, the pattern check \texttt{/n? w empty/} returns false because neither B2 nor B3 is empty\\
On A2, the pattern check \texttt{/(n | e) empty/} returns true only because A3 is empty\\
On C2, the pattern check \texttt{/e | w empty/} returns false because neither B2 nor C3 is empty\\
On B1, the pattern check \texttt{/(n n | e e) empty/} returns true only because D1 is empty\\
On A1, the pattern check \texttt{/(n w)* empty/} returns true both because A1 is empty and because D4 is empty\\

The \texttt{friend}-keyword is evaluated based on the current player. Suppose that we have a player called Green, who owns the 
green piece. If it is green's turn to move, any branch of a pattern check will return false
whenever it meets a keyword \texttt{friend} on a square that does not contain any of Green's pieces.
On B2, the pattern check \texttt{/n|e|(n e) friend/} will return true if it is Green's turn, since C3 contains a friendly piece of Green.
On C3, the pattern check \texttt{/(e | w)+/} will return false if it is Green's turn, since no square containing a piece of Green can be reached by going east or west from C3 one or more times.

The keyword \texttt{foe} does the opposite of \texttt{friend}. It makes a branch continue only if its current square contains at least one piece not owned by the player who has the turn.

Just like \texttt{foe} and \texttt{friend}, the name of a piece or square-type defined in a \productname{}-game can also be used. E.g, the keyword \texttt{White} can be used if a piece or square-type with the name \textit{White} has been define. In a such case, a branch survives only if its current square is of type \textit{White} or if the square contains a piece of type \textit{White}.

A pattern can also check that a specific piece exists on a specific square. This is done using the \texttt{this} keyword.
Before this check can be achieved, the pattern must be checked regarding to a specific piece. Suppose that we on A3, make the pattern check
\texttt{empty e this}. If this check is done in relation to the black piece on B3 it returns true. However, in relation to the black piece on B2, the pattern check returns false. To understand both how this function exactly works and why this is useful, consider the board in \figref{fig:patternboard}.
If we for any piece specify that it can move to a square for which the pattern check \texttt{/(n | s) empty/} is true, this means that it can move to any square except $\{B1, B4, C4\}$. These square does not have an empty square north or east from it. Recall that a branch going out of board dies.
However, the pattern check \texttt{/empty (n | s) this/} will in relation to the green piece return true only when checked on the squares $\{C2, C4\}$.
This can be used to specify that a piece can move to an empty square one north or one south from its current square. 
\section{Structural operational semantics}
\label{sec:structuraloperationalsemantics}

In the following subsections we will describe how programs, written in
\productname{}, behave. We begin by defining the abstract syntax for our
language which consists of a list of different syntactic categories followed by
formulation rules. When these have been presented we can move on to the
description of our operational semantics which is our big-step
semantics.\cite[pg. 27]{tt-hh}

\section{Abstract syntax}

The abstract syntax is the interpreter or compiler's internal representation of a program. It is represented
as an abstract syntax tree.

This section should cover all aspects of our abstract syntax tree, and how it differs from the
parse tree (e.g. there are no expressions or elements in the AST).


\section{Big-step semantics}
There are two kinds of operational semantics. we will only use and describe the
big-step semantics for \productname{}. The reason for not describing small-step
semantics is taht we do not find i necessary. The big-step semantics will
capture the operational semantics of programs written in \productname{}.

We have divided the following subsections into expressions, lists, variable
lists, and definitions. In each section we describe the construction of the
big-step semantics by showing the transition rules for each construct.

\subsection{Expressions}

\begin{table}[ht]
  \begin{center}
    \begin{tabular*}{\textwidth}{lc}
      $[\mbox{LET}]$ & \infrule{env_{V} \vdash E_{1} \ra
      v_{1} \quad env_{V}\left[x \mapsto v_{1}\right] \vdash E_{2} \ra
      v_{2}}{env_{V} \vdash \texttt{let} x = E_{1} \texttt{in} E_{2} \ra v_{2}}
    \end{tabular*}
  \end{center}
\end{table}



%      $[\mbox{MINUS}]$ & \infrule{s \vdash E_{1} \ra v_{1} \: s \vdash E_{2} \ra
%      v_{2}}{s \vdash E_{1} - E_{2} \ra v} & where $v=v_{1}-v_{2}$ \\
 %     $[\mbox{MULTIPLICATION}]$ & \infrule{s \vdash E_{1} \ra v_{1} \: s \vdash E_{2} \ra
 %     v_{2}}{s \vdash E_{1} * E_{2} \ra v} & where $v=v_{1}*v_{2}$ \\
 %     $[\mbox{DIVISION}]$ & \infrule{s \vdash E_{1} \ra v_{1} \: s \vdash E_{2} \ra
 %     v_{2}}{s \vdash E_{1} / E_{2} \ra v} & where $v=v_{1}/v_{2}$ \\
 %     $[\mbox{MODULO}]$ & \infrule{s \vdash E_{1} \ra v_{1} \: s \vdash E_{2} \ra
 %     v_{2}}{s \vdash E_{1} \% E_{2} \ra v} & where $ $ \\

%\begin{table}[ht]
%  \begin{center}
%    \begin{tabular*}{\textwidth}{lc}
%      $[\mbox{RULE}]$ & \infrule{\lag Something, Something \rag \ra
%      something}{\lag something, something, something \rag \ra something } \\
%    \end{tabular*}
%  \end{center}
%\end{table}


%\subsection{Lists}

%\begin{table}[ht]
%  \begin{center}
%    \begin{tabular*}{\textwidth}{lc}
%     $[\mbox{RULE}]$ & \infrule{\lag Something, Something \rag \ra
%      something}{\lag something, something, something \rag \ra something } \\
%    \end{tabular*}
%  \end{center}
%\end{table}


%\subsection{Variable lists}

%\begin{table}[ht]
%  \begin{center}
%    \begin{tabular*}{\textwidth}{lc}
%      $[\mbox{RULE}]$ & \infrule{\lag Something, Something \rag \ra
%      something}{\lag something, something, something \rag \ra something } \\
%    \end{tabular*}
%  \end{center}
%\end{table}


%\subsection{Definitions}

%\begin{table}[ht]
%  \begin{center}
%    \begin{tabular*}{\textwidth}{lc}
%      $[\mbox{RULE}]$ & \infrule{\lag Something, Something \rag \ra
%      something}{\lag something, something, something \rag \ra something } \\
%    \end{tabular*}
%  \end{center}
%\end{table}


