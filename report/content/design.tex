\chapter{Design}
\label{chap:design}

In this chapter we present the design of our programming language
(\productname{}). The abstract syntax of \productname{} will be presented in
\secref{sec:abstractsyntax} followed by the lexical structure of programs in
\secref{sec:lexicalstructure}. We also present the different expressions and
their grammar in \secref{sec:expressions} followed by the different definitions
and their grammar in \secref{sec:definitions}. Futhermore, we introduce patterns
in \secref{sec:patterns} which are very important in \productname{}.  Finally,
we present the predefined types and constants of \productname{} in
\secref{sec:predefined} where the standard and game environments are presented.

In this chapter a number of terms are used, when referring to different aspects
of the programming language. Essentially the language consists of
\emph{constants}, \emph{functions} and \emph{types}.

A type is a structure, that may inherit from a \emph{super
type}/\emph{parent type} and contains \emph{members} in the form of \emph{data
fields} and \emph{constants}. A constant may be a function in which case it is
referred to as a \emph{method}. Types and constants can be \emph{abstract}. A
type can be instantiated using its \emph{constructor}, this creates an
\emph{instance} or \emph{object} of that type. All values in \productname{} are
instances of types.

We begin this chapter by giving an example of a game implemented in
\productname{}:

\subsection*{Code example}
\label{codesample}

The following is an example of how a program is written in \productname{}. It is
an implementation of the board game Noughts and Crosses:

\codesample{noughtandcrosses.game}

\productname{} is a purely functional object-oriented programming language
designed explicitly for creating board games. We have developed \productname{}
by first brainstorming and writing a bunch of game implementations using a
``programming language'' which felt the most natural to us. This means that we 
were actually using the programming language before we had even constructed
it. We began by writing programs in the unfinished language to try to find out
how it should be built and what would be the easiest to write. 

In the following we will go through the \csref{noughtandcrosses.game}
and introduce some of the important concepts of \productname{}. These will be
further described in the rest of this chapter and in
\chapref{chap:implementation}.

The very first thing that is visible in \csref{noughtandcrosses.game} are the
two lines of comments. Comments are made with two forwardslashes. Comments are
described further in \secref{sec:comments}.

The next thing that happens is the declaration of a type \keyword{type}
\type{NacGame}[].  The types of \productname{} can be compared to classes as
seen in other object-oriented programming languages. The square brackets in
\productname{} are used to encapsulate parameters and to encapsulate the members
of lists.  \keyword{type} \type{NacGame}[] extends the super type \type{Game}
which is a built-in type in \productname{}. The \keyword{extends} keyword is
similar to the extends keyword of Java, and means the \type{NacGame}[] type
inherits members of the \type{Game} type.  \type{Game}'s constructor takes the
title of the game as input and it contains many useful constants and
functions which are described further in \secref{sec:predefined}. 

Constants and functions can be thought of as subprograms similar to methods.
They distinguish themselves from each other by the fact that constants cannot
take any parameters whereas functions can. Functions and constants are further
explained in \secref{sec:constantdefinitions}. One of the built-in constants is
\constant{players} which contains a list of players. In the code sample at line
four we see how the constant is defined. When a game, created in
\productname{}, is played, the turn is shifted between each of the players in
the list provided by the \constant{players} constant. This is true, unless the turn
order is specifically modified by the constant \constant{turnOrder}, which is
another built-in constant in the \type{Game} type. This however is not necessary
in a Noughts and Crosses game, since the turn order is very simple. For instance
if the first player must be able to make three turns before the second player
can make one, that would be defined in the \constant{turnOrder}.

Furthermore, \type{Game} contains the constant \constant{initialBoard} which in
this case is assigned a grid board (another built-in type that takes the height
and width of the board as parameters) of $3 \times 3$ squares. In other
programming languages the override keyword is used when implementing methods
from a super class but in \productname{} the override functionality
already exists in the define keyword. For instance in
\csref{noughtandcrosses.game} \constant{players} and \constant{initialBoard} are
overridden in line four and eight, respectively.

%A second type \type{NacPlayer}, is declared which extends the built-in
%\productname{} type \type{Player}. The \type{Player} type takes as input the name of
%the player and contains three important functions: \function{winCondition},

The next thing that is important in the code sample are the three functions: 
\function{winCondition}, \function{tieCondition} and \function{actions}. 

As the name indicates the \function{winCondition} checks if the current player
is in a winning state and returns a boolean value; true or false.
\function{winCondition} takes a game object as input. In Noughts and Crosses the
win condition is obtained if a player has three of his pieces in a row, in
either a vertical, horizontal or diagonal line. In \productname{} this is
specified by using what we call ``patterns'' which are explained in
\secref{sec:patterns}. Patterns begin and end with forwardslashes. A pattern for
\function{winCondition} can be seen through line 13 and 14.

The \function{tieCondition} function checks if a tie condition is obtained and
returns a boolean value; true or false. The tie condition is achieved whenever
the board is full. This is specified using the built-in function;
\function{isFull}. 

The last function \function{actions} also takes as input a game object and
contains a list of actions. In this case the only possible action is the
\function{addAction} function which makes it possible to add a piece to the
board of type \keyword{this}, which is the current player's piece type (crosses
or noughts), to an empty square on the board.

At first sight the code sample will look complicated. This is mostly due to the
overwhelming use of built-in functionality. This is however implemented to make
it easier and faster for programmers to write code in \productname{}, since they
don't have to implement all the functionality themselves.

\subsection*{Type system}

In \secref{sec:typesystemanalysis} we analysed the two main type system
approaches.  We chose the dynamic type system due to the fact that it increases
the writability of our programming language. As seen in the above code example
\secref{codesample} a Noughts and Crosses game can be created in only
approximately 20 lines of code.

The type system in \productname{} comprises a number of simple types, from which
every other type can be created. The simple types are; integers, character
strings, booleans, lists, directions (vectors), coordinates (points), patterns,
functions and types. Unlike in other programming languages, both functions and
types in \productname{} are first-class citizens, this means that they can
be passed around, used and returned as any other value. This adds even more
flexibility to the language. For instance in the Noughts and Crosses example
(see \csref{noughts-and-crosses.junta}) both the types \type{Noughts} and
\type{Crosses} are passed as values to the constructor of \type{NacPlayer}. The
simple types, their operators and their methods are further explained in the
rest of this chapter.

User types can be created using the \keyword{type}-keyword. They are very
similar to classes in traditional object-oriented languages, in that a type has
a constructor, attributes, constants and methods. An important aspect of the
type system of \productname{} however, is that all values are immutable. It
doesn't matter if it's an integer, a list, or an instance of a custom user type,
the value of the object can't be changed. They can however be cloned and
modified using various techniques depending on the type in question. For
instance, adding two integers using the \texttt{+}-operator returns a new
integer representing the sum of the two operands. For instances of user types a
new modified instance is returned when using the \keyword{set}-keyword (as
described in \secref{sec:setexpressions} and \secref{sec:typedefinitions}). The
reason for this functionality, is to prevent side-effects, since randomly
changing objects, could have undesirable influence on other functions or types
that depend on these objects.

The type system of \productname{} supports single inheritance, meaning that a
user type \type{B} can extend the type \type{A} and inherit all members and 
\productname{} supports single inheritance between types. An inheriting type, will
inherit all the members of its super type(s) (if \type{C} extends \type{B}, which
extends \type{A}, then \type{C} inherits all members from both \type{B} and
\type{A}). Visibility in \productname{} is implicit, in that all
constants/methods are public (they can be accessed from anywhere as long as an
instance of the type is available) while all data fields are private (they can
only be accessed/changed from within the specific type, not even from inheriting
types). Getters and setters are necessary in order to access data fields from
the outside or in inheriting types. More details on data, inheritance, members
and abstract members are available in \secref{sec:setexpressions} and
\secref{sec:typedefinitions}.

Another feature of \productname{} is implicit casting, when dealing with simple
types (such as integers and strings). If a user were to create a type
\type{MyInteger} extending the built-in type \type{Integer} (the type of
integers in \productname{}), then instances of \type{MyInteger} could be used in
place of simple integers, this works by casting the instance to a simple integer
value (simply by throwing away the extra information provided by the
\type{MyInteger}-type). Explicit casting is only really possible with simple
types, since their type constructors accepts one parameter of the same type. For
instance the constructor \type{Integer} accepts a parameter of type
\type{Integer}, meaning that it also accepts types that extend \type{Integer}.
This makes it possible to cast a value of type \type{MyInteger} to a raw
\type{Integer} value, albeit the usefulness of this functionality is dubious.
The constructors of all the simple types implement this functionality however
(as described in \secref{sec:standardenvironment}).

\subsection*{Scope rules}

A scope is the context in which one or more variables or constants exist. In
\productname{} we for instance have different expressions with their own scopes
where their variables live and die. By this we mean that when the scope of the
expression ends, the variables within the scope cannot be accessed anymore.
These expressions with scopes will be defined in \secref{sec:expressions}.

Furthermore, it is important to know that \productname{} uses statical scoping.
Scope rules where described in \secref{sec:anal-scoperules}.

\section{Abstract syntax}

The abstract syntax is the interpreter or compiler's internal representation of a program. It is represented
as an abstract syntax tree.

This section should cover all aspects of our abstract syntax tree, and how it differs from the
parse tree.

\subsection{Program}
\begin{figure}[H]\begin{center}\begin{tikzpicture}[level/.style={sibling distance=40mm/#1}]
\node [square] (z){Program}
  child {node [square] (a) {Function definition} edge from parent[dashed];}
  child {node [square] (aa) {Function definition} edge from parent[dashed];}
  child {node [square] (b) {Game decleration}};
\path (a)--(aa) node [midway] {$\cdots$};
\end{tikzpicture}\end{center}\end{figure}

\subsection{Variable list}
\begin{figure}[H]\begin{center}\begin{tikzpicture}[level/.style={sibling distance=30mm/#1}]
\node [square] (z){Variable list}
  child {node [square] (a) {Variable} edge from parent[dashed];}
  child {node [square] (aa) {Variable} edge from parent[dashed];};
\path (a)--(aa) node [midway] {$\cdots$};
\end{tikzpicture}\end{center}\end{figure}

\subsection{Function definition}
\begin{figure}[H]\begin{center}\begin{tikzpicture}[level/.style={sibling distance=30mm/#1}]
\node [square] (z){Function definition}
  child {node [square] (a) {Function}}
  child {node [square] (b) {Variable list}}
  child {node [ellipse,draw] (c) {\textit{Expression}}};
\end{tikzpicture}\end{center}\end{figure}

\subsection{Game decleration}
\begin{figure}[H]\begin{center}\begin{tikzpicture}[level/.style={sibling distance=30mm/#1}]
\node [square] (z){Game decleration}
  child {node [square] (a) {Decleration struct}};
\end{tikzpicture}\end{center}\end{figure}

\subsection{Decleration struct}
\begin{figure}[H]\begin{center}\begin{tikzpicture}[level/.style={sibling distance=30mm/#1}]
\node [square] (z) {Decleration struct}
  child {node [square] (a) {Decleration}}
  child {node [square] (b) {Decleration} edge from parent[dashed];}
  child {node [square] (c) {Decleration} edge from parent[dashed];};
  
\path (b)--(c) node [midway] {$\cdots$};
\end{tikzpicture}\end{center}\end{figure}

\subsection{Decleration}
\begin{figure}[H]\begin{center}\begin{tikzpicture}[level/.style={sibling distance=30mm/#1}]
\node [square] {Decleration}
  child {node [ellipse split,draw] {Keyword \nodepart{lower} Identifier}}
  child {node [ellipse,draw] {\textit{Structure}}};
\end{tikzpicture}\end{center}\end{figure}

\subsection{Assignment}
\begin{figure}[H]\begin{center}\begin{tikzpicture}
[level/.style={sibling distance=40mm},
level 1/.style={sibling distance = 39mm},
level 2/.style={sibling distance = 20mm}]
\pgfdeclareshape{z,a,b,c,x,y,d,o,p,e}

\node [square] (z) {Assignment}
  child {node [square,left of=b,xshift=-4cm] (a) {Variable}}
  child {node [ellipse,draw,left of=c,xshift=-4.5cm] (b) {\textit{Expression}}}
  child {node [square] (c) {Assignment} edge from parent[dashed]
  	child {node [square,xshift=-1cm] (x) {Variable} edge from parent[solid]}
  	child {node [ellipse,draw,solid,xshift=-1cm] (y) {\textit{Expression}} edge from parent[solid]}
  }
  child {node [square,xshift=-1cm] (d) {Assignment} edge from parent[dashed]
  	child {node [square,xshift=1cm] (o) {Variable} edge from parent[solid]}
  	child {node [ellipse,draw,solid,xshift=1cm] (p) {\textit{Expression}} edge from parent[solid]}
  }
  child {node [ellipse,draw,right of=d,xshift=1.5cm](e) {\textit{Expression}}};

\path (c)--(d) node [midway] {$\cdots$};
\end{tikzpicture}\end{center}\end{figure}

\subsection{If expression}
\begin{figure}[H]\begin{center}\begin{tikzpicture}[level/.style={sibling distance=30mm/#1}]
\node [square] (z){If expression}
  child {node [ellipse,draw] (a) {Expression}}
  child {node [ellipse,draw] (b) {Expression}}
  child {node [ellipse,draw] (c) {Expression}};
\end{tikzpicture}\end{center}\end{figure}

\subsection{Lambda expression}

\begin{figure}[H]\begin{center}\begin{tikzpicture}[level/.style={sibling distance=30mm/#1}]
\node [square] (z){Lambda expression}
  child {node [square] (a) {Variable list}}
  child {node [ellipse,draw] (b) {\textit{Expression}}};
\end{tikzpicture}\end{center}\end{figure}

\subsection{List}
\begin{figure}[H]\begin{center}\begin{tikzpicture}[level/.style={sibling distance=30mm/#1}]
\node [square] (z){List}
  child {node [ellipse,draw] (a) {Element} edge from parent[dashed]}
  child {node [ellipse,draw] (b) {Element} edge from parent[dashed]};

\path (a)--(b) node [midway] {$\cdots$};
\end{tikzpicture}\end{center}\end{figure}

\subsection{Pattern}
\begin{figure}[H]\begin{center}\begin{tikzpicture}[level/.style={sibling distance=40mm/#1}]
\node [square] (z){Pattern}
  child {node [square] (a) {Pattern expression}}
  child {node [square] (b) {Pattern expression} edge from parent[dashed]}
  child {node [square] (c) {Pattern expression} edge from parent[dashed]};

\path (b)--(c) node [midway] {$\cdots$};
\end{tikzpicture}\end{center}\end{figure}

\subsection{Pattern or-operator}
\begin{figure}[H]\begin{center}\begin{tikzpicture}[level/.style={sibling distance=40mm/#1}]
\node [square] {Pattern, or-operator}
  child {node [square] {Pattern value}}
  child {node [square] {Pattern expression}};
\end{tikzpicture}\end{center}\end{figure}

\subsection{Pattern multiplier-operator}
\begin{figure}[H]\begin{center}\begin{tikzpicture}[level/.style={sibling distance=40mm/#1}]
\node [square] {Pattern, multiplier-operator}
  child {node [square] {Pattern value}};
\end{tikzpicture}\end{center}\end{figure}

\subsection{Pattern not-operator}
\begin{figure}[H]\begin{center}\begin{tikzpicture}[level/.style={sibling distance=40mm/#1}]
\node [square] {Pattern, not-operator}
  child {node [square] {\textit{Pattern check}}};
\end{tikzpicture}\end{center}\end{figure}

\subsection{Not-operator}
\begin{figure}[ht]\begin{center}\begin{tikzpicture}[level/.style={sibling distance=40mm/#1}]
\node [square] {Not-operator}
  child {node [ellipse, draw] {\textit{Expression}}};
\end{tikzpicture}\end{center}\end{figure}
\section{Lexical structure}
\label{sec:lexicalstructure}

This section presents the low-level non-terminals of \productname{}.
We begin by describing the conventions we use throughout the report. Then we
describe the contents of the lexical structure of \productname{}, such as the
different reserved keywrods, identifiers, and literals that \productname{} contains.

\subsection{Notational Conventions}
We use a variant of Extended Backus-Naur Form (EBNF) to express the context-free
grammar of our programming language.

Each production rule assigns an expression of terminals, non-terminals and operations
to a non-terminal. E.g. in the following example the non-terminal $decimal$ is assigned
the possible terminals of $\gter{0}$ up to, and including, $\gter{9}$.

\begin{ebnf}
\grule{decimal}{\gter{0} \gor \gter{1} \gor \grange \gor \gter{9}}
\end{ebnf}

The following operations are used throughout this section to describe the
grammar of the programming language:

\begin{center}
\begin{tabular}{r l}
  $\gopt{pattern}$ & an optional pattern \\
  $\grep{pattern}$ & zero or more repititions of pattern \\
  $\ggrp{pattern}$ & a group \\
  $pattern_1 \gor pattern_2$ & a selection \\
  $\gter{0} \gor \grange \gor \gter{9}$ & a range of terminals \\
  $pattern_1 \gex pattern_2$ & matched by $pattern_1$ but not by $pattern_2$\\
  $pattern_1 \gcat pattern_2$ & concatenation of $pattern_1$ and $pattern_2$ \\
  $\gter{test}$ & a terminal \\
  $\gtsq$ & a terminal single quotation mark \\
  $\gtdq$ & a terminal double quotation mark \\
  $\gtbs$ & a terminal backslash character \\
\end{tabular}
\end{center}

The left-hand side is what will be seen in the grammar, while the right-hand
side is a description of what the operation means. For instance  a terminal will
always consist of $\gter{ }$ around the name of the terminal.

\subsection{Character Classes}
To be able to describe which characters or symbols a non-terminal can consist of
in a concise manner, we need to define some sets of symbols to specific names
which we can use in the description of our grammar.

The following classes of characters will be used throughout this section:

\begin{ebnf}
\grule{decimal}{\gter{0} \gor \gter{1} \gor \grange \gor \gter{9}}
\grule{lowercase}{\gter{a} \gor \gter{b} \gor \grange \gor \gter{z}}
\grule{uppercase}{\gter{A} \gor \gter{B} \gor \grange \gor \gter{Z}}
\grule{anycase}{lowercase \gor uppercase}
\grule{alphanum}{anycase \gor decimal}
\grule{quotebs}{\gtdq \gor \gtbs}
\grule{unichar}{\gcomment{any unicode character}}
\grule{strchar}{unichar \gex quotebs}
\end{ebnf}

For instance a string character $\left(strchar\right)$ can by any
unicode character besides a quotation mark or a backslash. 

\subsection{Comments}

In \productname{} a single-line comment begins with a sequence of at least two
forward slashes (\texttt{//}). The comment ends at the next newline. Everything
after the first two forward slashes and until the first next newline is
completely ignored at the leve of lexical analysis. Comments are valid
whitespace.

The following example shows a valid comment within an expression:

\codesample{validcomment.junta}

Unlike other programming languages, \productname{} does not have support for
multi-line comments (such as \texttt{/* */} in C-like languages).

\subsection{Reserved keywords}

The following list presents the keywords which are reserved in
\productname{}. These keywords cannot be used as identifiers in the language,
which is apparent in the following section.

\begin{ebnf}
  \grule{reserved}{\gter{define} \gor \gter{type} \gor \gter{abstract} \gor
        \gter{data} \gor \gter{extends} \gor \gter{let} \gor \gter{in}}
  \galt{\gter{set} \gor \gter{if} \gor \gter{then} \gor \gter{else} \gor
        \gter{not} \gor \gter{and} \gor \gter{or} \gor \gter{this} \gor \gter{super}}
  \galt{\gter{foe} \gor \gter{friend} \gor \gter{empty} \gor \gter{is}}
\end{ebnf}

\subsection{Identifiers}

\productname{} has three different identifiers which are defined as follows:

\begin{ebnf}
%Identifiers
\grule{constant}{\ggrp{lowercase \gcat \grep{alphanum}} \gex \ggrp{reserved \gor direction}}
\grule{type}{\ggrp{uppercase \gcat \grep{alphanum}} \gex coordinate}
\grule{variable}{\gter{\$} \gcat alphanum \gcat \grep{alphanum}}
\end{ebnf}

Constants, types, and variables cannot be defined with equal names because
they begin with different characters. 

\subsection{Literals}

The following list presents the literals of \productname{}:

\begin{ebnf}
%Literals
\grule{integer}{decimal \gcat \grep{decimal}}
\grule{direction}{\gter{n} \gor \gter{s} \gor \gter{e} \gor \gter{w} \gor
\gter{ne} \gor \gter{nw} \gor \gter{se} \gor \gter{sw}}
\grule{coordinate}{uppercase \gcat \grep{uppercase} \gcat decimal \gcat \grep{decimal}}
\end{ebnf}

The string literal can contain any unicode character if it is escaped. This lets
us construct any string. For instance if one needs to use a quotation mark
within a string, it is possible by escaping the quotation marks.

\begin{ebnf}
\grule{string}{\gtdq \gcat \grep{strchar \gor escape} \gcat \gtdq}
\grule{escape}{\gtbs \gcat unichar}
\end{ebnf}



\section{Expressions}
\label{sec:expressions}

Throughout this section we will present all the expressions that are included in
\productname{}. We have provided big-step semantics for six central expressions
in this section. These are the ones that we have deemed most necessary and they
are not as any other construct. Which sections that include semantics will be
explained in the following.

The smallest parts of expressions are presented in
\secref{sec:atomicexpressions}. In \secref{sec:lists} the notion of
lists is presented and explained. Here we present big-step semantics. Then we
present the let expressions in \secref{sec:letexpressions}. Also here we present
big-step semantics. Afterwards, we present the conditional expressions in 
\secref{sec:conditionalexpressions} followed by a lambda expressions in
\secref{sec:lambdaexpressions}. We provide big-step semantics for lambda
expressions. Furthermore, we present the concept of a set expression in
\secref{sec:setexpressions}. We also supply big-step semantics for set
expressions. Lastly, we present the different operators and calls in
\secref{sec:operatorsandcalls}. Here we provide big-step semantics for function
calls and for member access.

We begin here by listing the main expressions below this
paragraph:

\begin{ebnf}
%Expressions
\grule{expression}{let\_expr}
\galt{if\_expr}
\galt{set\_expr}
\galt{lambda\_expr}
\galt{\gter{not} \gcat expression}
\galt{lo\_sequence}
\end{ebnf}

Two statements hold about expressions in \productname{}:

\begin{nlist}
\item An expression \textbf{always} has a value.
\item An expression \textbf{cannot} have side effects.
\end{nlist}

\subsection{Atomic expressions}
\label{sec:atomicexpressions}

These are the smallest possible parts of expressions in \productname{}, defined by
the rule:

\begin{ebnf}
\grule{atomic}{\gter{(} \gcat expression \gcat \gter{)}}
\galt{constant}
\galt{type}
\galt{variable}
\galt{\gter{this}}
\galt{\gter{super}}
\galt{integer}
\galt{string}
\galt{direction}
\galt{coordinate}
\galt{\gter{/} \gcat pattern \gcat \gter{/}}
\galt{list}
\end{ebnf}

The first atomic expression is $\gter{(} \gcat expression \gcat \gter{)}$, which means
that it is possible to embed expressions within other expressions, and manually control
the precedence of operations. Consider the following two expressions:

\codesample{parentheses1.junta}
\codesample{parentheses2.junta}

In \csref{parentheses2.junta} the parentheses are in fact unnecessary because the
\texttt{*}-operator has precedence over the \texttt{+}-operator (see
\secref{sec:operatorsandcalls}).

Names (constants, types, and variables) are also atomic expressions, and are evaluated
to whatever value they are associated with, based on the current scope. The keywords
($\gter{this}$ and $\gter{super}$) are atomic, but only applicable within type
definitions, where $\gter{this}$ refers to the current object and $\gter{super}$
refers to the current object casted to its parent type (if it has one).

The literals ($integer$, $string$, $direction$, and $coordinate$) are evaluated to their
respective values, while patterns are evaluated to \type{Pattern}-values according to
the grammar in \secref{sec:patterns} and lists are evaluated to \type{List}-values
according to the grammar in \secref{sec:lists}.

\subsection{Lists}
\label{sec:lists}

The \type{List}-type is one of the basic types in \productname{}. A \type{List}-value is
created using the following syntax:

\begin{ebnf}
\grule{list}{\gter{[} \gcat \gopt{expression \gcat \grep{\gter{,} \gcat
expression}} \gcat \gter{]}}
\end{ebnf}

Essentially this means that a list is created from zero or more expressions (separated
by commas). The following statements can be made about lists:

\begin{nlist}
\item A list can be empty: \texttt{[]}
\item Lists are ordered ($\texttt{[1, 2]} \ne \texttt{[2, 1]}$).
\item Lists are immutable.
\end{nlist}

When a list is evaluated, each expression is evaluated to a value (the order of
evaluation does not matter, since no side-effects are possible, lazy-evaluation
of expressions could even be a possibility), and all values (in the same order
as the expressions) are added to the resulting \type{List}-value. When a
\type{List}-value is created, it can't be altered further (because of the
no-side-effects condition). All operations on that \type{List}-value will create
new \type{List}-values, and leave the original value intact.

Values within lists can be accessed using the \texttt{[]}-operation (same as
with function calls and type instantiation). Lists begin at offset $0$, so in
order to access the second element of a list, one can use the following:

\codesample{listaccess1.junta}

Ranges of elements can also be returned. For instance in the following
expression a new list is returned containing elements from offset 1 up
to and including offset 2 (the list $\texttt{["is", "a"]}$):

\codesample{listaccess2.junta}

An offset can be negative, which means that the offset is dependent on
the size of the list. This way offset $-1$ will always refer to the last element
of the list. For example in order to return the last two elements of a list one
could use the offsets, $-2$ and $-1$:

\codesample{listaccess3.junta}

Some other examples are:

\codesample{listaccess4.junta}

\subsubsection{Big-step semantics}

The semantics presented in \tableref{semantic:lists} are the transition rules
for lists.

\begin{table}[ht]
  \begin{center}
    \begin{tabular*}{\textwidth}{l p{\textwidth}}
      \hline \\
      \hspace{0.5cm} $[\mbox{LIST}_{\mbox{ACCESS-1}}]$ & \infrule{env_{T},
      env_{C}, env_{V} \vdash \lag e\; \rag \ra v_2  \qquad env_{T}, env_{C},
      env_{V} \vdash \lag i \rag \ra v_3}
      {env_{T}, env_{C}, env_{V} \vdash \lag e\; i \rag \ra v_1} \\
       & where $v_2 = \left(l_1, elem_1\right)$ \\
       & and $v_3 = (l_2,elem_2)$ \\
       & and $l_2 = 1$ \\
       & and $d = elem_2\; 0$ \vspace{0.1cm} \\
       & and $v_1 = \left\{
	 \begin{array}{l l}
           elem_1\; d         & \quad \text{if $d \geq 0$}\\
           elem_1\; (l_1 + d) & \quad \text{if $d < 0$}
	 \end{array} \right.$ \\
	 
      \hspace{0.5cm} $[\mbox{LIST}_{\mbox{ACCESS-2}}]$ & \infrule{env_{T},
      env_{C}, env_{V} \vdash \lag e \rag \ra v_2  \qquad env_{T}, env_{C},
      env_{V} \vdash \lag i \rag \ra v_3}
      {env_{T}, env_{C}, env_{V} \vdash \lag e\; i \rag \ra v_1} \\
       & where $v_2 = \left(l_1, elem_1\right)$ \\
       & and $v_3 = (l_2,elem_2)$ \\
       & and $l_2 = 2$ \vspace{0.1cm} \\
       & and $d = \left\{
	 \begin{array}{l l}
           elem_2\; 0         & \quad \text{if $elem_2\; 0 \geq 0$}\\
           l_1 + elem_2\; 0   & \quad \text{if $elem_2\; 0 < 0$}
	 \end{array} \right.$ \vspace{0.1cm} \\
       & and $j = \left\{
	 \begin{array}{l l}
           elem_2\; 1         & \quad \text{if $elem_2\; 1 \geq 0$}\\
           l_1 + elem_2\; 1   & \quad \text{if $elem_2\; 1 < 0$}
	 \end{array} \right.$ \vspace{0.1cm} \\
       & and $elem_3\; z = \left\{
	 \begin{array}{l l}
           elem_1\; d+1       & \quad \text{if $z = 0$}\\
	   \hspace{1cm} \vdots &   \\
           elem_1\; d+n-1     & \quad \text{if $z = n-1$}
	 \end{array} \right.$ \vspace{0.1cm} \\
       & and $n=j-d+1,\; elem_3$ \\	 
       & and $v_1 = n$ \\
       & \\
       \hline
    \end{tabular*}
    \capt{Transition rules for accessing lists.}
    \label{semantic:lists}
  \end{center}
\end{table}



List access requires two rules, since two cases are possible. The first case, is
when just one offset is requested, then that element should be returned. In the
second case, two offsets are requested, and a range of elements should be
returned, also in the form of a list.

\subsection{Let expressions}
\label{sec:letexpressions}

Variables in \productname{} are assigned using let expressions:

\begin{ebnf}
\grule{expression}{\grange \gor let\_expr}
\grule{let\_expr}{\gter{let} \gcat variable \gcat \gter{=} \gcat expression
\gcat \grep{\gter{,} \gcat variable \gcat \gter{=} \gcat expression} \gnl
\gcat \gter{in} \gcat expression}
\end{ebnf}

A let expression consists of one or more assignments and an expression. Each
assignment assigns the value of an expression to a variable name. These
variables can then be used in the expression after the \texttt{in}-keyword.
After the evaluation of the let-expression, the variables cease to exist.

\subsubsection{Informal scope rules for let expressions}

Destructive assignments are not possible \productname{}, meaning that it isn't
possible to reassign a variable. It is however possible to hide a variable.
Consider the following expression:

\codesample{lethiding.garry}

The value of this expression is $13$. This is because within the
\texttt{\variable{x} + 2}-expression the \variable{x}-variable evaluates to $6$.
But in the outer expression \variable{x} evaluates to $5$.

Nested \emph{let}-scopes are possible. Consider for instance:

\codesample{nestedlet.garry}

In the inner scope, both \variable{x} and \variable{y} are available. This is of
course equivalent to:

\codesample{nestedlet2.garry}

\subsubsection{Big-step semantics}

The semantics presented in \tableref{semantic:let} are the transition rules for
the let expression.  This is transition rule is defined recursively to best
illustrate the functionality of the expression.


\begin{table}[ht]
  \begin{tabular*}{\textwidth}{l l}
    \hline \\
    \hspace{1.2cm} $[\mbox{LET-1}]$ & $env_{T}, env_{C}, env_{V}
    \vdash E_{1} \ra v_{1} \vspace{-0.3cm}$ \\
    & \infrule{env_{T}, env_{C}, env_{V}[x_{1} \mapsto v_{1}] \vdash \lag
    \texttt{let}\; x_{2} = E_{2}, \cdots,\; x_{k} = E_{k}\; \texttt{in}\;
    E_{k+1} \rag \ra v_{k+1}}
    {env_{T}, env_{C}, env_{V} \vdash \lag \texttt{let}\; x_{1} = E_{1},\; x_{2}
    = E_{2}, \cdots, x_{k} = E_{k}\; \texttt{in}\; E_{k+1} \rag \ra v_{k+1}} \\
    & where $k \geq 2$\\
    & \\
    
    \hspace{1.2cm} $[\mbox{LET-2}]$ & \hspace{0.4cm} $env_{T}, env_{C}, env_{V}
    \vdash E_{1} \ra v_{1}$ \vspace{-0.3cm} \\
    & \infrule{env_{T}, env_{C}, env_{V}[x_{1} \mapsto v_{1}] \vdash \lag
    E_{2}\rag \ra v_{2}} {env_{T}, env_{C}, env_{V} \vdash \lag \texttt{let}\;
    x_{1} = E_{1}\; \texttt{in}\; E_{2} \rag \ra v_{2}} \\
    \hline \\
  \end{tabular*}
  \capt{Transition rules for let expressions.}
  \label{semantic:let}
\end{table}


The transition rules for $[\mbox{LET-1}]$ is recursive because we must
evaluate each expression $(x_{1}=E_{1})$ individually before we move on to the next one. This
is a must because of the fact that the next expressions can in fact make use of
the previous expressions value. As an example take a look at the following code
sample:

\codesample{letbigstep.junta}

So, each call where there is more than one expression to be evaluated, 
the transition rule $[\mbox{LET-1}]$ where $k \geq 2$ is used. Here the expression first
in line to be evaluated will be evaluated before a new call to one of the two
transition rules is made. When we reach a let expression with only one
expression, we then call the transition rule $[\mbox{LET-2}]$ where $k < 2$.

\subsection{Conditional expressions}
\label{sec:conditionalexpressions}

It is often desirable to base the result of an expression on some sort of condition.
In \productname{} this is achievable using \emph{if}-expressions, as defined by the
following syntax:

\begin{ebnf}
\grule{expression}{\grange \gor if\_expr}
\grule{if\_expr}{\gter{if} \gcat expression \gcat \gter{then} \gcat expression
\gcat \gter{else} \gcat expression}
\end{ebnf}

Unlike in most imperative languages, the conditional construct in \productname{}
is not a statement (\productname{} doesn't have statements) but an expression.
Since all expressions must have a value, the \emph{else}-part of en
if-expression is compulsory.

The if-expression first evaluates the condition (the first expression). The
resulting value must be of type \type{Boolean}. If the value is equal to the
boolean true-value, the \emph{then}-expression is evaluated, and the result is
returned. If the value is false, then the \emph{else}-expression is evaluate,
and the result returned.

\subsection{Lambda expressions}
\label{sec:lambdaexpressions}

Lambda expressions are expressions that evaluate to anonymous functions. In
\productname{} they are defined as:

\begin{ebnf}
\grule{expression}{\grange \gor lambda\_expr}
\grule{lambda\_expr}{\gter{\#} \gcat varlist \gcat \gter{=>} \gcat expression}
\end{ebnf}

The non-terminal $varlist$ represents a list formal parameters, and is further
explained in \secref{sec:constantdefinitions}.

When a lambda expression is created, a reference to the scope it was created in
is saved with it. This is known as a closure, and means that a lambda
expression may access variables outside of its own scope. The accessible
variables are the variables that were available at the time of the creation of
the lambda expression.

Consider the following example:

\codesample{closuredef.garry}

The function \function{getAdder} takes one argument (\variable{a}) and
returns a lambda expression, is defined. Notice how \variable{a} is used within
the lambda expression. This means that when the lambda expression is created, it
must remember the value of the variables that exist in the scope, in which it is
created. The use of the \function{getAdder}-function could look like this:

\codesample{closureuse.garry}

In the first line \function{getAdder} is called with the argument, $25$. A new
scope, $A$, is created, in which the variable \variable{a} is assigned the value
$25$. Then the function expression is evaluated, which results in a new lambda
expression (with a reference to scope $A$).  This is returned and assigned to
\variable{adder} in line 1 of the above example.

In the second line, \variable{adder} is called as a function, meaning
that a new scope, $B$, is created, where the variable \variable{b} is
assigned the value $5$. The important part is that $B$'s parent scope is set to
$A$ (which is saved with the lambda expression). The expression (the right side
of the lambda expression) is then evaluated. First the \variable{a}-variable is
encountered. The interpreter first searches the $B$-scope for \variable{a}, and
when unsuccessful, searches the parent-scope, $A$, for \variable{a}. In $A$ the
variable \variable{a} holds the value $25$, and this is returned. Then the
$B$-scope is searched for the \variable{b}-variable, and the value $5$ is
returned. The two integers are added, and the final return-value of the
lambda-expression ends up being $30$.

\subsubsection{Big-step semantics}

The semantics presented in \tableref{semantic:lambda} is the transition rule for
lambda expressions.


\begin{figure}
\begin{center}
\begin{tikzpicture}[level/.style={sibling distance=30mm/#1}]
\node [square] {Lambda expression}
  child {node [square] {Variable list}}
  child {node [ellipse,draw] {\textit{Expression}}};
\end{tikzpicture}
\end{center}
\capt{The abstract syntax for the lambda expression node.}
\label{ast:lambdaexpr}
\end{figure}


The three environments ($env_{T}, env_{C}, env_{V}$) must be known before it is
possible to execute a lambda expression. We need to know which types, constants
and different variables are given in the specific scope.

The lambda expressions evaluates to a value $v$. The side condition
of the transition rule explains that $v$ is assigned the $4$-tuple, a
function value.

\subsection{Set expressions}
\label{sec:setexpressions}

Set expressions look a bit like let expressions, but are only applicable within
type definitions:

\begin{ebnf}
\grule{expression}{\grange \gor set\_expr}
\grule{set\_expr}{\gter{set} \gcat variable \gcat \gter{=} \gcat expression
\gcat \grep{\gter{,} \gcat variable \gcat \gter{=} \gcat expression}}
\end{ebnf}

Set expressions are used to ``modify'' the value of data-members in objects (see
\secref{sec:typedefinitions} for an explanation of data). Each variable in the
set expression must exist in the current type as data-members. Since modifying a
data-member would be a side-effect, which is not allowed, the set expression
instead returns a clone of the object, with the specified data-members set to
their respective values. This is useful for making setters (or something that
looks like setters). Consider for example the following type:

\codesample{setget.junta}

In the example above, the method \constant{setMyData} returns a new instance of
\type{MyType}, with the data-member \variable{myData} set to something else. The
following example shows the use of a getter and a setter:

\codesample{setget2.junta}

The call to \constant{setMyData} does not change the state of the original
instance of \type{MyType}, instead it returns a new instance.

\subsubsection{Big-step semantics}

The semantics presented in \tableref{semantic:set} is the transition rule for
set expressions.

\begin{table}[ht]
  \begin{tabular*}{\textwidth}{l l}
    \hline \\
    \hspace{1.5cm} $[\mbox{SET}]$ & \infrule{env_C, env_V, env_T \vdash \lag e_1
      \rag\ra u_1 \quad
    \ldots \quad env_C, env_V, env_T \vdash \lag e_k \rag \ra u_k}
    {env_C, env_V, env_T \vdash \lag \texttt{set}\; x_1 = e_1, \ldots, x_k =
    e_k \rag \ra v_1} \\
    & where $env_C\; \texttt{this} = \left(t, env'_C, env'_V, v_2 \right)$ \\
  & and $env''_V = env'_V \left[ x_1 \mapsto u_1, \ldots, x_k \mapsto u_k \right]$ \\
    & and $v_1 = \left( t, env'_C, env''_V, v_2\right)$ \\ 
    & \\
    \hline
  \end{tabular*}
  \capt{Transition rules for set expressions.}
  \label{semantic:set}
\end{table}


The transition rule assumes that the current constant environment $env_C$
contains a pointer to the current object, $\texttt{this}$. It then returns a
copy of that object with a new variable environment, containing the new data.

\subsection{Operators and calls}
\label{sec:operatorsandcalls}

Operators are useful for doing calculations, and \productname{} supports
basic mathematical operators and precedence. In order to prevent
left-recursion, which makes it possible for us to construct an LL-parser
(this was discussed in \secref{subsec:llparsersandlrparsers}), but
preserve left-associativity we have created a hierarchy of operators,
taking advantage of LL-parsing by putting the operators with highest
precedence the lowest in any parse tree that includes them. The grammar
for the operators of \productname{} are described using operator
sequences. A sequence is essentially just a list of operations on that
particular precedence level. In this way all the precedence levels of
\productname{} are described formally:

\begin{ebnf}
\grule{expression}{\grange \gor \gter{not} \gcat expression \gor lo\_sequence}
\grule{lo\_sequence}{eq\_sequence \gcat \grep{\ggrp{\gter{and} \gor \gter{or}}
\gcat eq\_sequence}}
\grule{eq\_sequence}{cm\_sequence \gcat \grep{\ggrp{\gter{==} \gor \gter{!=}
\gor \gter{is}} \gcat cm\_sequence}}
\grule{cm\_sequence}{as\_sequence \gcat \grep{\ggrp{\gter{<} \gor \gter{>} \gor
\gter{<=} \gor \gter{>=}} \gcat as\_sequence}}
\grule{as\_sequence}{md\_sequence \gcat \grep{\ggrp{\gter{+} \gor \gter{-}}
\gcat md\_sequence}}
\grule{md\_sequence}{negation \gcat \grep{\ggrp{\gter{*} \gor \gter{/} \gor
\gter{\%}} \gcat negation}}
\grule{negation}{element}
\galt{\gter{-} \gcat negation}
\grule{element}{call\_sequence \gcat \grep{member\_access}}
\grule{member\_access}{\gter{.} \gcat constant \gcat \grep{list}}
\grule{call\_sequence}{atomic \gcat \grep{list}}
\end{ebnf}

In order to completely understand this grammar, we must first take a look at the list of
operators ordered by precedence. The precedence of operators is presented in
\tableref{table:operatorPrecedence}.

\tab[\textwidth]{operatorPrecedence}{2}{The precedence of operators in \productname{}.}
                  {Operator precedence}
           {Level}{Operator & Description}{
    \tabrow{1}{\texttt{f[]} & Function/constructor invocation and list access}
    \tabrow{2}{\texttt{r.m r.m[]} & Record member access and member invocation}
    \tabrow{3}{\texttt{-} & Unary negation operation}
    \tabrow{4}{\texttt{* / \%} & Multiplication, division, and modulo}
    \tabrow{5}{\texttt{+ -} & Addition and subtraction}
    \tabrow{6}{\texttt{< > <= >=} & Comparison operators}
    \tabrow{7}{\texttt{== != is} & Equality operators and type checking}
    \tabrow{8}{\texttt{and or} & Logical $and$ and $or$}
    \tabrow{9}{\texttt{not} & Logical $not$}
    \tabrow{10}{\texttt{if let set \#} & if-, let-, set-, and lambda-expressions}
}

Each precedence level will correspond to a certain rule in the grammar. For
instance, the fifth precedence level for addition and subtraction is expressed
using the $as\_sequence$-rule. Combined with some multiplication an expression
making use of the $as\_sequence$- and $md\_sequence$-rules could look like
\csref{asandmd.junta}:

\codesample{asandmd.junta}

The resulting parse tree, using the grammar of \productname{}, could look
somewhat like the tree in \figref{fig:parsetreesequences}.

\begin{figure}[ht]
  \begin{center}
      \begin{tikzpicture}[]
	%the nodes
     	\node[square,xshift=4em]      			    (as)     {as\_sequence};
     	\node[circle,draw,yshift=-3em] 		    (plus1)  {$+$};
     	\node[circle,draw,yshift=-3em,xshift=8em]   (minus1) {$-$};
     	\node[circle,draw,yshift=-3em,xshift=16em]  (plus2)  {$+$};
     	\node[square,yshift=-3em,xshift=-8em]       (int1)   {integer $\left(2\right)$};
     	\node[square,yshift=-6em]      		    (md1)    {md\_sequence};
	\node[square,yshift=-6em,xshift=8em]        (int2)   {integer $\left(3\right)$};
     	\node[square,yshift=-6em,xshift=16em]       (md2)    {md\_sequence};
     	\node[square,yshift=-9em,xshift=-4em]      (int3)   {integer $\left(3\right)$};
     	\node[circle,draw,yshift=-9em,xshift=4em]  (mult1)  {$*$};
     	\node[square,yshift=-9em,xshift=12em]      (int4)   {integer $\left(2\right)$};
     	\node[circle,draw,yshift=-9em,xshift=20em] (div1)   {$/$};
     	\node[square,yshift=-12em,xshift=4em]       (int5)   {integer $\left(5\right)$};
     	\node[square,yshift=-12em,xshift=20em]      (int6)   {integer $\left(3\right)$};

	%the solution
	\node[rectangle,yshift=-14.5em,xshift=-8em] (a) {$2$};
     	\node[rectangle,yshift=-14.5em,xshift=-6em]     {$+$};
     	\node[rectangle,yshift=-14.5em,xshift=-4em] (b) {$3$};
     	\node[rectangle,yshift=-14.5em]                 {$*$};
     	\node[rectangle,yshift=-14.5em,xshift=4em]  (c) {$5$};
     	\node[rectangle,yshift=-14.5em,xshift=6em]      {$-$};
	\node[rectangle,yshift=-14.5em,xshift=8em]  (d) {$3$};
     	\node[rectangle,yshift=-14.5em,xshift=10em]     {$+$};
     	\node[rectangle,yshift=-14.5em,xshift=12em] (e) {$2$};
     	\node[rectangle,yshift=-14.5em,xshift=16em]     {$/$};
	\node[rectangle,yshift=-14.5em,xshift=20em] (f) {$3$};

	%first level
	\draw[-,-|,-,thin,] (as.south) |-+(0,-0.75em)-| (int1.north);
	\draw[-,-|,-,thin,] (as.south) |-+(0,-0.75em)-| (plus1.north);
	\draw[-,-|,-,thin,] (as.south) |-+(0,-0.75em)-| (minus1.north);
	\draw[-,-|,-,thin,] (as.south) |-+(0,-0.75em)-| (plus2.north);

	%second level
	\draw[-,thin,] (plus1.south) -- (md1.north);
	\draw[-,thin,] (minus1.south) -- (int2.north);
	\draw[-,thin,] (plus2.south) -- (md2.north);

	%third level
	\draw[-,-|,-,thin,] (md1.south) |-+(0,-0.75em)-| (int3.north);
	\draw[-,thin,] (md1.south) |-+(0,-0.75em)-| (mult1.north);
	\draw[-,-|,-,thin,] (md2.south) |-+(0,-0.75em)-| (int4.north);
	\draw[-,thin,] (md2.south) |-+(0,-0.75em)-| (div1.north);

	%fourth level
	\draw[-,thin,] (mult1.south) -- (int5.north);
	\draw[-,thin,] (div1.south) -- (int6.north);

	%solution level
	\draw[-,dashed,] (int1) -- (a);
	\draw[-,dashed,] (int2) -- (d);
	\draw[-,dashed,] (int3) -- (b);
	\draw[-,dashed,] (int4) -- (e);
	\draw[-,dashed,] (int5) -- (c);
	\draw[-,dashed,] (int6) -- (f);

    \end{tikzpicture}
  \end{center}
  \capt{A parse tree for the expression $2 + 3 * 5 - 3 + 2 / 3$.}
  \label{fig:parsetreesequences}
\end{figure}


The figure clearly shows the precedence, because the lower nodes will be
evaluated before the nodes that are higher in the parse tree. E.g.\ all of the
multiplication and division nodes will be calculated before the additions and
subtraction. This hierarchy is achieved because of the formally described
precedence rules in the previous grammar about sequences.

\subsubsection{Big-step semantics}

The semantics presented in \tableref{semantic:callfun} is transition
rule for function calls. Big-step semantics for the others are left out,
as they are mostly trivial.

\begin{table}[ht]
  \begin{tabular*}{\textwidth}{l l}
    \hline \\
    \hspace{3cm} $[\mbox{CALL}_{\mbox{FUN}}]$ & \hspace{0.1cm} $env_C, env_V,
    env_T \vdash \lag e \rag \ra v_2$ \\
    & \hspace{0.1cm} $env_C, env_V, env_T \vdash \lag i \rag \ra v_3$
    \vspace{-0.3cm} \\
    & \infrule{env'_C, env''_V, env_T \vdash \lag e' \rag \ra v_1}{env_C, env_V,
    env_T \vdash \lag e\; i\; \rag \ra v_1} \\
    & where $v_2 = \left(g, e', env'_V, env'_C\right)$ \\
    & and $v_3 = \left(l, elem\right)$ \\
    & and $env''_V = \left[x_1 \mapsto elem\; 1, \ldots, x_n \mapsto elem\; n \right]$ \\
    & \\
    \hline
  \end{tabular*}
  \capt{Transition rules for function calls.}
  \label{semantic:callfun}
\end{table}



In this rule the expression $E$ is evaluated to a function value, $v_2$, which
has its own variable and constant environments, as per the static scope rules.
The variable environment is then updated with the actual parameters assigned to
the formal parameters, after which the expression, contained within the function
value, is evaluated.

The semantics presented in \tableref{semantic:memaccess} is the transition rule
for member access, also known as dot-notation.

\begin{table}[ht]
  \begin{tabular*}{\textwidth}{l l}
    \hline \\
    \hspace{3cm} $[\mbox{MEMBER}_{\mbox{ACCESS}}]$ & \infrule{env_C, env_V, env_T
    \vdash \lag e \rag \ra v_1}{env_C, env_V, env_T \vdash \lag e\texttt{.}C
  \rag \ra v_3} \\
     & where $v_2 = \left(t, env'_C, env'_V, v_2 \right)$ \\
     & and $env'_C\; C = v_3$ \\
     & \\
     \hline
  \end{tabular*}
  \capt{Transition rules for member access.}
  \label{semantic:memaccess}
\end{table}



A member access is as simple as evaluating the left-side of the object
first (the constant), and then accessing the evaluated constant in that
objects constant environment.

\subsubsection{Valid operands}
\label{sec:validoperands}
In the following paragraphs we present the valid operands of
\productname{}. These will be grouped in the following categories; boolean,
comparison, integer, string, list, and direection and coordinate operators.

\paragraph{Boolean operators}

These operators only accept boolean operands and only return boolean values:

\begin{dlist}
  \item \operator[Boolean]{and}{Boolean}{Boolean}\\
    Returns true when both operands are true and false otherwise. 
  \item \operator[Boolean]{or}{Boolean}{Boolean}\\
    Returns true when at least one of the operands are true and false otherwise.
  \item \operator{not}{Boolean}{Boolean}\\
    Returns true if the single operand is false and false otherwise.
\end{dlist}

\paragraph{Comparison operators}

These operators are used to compare two values, and always returns a boolean value:

\begin{dlist}
  \item \operator[Integer]{<}{Integer}{Boolean}\\
    Returns true if the left operand is less than the right one.
  \item \operator[Integer]{>}{Integer}{Boolean}\\
    Returns true if the left operand is greater than the right one.
  \item \operator[Integer]{<=}{Integer}{Boolean}\\
    Returns true if the left operand is less than or equal to the right one.
  \item \operator[Integer]{>=}{Integer}{Boolean}\\
    Returns true if the left operand is greater than or equal to the right one.
  \item \operator[\opstar]{==}{\opstar}{Boolean}\\
    Returns true if the left operand is equal to the right one.
  \item \operator[\opstar]{!=}{\opstar}{Boolean}\\
    Returns true if the left operand is not equal to the right one.
  \item \operator[\opstar]{is}{Type}{Boolean}\\
    Returns true if the type of the first operand is equal to or inherits from
    the type operand.
\end{dlist}

\paragraph{Integer operators}

The following operations are possible on integers:

\begin{dlist}
  \item \operator{-}{Integer}{Integer} \\
    Integer negation.
  \item \operator[Integer]{+}{Integer}{Integer} \\
    Integer addition.
  \item \operator[Integer]{-}{Integer}{Integer} \\
    Integer subtraction.
  \item \operator[Integer]{*}{Integer}{Integer} \\
    Integer multiplication.
  \item \operator[Integer]{/}{Integer}{Integer} \\
    Integer division.
  \item \operator[Integer]{\%}{Integer}{Integer} \\
    Integer modulo operation.
\end{dlist}

\paragraph{String operators}

It is possible to concatenate strings:

\begin{dlist}
  \item \operator[String]{+}{String}{String} \\
    Returns the concatenation of two strings.
  \item \operator[String]{+}{\opstar}{String} \\
   \operator[\opstar]{+}{String}{String} \\
    Returns the concatenation of a string and the string-representation of another type
\end{dlist}

\paragraph{List operators}

Some operators are available for list values as well:

\begin{dlist}
\item \operator[List]{+}{List}{List} \\
  Returns a list containing all elements from the first list followed
  by all elements from the second list.
\item \operator[List]{-}{List}{List} \\
  Returns a list containing all the elements from the first list that
  do not exist in the second list.
\item \operator[List]{+}{\opstar}{List} \\
  Appends any element on to the end of a list, and returns the resulting list.
\item \operator[List]{-}{\opstar} \\
  Returns a list containing the elements that do not match the right operand.
\item \operator[\opstar]{+}{List}{List} \\
  Prepends any element on to the start of a list, and returns the resulting list.
\end{dlist}

\paragraph{Direction and coordinate operators}

The following operators can manipulate directions and coordinates:

\begin{dlist}
  \item \operator[Direction]{+}{Direction}{Direction} \\
    Add a direction (vector) to another direction.
  \item \operator[Direction]{-}{Direction}{Direction} \\
    Subtract a direction from another direction.
  \item \operator[Direction]{+}{Coordinate}{Coordinate} \\
    Add a coordinate to a direction.
  \item \operator{-}{Direction}{Direction} \\
    Negate a direction.
  \item \operator[Coordinate]{-}{Coordinate}{Direction} \\
    Returns the distance between two coordinates as a direction.
  \item \operator[Coordinate]{+}{Direction}{Coordinate} \\
    Add a direction to a coordinate. 
  \item \operator[Coordinate]{-}{Direction}{Coordinate} \\
    Subtract a direction from a coordinate.
\end{dlist}

For instance adding the directions \texttt{n} and \texttt{e} produces a
direction equivalent with the direction \texttt{ne}. Adding a coordinate and
direction (and vice versa) gives a coordinate. As an example, $\texttt{A2}
\verb!+! \texttt{e}$ gives \texttt{B2}. More information about the coordinate
and direction types is available in \secref{sec:standardenvironment}.


\section{Definitions}
\label{sec:definitions}


\subsection{Program structure}


\subsection{Constant definitions}


\subsection{Type definitions}

\section{Patterns}
\label{sec:patterns}

This section covers how to use patterns and what to use them for. The operators of a pattern looks like and behaves a little like regular expressions. This EBNF-grammar shows how a pattern is constructed:

\begin{ebnf}
\grule{pattern}{pattern\_expr \gcat \grep{pattern\_expr}}
\grule{pattern\_expr}{pattern\_val \gcat \gopt{\gter{*} \gor \gter{?} \gor \gter{+}}}
\galt{pattern\_val \gcat \gter{|} \gcat pattern\_expr}
\grule{pattern\_val}{direction}
\galt{variable}
\galt{pattern\_check}
\galt{\gter{!} \gcat pattern\_check}
\galt{\gter{(} \gcat pattern \gcat \gter{)} \gcat \gopt{integer}}
\grule{pattern\_check}{\gter{friend}}
\galt{\gter{foe}}
\galt{\gter{empty}}
\galt{\gter{this}}
\galt{type}
\end{ebnf}


A pattern is checked on a particular square, and returns either true or false. An example of a pattern is \text{/n n e empty/}. This pattern can be checked on the board seen in \figref{patternboard} on the field \textbf{A1}. The pattern says ``go one square north, go one square north, go one square east, check if square is empty''. This means that the square \textbf{B3} will be checked for emptiness. Since the square is occupied by a piece, the pattern will return false if checked on \textbf{A1}. However, the same pattern checked on \textbf{C1} will return true since the square at \textbf{D3} is empty.
\fig[scale=2]{patternboard}{A simple 4 X 4 board with 3 pieces}

With this basic introduction to a simple pattern check, the table (insert ref and convert list below to a table) describes briefly how the different pattern operators work. For each operator, an example of the use in a context are given in \secref{sec:patternexamples}. Note that the description of patterns assumes a minor understanding of regular expressions, see \cite{regex}.

\texttt{empty} : current square contains no pieces\\
\texttt{n} : north \\
\texttt{e} : east \\
\texttt{s} : south \\
\texttt{w} : west \\
\texttt{*} : zero-to-many times\\
\texttt{+} : one-to-many times\\
\texttt{?} : zero-or-one time\\
\texttt{|} : or\\
\texttt{!} : not\\
\texttt{(} \textit{pattern} \texttt{)} : encapsulation\\
\texttt{friend} :  current square contains at least one friendly piece of the current player\\
\texttt{foe} : current square contains at least one enemy piece of the current player\\
\textit{type} : the current square is of the given type or a piece of the given type is residing on the current square.
\texttt{this} : current square contains the piece for which the check is being performed\\

\subsection{Pattern examples}
\label{sec:patternexamples}
All these examples of pattern checks are performed on the board and pieces seen in \figref{patternboard}. For each operator, two examples of a pattern check on a particular square is given. The first check returns true and the second check false.

On A1, the pattern check \texttt{/empty/} returns true because A1 is empty\\
On B2, the pattern check \texttt{/empty/} returns false because B2 is not empty\\
On A1, the pattern check \texttt{/n empty/} returns true because A2 is empty\\
On B1, the pattern check \texttt{/n empty/} returns false because B2 is not empty\\
On C3, the pattern check \texttt{/e empty/} returns true because C4 is empty\\
On C4, the pattern check \texttt{/e empty/} returns false because C5 is out of board\\
On A1, the pattern check \texttt{/n* n e empty/} returns true because moving north 2 times then north and east hits an empty square on B4\\

Notice that the *-operator causes many branches to be made. The previous pattern check, \texttt{/n* n e empty/} done on A1, has a branch checking \texttt{/n e empty/}. The branch dies because B2 is not empty. If just one branch survives, the pattern check returns true. In the example, the only branch surviving is the \texttt{n n n e empty} branch. The same rules for branching counts for the \texttt{+}, \texttt{?} and \texttt{|} operator. When a branch moves out of board it dies immediately.

On C3, the pattern check \texttt{/n* s s !empty/} returns false because neither A1 nor A2 contains a piece\\
On A1, the pattern check \texttt{/n+ e empty/} returns true only because B4 is empty\\
On B1, the pattern check \texttt{/n+ empty e !empty/} returns false because B4 is the only empty square north of B1 and C4 is empty\\
On B3, the pattern check \texttt{/s? e empty/} returns true only because C2 is empty\\
On C2, the pattern check \texttt{/n? w empty/} returns false because neither B2 nor B3 is empty\\
On A2, the pattern check \texttt{/(n | e) empty/} returns true only because A3 is empty\\
On C2, the pattern check \texttt{/e | w empty/} returns false because neither B2 nor C3 is empty\\
On B1, the pattern check \texttt{/(n n | e e) empty/} returns true only because D1 is empty\\
On A1, the pattern check \texttt{/(n w)* empty/} returns true both because A1 is empty and because D4 is empty\\

The \texttt{friend}-keyword is evaluated based on the current player. Suppose that we have a player called Green, who owns the 
green piece. If it is green's turn to move, any branch of a pattern check will return false
whenever it meets a keyword \texttt{friend} on a square that does not contain any of Green's pieces.
On B2, the pattern check \texttt{/n|e|(n e) friend/} will return true if it is Green's turn, since C3 contains a friendly piece of Green.
On C3, the pattern check \texttt{/(e | w)+/} will return false if it is Green's turn, since no square containing a piece of Green can be reached by going east or west from C3 one or more times.

The keyword \texttt{foe} does the opposite of \texttt{friend}. It makes a branch continue only if its current square contains at least one piece not owned by the player who has the turn.

Just like \texttt{foe} and \texttt{friend}, the name of a piece or square-type defined in a \productname{}-game can also be used. E.g, the keyword \texttt{White} can be used if a piece or square-type with the name \textit{White} has been define. In a such case, a branch survives only if its current square is of type \textit{White} or if the square contains a piece of type \textit{White}.

A pattern can also check that a specific piece exists on a specific square. This is done using the \texttt{this} keyword.
Before this check can be achieved, the pattern must be checked regarding to a specific piece. Suppose that we on A3, make the pattern check
\texttt{empty e this}. If this check is done in relation to the black piece on B3 it returns true. However, in relation to the black piece on B2, the pattern check returns false. To understand both how this function exactly works and why this is useful, consider the board in \figref{fig:patternboard}.
If we for any piece specify that it can move to a square for which the pattern check \texttt{/(n | s) empty/} is true, this means that it can move to any square except $\{B1, B4, C4\}$. These square does not have an empty square north or east from it. Recall that a branch going out of board dies.
However, the pattern check \texttt{/empty (n | s) this/} will in relation to the green piece return true only when checked on the squares $\{C2, C4\}$.
This can be used to specify that a piece can move to an empty square one north or one south from its current square. 
\section{Predefined types and constants}
\label{sec:predefined}

In order to write programs in a programming language, it is often necessary to use a number of built-in
functions and types. \productname{} provides a number of built-in functions and constants, as well as 
a number of simple types for representing values such as integers and strings. \productname{} also provides
a type-hierarchy designed for implementing and expressing board games. The built-ins will in many cases make
the lives of \productname{}-programmers easier since he/she won't have to implement the functionality the built-ins
provide for themselves. 

Since \productname{} doesn't have a module, package or name space system, the distinction between a standard-
and game environment doesn't actually exist in the language, and all types and constants exist in the same
global name space. The distinction between the two is merely formal and based on the sort of types and functionality
that each provide.
 
\subsection{Standard environment}
\label{sec:standardenvironment}

We now introduce the standard environment of \productname{}. The standard environment provides the simple types, such as
integers, strings, boolean values, etc.\ and their related functions and constants for working with the these.

The following global constants are available:

\begin{dlist}
  \item \constdef{typeOf}{[\farg{value}{\opstar}]}{Type}\\
    A function that returns the type of any value.
  \item \constdef{union}{[\farg{list}{List}, ... \farg{lists}{List}]}{List}\\
    A function that returns the union of a number of lists.
  \item \constdef{true}{}{Boolean}\\
    The boolean true value.
  \item \constdef{false}{}{Boolean}\\
    The boolean false value.
\end{dlist}

\subsubsection{Integer}

\begin{dlist}
  \item \type{Integer}[\variable{integer} : \type{Integer}]\\
The standard environment provides the Integer type, which is implemented as Java's primitive data type, integer. That is
it's a 32-bit signed two's complement integer. When the interpreter detects a numeral it returns an integer value object. If
for instance the numeral exceeds the highest possible value a TypeError is thrown.
\end{dlist}

\subsubsection{Boolean}

\begin{dlist}
  \item \type{Boolean}[\variable{boolean} : \type{Boolean}]\\
  The standard environment provides the Boolean type, which is implemented as Java's primitive data type, boolean. That is, it only has two possible values: true and false. Even though the data type represents only one bit of information, according to the Java documentation, the ``size'' isn't precisely defined. 
\end{dlist}

\subsubsection{String}
\begin{dlist}
  \item \type{String}[\variable{string} : \type{String}]\\
    The standard environment provides the String type, which is implemented as Java's data type, String. That is, it may contain any unicode (UTF-16) characters. Though it is not possible to writ unicode characters of the form ``\textbackslash{}uXXXX'' as in Java (for instance ``\textbackslash{}u0108'', which is the capital C with circumflex, Ĉ). The \type{String} type contains one built-in constant:  
  \begin{dlist}
  \item \constant{size} : \type{Integer}\\
  The size constant returns the number of characters in the string, which is an integer value. For example ``test\_string''.size = $11$
  \end{dlist}
\end{dlist}
\subsubsection{List}

\begin{dlist}
  \item \type{List}[\variable{list} : \type{List}]\\
  The standard environment provides the List type. A list object can contain a mix of any types: strings, integers, other lists, game objects etc.
  This has both advantages and disadvantages. It increases the orthogonality of the programming language but it increases the risk of getting
  errors, which doesn't show until at run-time. The List type is similar to the ArrayList of Java and it's resizeable, which means that types can be added to the List. The type comes with a number of built-in constants and functions. 
  \item \constant{size} : \type{Integer}\\
  The size constant returns the number of elements in the list, which is an integer value. For example [``hi'', 2, 4].size = 3.
  \item \constant{sort}[\variable{comparator} : \type{Function}] : \type{List} \\
  The sort function sorts a list using a function that must take two parameters as input and return an integer value. For example [1, 6, 2, 5, 4, 3].sort[\#[\$a, \$b] $=>$ if \$a > \$b then 1 else if \$a == \$b then 0 else -1]. Will sort the list in ascending order. That is [1, 2, 3, 4, 5, 6].
  \item \constant{map}[\variable{mapper} : \type{Function}] : \type{List} \\
  The map function maps each element of the list with a function of style \#[\$a] => \$a. The function must take one parameter. For example [1, 2, 3, 4, 5, 6].map[\#[\$a] => \$a + 1] will return the list: [2, 3, 4, 5, 6, 7].
  \item \constant{filter}[\variable{filter} : \type{Function}] : \type{List} \\
  The filter function filters a list by feeding it with a function of style \#[\$a] => \$a >= 5, and returns a list with only the elements which comply with the function. The function fed to the filter function must take one parameter and return a boolean value. For example [1, 2, 3, 4, 5, 6].filter[\#[\$a] => \$a >= 5] will return [5, 6]. 
\end{dlist}

\subsubsection{Direction}
\begin{dlist}
  \item \type{Direction}[\variable{direction} : \type{Direction}]\\
  The standard environment provides the Direction type, which can be compared to a vector. There are eight different directions: n (north), s (south), w (west), e (east), nw, ne, sw, se. The type consist of an $x$ value and a $y$ value. For example n has value $y = 1$ and $x = 0$, s has value $y = -1$ and $x = 0$, w has value $y = 0$ and $x = -1$, etc.\ The Direction type is meant as a practical tool for use in patterns. 
\end{dlist}

\subsubsection{Coordinate}
\begin{dlist}
  \item \type{Coordinate}[\variable{coordinate} : \type{Coordinate}]\\
  The standard environment provides the Coordinate type. The Coordinate type is closely related to the Direction type in the way that it also consist of a $x$ value and an $y$ value. When the interpreter detects a number of capital letters followed by a number a numerals it returns a coordinate value object. Examples of coordinate values are A1, Z99 and ABCD1234. The coordinate value A1 corresponds to the $x$ value $1$ and $y$ value $1$, which is the top-left square on a board. The coordinate type is means as a practical tool to specify squares on a grid-formed board. Coordinate values must be positive, as negative $x$ and $y$ values make no sense representing coordinates off of the board.
\end{dlist}
\subsubsection{Type}
\begin{dlist}
  \item \type{Type}[\variable{type} : \type{Type}]\\
\end{dlist}
\subsubsection{Function}

\begin{dlist}
  \item \constdef{call}{[\farg{parameters}{List}]}{\opstar}\\
    Calls the function with the specified parameter list. 
\end{dlist}

\subsubsection{Pattern}

\subsection{Game environment}
\label{sec:gameenvironment}

The game environment provides a class hierarchy for describing a board game in an object-oriented manner.
In the game environment the following global functions are available:

\begin{dlist}
  \item \constant{addAction}[\variable{piece} : \type{Piece}, \variable{squares} : \type{List}] : \type{List}\\
    A function that returns a list of \type{AddAction}s to where it's possible to add a piece (\variable{piece}. The functions
    takes two parameters. The first parameter contains information on which type of piece the actions applies to. The second parameter is
    the list of squares where the type of piece can be added to.
    
    In the code example in \secref{codesample} in the beginning of the chapter, \function{addAction} is used in the following
    way: \\
    \begin{center}
    {addAction}[\variable{pieceType}[\keyword{this}], \variable{gameState}.\constant{board}.\constant{emptySquares}]
    \end{center}
    
    Here \constant{addAction} returns a list of empty squares to where it's possible to add a piece of the type \keyword{this}, which in this case was
    either a crosses piece or noughts piece depending whose turn it is.
    
  \item \constant{moveAction}[\variable{piece} : \type{Piece}, \variable{squares} : \type{List}] : \type{List}\\
  \constant{moveAction} works like \constant{addAction}, but instead of returning a \type{List} of \type{AddAction}s it returns a \type{List} of \type{MoveAction}s.
  	
    
\end{dlist}



\subsubsection{Game}
The \type{Game} type contains all information to describe a board game at a specific point at time.

\begin{dlist}
  \item \type{Game}[\variable{title} : \type{String}]\\
  Creates a instance of the \type{Game} type with a Game title of \variable{title}, \constant{board} set to \constant{initialBoard} and \constant{currentPlayer} set to \constant{turnOrder}[0].
  
  \item \constant{players} : \type{List}\\
  List of all \type{Player}s which are a part of this game.
  
  \item \constant{currentPlayer} : \type{Player}\\
  The \type{Player} from \constant{players} which currently have the turn.
  
  \item \constant{turnOrder} : \type{List}\\
  The order of \type{Player}s which determines in which order each \type{Player} from \constant{players} has their turn.
  
  \item \constant{initialBoard} : \type{Board}\\
  The value of \constant{board} at the beginning of each game.
  
  \item \constant{board} : \type{Board}\\
  The current state of a \type{Board} for this game.
  
  \item \constant{title} : \type{String}\\
  The title of the game which users can indentify the game with.
  
  \item \constant{description} : \type{String}\\
  An short explanation of the game and/or its rules.
  
  \item \constant{matchSquare}[ \variable{position} : \type{Coordinate}, \variable{pattern} : \type{Pattern} ] : \type{Boolean}\\
  Is true if \variable{pattern} is valid for \variable{position}.
  
  \item \constant{matchSquares}[ \variable{positions} : \type{List}, \variable{pattern} : \type{Pattern} ] : \type{Boolean}\\
  Is true if and only if all \type{Coordinate}s in \variable{positions} is true for \constant{matchSquare} with \variable{pattern}.
  
  \item \constant{findSquares}[ \variable{pattern} : \type{Pattern} ] : \type{List}\\
  \type{List} of all \type{Square}s where its \constant{position} matches \variable{pattern}.
  
  \item \constant{findSquaresIn}[ \variable{positions} : \type{List}, \variable{pattern} : \type{Pattern} ] : \type{List}\\
  \type{List} of \type{Square}s where its \constant{position} matches \variable{pattern}, but only \type{Squares} which \type{Coordinate} exists in \variable{positions}.
  
  \item \constant{history} : \type{List}\\
  \type{List} of all applied \type{Action}s.
  
  \item \constant{applyAction}[ \variable{action} : \type{Action} ] : \type{Game}\\
  A \type{Game} where \constant{board} have been updated according to \variable{action} and where \variable{action} is appended to \constant{history}.
  
  \item \constant{undoAction}[ \variable{action} : \type{Action} ] : \type{Game}\\
  A \type{Game} where \constant{board} have been reset to its state before \type{Action} was applied and with \constant{history} updated accordantly.
  
  \item \constant{setHistory}[ \variable{history} : \type{List} ] : \type{Game}\\
  A \type{Game} where \constant{history} is equaliant to \variable{history}.
  
  \item \constant{setBoard}[ \variable{board} : \type{GridBoard} ] : \type{Game}\\
  A \type{Game} where \constant{board} is equaliant to \variable{board}.
  
  \item \constant{setCurrentPlayer}[ \variable{i} : \type{Integer} ] : \type{Game}\\
  A \type{Game} where \constant{currentPlayer} is \constant{turnOrder}[\variable{i}].
  
  \item \constant{nextTurn}[] : \type{Game}\\
  The \type{Player} which has the turn after \constant{currentPlayer}.
\end{dlist}

\subsubsection{Board}
\begin{dlist}
  \item \type{Board}[]\\
  A \type{Board} with no \type{Piece}s.
  
  \item \constant{pieces} : \type{List}\\
  A \type{List} containing all \type{Piece}s associated with the \type{Board}.
  
  \item \constant{setPieces}[ \variable{pieces} : \type{List} ] : \type{Board}\\
  A \type{Board} where \constant{pieces} is equaliant to \variable{pieces}.
\end{dlist}

\subsubsection{GridBoard}
\type{GridBoard} \keyword{extends} \type{Board} to provide an easy way to describe rectangular \type{Board}s.

\begin{dlist}
  \item \type{GridBoard}[ \variable{width} : \type{Integer}, \variable{height} : \type{Integer} ]\\
  A \type{GridBoard} with \constant{width} and \constant{height} being \variable{width} and \variable{height} respectively.
  
  \item \constant{width} : \type{Integer}\\
  The width of the rectangular \type{Board}.
  
  \item \constant{height} : \type{Integer}\\
  The height of the rectangular \type{Board}.
  
  \item \constant{squares} : \type{List}\\
  A \type{List} of all associated \type{Square}s.
  
  \item \constant{setSqaures}[ \variable{squares} : \type{List} ] : \type{GridBoard}\\
  A \type{GridBoard} where \constant{squares} is equaliant to \variable{squares}.
  
  \item \constant{addPiece}[ \variable{piece} : \type{Piece}, \variable{position} : \type{Coordinate} ] : \type{GridBoard}\\
  A \type{GridBoard} where \variable{piece} is appended to \constant{pieces} and added to the \type{Square} at \variable{position}.
  
  \item \constant{addPieces}[ \variable{piece} : \type{Piece}, \variable{positions} : \type{List} ] : \type{GridBoard}\\
  A \type{GridBoard} where \variable{piece} is appended to \constant{pieces} and added to all the \type{Square}s at any of \variable{positions}.
  
  \item \constant{removePiece}[ \variable{piece} : \type{Piece} ] : \type{GridBoard}\\
  A \type{GridBoard} where \variable{piece} is off-board.
  
  \item \constant{movePiece}[ \variable{piece} : \type{Piece}, \variable{position} : \type{Coordinate} ] : \type{GridBoard}\\
  A \type{GridBoard} where \variable{piece} (which is already contained in \constant{pieces}) is \constant{onBoard} and is only included in one \type{Square}'s \constant{pieces}.
  
  \item \constant{squareAt}[ \variable{position} : \type{Coordinate} ] : \type{Square}\\
  The \type{Square} at \variable{position} in the rectangular grid of \type{GridBoard}.
  
  \item \constant{setSqauresAt}[ \variable{square} : \type{Square}, \variable{position} : \type{List} ] : \type{Square}\\
  A \type{GridBoard} where \constant{squareAt}[ \variable{position} ] is equaliant to \variable{square}.
  
  \item \constant{isFull} : \type{Boolean}\\
  Is true if \constant{emptySquares}.size is 0.
  
  \item \constant{emptySquares} : \type{List}\\
  A \type{List} with \type{Square}s from \constant{squares} where \constant{isEmpty} is false.
  
  \item \constant{squareTypes} : \type{List}\\
  A \type{List} with default \type{Square}s which will be used to create a checkered pattern of \type{Square}s in the grid of \type{Square}s.
\end{dlist}

\subsubsection{Square}
\type{Square} describes a position on the \type{Board} where 0-to-many \type{Piece}s can be placed.

\begin{dlist}
	\item \type{Square}[]\\
	\type{Square} with no \type{Piece}s.
	
	\item \constant{position} : \type{Coordinate}\\
	\type{Coordinate} describing the position on a \type{GridBoard}.
	
	\item \constant{pieces} : \type{List}\\
	A \type{List} with \type{Piece}s located on this \type{Square}.
	
	\item \constant{addPiece}[ \variable{piece} : \type{Piece} ] : \type{Square}\\
	A \type{Square} where \variable{piece} is appended to \constant{pieces}.
	
	\item \constant{removePiece}[ \variable{piece} : \type{Piece} ] : \type{Square}\\
	A \type{Square} where \variable{piece} is not contained in \constant{pieces}.
	
	\item \constant{setPieces}[ \variable{pieces} : \type{List} ] : \type{Square}\\
	A \type{Square} where \constant{pieces} is equaliant to \variable{pieces}.
	
	\item \constant{image} : \type{String}\\
	Path to an image file used for visualizing the \type{Square}.
	
	\item \constant{isOccupied} : \type{Boolean}\\
	Is true if \constant{pieces}.\constant{size} is larger than 0.
	
	\item \constant{isEmpty} : \type{Boolean}\\
	Is true if \constant{pieces}.\constant{size} is 0.
	
	\item \constant{setPosition}[ \variable{position} : \type{Coordinate} ] : \type{Square}\\
	A \type{Square} where \constant{position} is equaliant to \variable{position}.
\end{dlist}

\subsubsection{Piece}
\type{Piece} describes an item associated to a \type{Player} which the \type{Player} can manipulate in order to progress the game.

\begin{dlist}
  \item \type{Piece}[ \variable{owner} : \type{Player} ]\\
  \type{Piece} with \constant{owner} set to \variable{owner}.
  
  \item \constant{owner} : \type{Player}\\
  \type{Player} which owns this \type{Piece}.
  
  \item \constant{image} : \type{String}\\
  Path to an image file used for visualizing the \type{Piece}.
  
  \item \constant{position} : \type{Coordinate}\\
  \type{Coordinate} for the \type{Square} this \type{Piece} is located on.
  
  \item \constant{move}[ \variable{position} : \type{Coordinate} ] : \type{Piece}\\
  A \type{Piece} with \constant{position} set to \variable{position} and \constant{onBoard} set to true.
  
  \item \constant{remove}[] : \type{Piece}\\
  A \type{Piece} where \constant{position} is invalid and \constant{onBoard} is false.
  
  \item \constant{onBoard} : \type{Boolean}\\
  Is true if \type{Piece} is on the \type{GridBoard}.
  
  \item \constant{actions}[ \variable{game} : \type{Game} ] : \type{List}\\
  A \type{List} of possible \type{Action}s the \type{Piece} can make on its \constant{owner}'s turn.
\end{dlist}

\subsubsection{Player}
\begin{dlist}
  \item \type{Player}[ \variable{name} : \type{String} ]\\
  \type{Player} with \constant{name} set to \variable{name}
  
  \item \constant{name} : \type{String}\\
  The name of the \type{Player}.
  
  \item \constant{winCondition}[ \variable{game} : \type{Game} ] : \type{Boolean}\\
  Is true if the \type{Player} has won at the ending of this turn.
  
  \item \constant{tieCondition}[ \variable{game} : \type{Game} ] : \type{Boolean}\\
  Is true if the game ended without a winner.
  
  \item \constant{actions}[ \variable{game} : \type{Game} ] : \type{List}\\
  A \type{List} of \type{Action}s that the \type{Player} can do during his turn.
\end{dlist}

\subsubsection{Action}
\begin{dlist}
  \item \type{Action}[]\\
  Empty \type{Action}.
\end{dlist}

\subsubsection{UnitAction}
\type{UnitAction} \keyword{extends} \type{Action} to provide a basic change to be performed on \type{Game}.

\begin{dlist}
  \item \type{UnitAction}[ \variable{piece} : \type{Piece} ]\\
  A \type{UnitAction} with \constant{piece} set to \variable{piece}.
  
  \item \constant{piece} : \type{Piece}\\
  The \type{Piece} this \type{UnitAction} affects.
\end{dlist}

\subsubsection{AddAction}
\type{UnitAction} \keyword{extends} \type{Action} to add a \type{Piece} to a \type{Game}.

\begin{dlist}
  \item \type{AddAction}[ \variable{piece} : \type{Piece}, \variable{to} : \type{Square} ]\\
  An \type{AddAction} which adds \variable{piece} to \variable{to}.
  
  \item \constant{to} : \type{Square}\\
  \type{Square} to add \constant{piece} to.
\end{dlist}

\subsubsection{RemoveAction}
\type{UnitAction} \keyword{extends} \type{Action} to remove a \type{Piece} from a \type{Game}.

\begin{dlist}
  \item \type{RemoveAction}[ \variable{piece} : \type{Piece} ]\\
  A \type{RemoveAction} which removes \variable{piece}.
\end{dlist}

\subsubsection{MoveAction}
\type{UnitAction} \keyword{extends} \type{Action} to move a \type{Piece} to another \type{Square}.

\begin{dlist}
  \item \type{MoveAction}[ \variable{piece} : \type{Piece}, \variable{to} : \type{Square} ]\\
  A \type{MoveAction} which moves \variable{piece} to \variable{to}.
  
  \item \constant{to} : \type{Square}\\
  \type{Square} to add \constant{piece} to.
\end{dlist}

\subsubsection{ActionSequence}
\type{ActionSequence} \keyword{extends} \type{Action} to provide a sequence of \type{UnitAction}s to be performed in order.

\begin{dlist}
  \item \type{ActionSeqence}[ \ldots \variable{actions} : \type{UnitAction} ]\\
  \type{ActionSequence} with \constant{actions} set to [ \ldots \variable{actions} ].
  
  \item \constant{actions} : \type{List}\\
  A \type{List} of \type{UnitAction} to be performed in order.
  
  \item \constant{addAction}[ \variable{action} : \type{UnitAction} ] : \type{ActionSequence}\\
  A \type{ActionSequence} where \variable{action} is appended to constant{actions}.
\end{dlist}

\subsubsection{TestCase}
An abstract type for unit testing.

