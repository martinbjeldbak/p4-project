\chapter{Design}
\label{chap:design}

%should be possible to construct the language by reading this design document..

Since the dawn of time, programming languages have been developped far and wide.

We have developped \productname{} by first writing a bunch of implementations
using the young programming language. E.g. the following is an implementation of
the board game Noughts and Crosses:

\codesample{noughtandcrosses.game}

Based on our implementations, we first define the formal grammar of
\productname{}. Then the contextual constraints are informally described.
Lastly, the formal semantics are described.

\section{Grammar}

\subsection{Notational Conventions}
We use the following notational conventions, to present the
syntax of our language in this section:

\begin{center}
\begin{tabular}{r l}
  $\gopt{pattern}$ & an optional pattern \\
  $\grep{pattern}$ & zero or more repititions of pattern \\
  $\ggrp{pattern}$ & a group \\
  $pattern_1 \gor pattern_2$ & a selection \\
  $pattern_1 \gex pattern_2$ & matched by $pattern_1$ but not by $pattern_2$\\
  $\gter{ test }$ & a terminal \\
  $\gtsq$ & a terminal single quotation mark \\
  $\gtdq$ & a terminal double quotation mark \\
  $\gtbs$ & a terminal backslash character \\
\end{tabular}
\end{center}

\subsection{Character Classes}
\begin{ebnf}
\grule{decimal}{\gter{0} \gor \gter{1} \gor \grange \gor \gter{9}}
\grule{lowercase}{\gter{a} \gor \gter{b} \gor \grange \gor \gter{z}}
\grule{uppercase}{\gter{A} \gor \gter{B} \gor \grange \gor \gter{Z}}
\grule{anycase}{lowercase \gor uppercase}
\grule{quotebs}{\gtdq \gor \gtbs}
\grule{unichar}{\gcomment{any unicode character}}
\grule{strchar}{unichar \gex quotebs}
\end{ebnf}
\subsection{Reserved Tokens}
\begin{ebnf}
%reserved
\grule{keyword}{\gter{game} \gor \gter{piece} \gor \gter{this} \gor \gter{width} \gor \gter{height}}
\galt{\gter{title} \gor \gter{players} \gor \gter{turnOrder} \gor \gter{board}}
\galt{\gter{grid} \gor \gter{setup} \gor \gter{wall} \gor \gter{name} \gor \gter{possibleDrops}}
\galt{\gter{possibleMoves} \gor \gter{winCondition} \gor \gter{tieCondition}}
\grule{operator}{\gter{and} \gor \gter{or}}
\grule{pattern\_keyword}{\gter{friend} \gor \gter{foe} \gor \gter{this} \gor \gter{empty}}
\grule{pattern\_operator}{\gter{*} \gor \gter{?} \gor \gter{+} \gor \gter{!}}
\end{ebnf}
\subsection{Literals}
\begin{ebnf}
%Literals
\grule{integer}{deimal \grep{decimal}}
\grule{direction}{\gter{n} \gor \gter{s} \gor \gter{e} \gor \gter{w} \gor \gter{ne} \gor \gter{nw}}
\galt{\gter{se} \gor \gter{sw}}
\grule{coordinate}{upperccase \grep{uppercase} decimal \grep{decimal}}
\grule{string}{\gtdq \grep{strchar \gor \gtbs unichar} \gtdq}
\end{ebnf}
\subsection{Identifiers}
\begin{ebnf}
%Identifiers
\grule{function}{lowercase anycase \grep{anycase}}
\grule{identifier}{uppercase \grep{anycase}}
\grule{variable}{\gter{\$} anycase \grep{anycase}}
\end{ebnf}
\subsection{Program Structure}
\begin{ebnf}
%Program structure
\grule{program}{\grep{function\_def} game\_decl}
\grule{function\_def}{\gter{define} function \gter{$[$} \grep{variable} \gter{$]$} expression}
\grule{game\_decl}{\gter{game} declaration\_struct}
\grule{declaration\_struct}{\gter{\{} declaration \grep{declaration} \gter{\}}}
\grule{declaration}{( keyword \gor identifier ) structure}
\grule{structure}{declaration\_struct \gor expression}
\end{ebnf}
\subsection{Expressions}
\begin{ebnf}
%Expressions
\grule{expression}{function\_call}
\galt{element operator expression}
\galt{if\_expr}
\galt{lambda\_expr}
\galt{element}
\grule{element}{\gter{(} expression \gter{)}}
\galt{variable}
\galt{list}
\galt{pattern}
\galt{keyword}
\galt{direction}
\galt{coordinate}
\galt{integer}
\galt{string}
\galt{identifier}
\grule{function\_call}{function list}
\grule{if\_expr}{\gter{if} expression \gter{then} expression \gter{else} expression}
\grule{lambda\_expr}{\gter{\#} \grep{variable} \gter{=>} expression}
\grule{list}{\gter{[} \grep{element} \gter{]}}
\end{ebnf}
\subsection{Patterns}
\begin{ebnf}
%Patterns
\grule{pattern}{\gter{/} pattern\_expr \gter{/}}
\grule{pattern\_expr}{pattern\_expr \grep{pattern\_expr}}
\galt{\gter{(} pattern\_expr \gter{)}}
\galt{\gter{(} pattern\_expr \gter{)} integer}
\galt{pattern\_expr \gter{*}}
\galt{pattern\_expr \gter{?}}
\galt{pattern\_expr \gter{+}}
\galt{pattern\_val}
\grule{pattern\_val}{direction}
\galt{pattern\_check}
\galt{\gter{!} pattern\_check}
\grule{patterh\_check}{\gter{friend}}
\galt{\gter{foe}}
\galt{\gter{empty}}
\galt{\gter{this}}
\galt{identifier}
\end{ebnf}

\section{Types}

The language should have support for the following types:

\begin{description}[noitemsep]
\item[Integer] a 32-bit signed integer.
\item[String] a UTF-8 encoded string.
\item[Direction] a directional value.
\item[Coordinate] a 2-dimensional vector consisting of an $x$-integer value and
a $y$-integer value, representing a position in a grid.
\item[Lambda] a callable lambda expression.
\item[List] a list of other values.
\end{description}

\section{Scoping}
\label{sec:scoping}

A scope is the context in which one or more variables exist.
There are three types of scopes in \productname{}. Function scopes, lambda scopes and
``let''-scopes.

\subsection{Function scope}
Consider a function definition such as:

\codesample{functiondef.garry}

The variables \variable{a} and \variable{b} only exist within the function \function{max}.
When calling the function:

\codesample{functioncall.garry}

A new scope will be created and the values $5$ and $23$
are assigned to \variable{a} and \variable{b}, respectively.

Named functions (such as \function{max}) always exist in the global scope.

\subsection{Lambda expression scope}

When a lambda expression is created, a reference to the scope it was created
in is saved with it. This is known as a closure, and means that a lambda
expression may access variables outside of its own scope. The accessible
variables are the variables that were available at the time of the creation
of the lambda expression.

Consider the following example:

\codesample{closuredef.garry}

A function \function{getAdder}, which takes one argument (\variable{a}) and
returns a lambda expression, is defined. Notice how \variable{a} is used within
the lambda expression. This means that when the lambda expression is created, it
must remember the value of the variables that exist in the scope, in which it is
created. The use of the \function{getAdder}-function could look like this:

\codesample{closureuse.garry}

In the first line \function{getAdder} is called with the argument, $25$. A new scope, $A$,
is created, in which the variable \variable{a} is assigned the value $25$. Then the function
expression is evaluated, which results in a new lambda expression (with a reference to scope $A$).
This is returned and assigned to \variable{adder} in line 1 of the above example.

In the second line the \variable{adder} is called as a function, which means that a new scope, $B$,
is created, in which the variable \variable{b} is assigned the value $5$. The important part is
that $B$'s parent scope is set to $A$ (which is saved with the lambda expression). The expression
(the right side of the lambda expression) is then evaluated. First the \variable{a}-variable is
encountered. The interpreter first searches the $B$-scope for \variable{a}, and when unsuccessful,
searches the parent-scope, $A$, for \variable{a}. In $A$ the variable \variable{a} holds the value
$25$, and this is returned. Then the $B$-scope is searched for the \variable{b}-variable, and the value
$5$ is returned. The two integers are added, and the final return-value of the lambda-expression
ends up beinng $30$.

\subsection{Let-expressions}


Outline:

\begin{dlist}
\item What is scoping?
\item Examples of static/lexical versus dynamic scoping
\item Why do we want to use dynamic scoping
\item What does that mean for \productname?
\end{dlist}

\section{Operators}

\productname{} supports a number of built-in operators. In this section the operators
of \productname{} are described using the format:

\operator[LeftOperandType]{operator}{RightOperandType}{ReturnType}

The available types are described in \secref{sec:types}.
Operations that are not described in this section can be considered invalid.

\subsection{Boolean operators}

These operators only accept boolean operands and only return boolean values.

\begin{itemize}[noitemsep]
  \item \operator[Boolean]{and}{Boolean}{Boolean}\\
    Returns true when both operands are true and false otherwise. 
  \item \operator[Boolean]{or}{Boolean}{Boolean}\\
    Returns true when at least one of the operands are true and false otherwise.
  \item \operator{not}{Boolean}{Boolean}\\
    Returns true if the single operand is false and false otherwise.
\end{itemize}

\subsection{Comparison operators}

These operators are used when comparing other values, they will always return
boolean values.

\begin{itemize}[noitemsep]
  \item \operator[Integer]{<}{Integer}{Boolean}\\
    Returns true if the left operand is less than the right one.
  \item \operator[Integer]{>}{Integer}{Boolean}\\
    Returns true if the left operand is greater than the right one.
  \item \operator[Integer]{<=}{Integer}{Boolean}\\
    Returns true if the left operand is less than or equal to the right one.
  \item \operator[Integer]{>=}{Integer}{Boolean}\\
    Returns true if the left operand is greater than or equal to the right one.
  \item \operator[Integer]{==}{Integer}{Boolean}\\
    \operator[String]{==}{String}{Boolean}\\
    \operator[Boolean]{==}{Boolean}{Boolean}\\
    Returns true if the left operand is equal to the right one.
  \item \operator[Integer]{!=}{Integer}{Boolean}\\
    \operator[String]{!=}{String}{Boolean}\\
    \operator[Boolean]{!=}{Boolean}{Boolean}\\
    Returns true if the left operand is not equal to the right one.
\end{itemize}

\subsection{Integer operators}

The following operations are possible on integers:

\begin{itemize}[noitemsep]
  \item \operator[Integer]{+}{Integer}{Integer} \\
    Integer addition.
  \item \operator[Integer]{-}{Integer}{Integer} \\
    Integer subtraction.
  \item \operator[Integer]{*}{Integer}{Integer} \\
    Integer multiplication.
  \item \operator[Integer]{/}{Integer}{Integer} \\
    Integer division.
\end{itemize}

\subsection{String operators}

It is possible to concatenate strings:

\begin{itemize}[noitemsep]
  \item \operator[String]{+}{String}{String} \\
    Returns the concatenation of two strings.
  \item \operator[String]{+}{Integer}{String} \\
    Returns the concatenation of a string and the string-representation of an integer.
\end{itemize}

\section{Functions}

Many functions are made available to the programmer in \productname.

\section{Patterns}
\label{sec:patterns}

This section covers how to use patterns and what to use them for. The operators of a pattern looks like and behaves a little like regular expressions. This EBNF-grammar shows how a pattern is constructed:

\begin{ebnf}
\grule{pattern}{pattern\_expr \gcat \grep{pattern\_expr}}
\grule{pattern\_expr}{pattern\_val \gcat \gopt{\gter{*} \gor \gter{?} \gor \gter{+}}}
\galt{pattern\_val \gcat \gter{|} \gcat pattern\_expr}
\grule{pattern\_val}{direction}
\galt{variable}
\galt{pattern\_check}
\galt{\gter{!} \gcat pattern\_check}
\galt{\gter{(} \gcat pattern \gcat \gter{)} \gcat \gopt{integer}}
\grule{pattern\_check}{\gter{friend}}
\galt{\gter{foe}}
\galt{\gter{empty}}
\galt{\gter{this}}
\galt{type}
\end{ebnf}


A pattern is checked on a particular square, and returns either true or false. An example of a pattern is \text{/n n e empty/}. This pattern can be checked on the board seen in \figref{patternboard} on the field \textbf{A1}. The pattern says ``go one square north, go one square north, go one square east, check if square is empty''. This means that the square \textbf{B3} will be checked for emptiness. Since the square is occupied by a piece, the pattern will return false if checked on \textbf{A1}. However, the same pattern checked on \textbf{C1} will return true since the square at \textbf{D3} is empty.
\fig[scale=2]{patternboard}{A simple 4 X 4 board with 3 pieces}

With this basic introduction to a simple pattern check, the table (insert ref and convert list below to a table) describes briefly how the different pattern operators work. For each operator, an example of the use in a context are given in \secref{sec:patternexamples}. Note that the description of patterns assumes a minor understanding of regular expressions, see \cite{regex}.

\texttt{empty} : current square contains no pieces\\
\texttt{n} : north \\
\texttt{e} : east \\
\texttt{s} : south \\
\texttt{w} : west \\
\texttt{*} : zero-to-many times\\
\texttt{+} : one-to-many times\\
\texttt{?} : zero-or-one time\\
\texttt{|} : or\\
\texttt{!} : not\\
\texttt{(} \textit{pattern} \texttt{)} : encapsulation\\
\texttt{friend} :  current square contains at least one friendly piece of the current player\\
\texttt{foe} : current square contains at least one enemy piece of the current player\\
\textit{type} : the current square is of the given type or a piece of the given type is residing on the current square.
\texttt{this} : current square contains the piece for which the check is being performed\\

\subsection{Pattern examples}
\label{sec:patternexamples}
All these examples of pattern checks are performed on the board and pieces seen in \figref{patternboard}. For each operator, two examples of a pattern check on a particular square is given. The first check returns true and the second check false.

On A1, the pattern check \texttt{/empty/} returns true because A1 is empty\\
On B2, the pattern check \texttt{/empty/} returns false because B2 is not empty\\
On A1, the pattern check \texttt{/n empty/} returns true because A2 is empty\\
On B1, the pattern check \texttt{/n empty/} returns false because B2 is not empty\\
On C3, the pattern check \texttt{/e empty/} returns true because C4 is empty\\
On C4, the pattern check \texttt{/e empty/} returns false because C5 is out of board\\
On A1, the pattern check \texttt{/n* n e empty/} returns true because moving north 2 times then north and east hits an empty square on B4\\

Notice that the *-operator causes many branches to be made. The previous pattern check, \texttt{/n* n e empty/} done on A1, has a branch checking \texttt{/n e empty/}. The branch dies because B2 is not empty. If just one branch survives, the pattern check returns true. In the example, the only branch surviving is the \texttt{n n n e empty} branch. The same rules for branching counts for the \texttt{+}, \texttt{?} and \texttt{|} operator. When a branch moves out of board it dies immediately.

On C3, the pattern check \texttt{/n* s s !empty/} returns false because neither A1 nor A2 contains a piece\\
On A1, the pattern check \texttt{/n+ e empty/} returns true only because B4 is empty\\
On B1, the pattern check \texttt{/n+ empty e !empty/} returns false because B4 is the only empty square north of B1 and C4 is empty\\
On B3, the pattern check \texttt{/s? e empty/} returns true only because C2 is empty\\
On C2, the pattern check \texttt{/n? w empty/} returns false because neither B2 nor B3 is empty\\
On A2, the pattern check \texttt{/(n | e) empty/} returns true only because A3 is empty\\
On C2, the pattern check \texttt{/e | w empty/} returns false because neither B2 nor C3 is empty\\
On B1, the pattern check \texttt{/(n n | e e) empty/} returns true only because D1 is empty\\
On A1, the pattern check \texttt{/(n w)* empty/} returns true both because A1 is empty and because D4 is empty\\

The \texttt{friend}-keyword is evaluated based on the current player. Suppose that we have a player called Green, who owns the 
green piece. If it is green's turn to move, any branch of a pattern check will return false
whenever it meets a keyword \texttt{friend} on a square that does not contain any of Green's pieces.
On B2, the pattern check \texttt{/n|e|(n e) friend/} will return true if it is Green's turn, since C3 contains a friendly piece of Green.
On C3, the pattern check \texttt{/(e | w)+/} will return false if it is Green's turn, since no square containing a piece of Green can be reached by going east or west from C3 one or more times.

The keyword \texttt{foe} does the opposite of \texttt{friend}. It makes a branch continue only if its current square contains at least one piece not owned by the player who has the turn.

Just like \texttt{foe} and \texttt{friend}, the name of a piece or square-type defined in a \productname{}-game can also be used. E.g, the keyword \texttt{White} can be used if a piece or square-type with the name \textit{White} has been define. In a such case, a branch survives only if its current square is of type \textit{White} or if the square contains a piece of type \textit{White}.

A pattern can also check that a specific piece exists on a specific square. This is done using the \texttt{this} keyword.
Before this check can be achieved, the pattern must be checked regarding to a specific piece. Suppose that we on A3, make the pattern check
\texttt{empty e this}. If this check is done in relation to the black piece on B3 it returns true. However, in relation to the black piece on B2, the pattern check returns false. To understand both how this function exactly works and why this is useful, consider the board in \figref{fig:patternboard}.
If we for any piece specify that it can move to a square for which the pattern check \texttt{/(n | s) empty/} is true, this means that it can move to any square except $\{B1, B4, C4\}$. These square does not have an empty square north or east from it. Recall that a branch going out of board dies.
However, the pattern check \texttt{/empty (n | s) this/} will in relation to the green piece return true only when checked on the squares $\{C2, C4\}$.
This can be used to specify that a piece can move to an empty square one north or one south from its current square. 
\section{Structural operational semantics}
\label{sec:structuraloperationalsemantics}

In the following subsections we will describe how programs, written in
\productname{}, behave. We begin by defining the abstract syntax for our
language which consists of a list of different syntactic categories followed by
formulation rules. When these have been presented we can move on to the
description of our operational semantics which is our big-step
semantics.\cite[pg. 27]{tt-hh}

\section{Abstract syntax}

The abstract syntax is the interpreter or compiler's internal representation of a program. It is represented
as an abstract syntax tree.

This section should cover all aspects of our abstract syntax tree, and how it differs from the
parse tree.

\subsection{Program}
\begin{figure}[H]\begin{center}\begin{tikzpicture}[level/.style={sibling distance=40mm/#1}]
\node [square] (z){Program}
  child {node [square] (a) {Function definition} edge from parent[dashed];}
  child {node [square] (aa) {Function definition} edge from parent[dashed];}
  child {node [square] (b) {Game decleration}};
\path (a)--(aa) node [midway] {$\cdots$};
\end{tikzpicture}\end{center}\end{figure}

\subsection{Variable list}
\begin{figure}[H]\begin{center}\begin{tikzpicture}[level/.style={sibling distance=30mm/#1}]
\node [square] (z){Variable list}
  child {node [square] (a) {Variable} edge from parent[dashed];}
  child {node [square] (aa) {Variable} edge from parent[dashed];};
\path (a)--(aa) node [midway] {$\cdots$};
\end{tikzpicture}\end{center}\end{figure}

\subsection{Function definition}
\begin{figure}[H]\begin{center}\begin{tikzpicture}[level/.style={sibling distance=30mm/#1}]
\node [square] (z){Function definition}
  child {node [square] (a) {Function}}
  child {node [square] (b) {Variable list}}
  child {node [ellipse,draw] (c) {\textit{Expression}}};
\end{tikzpicture}\end{center}\end{figure}

\subsection{Game decleration}
\begin{figure}[H]\begin{center}\begin{tikzpicture}[level/.style={sibling distance=30mm/#1}]
\node [square] (z){Game decleration}
  child {node [square] (a) {Decleration struct}};
\end{tikzpicture}\end{center}\end{figure}

\subsection{Decleration struct}
\begin{figure}[H]\begin{center}\begin{tikzpicture}[level/.style={sibling distance=30mm/#1}]
\node [square] (z) {Decleration struct}
  child {node [square] (a) {Decleration}}
  child {node [square] (b) {Decleration} edge from parent[dashed];}
  child {node [square] (c) {Decleration} edge from parent[dashed];};
  
\path (b)--(c) node [midway] {$\cdots$};
\end{tikzpicture}\end{center}\end{figure}

\subsection{Decleration}
\begin{figure}[H]\begin{center}\begin{tikzpicture}[level/.style={sibling distance=30mm/#1}]
\node [square] {Decleration}
  child {node [ellipse split,draw] {Keyword \nodepart{lower} Identifier}}
  child {node [ellipse,draw] {\textit{Structure}}};
\end{tikzpicture}\end{center}\end{figure}

\subsection{Assignment}
\begin{figure}[H]\begin{center}\begin{tikzpicture}
[level/.style={sibling distance=40mm},
level 1/.style={sibling distance = 39mm},
level 2/.style={sibling distance = 20mm}]
\pgfdeclareshape{z,a,b,c,x,y,d,o,p,e}

\node [square] (z) {Assignment}
  child {node [square,left of=b,xshift=-4cm] (a) {Variable}}
  child {node [ellipse,draw,left of=c,xshift=-4.5cm] (b) {\textit{Expression}}}
  child {node [square] (c) {Assignment} edge from parent[dashed]
  	child {node [square,xshift=-1cm] (x) {Variable} edge from parent[solid]}
  	child {node [ellipse,draw,solid,xshift=-1cm] (y) {\textit{Expression}} edge from parent[solid]}
  }
  child {node [square,xshift=-1cm] (d) {Assignment} edge from parent[dashed]
  	child {node [square,xshift=1cm] (o) {Variable} edge from parent[solid]}
  	child {node [ellipse,draw,solid,xshift=1cm] (p) {\textit{Expression}} edge from parent[solid]}
  }
  child {node [ellipse,draw,right of=d,xshift=1.5cm](e) {\textit{Expression}}};

\path (c)--(d) node [midway] {$\cdots$};
\end{tikzpicture}\end{center}\end{figure}

\subsection{If expression}
\begin{figure}[H]\begin{center}\begin{tikzpicture}[level/.style={sibling distance=30mm/#1}]
\node [square] (z){If expression}
  child {node [ellipse,draw] (a) {Expression}}
  child {node [ellipse,draw] (b) {Expression}}
  child {node [ellipse,draw] (c) {Expression}};
\end{tikzpicture}\end{center}\end{figure}

\subsection{Lambda expression}

\begin{figure}[H]\begin{center}\begin{tikzpicture}[level/.style={sibling distance=30mm/#1}]
\node [square] (z){Lambda expression}
  child {node [square] (a) {Variable list}}
  child {node [ellipse,draw] (b) {\textit{Expression}}};
\end{tikzpicture}\end{center}\end{figure}

\subsection{List}
\begin{figure}[H]\begin{center}\begin{tikzpicture}[level/.style={sibling distance=30mm/#1}]
\node [square] (z){List}
  child {node [ellipse,draw] (a) {Element} edge from parent[dashed]}
  child {node [ellipse,draw] (b) {Element} edge from parent[dashed]};

\path (a)--(b) node [midway] {$\cdots$};
\end{tikzpicture}\end{center}\end{figure}

\subsection{Pattern}
\begin{figure}[H]\begin{center}\begin{tikzpicture}[level/.style={sibling distance=40mm/#1}]
\node [square] (z){Pattern}
  child {node [square] (a) {Pattern expression}}
  child {node [square] (b) {Pattern expression} edge from parent[dashed]}
  child {node [square] (c) {Pattern expression} edge from parent[dashed]};

\path (b)--(c) node [midway] {$\cdots$};
\end{tikzpicture}\end{center}\end{figure}

\subsection{Pattern or-operator}
\begin{figure}[H]\begin{center}\begin{tikzpicture}[level/.style={sibling distance=40mm/#1}]
\node [square] {Pattern, or-operator}
  child {node [square] {Pattern value}}
  child {node [square] {Pattern expression}};
\end{tikzpicture}\end{center}\end{figure}

\subsection{Pattern multiplier-operator}
\begin{figure}[H]\begin{center}\begin{tikzpicture}[level/.style={sibling distance=40mm/#1}]
\node [square] {Pattern, multiplier-operator}
  child {node [square] {Pattern value}};
\end{tikzpicture}\end{center}\end{figure}

\subsection{Pattern not-operator}
\begin{figure}[H]\begin{center}\begin{tikzpicture}[level/.style={sibling distance=40mm/#1}]
\node [square] {Pattern, not-operator}
  child {node [square] {\textit{Pattern check}}};
\end{tikzpicture}\end{center}\end{figure}

\subsection{Not-operator}
\begin{figure}[ht]\begin{center}\begin{tikzpicture}[level/.style={sibling distance=40mm/#1}]
\node [square] {Not-operator}
  child {node [ellipse, draw] {\textit{Expression}}};
\end{tikzpicture}\end{center}\end{figure}
\section{Big-step semantics}
There are two kinds of operational semantics. we will only use and describe the
big-step semantics for \productname{}. The reason for not describing small-step
semantics is taht we do not find i necessary. The big-step semantics will
capture the operational semantics of programs written in \productname{}.

We have divided the following subsections into expressions, lists, variable
lists, and definitions. In each section we describe the construction of the
big-step semantics by showing the transition rules for each construct.

\subsection{Expressions}

\begin{table}[ht]
  \begin{center}
    \begin{tabular*}{\textwidth}{lc}
      $[\mbox{LET}]$ & \infrule{env_{V} \vdash E_{1} \ra
      v_{1} \quad env_{V}\left[x \mapsto v_{1}\right] \vdash E_{2} \ra
      v_{2}}{env_{V} \vdash \texttt{let} x = E_{1} \texttt{in} E_{2} \ra v_{2}}
    \end{tabular*}
  \end{center}
\end{table}



%      $[\mbox{MINUS}]$ & \infrule{s \vdash E_{1} \ra v_{1} \: s \vdash E_{2} \ra
%      v_{2}}{s \vdash E_{1} - E_{2} \ra v} & where $v=v_{1}-v_{2}$ \\
 %     $[\mbox{MULTIPLICATION}]$ & \infrule{s \vdash E_{1} \ra v_{1} \: s \vdash E_{2} \ra
 %     v_{2}}{s \vdash E_{1} * E_{2} \ra v} & where $v=v_{1}*v_{2}$ \\
 %     $[\mbox{DIVISION}]$ & \infrule{s \vdash E_{1} \ra v_{1} \: s \vdash E_{2} \ra
 %     v_{2}}{s \vdash E_{1} / E_{2} \ra v} & where $v=v_{1}/v_{2}$ \\
 %     $[\mbox{MODULO}]$ & \infrule{s \vdash E_{1} \ra v_{1} \: s \vdash E_{2} \ra
 %     v_{2}}{s \vdash E_{1} \% E_{2} \ra v} & where $ $ \\

%\begin{table}[ht]
%  \begin{center}
%    \begin{tabular*}{\textwidth}{lc}
%      $[\mbox{RULE}]$ & \infrule{\lag Something, Something \rag \ra
%      something}{\lag something, something, something \rag \ra something } \\
%    \end{tabular*}
%  \end{center}
%\end{table}


%\subsection{Lists}

%\begin{table}[ht]
%  \begin{center}
%    \begin{tabular*}{\textwidth}{lc}
%     $[\mbox{RULE}]$ & \infrule{\lag Something, Something \rag \ra
%      something}{\lag something, something, something \rag \ra something } \\
%    \end{tabular*}
%  \end{center}
%\end{table}


%\subsection{Variable lists}

%\begin{table}[ht]
%  \begin{center}
%    \begin{tabular*}{\textwidth}{lc}
%      $[\mbox{RULE}]$ & \infrule{\lag Something, Something \rag \ra
%      something}{\lag something, something, something \rag \ra something } \\
%    \end{tabular*}
%  \end{center}
%\end{table}


%\subsection{Definitions}

%\begin{table}[ht]
%  \begin{center}
%    \begin{tabular*}{\textwidth}{lc}
%      $[\mbox{RULE}]$ & \infrule{\lag Something, Something \rag \ra
%      something}{\lag something, something, something \rag \ra something } \\
%    \end{tabular*}
%  \end{center}
%\end{table}


