\chapter{Design}
\label{chap:design}

In this chapter we present the design of our programming language
(\productname{}). We begin this chapter by giving an example of a game implemented in Junta. 
The abstract syntax of \productname{} will be presented in
\secref{sec:abstractsyntax}, followed by the lexical structure of programs in
\secref{sec:lexicalstructure}. We also present the different expressions and
their grammar in \secref{sec:expressions} followed by the different definitions
and their grammar in \secref{sec:definitions}. Futhermore, we introduce patterns
in \secref{sec:patterns} which are very important in \productname{}.  Finally,
we present the predefined types and constants of \productname{} in
\secref{sec:predefined}, where the standard and game environments are presented.

In this chapter a number of terms are used, when referring to different aspects
of the programming language. Essentially the language consists of
\emph{constants}, \emph{functions} and \emph{types}.

A type is a structure, that may inherit from a \emph{super
type}/\emph{parent type} and contains \emph{members} in the form of \emph{data
fields} and \emph{constants}. A constant may be a function in which case it is
referred to as a \emph{method}. Types and constants can be \emph{abstract}. A
type can be instantiated using its \emph{constructor}, this creates an
\emph{instance} or \emph{object} of that type. All values in \productname{} are
instances of types.

\subsection*{Code example}

\productname{} is a purely functional object-oriented programming language
designed explicitly for creating board games. We have developed \productname{}
by first brainstorming and writing a handful of (partial) game implementations using a
``programming language'' which felt the most natural to us. This means that we 
were actually using the programming language before we had even constructed
it. We began by writing programs in the unfinished language to try to find out
how it should be built and what would be the easiest to write and understand. 

In the following we go through the \csref{noughtandcrosses.game}. The example feature
an implementation of the Noughts and Crosses game and introduces some of the important concepts of \productname{}. These will be
further described in the rest of this chapter and in \chapref{chap:implementation}.

The very first thing that is visible in \csref{noughtandcrosses.game} are the
two lines of comments. Comments are made with two forward slashes. Comments are
described further in \secref{sec:comments}.

The next thing that happens is the declaration of a type: \keyword{type}
\type{NacGame}[].  The types of \productname{} can be compared to classes as
seen in other object-oriented programming languages. The square brackets in
\productname{} are used to encapsulate parameters and members
of lists. \keyword{type} \type{NacGame}[] is a sub-type of the super type \type{Game}
which is a built-in type in \productname{}. The \keyword{extends} keyword is
similar to the extends keyword of Java, and means the \type{NacGame}[] type
inherits members of the \type{Game} type.  \type{Game}'s constructor takes the
title of the game as input and it contains many useful constants and
functions, described further in \secref{sec:predefined}. 

\codesample{noughtandcrosses.game}

Constants and functions can be thought of as subprograms similar to methods.
They distinguish themselves from each other by the fact that constants cannot
take any parameters whereas functions can. Functions and constants are further
explained in \secref{sec:constantdefinitions}. One of the built-in constants is
\constant{players} which contains a list of players. In the code sample at line
four we see how the constant is defined. When a game written in
\productname{} is played, the turn is shifted between each of the players in
the list provided by the \constant{players} constant. This is true, unless the turn
order is specifically modified by the constant \constant{turnOrder}, which is
another built-in constant in the \type{Game} type. This however is not necessary
in a Noughts and Crosses game, since the default turn order is desired. For instance,
if the first player must be able to make three turns before the second player
can make one, that would be defined in the \constant{turnOrder}.

Furthermore, \type{Game} contains the constant \constant{initialBoard} which in
this case is assigned a grid board (another built-in type that takes the height
and width of the board as parameters) of $3 \times 3$ squares. In other
programming languages the override keyword is used when implementing methods
from a super class but in \productname{} the override functionality
already exists in the define keyword. For instance in
\csref{noughtsandcrosses.game} \constant{players} and \constant{initialBoard} are
overridden in line four and eight, respectively.

%A second type \type{NacPlayer}, is declared which extends the built-in
%\productname{} type \type{Player}. The \type{Player} type takes as input the name of
%the player and contains three important functions: \function{winCondition},

The next thing that is important in the code sample are the three functions: 
\function{winCondition}, \function{tieCondition} and \function{actions}. 
As the name indicates, the function \function{winCondition} checks if the current player
is in a winning state and returns a boolean value: true or false.
\function{winCondition} takes a game object as input. In Noughts and Crosses the
win condition is obtained if a player has three of his pieces in a row, in
either a vertical, horizontal or diagonal line. In \productname{} this is
specified by using what we call ``patterns'', described in
\secref{sec:patterns}. Patterns begin and end with forward slashes. A pattern for
\function{winCondition} can be seen through line $13$ and $14$.

The \function{tieCondition} function checks if a tie condition is obtained and
returns a boolean value: true or false. The tie condition is achieved whenever
the board is full. This is specified using the built-in function:
\function{isFull}. 

The last function \function{actions} also takes as input a game object and
contains a list of actions. In this case the only possible action is the
\function{addAction} function which makes it possible to add a piece to the
board of type \keyword{this}, which is the current player's piece type (crosses
or noughts), to an empty square on the board.

At first sight the code sample will look complicated. This is mostly due to the
overwhelming use of built-in functionality. This is however implemented to make
it easier and faster for programmers to write code in \productname{}, since they
don't have to implement all the functionality themselves.

\subsection*{Type system}

In \secref{sec:typesystemanalysis} we analysed the two main type system
approaches.  We chose the dynamic type system due to the fact that it increases
the writability of our programming language. As seen in the above code example
above a Noughts and Crosses game can be created in only
approximately $20$ lines of code.

The type system in \productname{} comprises a number of simple types, from which
every other type can be created. The simple types are: integers, character
strings, booleans, lists, directions (vectors), coordinates (points), patterns,
functions and types. Unlike in other programming languages, both functions and
types in \productname{} are first-class citizens, this means that they can
be passed like any other value. This adds even more
flexibility to the language. For instance in the Noughts and Crosses example
(see \csref{noughts-and-crosses.junta}) both the types \type{Noughts} and
\type{Crosses} are passed as values to the constructor of \type{NacPlayer}. The
simple types, their operators and their methods are further explained in the
rest of this chapter.

User types can be created using the \keyword{type}-keyword. They are very
similar to classes in traditional object-oriented languages, in that a type has
a constructor, attributes, constants and methods. An important aspect of the
type system of \productname{} however, is that all values are immutable. It
doesn't matter if it's an integer, a list, or an instance of a custom user type,
the value of the object can't be changed. They can however be cloned and
modified using various techniques depending on the type in question. For
instance, adding two integers using the \texttt{+}-operator returns a new
integer representing the sum of the two operands. For instances of user types a
new modified instance is returned when using the \keyword{set}-keyword (as
described in \secref{sec:setexpressions} and \secref{sec:typedefinitions}). The
reason for this functionality, is to prevent side-effects, since randomly
changing objects, could have undesirable influence on other functions or types
that depend on these objects.

The type system of \productname{} supports single inheritance, meaning that a
user of type \type{B} can extend the type \type{A} and inherit all members and 
\productname{} supports single inheritance between types. An inheriting type, will
inherit all the members of its super type(s) (if \type{C} extends \type{B}, which
extends \type{A}, then \type{C} inherits all members from both \type{B} and
\type{A}). Visibility in \productname{} is implicit, in that all
constants/methods are public (they can be accessed from anywhere as long as an
instance of the type is available) while all data fields are private (they can
only be accessed/changed from within the specific type, not even from inheriting
types). Getters and setters are necessary in order to access data fields from
the outside or in inheriting types. More details on data, inheritance, members
and abstract members are available in \secref{sec:setexpressions} and
\secref{sec:typedefinitions}.

Another feature of \productname{} is implicit casting, when dealing with simple
types (such as integers and strings). If a user were to create a type
\type{MyInteger} extending the built-in type \type{Integer} (the type of
integers in \productname{}), then instances of \type{MyInteger} could be used in
place of simple integers, this works by casting the instance to a simple integer
value (simply by throwing away the extra information provided by the
\type{MyInteger}-type). Explicit casting is only really possible with simple
types, since their type constructors accepts one parameter of the same type. For
instance the constructor \type{Integer} accepts a parameter of type
\type{Integer}, meaning that it also accepts types that extend \type{Integer}.
This makes it possible to cast a value of type \type{MyInteger} to a raw
\type{Integer} value, albeit the usefulness of this functionality is dubious.
The constructors of all the simple types implement this functionality however, shown later in \secref{sec:standardenvironment}.

\subsection*{Scope rules}

A scope is the context in which one or more variables or constants exist. In
\productname{} we for instance have different expressions with their own scopes
where their variables live and die. By this we mean that when the scope of the
expression ends, the variables within the scope cannot be accessed anymore.
These expressions with scopes will be defined in \secref{sec:expressions}.

Furthermore, it is important to know that \productname{} uses static scoping.
The different kinds of scope rules were described in \secref{sec:anal-scoperules}.

\section{Abstract syntax}

The abstract syntax is the interpreter or compiler's internal representation of a program. It is represented
as an abstract syntax tree.

This section should cover all aspects of our abstract syntax tree, and how it differs from the
parse tree (e.g. there are no expressions or elements in the AST).


\section{Lexical structure}
\label{sec:lexicalstructure}

This section presents the low-level non-terminals of \productname{}.

\subsection{Notational Conventions}
We use a variant of Extended Backus-Naur Form to express the context-free grammar of
our programming language.

Each production rule assigns an expression of terminals, non-terminals and operations
to a non-terminal. E.g. in the following example the non-terminal $decimal$ is assigned
the possible terminals of $\gter{0}$ up to and including $\gter{9}$.

\begin{ebnf}
\grule{decimal}{\gter{0} \gor \gter{1} \gor \grange \gor \gter{9}}
\end{ebnf}

The following operations are used throughout this section to describe the grammar of the programming language:

\begin{center}
\begin{tabular}{r l}
  $\gopt{pattern}$ & an optional pattern \\
  $\grep{pattern}$ & zero or more repititions of pattern \\
  $\ggrp{pattern}$ & a group \\
  $pattern_1 \gor pattern_2$ & a selection \\
  $\gter{0} \gor \grange \gor \gter{9}$ & a range of terminals \\
  $pattern_1 \gex pattern_2$ & matched by $pattern_1$ but not by $pattern_2$\\
  $pattern_1 \gcat pattern_2$ & concatenation of $pattern_1$ and $pattern_2$ \\
  $\gter{test}$ & a terminal \\
  $\gtsq$ & a terminal single quotation mark \\
  $\gtdq$ & a terminal double quotation mark \\
  $\gtbs$ & a terminal backslash character \\
\end{tabular}
\end{center}

\subsection{Character Classes}
To be able to describe which characters or symbols a non-terminal can consist of in a concise manner, we need to define some sets of symbols to specific names which we can use in the description of our grammar.

The following classes of characters will be used throughout this section:

\begin{ebnf}
\grule{decimal}{\gter{0} \gor \gter{1} \gor \grange \gor \gter{9}}
\grule{lowercase}{\gter{a} \gor \gter{b} \gor \grange \gor \gter{z}}
\grule{uppercase}{\gter{A} \gor \gter{B} \gor \grange \gor \gter{Z}}
\grule{anycase}{lowercase \gor uppercase}
\grule{alphanum}{anycase \gor decimal}
\grule{quotebs}{\gtdq \gor \gtbs}
\grule{unichar}{\gcomment{any unicode character}}
\grule{strchar}{unichar \gex quotebs}
\end{ebnf}

\subsection{Comments}

In \productname{} single line comments begin with a sequence of at least two forward slashes (\texttt{//}).
The comment ends at the next newline. Everything after the first two forward slashes and until the first next
newline is completely ignored at the lexical analysis level.

The following example shows a valid comment within an expression:

\codesample{validcomment.junta}

Unlike other programming languages \productname{} does not have support for multi line comments (such as
\texttt{/* */} in C-like languages).

\subsection{Reserved keywords}

\begin{ebnf}
\grule{reserved}{\gter{define} \gor \gter{type}}
\end{ebnf}

\subsection{Identifiers}

\begin{ebnf}
%Identifiers
\grule{constant}{\ggrp{lowercase \gcat \grep{alphanum}} \gex \ggrp{reserved \gor direction}}
\grule{type}{\ggrp{uppercase \gcat \grep{alphanum}} \gex coordinate}
\grule{variable}{\gter{\$} \gcat alphanum \gcat \grep{alphanum}}
\end{ebnf}

\subsection{Literals}

\begin{ebnf}
%Literals
\grule{integer}{decimal \gcat \grep{decimal}}
\grule{direction}{\gter{n} \gor \gter{s} \gor \gter{e} \gor \gter{w} \gor \gter{ne} \gor \gter{nw} \gor \gter{se} \gor \gter{sw}}
\grule{coordinate}{uppercase \gcat \grep{uppercase} \gcat decimal \gcat \grep{decimal}}
\end{ebnf}

\begin{ebnf}
\grule{string}{\gtdq \gcat \grep{strchar \gor escape} \gcat \gtdq}
\grule{escape}{\gtbs \gcat unichar}
\end{ebnf}

\section{Expressions}
\label{sec:expressions}

\begin{ebnf}
%Expressions
\grule{expression}{assignment}
\galt{if\_expr}
\galt{set\_expr}
\galt{lambda\_expr}
\galt{\gter{not} \gcat expression}
\galt{lo\_sequence}
\end{ebnf}

\subsection{Atomic expressions}

These are the smalles possible parts of expressions in \productname{}.

\begin{ebnf}
\grule{atomic}{\gter{(} \gcat expression \gcat \gter{)}}
\galt{variable}
\galt{list}
\galt{\gter{/} \gcat pattern \gcat \gter{/}}
\galt{\gter{this}}
\galt{\gter{super}}
\galt{direction}
\galt{coordinate}
\galt{integer}
\galt{string}
\galt{type}
\galt{constant}
\end{ebnf}

This group also contains $expression$ contained within parentheses.


\subsection{Lists}

\begin{ebnf}
\grule{list}{\gter{[} \gcat \gopt{expression \gcat \grep{\gter{,} \gcat expression}} \gcat \gter{]}}
\end{ebnf}

\subsection{Let expressions}

\begin{ebnf}
\grule{expression}{\grange \gor let\_expr}
\grule{let\_expr}{\gter{let} \gcat variable \gcat \gter{=} \gcat expression \gcat \grep{\gter{,} \gcat variable \gcat \gter{=} \gcat expression} \gnl
\gcat \gter{in} \gcat expression}
\end{ebnf}

\subsection{Conditional expressions}

\begin{ebnf}
\grule{expression}{\grange \gor if\_expr}
\grule{if\_expr}{\gter{if} \gcat expression \gcat \gter{then} \gcat expression \gcat \gter{else} \gcat expression}
\end{ebnf}

\subsection{Lambda expressions}

\begin{ebnf}
\grule{expression}{\grange \gor lambda\_expr}
\grule{lambda\_expr}{\gter{\#} \gcat varlist \gcat \gter{=>} \gcat expression}
\end{ebnf}

\subsection{Set expressions}

\begin{ebnf}
\grule{expression}{\grange \gor set\_expr}
\grule{set\_expr}{\gter{set} \gcat variable \gcat \gter{=} \gcat expression \gcat \grep{\gter{,} \gcat variable \gcat \gter{=} \gcat expression}}
\end{ebnf}

\subsection{Operators and calls}

\begin{ebnf}
\grule{expression}{\grange \gor \gter{not} \gcat expression \gor lo\_sequence}
\grule{lo\_sequence}{eq\_sequence \gcat \grep{\ggrp{\gter{and} \gor \gter{or}} \gcat eq\_sequence}}
\grule{eq\_sequence}{cm\_sequence \gcat \grep{\ggrp{\gter{==} \gor \gter{!=} \gor \gter{is}} \gcat cm\_sequence}}
\grule{cm\_sequence}{as\_sequence \gcat \grep{\ggrp{\gter{<} \gor \gter{>} \gor \gter{<=} \gor \gter{>=}} \gcat as\_sequence}}
\grule{as\_sequence}{md\_sequence \gcat \grep{\ggrp{\gter{+} \gor \gter{-}} \gcat md\_sequence}}
\grule{md\_sequence}{negation \gcat \grep{\ggrp{\gter{*} \gor \gter{/} \gor{\%}} \gcat negation}}
\grule{negation}{element}
\galt{\gter{-} \gcat negation}
\grule{element}{call\_sequence \gcat \grep{member\_access}}
\grule{member\_access}{\gter{.} \gcat constant \gcat \grep{list}}
\grule{call\_sequence}{atomic \gcat \grep{list}}
\end{ebnf}

\subsubsection{Operator precedence}

\tab[\textwidth]{operatorPrecedence}{2}{The precedence of operators in \productname{}.}
         {Operator precedence}
  {Level}{Operator & Description}{
    \tabrow{1}{\texttt{f[]} & Function/constructor invocation and list access}
    \tabrow{2}{\texttt{r.m r.m[]} & Record member access and member invocation}
    \tabrow{3}{\texttt{-} & Unary negation operation}
    \tabrow{4}{\texttt{* / \%} & Multiplication, division, and modulo}
    \tabrow{5}{\texttt{+ -} & Addition and subtraction}
    \tabrow{6}{\texttt{< > <= >=} & Comparison operators}
    \tabrow{7}{\texttt{== != is} & Equality operators and type checking}
    \tabrow{8}{\texttt{and or} & Logical $and$ and $or$}
    \tabrow{9}{\texttt{not} & Logical $not$}
    \tabrow{10}{\texttt{if let \#} & if-, let-, and lambda-expressions}
}

\section{Definitions}
\label{sec:definitions}
In this section we present how programs written in \productname{} can be
structured with definitions of constants and types.

We begin by presenting what a program can consist of and then we further specify
how these different definitions are built. We present constant definitions
followed by type definitions. We provide big-step semantics for type definitions
in \secref{sec:typedefinitions}.

\subsection{Program structure}

The outermost layer of a \productname{} program is a list of definitions:

\begin{ebnf}
\grule{program}{\grep{definition}}
\grule{definition}{constant\_def}
\galt{type\_def}
\end{ebnf}

A definition is either a constant definition or a type definition. So, when a
program has been run, we are left with a symbol table full of types and
constants.

As the above grammar shows, an empty program is valid in \productname{}, because
it is possible to have zero, one, or more definitions in the outermost layer of
the structuring of programs.

\subsection{Constant definitions}
\label{sec:constantdefinitions}

Functions are also constants and are defined with the following definition:

\begin{ebnf}
\grule{constant\_def}{\gter{define} \gcat constant \gcat \gopt{varlist} \gcat
\gter{=} \gcat \gcat expression}
\end{ebnf}

A function definition needs a list of formal parameters, formal parameters (a
$varlist$) are described with the following grammar:

\begin{ebnf}
\grule{varlist}{\gter{[} \gcat \gopt{variable \gcat \grep{\gter{,} \gcat
variable} \gcat \gopt{\gter{,} \gcat vars} \gor vars} \gcat \gter{]}}
\grule{vars}{\gter{...} \gcat variable}  
\end{ebnf}

Creating constants and functions outside of type definitions adds them to the
global scope, meaning that they are essentially accessible from anywhere in the
program (provided that they are not hidden by a constant within a type).

An example of a global function is the following implementation of function for
computing the greatest of two numbers:

\codesample{functiondef.garry}

Essentially this creates a constant in the global scope named \constant{max},
which when used, returns a value of type \type{Function}. Since function values
can also be created with lambda expressions (as described in
\secref{sec:lambdaexpressions}), the following constant definition is equivalent
to \csref{functiondef.garry}:

\codesample{functiondef.junta}

This equivalence only holds for global constants, since type constants/methods
are a bit more special as described in \secref{sec:typedefinitions}.

Constants and functions are useful for putting frequently used expressions or
values in one place.

The non-terminal $vars$ is used for variadic functions, another feature of
\productname{}. Variadic functions are functions with indefinite arity, meaning
they will accept any number of actual parameters. In \productname{} this is
supported for both lambda expressions and functions. The following code sample
is an example of two variadic functions:

\codesample{variadic.junta}

The use of the \texttt{"..."}-terminal marks that the following variable
represents a list that holds all additional parameters passed to the function.
In the above example, calling the function \constant{last} with no parameters,
is possible and results in the formal parameter \variable{args} holding the
value of an empty list, \texttt{[]}. Additionally, \constant{last} can be called
with any number of parameters, which will then be appended to the list in
\variable{args}. In the second function, \constant{myMap}, at least one
actual parameter must be provided (since the variadic parameter is the second
one). But other than that, the parameter passing works in the same way as with
\constant{last}, in that the function accepts any number of parameters greater
than or equal to $1$. Some uses and result of the two functions are shown in the
following two examples:

\codesample{variadicuse2.junta}
\codesample{variadicuse1.junta}

One limit is that the variadic parameter (the one after \texttt{"..."}) must be
the last one in the list of formal parameters, and there can only be one. This
is expressed in the grammar.

\subsubsection{Scope rules for functions and constants}

Consider the \constant{max}-function in \csref{functiondef.garry}. Its formal
parameters \variable{a} and \variable{b} only exist, and are only available
within the \constant{max}-function. The following example shows a call of the
\constant{max}-function:

\codesample{functioncall.garry}

When called with the actual parameters $5$ and $23$, a new scope is created and
the actual parameters are assigned to the formal parameters \variable{a} and
\variable{b}, respectively. The body of the function (the if expression) is then
evaluated and the result is returned. When returning, the variables
\variable{a} and \variable{b} cease to exist. 

\subsection{Type definitions}
\label{sec:typedefinitions}

As described the introduction of this chapter, types are a central part of
\productname{} and being able to define custom user types is essential when
creating board games.
The following grammar rules present how type definitions work in
\productname{} followed by the associated definitions which can be used within
a type definition.

\begin{ebnf}
\grule{type\_def}{\gter{type} \gcat type \gcat varlist \gcat
\gopt{\gter{extends} \gcat type \gcat list} \gopt{type\_body}}
\grule{type\_body}{\gter{\{} \gcat \grep{member\_def} \gcat \gter{\}}}
\grule{member\_def}{abstract\_def}
\galt{constant\_def}
\galt{data\_def}
\grule{abstract\_def}{\gter{define} \gcat \gter{abstract} \gcat constant \gcat
\gopt{varlist}} \grule{data\_def}{\gter{data} \gcat variable \gcat \gter{=}
\gcat expression}
\end{ebnf}

The simplest type that can be created is a type such as:

\codesample{simplesttype.junta}

The type \type{A} is a type without any data, constructor parameters, constants
or parent type. This type is truly useless, since it has no identity or
behaviour. It can however be instantiated, and instances of it can be compared
using the \texttt{==}, \texttt{!=}, and \texttt{is}-operators. But since the
type has no identity in any way, all instances will be equal:

\codesample{simplesttypeuse.junta}

\subsubsection{The constructor}

One way to add identity to objects, is with the constructor. In the previous
example of a very simple type, the constructor takes no parameters (the empty
parameter list \texttt{[]} after the type name). If we were to add some formal
parameters to the type definition, it could look like:

\codesample{typedef1.junta}

Now in order to instantiate \type{A}, we must provide the constructor with two
parameters: 

\codesample{typedef1use.junta}

Notice how in the first line, the two objects are equal to each other, while in
the second line they are not. This means that we have successfully added
identity to the \type{A}-type.

\subsubsection{The constants}

Constants within types are defined in the same way as constants outside of
types. The difference is that constants defined within a type can only be
accessed within that type (and inheriting types) or by using the dot-notation
outside of the type.

\codesample{typedef2.junta}

In the example above, two constants are defined within type \type{A};
\constant{b} and \constant{calculate}. The first one \constant{b} is a simple
constant holding the value of \variable{b}, the constructor parameter, plus one.
It can be seen as a getter, since it makes the value of \variable{b} visible to
the outside. \constant{calculate} is a method that returns the sum of some
numbers. In order to call the constants contained within a type, we use the dot
notation on an object of that type:

\codesample{typedef2use.junta}

The variable \variable{obj} is assigned an instance of the \type{A}-type. Using
the dot-notation, the method \constant{calculate} is called on the object.

\subsubsection{The data}

Another way to add identity to objects, is to add data fields to the type.
Unlike constants, data fields contain private semi-mutable data. In the
strictest sense, the data is still immutable, but using the
\keyword{set}-keyword, it is possible to clone the current object, and return a
new one with the selected data fields set to new values. The following example
shows a new version of type \type{A}, with a data field:

\codesample{typedef3.junta}

In this example, we define the data field \variable{att} with the default value
of $15$. Since data fields are not accessible from outside of the type, we must
define a getter, the constant \constant{att}, in order to make the value
visible. Using the \keyword{set}-keyword, we can also define at setter, the
\constant{setAtt}-method, which returns a new instance of \type{A} with
\variable{att} set to whatever parameter \constant{setAtt} was called with.

The following example show the use of this setter, to create a clone of an
instance of \type{A}:

\codesample{typedef3use.junta}

This time, we create an instance of \type{A}, and then calls \constant{setAtt}
on that instance, in order to get a new instance of \type{A} with \variable{att}
set to $2$. After that, the values are accessed using the getter. Note that,
again, the two objects are not equal, since the value of \variable{obj2} is
different than \variable{obj1}.

\subsubsection{Inheritance}

\todo{repeating stuff from introduction}

\productname{} supports single inheritance between types. An inheriting type, will
inherit all the members of its super type(s) (if \type{C} extends \type{B}, which
extends \type{A}, then \type{C} inherits all members from both \type{B} and \type{A}).
This also introduces the \keyword{super}-keyword, which can be used to access members
in superclasses. The following example shows how inheritance works:

\codesample{typescope2.junta}

\productname{} doesn't have an \keyword{abstract}-keyword for type definitions, only
for constant/function definitions. A type is implicitly marked as abstract if it has
unimplemented abstract members. In the above example the type \type{A} is abstract,
since because it's member, \constant{constantA}, is abstract. Likewise, the type
\type{B} is abstract because it extends \type{A}, but doesn't implement the abstract
member of \type{A}. The type \type{C} on the other hand is not abstract, since it
implements the abstract constant. Abstract types can't be constructed, albeit the
constructor for an abstract type has to be used when extending the type (after the
\keyword{extends}-keyword).

\subsubsection{Big-step semantics}

The semantics presented in \tableref{semantic:typedef} are the transition rules for type definitions.
These type definitions have some optional arguments which correspond with the
written grammar for these definitions, and this is why there are four transition
rules described.

\begin{table}[ht]
    \begin{tabular*}{\textwidth}{l l}
      \hline \\
      \hspace{0.4cm} $\left[\mbox{TYPEDEF}\right]$ & \infrule{env_{C} \vdash
      \lag D_{G}, env_{T}[T \mapsto \left(T, X, \varepsilon, \varepsilon,
      \varepsilon \right)] \rag \ra env_{T}'}
      {env_{C} \vdash \lag \texttt{type}\; T\; X\; D_{G},\; env_{T} \rag \ra
      env_{T}'} \\

      \hspace{0.4cm} $\left[\mbox{TYPEDEF}_{\mbox{BODY}}\right]$ &
      \infrule{env_{C} \vdash \lag D_{G}, env_{T}[T \mapsto \left(T, X, D_{M},
      \varepsilon, \varepsilon \right)] \rag \ra env_{T}'}
      {env_{C} \vdash \lag \texttt{type}\; T\; X\; \left\{D_{M}\right\}\;
      D_{G},\; env_{T} \rag \ra env_{T}'} \\

      \hspace{0.4cm} $\left[\mbox{TYPEDEF}_{\mbox{EXTEND}}\right]$ &
      \infrule{env_{C} \vdash \lag D_{G}, env_{T}[T_{1} \mapsto \left(T_{1}, X,
      \varepsilon, L, T_{2} \right)] \rag \ra env_{T}'}
      {env_{C} \vdash \lag \texttt{type}\; T_{1}\; X\; \texttt{extends}\;
      T_{2}\; L\; D_{G},\; env_{T} \rag \ra env_{T}'} \\

      \hspace{0.4cm} $\left[\mbox{TYPEDEF}_{\mbox{EXTEND-BODY}}\right]$ &
      \infrule{env_{C} \vdash \lag D_{G}, env_{T}[T_{1} \mapsto \left(T_{1}, X,
      D_{M}, L, T_{2} \right)] \rag \ra env_{T}'}
      {env_{C} \vdash \lag \texttt{type}\; T_{1}\; X\; \texttt{extends}\;
      T_{2}\; L\; \left\{D_{M}\right\}\; D_{G},\; env_{T} \rag \ra env_{T}'} \\
      \hline \\
    \end{tabular*}
    \capt{Transition rules for type definitions.}
    \label{semantic:typedef}
\end{table}

In the premises of the rules we present a 5-tuple where $env_{T}$ is updated
according to the rule. In three of the four 5-tuples we include the symbol
$\varepsilon$, which denotes that the given position of the symbol is an empty
slot. This is again due to the fact that we have some optional arguments.

The 5-tuple is ordered as follows:

\begin{nlist}
\item $\mathbf{T_{k}}$ - current type
  \item $\mathbf{X}$ - current type's formal parameters
  \item $\mathbf{D_{M}}$ - member definitions
  \item $\mathbf{L}$ - super type's parameters
  \item $\mathbf{T_{k+1}}$ - super type
\end{nlist}

Throughout the transition rules we use the 5-tuple to update the type environment.

\section{Patterns}
\label{sec:patterns}

This section covers how to use patterns and what to use them for. The operators of a pattern looks like and behaves a little like regular expressions. This EBNF-grammar shows how a pattern is constructed:

\begin{ebnf}
\grule{pattern}{pattern\_expr \gcat \grep{pattern\_expr}}
\grule{pattern\_expr}{pattern\_val \gcat \gopt{\gter{*} \gor \gter{?} \gor \gter{+}}}
\galt{pattern\_val \gcat \gter{|} \gcat pattern\_expr}
\grule{pattern\_val}{direction}
\galt{variable}
\galt{pattern\_check}
\galt{\gter{!} \gcat pattern\_check}
\galt{\gter{(} \gcat pattern \gcat \gter{)} \gcat \gopt{integer}}
\grule{pattern\_check}{\gter{friend}}
\galt{\gter{foe}}
\galt{\gter{empty}}
\galt{\gter{this}}
\galt{type}
\end{ebnf}


A pattern is checked on a particular square, and returns either true or false. An example of a pattern is \text{/n n e empty/}. This pattern can be checked on the board seen in \figref{patternboard} on the field \textbf{A1}. The pattern says ``go one square north, go one square north, go one square east, check if square is empty''. This means that the square \textbf{B3} will be checked for emptiness. Since the square is occupied by a piece, the pattern will return false if checked on \textbf{A1}. However, the same pattern checked on \textbf{C1} will return true since the square at \textbf{D3} is empty.
\fig[scale=2]{patternboard}{A simple 4 X 4 board with 3 pieces}

With this basic introduction to a simple pattern check, the table (insert ref and convert list below to a table) describes briefly how the different pattern operators work. For each operator, an example of the use in a context are given in \secref{sec:patternexamples}. Note that the description of patterns assumes a minor understanding of regular expressions, see (ref http://www.regular-expressions.info/).

\texttt{empty} : current square contains no pieces\\
\texttt{n} : north \\
\texttt{e} : east \\
\texttt{s} : south \\
\texttt{w} : west \\
\texttt{*} : zero-to-many times\\
\texttt{+} : one-to-many times\\
\texttt{?} : zero-or-one time\\
\texttt{|} : or\\
\texttt{!} : not\\
\texttt{(} \textit{pattern} \texttt{)} : encapsulation\\
\texttt{friend} :  current square contains at least one friendly piece of the current player\\
\texttt{foe} : current square contains at least one enemy piece of the current player\\
\textit{type} : the current square is of the given type or a piece of the given type is residing on the current square.
\texttt{this} : current square contains the piece for which the check is being performed\\

\subsection{Pattern examples}
\label{sec:patternexamples}
All these examples of pattern checks are performed on the board and pieces seen in \figref{patternboard}. For each operator, two examples of a pattern check on a particular square is given. The first check returns true and the second check false.

On A1, the pattern check \texttt{/empty/} returns true because A1 is empty\\
On B2, the pattern check \texttt{/empty/} returns false because B2 is not empty\\
On A1, the pattern check \texttt{/n empty/} returns true because A2 is empty\\
On B1, the pattern check \texttt{/n empty/} returns false because B2 is not empty\\
On C3, the pattern check \texttt{/e empty/} returns true because C4 is empty\\
On C4, the pattern check \texttt{/e empty/} returns false because C5 is out of board\\
On A1, the pattern check \texttt{/n* n e empty/} returns true because moving north 2 times then north and east hits an empty square on B4\\

Notice that the *-operator causes many branches to be made. The previous pattern check, \texttt{/n* n e empty/} done on A1, has a branch checking \texttt{/n e empty/}. The branch dies because B2 is not empty. If just one branch survives, the pattern check returns true. In the example, the only branch surviving is the \texttt{n n n e empty} branch. The same rules for branching counts for the \texttt{+}, \texttt{?} and \texttt{|} operator. When a branch moves out of board it dies immediately.

On C3, the pattern check \texttt{/n* s s !empty/} returns false because neither A1 nor A2 contains a piece\\
On A1, the pattern check \texttt{/n+ e empty/} returns true only because B4 is empty\\
On B1, the pattern check \texttt{/n+ empty e !empty/} returns false because B4 is the only empty square north of B1 and C4 is empty\\
On B3, the pattern check \texttt{/s? e empty/} returns true only because C2 is empty\\
On C2, the pattern check \texttt{/n? w empty/} returns false because neither B2 nor B3 is empty\\
On A2, the pattern check \texttt{/(n | e) empty/} returns true only because A3 is empty\\
On C2, the pattern check \texttt{/e | w empty/} returns false because neither B2 nor C3 is empty\\
On B1, the pattern check \texttt{/(n n | e e) empty/} returns true only because D1 is empty\\
On A1, the pattern check \texttt{/(n w)* empty/} returns true both because A1 is empty and because D4 is empty\\

The \texttt{friend}-keyword is evaluated based on the current player. Suppose that we have a player called Green, who owns the 
green piece. If it is green's turn to move, any branch of a pattern check will return false
whenever it meets a keyword \texttt{friend} on a square that does not contain any of Green's pieces.
On B2, the pattern check \texttt{/n|e|(n e) friend/} will return true if it is Green's turn, since C3 contains a friendly piece of Green.
On C3, the pattern check \texttt{/(e | w)+/} will return false if it is Green's turn, since no square containing a piece of Green can be reached by going east or west from C3 one or more times.

The keyword \texttt{foe} does the opposite of \texttt{friend}. It makes a branch continue only if its current square contains at least one piece not owned by the player who has the turn.

Just like \texttt{foe} and \texttt{friend}, the name of a piece or square-type defined in a \productname{}-game can also be used. E.g, the keyword \texttt{White} can be used if a piece or square-type with the name \textit{White} has been define. In a such case, a branch survives only if its current square is of type \textit{White} or if the square contains a piece of type \textit{White}.

A pattern can also check that a specific piece exists on a specific square. This is done using the \texttt{this} keyword.
Before this check can be achieved, the pattern must be checked regarding to a specific piece. Suppose that we on A3, make the pattern check
\texttt{empty e this}. If this check is done in relation to the black piece on B3 it returns true. However, in relation to the black piece on B2, the pattern check returns false. To understand both how this function exactly works and why this is useful, consider the board in \figref{fig:patternboard}.
If we for any piece specify that it can move to a square for which the pattern check \texttt{/(n | s) empty/} is true, this means that it can move to any square except $\{B1, B4, C4\}$. These square does not have an empty square north or east from it. Recall that a branch going out of board dies.
However, the pattern check \texttt{/empty (n | s) this/} will in relation to the green piece return true only when checked on the squares $\{C2, C4\}$.
This can be used to specify that a piece can move to an empty square one north or one south from its current square. 
\section{Predefined types and constants}
\label{sec:predefined}

In order to write programs in a programming language, it is often necessary to use a number of built-in
functions and types. \productname{} provides a number of built-in functions and constants, as well as 
a number of simple types for representing values such as integers and strings. \productname{} also provides
a type-hierarchy designed for implementing and expressing board games. The built-ins will in many cases make
the lives of \productname{}-programmers easier since he/she won't have to implement the functionality the built-ins
provide for themselves. 

Since \productname{} doesn't have a module, package or name space system, the distinction between a standard-
and game environment doesn't actually exist in the language, and all types and constants exist in the same
global name space. The distinction between the two is merely formal and based on the sort of types and functionality
that each provide.
 
\subsection{Standard environment}
\label{sec:standardenvironment}

We now introduce the standard environment of \productname{}. The standard environment provides the simple types, such as
integers, strings, boolean values, etc.\ and their related functions and constants for working with the these.

The following global constants are available:

\begin{dlist}
  \item \constdef{typeOf}{[\farg{value}{\opstar}]}{Type}\\
    A function that returns the type of any value.
  \item \constdef{union}{[\farg{list}{List}, ... \farg{lists}{List}]}{List}\\
    A function that returns the union of a number of lists.
  \item \constdef{true}{}{Boolean}\\
    The boolean true value.
  \item \constdef{false}{}{Boolean}\\
    The boolean false value.
\end{dlist}

\subsubsection{Integer}

\begin{dlist}
  \item \type{Integer}[\variable{integer} : \type{Integer}]\\
    The standard environment provides the \type{Integer}-type, which is implemented as Java's primitive data type, integer. That is
it's a 32-bit signed two's complement integer. When the interpreter detects a numeral it returns an integer value object. If
for instance the numeral exceeds the highest possible value a TypeError is thrown.
\end{dlist}

\subsubsection{Boolean}

\begin{dlist}
  \item \type{Boolean}[\variable{boolean} : \type{Boolean}]\\
  The standard environment provides the Boolean type, which is implemented as Java's primitive data type, boolean. That is, it only has two possible values: true and false. Even though the data type represents only one bit of information, according to the Java documentation, the ``size'' isn't precisely defined. 
\end{dlist}

\subsubsection{String}
\begin{dlist}
  \item \type{String}[\variable{string} : \type{String}]\\
    The standard environment provides the String type, which is implemented as Java's data type, String. That is, it may contain any unicode (UTF-16) characters. Though it is not possible to writ unicode characters of the form ``\textbackslash{}uXXXX'' as in Java (for instance ``\textbackslash{}u0108'', which is the capital C with circumflex, Ĉ). The \type{String} type contains one built-in constant:  
  \begin{dlist}
  \item \constant{size} : \type{Integer}\\
  The size constant returns the number of characters in the string, which is an integer value. For example ``test\_string''.size = $11$
  \end{dlist}
\end{dlist}
\subsubsection{List}

\begin{dlist}
  \item \type{List}[\variable{list} : \type{List}]\\
  The standard environment provides the List type. A list object can contain a mix of any types: strings, integers, other lists, game objects etc.
  This has both advantages and disadvantages. It increases the orthogonality of the programming language but it increases the risk of getting
  errors, which doesn't show until at run-time. The List type is similar to the ArrayList of Java and it's resizeable, which means that types can be added to the List. The type comes with a number of built-in constants and functions. 
  \item \constant{size} : \type{Integer}\\
  The size constant returns the number of elements in the list, which is an integer value. For example [``hi'', 2, 4].size = 3.
  \item \constant{sort}[\variable{comparator} : \type{Function}] : \type{List} \\
  The sort function sorts a list using a function that must take two parameters as input and return an integer value. For example [1, 6, 2, 5, 4, 3].sort[\#[\$a, \$b] $=>$ if \$a > \$b then 1 else if \$a == \$b then 0 else -1]. Will sort the list in ascending order. That is [1, 2, 3, 4, 5, 6].
  \item \constant{map}[\variable{mapper} : \type{Function}] : \type{List} \\
  The map function maps each element of the list with a function of style \#[\$a] => \$a. The function must take one parameter. For example [1, 2, 3, 4, 5, 6].map[\#[\$a] => \$a + 1] will return the list: [2, 3, 4, 5, 6, 7].
  \item \constant{filter}[\variable{filter} : \type{Function}] : \type{List} \\
  The filter function filters a list by feeding it with a function of style \#[\$a] => \$a >= 5, and returns a list with only the elements which comply with the function. The function fed to the filter function must take one parameter and return a boolean value. For example [1, 2, 3, 4, 5, 6].filter[\#[\$a] => \$a >= 5] will return [5, 6]. 
\end{dlist}

\subsubsection{Direction}
\begin{dlist}
  \item \type{Direction}[\variable{direction} : \type{Direction}]\\
  The standard environment provides the Direction type, which can be compared to a vector. There are eight different directions: n (north), s (south), w (west), e (east), nw, ne, sw, se. The type consist of an $x$ value and a $y$ value. For example n has value $y = 1$ and $x = 0$, s has value $y = -1$ and $x = 0$, w has value $y = 0$ and $x = -1$, etc.\ The Direction type is meant as a practical tool for use in patterns. 
\end{dlist}

\subsubsection{Coordinate}
\begin{dlist}
  \item \type{Coordinate}[\variable{coordinate} : \type{Coordinate}]\\
  The standard environment provides the Coordinate type. The Coordinate type is closely related to the Direction type in the way that it also consist of a $x$ value and an $y$ value. When the interpreter detects a number of capital letters followed by a number a numerals it returns a coordinate value object. Examples of coordinate values are A1, Z99 and ABCD1234. The coordinate value A1 corresponds to the $x$ value $1$ and $y$ value $1$, which is the top-left square on a board. The coordinate type is means as a practical tool to specify squares on a grid-formed board. Coordinate values must be positive, as negative $x$ and $y$ values make no sense representing coordinates off of the board.
\end{dlist}
\subsubsection{Type}
\begin{dlist}
  \item \type{Type}[\variable{type} : \type{Type}]\\
\end{dlist}
\subsubsection{Function}

\begin{dlist}
  \item \constdef{call}{[\farg{parameters}{List}]}{\opstar}\\
    Calls the function with the specified parameter list. 
\end{dlist}

\subsubsection{Pattern}

\subsection{Game environment}
\label{sec:gameenvironment}

The game environment provides a class hierarchy for describing a board game in an object-oriented manner.
In the game environment the following global functions are available:

\begin{dlist}
  \item \constant{addAction}[\variable{piece} : \type{Piece}, \variable{squares} : \type{List}] : \type{List}\\
    A function that returns a list of \type{AddAction}s to where it's possible to add a piece (\variable{piece}. The functions
    takes two parameters. The first parameter contains information on which type of piece the actions applies to. The second parameter is
    the list of squares where the type of piece can be added to.
    
    In the code example in \secref{codesample} in the beginning of the chapter, \function{addAction} is used in the following
    way: \\
    \begin{center}
    {addAction}[\variable{pieceType}[\keyword{this}], \variable{gameState}.\constant{board}.\constant{emptySquares}]
    \end{center}
    
    Here \constant{addAction} returns a list of empty squares to where it's possible to add a piece of the type \keyword{this}, which in this case was
    either a crosses piece or noughts piece depending whose turn it is.
    
  \item \constant{moveAction}[\variable{piece} : \type{Piece}, \variable{squares} : \type{List}] : \type{List}\\
  \constant{moveAction} works like \constant{addAction}, but instead of returning a \type{List} of \type{AddAction}s it returns a \type{List} of \type{MoveAction}s.
  	
    
\end{dlist}



\subsubsection{Game}
The \type{Game} type contains all information to describe a board game at a specific point at time.

\begin{dlist}
  \item \type{Game}[\variable{title} : \type{String}]\\
  Creates a instance of the \type{Game} type with a Game title of \variable{title}, \constant{board} set to \constant{initialBoard} and \constant{currentPlayer} set to \constant{turnOrder}[0].
  
  \item \constant{players} : \type{List}\\
  List of all \type{Player}s which are a part of this game.
  
  \item \constant{currentPlayer} : \type{Player}\\
  The \type{Player} from \constant{players} which currently have the turn.
  
  \item \constant{turnOrder} : \type{List}\\
  The order of \type{Player}s which determines in which order each \type{Player} from \constant{players} has their turn.
  
  \item \constant{initialBoard} : \type{Board}\\
  The value of \constant{board} at the beginning of each game.
  
  \item \constant{board} : \type{Board}\\
  The current state of a \type{Board} for this game.
  
  \item \constant{title} : \type{String}\\
  The title of the game which users can indentify the game with.
  
  \item \constant{description} : \type{String}\\
  An short explanation of the game and/or its rules.
  
  \item \constant{matchSquare}[ \variable{position} : \type{Coordinate}, \variable{pattern} : \type{Pattern} ] : \type{Boolean}\\
  Is true if \variable{pattern} is valid for \variable{position}.
  
  \item \constant{matchSquares}[ \variable{positions} : \type{List}, \variable{pattern} : \type{Pattern} ] : \type{Boolean}\\
  Is true if and only if all \type{Coordinate}s in \variable{positions} is true for \constant{matchSquare} with \variable{pattern}.
  
  \item \constant{findSquares}[ \variable{pattern} : \type{Pattern} ] : \type{List}\\
  \type{List} of all \type{Square}s where its \constant{position} matches \variable{pattern}.
  
  \item \constant{findSquaresIn}[ \variable{positions} : \type{List}, \variable{pattern} : \type{Pattern} ] : \type{List}\\
  \type{List} of \type{Square}s where its \constant{position} matches \variable{pattern}, but only \type{Squares} which \type{Coordinate} exists in \variable{positions}.
  
  \item \constant{history} : \type{List}\\
  \type{List} of all applied \type{Action}s.
  
  \item \constant{applyAction}[ \variable{action} : \type{Action} ] : \type{Game}\\
  A \type{Game} where \constant{board} have been updated according to \variable{action} and where \variable{action} is appended to \constant{history}.
  
  \item \constant{undoAction}[ \variable{action} : \type{Action} ] : \type{Game}\\
  A \type{Game} where \constant{board} have been reset to its state before \type{Action} was applied and with \constant{history} updated accordantly.
  
  \item \constant{setHistory}[ \variable{history} : \type{List} ] : \type{Game}\\
  A \type{Game} where \constant{history} is equaliant to \variable{history}.
  
  \item \constant{setBoard}[ \variable{board} : \type{GridBoard} ] : \type{Game}\\
  A \type{Game} where \constant{board} is equaliant to \variable{board}.
  
  \item \constant{setCurrentPlayer}[ \variable{i} : \type{Integer} ] : \type{Game}\\
  A \type{Game} where \constant{currentPlayer} is \constant{turnOrder}[\variable{i}].
  
  \item \constant{nextTurn}[] : \type{Game}\\
  The \type{Player} which has the turn after \constant{currentPlayer}.
\end{dlist}

\subsubsection{Board}
\begin{dlist}
  \item \type{Board}[]\\
  A \type{Board} with no \type{Piece}s.
  
  \item \constant{pieces} : \type{List}\\
  A \type{List} containing all \type{Piece}s associated with the \type{Board}.
  
  \item \constant{setPieces}[ \variable{pieces} : \type{List} ] : \type{Board}\\
  A \type{Board} where \constant{pieces} is equaliant to \variable{pieces}.
\end{dlist}

\subsubsection{GridBoard}
\type{GridBoard} \keyword{extends} \type{Board} to provide an easy way to describe rectangular \type{Board}s.

\begin{dlist}
  \item \type{GridBoard}[ \variable{width} : \type{Integer}, \variable{height} : \type{Integer} ]\\
  A \type{GridBoard} with \constant{width} and \constant{height} being \variable{width} and \variable{height} respectively.
  
  \item \constant{width} : \type{Integer}\\
  The width of the rectangular \type{Board}.
  
  \item \constant{height} : \type{Integer}\\
  The height of the rectangular \type{Board}.
  
  \item \constant{squares} : \type{List}\\
  A \type{List} of all associated \type{Square}s.
  
  \item \constant{setSqaures}[ \variable{squares} : \type{List} ] : \type{GridBoard}\\
  A \type{GridBoard} where \constant{squares} is equaliant to \variable{squares}.
  
  \item \constant{addPiece}[ \variable{piece} : \type{Piece}, \variable{position} : \type{Coordinate} ] : \type{GridBoard}\\
  A \type{GridBoard} where \variable{piece} is appended to \constant{pieces} and added to the \type{Square} at \variable{position}.
  
  \item \constant{addPieces}[ \variable{piece} : \type{Piece}, \variable{positions} : \type{List} ] : \type{GridBoard}\\
  A \type{GridBoard} where \variable{piece} is appended to \constant{pieces} and added to all the \type{Square}s at any of \variable{positions}.
  
  \item \constant{removePiece}[ \variable{piece} : \type{Piece} ] : \type{GridBoard}\\
  A \type{GridBoard} where \variable{piece} is off-board.
  
  \item \constant{movePiece}[ \variable{piece} : \type{Piece}, \variable{position} : \type{Coordinate} ] : \type{GridBoard}\\
  A \type{GridBoard} where \variable{piece} (which is already contained in \constant{pieces}) is \constant{onBoard} and is only included in one \type{Square}'s \constant{pieces}.
  
  \item \constant{squareAt}[ \variable{position} : \type{Coordinate} ] : \type{Square}\\
  The \type{Square} at \variable{position} in the rectangular grid of \type{GridBoard}.
  
  \item \constant{setSqauresAt}[ \variable{square} : \type{Square}, \variable{position} : \type{List} ] : \type{Square}\\
  A \type{GridBoard} where \constant{squareAt}[ \variable{position} ] is equaliant to \variable{square}.
  
  \item \constant{isFull} : \type{Boolean}\\
  Is true if \constant{emptySquares}.size is 0.
  
  \item \constant{emptySquares} : \type{List}\\
  A \type{List} with \type{Square}s from \constant{squares} where \constant{isEmpty} is false.
  
  \item \constant{squareTypes} : \type{List}\\
  A \type{List} with default \type{Square}s which will be used to create a checkered pattern of \type{Square}s in the grid of \type{Square}s.
\end{dlist}

\subsubsection{Square}
\type{Square} describes a position on the \type{Board} where 0-to-many \type{Piece}s can be placed.

\begin{dlist}
	\item \type{Square}[]\\
	\type{Square} with no \type{Piece}s.
	
	\item \constant{position} : \type{Coordinate}\\
	\type{Coordinate} describing the position on a \type{GridBoard}.
	
	\item \constant{pieces} : \type{List}\\
	A \type{List} with \type{Piece}s located on this \type{Square}.
	
	\item \constant{addPiece}[ \variable{piece} : \type{Piece} ] : \type{Square}\\
	A \type{Square} where \variable{piece} is appended to \constant{pieces}.
	
	\item \constant{removePiece}[ \variable{piece} : \type{Piece} ] : \type{Square}\\
	A \type{Square} where \variable{piece} is not contained in \constant{pieces}.
	
	\item \constant{setPieces}[ \variable{pieces} : \type{List} ] : \type{Square}\\
	A \type{Square} where \constant{pieces} is equaliant to \variable{pieces}.
	
	\item \constant{image} : \type{String}\\
	Path to an image file used for visualizing the \type{Square}.
	
	\item \constant{isOccupied} : \type{Boolean}\\
	Is true if \constant{pieces}.\constant{size} is larger than 0.
	
	\item \constant{isEmpty} : \type{Boolean}\\
	Is true if \constant{pieces}.\constant{size} is 0.
	
	\item \constant{setPosition}[ \variable{position} : \type{Coordinate} ] : \type{Square}\\
	A \type{Square} where \constant{position} is equaliant to \variable{position}.
\end{dlist}

\subsubsection{Piece}
\type{Piece} describes an item associated to a \type{Player} which the \type{Player} can manipulate in order to progress the game.

\begin{dlist}
  \item \type{Piece}[ \variable{owner} : \type{Player} ]\\
  \type{Piece} with \constant{owner} set to \variable{owner}.
  
  \item \constant{owner} : \type{Player}\\
  \type{Player} which owns this \type{Piece}.
  
  \item \constant{image} : \type{String}\\
  Path to an image file used for visualizing the \type{Piece}.
  
  \item \constant{position} : \type{Coordinate}\\
  \type{Coordinate} for the \type{Square} this \type{Piece} is located on.
  
  \item \constant{move}[ \variable{position} : \type{Coordinate} ] : \type{Piece}\\
  A \type{Piece} with \constant{position} set to \variable{position} and \constant{onBoard} set to true.
  
  \item \constant{remove}[] : \type{Piece}\\
  A \type{Piece} where \constant{position} is invalid and \constant{onBoard} is false.
  
  \item \constant{onBoard} : \type{Boolean}\\
  Is true if \type{Piece} is on the \type{GridBoard}.
  
  \item \constant{actions}[ \variable{game} : \type{Game} ] : \type{List}\\
  A \type{List} of possible \type{Action}s the \type{Piece} can make on its \constant{owner}'s turn.
\end{dlist}

\subsubsection{Player}
\begin{dlist}
  \item \type{Player}[ \variable{name} : \type{String} ]\\
  \type{Player} with \constant{name} set to \variable{name}
  
  \item \constant{name} : \type{String}\\
  The name of the \type{Player}.
  
  \item \constant{winCondition}[ \variable{game} : \type{Game} ] : \type{Boolean}\\
  Is true if the \type{Player} has won at the ending of this turn.
  
  \item \constant{tieCondition}[ \variable{game} : \type{Game} ] : \type{Boolean}\\
  Is true if the game ended without a winner.
  
  \item \constant{actions}[ \variable{game} : \type{Game} ] : \type{List}\\
  A \type{List} of \type{Action}s that the \type{Player} can do during his turn.
\end{dlist}

\subsubsection{Action}
\begin{dlist}
  \item \type{Action}[]\\
  Empty \type{Action}.
\end{dlist}

\subsubsection{UnitAction}
\type{UnitAction} \keyword{extends} \type{Action} to provide a basic change to be performed on \type{Game}.

\begin{dlist}
  \item \type{UnitAction}[ \variable{piece} : \type{Piece} ]\\
  A \type{UnitAction} with \constant{piece} set to \variable{piece}.
  
  \item \constant{piece} : \type{Piece}\\
  The \type{Piece} this \type{UnitAction} affects.
\end{dlist}

\subsubsection{AddAction}
\type{UnitAction} \keyword{extends} \type{Action} to add a \type{Piece} to a \type{Game}.

\begin{dlist}
  \item \type{AddAction}[ \variable{piece} : \type{Piece}, \variable{to} : \type{Square} ]\\
  An \type{AddAction} which adds \variable{piece} to \variable{to}.
  
  \item \constant{to} : \type{Square}\\
  \type{Square} to add \constant{piece} to.
\end{dlist}

\subsubsection{RemoveAction}
\type{UnitAction} \keyword{extends} \type{Action} to remove a \type{Piece} from a \type{Game}.

\begin{dlist}
  \item \type{RemoveAction}[ \variable{piece} : \type{Piece} ]\\
  A \type{RemoveAction} which removes \variable{piece}.
\end{dlist}

\subsubsection{MoveAction}
\type{UnitAction} \keyword{extends} \type{Action} to move a \type{Piece} to another \type{Square}.

\begin{dlist}
  \item \type{MoveAction}[ \variable{piece} : \type{Piece}, \variable{to} : \type{Square} ]\\
  A \type{MoveAction} which moves \variable{piece} to \variable{to}.
  
  \item \constant{to} : \type{Square}\\
  \type{Square} to add \constant{piece} to.
\end{dlist}

\subsubsection{ActionSequence}
\type{ActionSequence} \keyword{extends} \type{Action} to provide a sequence of \type{UnitAction}s to be performed in order.

\begin{dlist}
  \item \type{ActionSeqence}[ \ldots \variable{actions} : \type{UnitAction} ]\\
  \type{ActionSequence} with \constant{actions} set to [ \ldots \variable{actions} ].
  
  \item \constant{actions} : \type{List}\\
  A \type{List} of \type{UnitAction} to be performed in order.
  
  \item \constant{addAction}[ \variable{action} : \type{UnitAction} ] : \type{ActionSequence}\\
  A \type{ActionSequence} where \variable{action} is appended to constant{actions}.
\end{dlist}

\subsubsection{TestCase}
An abstract type for unit testing.


