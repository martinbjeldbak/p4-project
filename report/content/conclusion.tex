\chapter{Conclusion}
\label{chap:conclusion}

In this chapter, we draw concluding thoughts on fulfilling the
curriculum for this project, and we also verify or falsify the
requirements described in \chapref{chap:requirements}.

We have accomplished the goal of formally implementing a programming
language in which it is possible to program board games. The programming
language is called \productname{}. Our implementation includes an
interpreter instead of a compiler, which would merely have resulted in a
different chapter on implementation.

We have respected the goals of the curriculum throughout the
report. In the analysis chapter (\chapref{chap:analysis}), we
have for instance accounted for different techniques and phases behind the
construction of compilation and interpretation. In the design
chapter (\chapref{chap:design}) and the implementation chapter
(\chapref{chap:implementation}), we have documented formally and
informally how we have designed, defined, and implemented a programming
language.

At a relatively early stage, we concluded that we needed a simulator
in which the games written in \productname{} could be visualised in
and played. This decision was based upon the conclusion that it would
be very awkward to play a game in a console window. We constructed a
game simulator which is presented in \secref{sec:simulator}. We also
implemented an application programming interface other simulator-like
programs can take advantage of.

We have unfortunately not accomplished to fulfil all the requirements 
(\chapref{chap:requirements}) we set for this project. We did not
fulfil requirement $1$a. This was due to special rules of Chess.
The \textit{En Passant} and \textit{Castling} moves, turned out to be more
difficult to implement than we had expected. Although, we still believe that
\productname{} provides the features necessary to implement them.

Based on the games we have programmed in \productname{} (Noughts and
Crosses and Connect Four), which were possible to create in around $20$
lines of code, we conclude that we have accomplished, where a similar
language uses almost double as many lines for the same implementation.

% This paragraph is about writability?
The speed in which a game can be created in \productname{} depends on a number
of parameters. First of all, the programmers familiarity with the programming
language and the programmer's experience with programming overall. Also the
programming language's support for relevant functionality and its overall
writability is an important parameter. The fact that we were able to create a
fully functional Connect Four game in $5$ minutes can be considered as a quick
implementation and witnesses that it is easy to develop board games in
\productname{}.

\section{Discussion}
\label{sec:discussion}

In this section we will discuss what we could have done differently throughout
the project and also discuss future expansion possibilities for \productname{}.

%
\begin{figure}[ht]
  \begin{center}
    \begin{tikzpicture}[level/.style={sibling distance=30mm/#1}]
      %m+n
      \node [square] (a) {Language $1$};
      \node [square,xshift=8em] (b) {Language $2$};
      \node [square,xshift=16em] (c) {Language $m$};

      \node [ellipse,draw,xshift=8em,yshift=-4em] (il) {Intermediate language};
      
      \node [square,yshift=-8em] (aa) {Platform $1$};
      \node [square,xshift=8em,yshift=-8em] (bb) {Platform $2$};
      \node [square,xshift=16em,yshift=-8em] (cc) {Platform $n$};

      \draw[->, thick,] (a) -- (il);
      \draw[->, thick,] (b) -- (il);
      \draw[->, dashed,] (c) -- (il);
      
      \draw[->, thick,] (il) -- (aa);
      \draw[->, thick,] (il) -- (bb);
      \draw[->, dashed,] (il) -- (cc);
      
      \path (b)--(c) node [midway] {$\cdots$};
      \path (bb)--(cc) node [midway] {$\cdots$};
      
      %m*n
      \node [square,xshift=23em,yshift=-1em] (x) {Language $1$};
      \node [square,xshift=31em,yshift=-1em] (y) {Language $2$};
      \node [square,xshift=39em,yshift=-1em] (z) {Language $m$};
      
      \node [square,xshift=23em,yshift=-7em] (xx) {Platform $1$};
      \node [square,xshift=31em,yshift=-7em] (yy) {Platform $2$};
      \node [square,xshift=39em,yshift=-7em] (zz) {Platform $n$};

      \draw[->, thick,] (x) -- (xx);
      \draw[->, thick,] (x) -- (yy);
      \draw[->, dashed,] (x) -- (zz);

      \draw[->, thick,] (y) -- (xx);
      \draw[->, thick,] (y) -- (yy);
      \draw[->, dashed,] (y) -- (zz);
      
      \draw[->, dashed,] (z) -- (xx);
      \draw[->, dashed,] (z) -- (yy);
      \draw[->, dashed,] (z) -- (zz);
     
      \path (y)--(z) node [midway] {$\cdots$};
      \path (yy)--(zz) node [midway] {$\cdots$};
    \end{tikzpicture}
  \end{center}
  \capt{Difference between compiling to an intermediate language.}
  \label{fig:mtimesn}
\end{figure}


\subsection{Alternative integer and decimal data structures}
\label{sec:bigintegers}

Our only primitive data type to represent numerals with, \type{Integer},
corresponds to Java's primitive data type, going by the same name. We
do not however have an alternative for representing decimal numbers
or for representing arbitrary-precision integers, if for instance a
number exceeds what's possible to represent with a $32$-bit signed
two's complement integer (Java's specification). It is imaginable
that a programmer of a board game would want to use decimals or very
big numbers, and therefore implementing this in a later version of
\productname{} could be an idea. A possible solution could be a sort of
implementation like \classref{BigInteger} and \classref{BigDecimal} data
types of Java, maybe even using these as underlying data structures.
The \classref{BigInteger} data type provides analogues to all of Java's
primitive integer operators, exactly the same as the primitive data
type \classref{Integer} \cite{javabigint}, but at the same time it
can represent arbitrary-precision integers. That is integers of any
size limited only by the memory of the computer. The same holds for
\classref{BigDecimal}s.

Decimals do not exist in \productname{}, due to the fact that from all
of the board games we've analysed and discussed, not one uses floating
point numbers to represent any aspect of the game. We have also chosen
not to use either of these, because we do not see a specific need to use
such massive numbers. And given that our \productname{} implementation
is already slow, using these arbitrary-length data strutures will
only bog it farther down. It would be rather simple if we ever wished
to implement floating point numerals, we would need to update the
\methodref{isDigit} method in the Scanner to continue picking up numbers
when seeing a period. The parser would then need to create a new type
of node for it, feeding it to the newly updated Interpreter that builds
a new type when seeing a floating point node (granted, the operations
and types we wish to support floating points on would need to be updated
too).

\subsection{Artefacts}
\label{sec:artefacts}

In \productname{}, the type of games that can be implemented are limited by the fact that randomness is not supported. 
A game like minesweeper cannot be implemented with randomly generated levels. The board game monopoly would need randomness 
to implement a good die. A random card cannot be drawn and a deck of card cannot be shuffled. In our game environment described in \secref{sec:gameenvironment} many board game related types, constants and functions are implemented. Here it would be natural to also include types like
a die type, a card type, and a type that can generate random numbers. If the random type would generate random numbers based on a seed, some kind of system time constant should also be available in \productname{}. Another approach would be to automatically seed the random type based on a system time constant every time a game is loaded. 
\subsection{Stronger action system}
\label{sec:strongeractionsystem}

The actions provide everything needed for working with pieces, however
when new artefacts are added, actions which effect those are needed.
An action to take a card from a deck, an action to play a card, etc.
Some artefacts could be generalized: a card could be a \type{Piece}
and a deck could be a \type{Square}, where the order of multiple pieces
is given significance. Generalizing should be done with care though, as
over-generalizing will be counter productive with the idea of using a
domain specific language. However if cards were pieces, new action types
would not need to be created and learned by the user.

Instead of generalizing the artefacts, we could generalize actions. For example, instead of \type{AddAction} adding a \type{Piece}, \type{AddAction} could add any type and \type{MoveAction} could move both pieces and cards. However now \type{MoveAction} can depend on an \type{AddAction} having created an instance of a type, which is not expressed well in \type{ActionSequence}.

\subsection{Board types}
\label{sec:boardtypes}

The game environment of \productname{} makes it easy for programmers to create grid boards, which is a
very basic kind of board defined by a width and a height. But what if the programmer wanted to create a
circular kind of square or a hexagonal square? This is currently not possible but it would be practical 
in a future version of \productname{}. A possible solution would be a graphical representation of the board where
the edges represent squares and transitions represent the border between squares.
   
\subsection{Efficiency of pattern matching}
\label{sec:patternmatchingefficiency}
The design of pattern matching is undoubtedly a strong mechanism for describing moves, win condition, or any other check that depends on a particular pattern. The implementation of the pattern matching is highly inefficient. Consider the piece placed at the square D5 in \figref{fig:inefficientpatterns}. The blue piece can make the moves of a knight from a chess game. This can be specified in \productname{} with the pattern \texttt{/(n n e|w) | (s s e|w) | (w w n|s) | (e e n|w) this/}.

\fig[scale=0.3]{inefficientpatterns}{Pattern matching done on the 8 green squares will return true, given the moves of a knight and the blue square (D5) as input.}

To find the moves of the piece on D5, the pattern must be matched on all 64 squares given D5 as input. Those squares for which the pattern matching returns true are those squares the piece can move to. In \figref{fig:inefficientpatterns}, only 8 of those 64 checks performed are depicted. For a game of chess starting with 32 pieces, the pattern matching is actually done on all 64 squares \textbf{for all 32 pieces}. If we for simplicity assumed that each piece had 8 possible moves, the amount of work related to pattern checks for the first move in chess can be calculated to be $32 * 64 * 8 = 16384$.
Or more generally, $O(p * n * m * c)$, hvor $p$ = the number of pieces, $(n, m)$ = the size of the gridboard, and $c$ is the complexity of each move. It is easy to see how inefficient this approach is so lets now consider a better and even a more intuitive approach.

Consider the green piece placed at the square D5 in \figref{fig:efficientpatterns}. The 8 arrows shows the moves a knight can make. An easy way to find these squares is simply to start at the knight's square (D5), move two squares in an arbitrary direction and then one square in an orthogonal direction. This also seems like a quite efficient approach. This can implemented by modifying the pattern matching to take a square as input and return a list of squares that satisfy a given pattern. A pattern for the knights move could look like \texttt{\%this (n n e|w) | (s s e|w) | (w w n|s) | (e e n|w)\%}. Notice that the \% encapsulation is used to distinguish between this modified pattern matching mechanism and the actual pattern mechanism used in \productname{} which encapsulates a pattern using \\. Such a pattern matching done on D5 would return a list containing the green squares in \figref{fig:efficientpatterns}.
Compared to the pattern matching in \productname{}, this approach will have the complexity $O(p * c)$. Comparing to the previous example, the amount of work related to the first move of a game of chess would be $32 * 8 = 256 < 163840.$ The complexity here seems to not depend on the actual size of the gridboard, but this is only true in some cases, e.g. the moves of a knight. Considering the moves of a rook, the complexity of a move (the constant $c$) will depend on the size of the gridboard since this will increase the amount of squares a piece can slide to. Generally, patterns containing the pattern-value \texttt{*} or \texttt{+} will have a complexity depending on the size of the gridboard.

\fig[scale=0.3]{efficientpatterns}{The intuitive way to check the moves of a knight-piece in a chess game.}  

