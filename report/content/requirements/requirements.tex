\section{Requirements}
This section presents a set of requirements for this project which has been structured using a method published by Stig Andersen\cite{dengodekravspecifikation}. The purpose of a requirements specification is to make sure that the final product does what is was intended to do and meets the specified requirements. It is used throughout the development phases and requirements are added as the project moves along and new challenges arise.

The requirements specification consists of three main points: functional requirements, non-functional requirements and solution goals. Functional requirements define what the final system should be able to do. Non-functional requirements define different constraints and boundaries for the entire project. Lastly, solution goals are overall requirements that help us define the correct solution to our problem statement \cite{requirementsGuide}.

%Besvar spørgsmålet: “hvorfor?” for hvert krav
%Prioritere krav ved hjælp af MoSCoW (Must have, should have, could have, won’t have)

\paragraph*{Functional requirements:}
\begin{enumerate}
  \item The programming language will be used to program board games
  \begin{enumerate}
    \item It must be possible to implement chess including the special rules of chess
  \end{enumerate}
  \item It must be possible to define what pieces the game consists of
  \item It must be possible to define which pieces a specific player controls
  \item It must be possible to define the possible squares a piece can be moved to
  \item It must be possible to represent list structures
  \begin{enumerate}
    \item It must be possible to perform list unions
    \item It must be possible to perform list intersections
  \end{enumerate}
  \item It must be possible to use the language’s built-in functions to do the following:
  \begin{enumerate}
    \item determine if a square is empty
    \item determine if a square is occupied, and by who
    \item check a condition for all objects in a collection
    \item find all squares that match a specific pattern
    \item combine lists
    \item perform a lambda expression on each element in a list
    \item move a piece to a square
    \item capture the old piece on a specific square while moving a new piece to the same square
    \item return which players turn it is
    \item check if the current move about to be made for a piece is the first move made by that piece
  \end{enumerate}
  \item It must be possible to determine which legal moves a player has
  \item It must be possible to define winning conditions
  \item It must be possible to define draw conditions
  \item It must be possible to perform integer arithmetic
  \begin{enumerate}
    \item Addition, subtraction, multiplication, and division
    \item The programming language must have boolean operators
    \item The programming language must have comparison operators
  \end{enumerate}
  \item It must be possible to perform string concatenation
  \item No function nor expression may produce side effects
  \item There must be an action type that handles game state changes
  \item It must be possible to create lambda expressions
  \item It must be possible to declare functions
  \item It must be possible to reference functions
  \begin{enumerate}
    \item Functions must be first-class citizens
    \item It must be possible to call functions
  \end{enumerate}
  \item It must be possible to non-destructively assign any value to variables
  \item The created board games must be playable in a graphical simulator
  \item The simulator must be able to remember move history
  \begin{enumerate}
    \item It must be possible to undo/redo moves
    \item It must be possible to save the move history
    \begin{enumerate}
      \item It must be possible to start a game from a saved move history
    \end{enumerate}
  \item It must be possible to play over a network
  \end{enumerate}
\end{enumerate}

The list of requirements also have \textbf{non-functional requirements} which is split into two topics - performance limitations and project limitations.

\paragraph*{Performance limitations:}
\begin{enumerate}
% How do we follow up on the requirement below?
  \item It must be easy to learn how to program in the programming language
  \item The programmer must be able to implement board games with relatively few lines of code
  \item The programming language must not be an extension of another programming language
  \item The board games should as a minimum consist of two players
  \item The source code of a single board game must be written in one file
  \item The formal definition of the programming language must be described in Extended Backus-Naur Form (EBNF)
  \item The programming language will be interpreted (not compiled)
\end{enumerate}

\paragraph*{Project limitations:}
\begin{enumerate}
  \item The programming language must be functional and operable no later than 29th of May 2013
  \item The group has approximately 20 hours per week to work on the project
  \item The project is limited by the group members' skill in the design and development of programming languages
  \item The project (and hence the programming language) must have a catchy name and logo
\end{enumerate}

\paragraph*{Solution goals:}
\begin{enumerate}
  \item The programming language must make it easy and quick for programmers to develop board games which are within the scope of the defined problem statement
  \item The board games must be playable on different operating systems
\end{enumerate}

\todo{Sum up here. What are the most important points?}

