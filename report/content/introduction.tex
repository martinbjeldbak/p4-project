\chapter{Introduction}
\label{chap:introduction}

\section{A new programming language?}
So, are we really going to create a new programming language? At this very moment, hundreds, maybe thousands of new programming languages are beeing created. A Wikipedia article about the programming languages that exists, currently contains a list of 664 different programming language \cite{listofprogramminglanguages}, and alot more languages exists, since this is only a list of the more well-known languages. So why is yet another programming language needed when their exists so many programming languages already, and when it's possible in, programming languages like Java, C\# and others, to basicly program anything one would like (refered to as general purpose programming languages). The answer lies within the powers of domain-specific programming languages - a language designed to express solutions to problems in a specific domain \cite{domainspecificprogramminglanguagedefinition}. 

What we intend to do in this project is develop a programming language which can describe board games. That is designing it, defining it and implementing it. We choose to develop a programming language for programming board games in, because of our personal interest in board games. Board games are first of all fun and entertaining. They can also be rich on learning opportunities \cite{whyboardgames1?} and they can give a certain satisfaction \cite{whyboardgames2?}. In fact, board games can be almost anything as long as the creator of the game has the right amount of creativity and the building blocks needed to create the game. These are some of the reasons why it would be pratical with a programming language which made it easy to program board games in. Yes, it is possible to program board games in Java, C\# and the other general purpose programming languages, but is it easy? Definetly not as easy as in a well crafted programming language specifically designed to program board games in. If a language purely concentrates on allowing the programmers using it to express how his game works, without having to reimplement everything from scratch as he would in a general purpose programming language, it would be quicker and easier to develop shorter and more precise programs.  

Furthermore data structures and special statements specifically designed to help define board games would greatly increase the readability and writability of such a program. A language designed with board games in mind would allow the programmers to, relatively quickly, explore new ideas for a board game with a simple implementation. He/she could then efficiently modify the code according to a new rule or idea that suddenly came up, without having to make the whole thing over. If the language also took multiple platforms into account, it would open up the possibility to run the same game across multiple devices. So there are definitly many advantages of creating a new domain-specific programming language. 

Now the question is, how to develop a new domain-specific programming language in which one can program board games? First we will have to have to do some research about board games and the elements and components they consist of. We will have to know about the different programming paradigms there exists, we will have to know about compilers and/or interpreters and many other things. All of these things are included in \chapref{chap:analysis}

%\input{content/introduction/projectgoals}
%\subsection{Why board games?}
\label{whyboardgames}

%Board games have been played throughout almost all 
%of history, within different cultures, societies, 
%and countries. Backgammon is known to be at least 
%5,000 years old, and is still played today. Board 
%games can consist of a board, some player pieces or 
%tokens, a  deck of cards, and/or dice. The given 
%player or players follow a set of rules in an attempt 
%to achieve a goal. Strategy and luck are usually 
%involved in some form so playing the game more than 
%once is engaging and dynamic. Most board games only 
%involve one winner at the end of the game.

So what do board games have to do with designing,
defining, and implementing a programming language?
If you're able to describe how board games work
on a very generic and general level, it would
theoretically be possible to abstract away from
that to describe all games in general.
%; something that could be very beneficial for 
%game designers in many different ways.

If a language purely concentrates on allowing the
programmer (i.e. game designer) to express how his
game works, it would be quicker and easier to pick
up and develop shorter and more precise programs
rather than having to reimplement everything
from scratch in an already existing high-level
language. Furthermore, data structures and special
statements specifically designed to help define
board games would greatly increase the readability
and writability of such a program.

A language designed with board games in mind
would also allow the game designer to, relatively
quickly, explore new ideas for a board game
with a simple implementation. He/she could then
efficiently modify the code according to a new
rule or idea, and the implementation would stay
exactly the same. If the language also took
multiple platforms into account, it would open
up the possibility to run the same game across
multiple devices.

This could enable AI creation for the language
that understands the rules, so you can test an
early implementation of your game without the need
of other human players.

An example could be four-person chess. If you
already had an implementation of chess in the
programming language set up with a board and
separate rules for each piece, then it could be as
simple as changing the piece location and player
count (and maybe editing the rules for one or two
of the pieces so it's a little more fair) followed
by running the program again.

Programming languages exist to create programs that 
express algorithms to control the behaviour of machines.
Most board games, as we will demonstrate, have specific 
rules and exact winning conditions to follow, which can
be described to a very detailed degree.

Different categories of board games exist and board games 
can be placed into different genres, such as: strategy, 
alignment, chess variants, paper-and-pencil, territory, 
race, trivia, wargames, word games, and dozens of others. 
Obviously some games can overlap genres. Chess is an 
example of this. It is obviously a chess variant and is 
very strategy-heavy.

Before we begin to develop our programming language
we have to do some research about board games, the different 
programming paradigms, and then we will give an overview 
of what a compilers job really is, then we will present our 
findings about scanning and parsing methods, followed by 
a comparison of compilers and interpreters, and finally
we will define what a game simulator is and why we need one.
When the analysis is done we present our problem statement at
the end of this chapter.
