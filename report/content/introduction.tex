\chapter{Introduction}
\label{chap:introduction}

\section{A new programming language?}
So, are we really going to create a new programming language? At this very moment, hundreds, maybe thousands of new programming languages are beeing created. A Wikipedia article about the programming languages that exists, currently contains a list of 664 different programming language \cite{listofprogramminglanguages}, and alot more languages exists, since this is only a list of the more well-known languages. So why is yet another programming language needed when their exists so many programming languages already, and when it's possible in, programming languages like Java, C\# and others, to basicly program anything one would like (refered to as general purpose programming languages). The answer lies within the powers of domain-specific programming languages - a language designed to express solutions to problems in a specific domain \cite{domainspecificprogramminglanguagedefinition}. 

What we intend to do in this project is develop a programming language which can describe board games. That is designing it, defining it and implementing it. We choose to develop a programming language for programming board games in, because of our personal interest in board games. Board games are first of all fun and entertaining. They can also be rich on learning opportunities \cite{whyboardgames1?} and they can give a certain satisfaction \cite{whyboardgames2?}. In fact, board games can be almost anything as long as the creator of the game has the right amount of creativity and the building blocks needed to create the game. These are some of the reasons why it would be pratical with a programming language which made it easy to program board games in. Yes, it is possible to program board games in Java, C\# and the other general purpose programming languages, but is it easy? Definetly not as easy as in a well crafted programming language specifically designed to program board games in. If a language purely concentrates on allowing the programmers using it to express how his game works, without having to reimplement everything from scratch as he would in a general purpose programming language, it would be quicker and easier to develop shorter and more precise programs.  

Furthermore data structures and special statements specifically designed to help define board games would greatly increase the readability and writability of such a program. A language designed with board games in mind would allow the programmers to, relatively quickly, explore new ideas for a board game with a simple implementation. He/she could then efficiently modify the code according to a new rule or idea that suddenly came up, without having to make the whole thing over. If the language also took multiple platforms into account, it would open up the possibility to run the same game across multiple devices. So there are definitly many advantages of creating a new domain-specific programming language. 

Now the question is, how to develop a new domain-specific programming language in which one can program board games? First we will have to have to do some research about board games and the elements and components they consist of. We will have to know about the different programming paradigms there exists, we will have to know about compilers and/or interpreters and many other things. All of these things are included in \chapref{chap:analysis}

%\subsection{Project goals}
\label{projectgoals}

So what is the goal of this project? Why do we need a new programming language for generic game playing - is it not possible to code games in C or Java? Yes it is possible to code games in already existing programming languages but the goal of the project is that the students gain knowledge of important underlying concepts in the world of programming languages. How are these concepts derived? How are they formally described and represented in an implementation? 

Obviously, all software is written in some kind of a programming language and compiled or interpreted so it can be executed. Design, definition and implementation of programming languages is a central topic of Computer Science. By gaining a better understanding of these topics the student will be able to grasp the possibilities of different programming languages and programming paradigms and what their differences are.\cite[p. 22]{dat-stud-ordning} We will discuss different paradigms in \todo{refer to correct section}. 

The goal is that:

\begin{quote}
the student must learn how to design and implement a programming language and how this process can be supported by formal definitions of the languages syntax and semantics and the techniques and methods to construct a translater for the language.\cite[p. 22]{dat-stud-ordning}
\end{quote}

This report presents and documents the process and work of which we've been through to reach this goal.
%\section{Why board games?}
So what do board games have to do with designing, defining, and implementing a programming language? If you're able to describe how board games work on a very generic and general level, it would theoretically be possible to abstract away from that to describe all games in general; something that could be very beneficial for game designers in many different ways.

If a language purely concentrates on allowing the programmer (i.e. game designer) to express how his game works, it would be quicker and easier to pick up and develop shorter and more precise programs rather than having to reimplement everything from scratch in an already existing high-level language. Furthermore, data structures and special statements specifically designed to help define board games would greatly increase the readability and writability of such a program.

A language designed with board games in mind would also allow the game designer to, relatively quickly, explore new ideas for a board game with a simple implementation. He/she could then efficiently modify the code according to a new rule or idea, and the implementation would stay exactly the same.  If the language also took multiple platforms into account, it would open up the possibility to run the same game across multiple devices.

This could enable AI creation for the language that understands the rules, so you can test an early implementation of your game without the need of other human players.

An example could be four-person chess. If you already had an implementation of chess in the programming language set up with a board and separate rules for each piece, then it could be as simple as changing the piece location and player count (and maybe editing the rules for one or two of the pieces so it's a little more fair) followed by running the program again.
