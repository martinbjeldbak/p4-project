\chapter{Evaluation}
\label{chap:evaluation}

\section{Unit testing}
A \productname{}-language construct makes it easy to unit test a \productname{}-game.
If a source code contains a type which extends the \typeref{TestCase}-type, it means that any constant defined in the type must evaluate to true.

An example of a unit test of Naughts and Crosses can be seen in \todo{Input codesamples/nactest}.
The data member \varref{state1} contain a Naughts and Crosses game in its starting state.
The first constant \methodref{testTitle} verifies that the game title really is set to ``Noughts and Crosses''.
The data member \varref{state2} contains the game state after the first player has 
performed the first action in the list of possible actions. This first action will cause a piece to be placed on one of the squares, which means 
that $3*3 - 1$ squares must still be empty. This is verified by the test constant \varref{testEmptySquares}.

\section{Writing games in \productname}
We have written a handful of games in \productname{}. The games vary from classic board games to unknown puzzle games invented by individual group members.

The Ice-game is a single-player puzzle game written in no more than 5 minutes.\figref{icegame} shows a screenshot of the game. The goal is to move the 

%kent-game
%mine-sweeper
%ice

