\section{Scope}
\label{sec:scoping}

A scope is the context in which one or more variables exist.
There are five types of scopes in \productname{}. The global scope,
function scopes, type scopes, lambda scopes, and ``let''-scopes.

\subsection{The global scope}

All named constants/functions defined outside of types exist in the global scope.
Also, it is not necessary to define a function before use, so something
like the following is valid:

\codesample{usebeforedefinition.junta}

All types also exist in the global scope, and again it isn't necessary to define
a type before using (extending or constructing):

\codesample{typebeforetype.junta}

\subsection{Function scope}
Consider a function definition such as:

\codesample{functiondef.garry}

The variables \variable{a} and \variable{b} only exist within the function \function{max}.
When calling the function:

\codesample{functioncall.garry}

A new scope will be created and the values $5$ and $23$
are assigned to \variable{a} and \variable{b}, respectively.

%Named functions defined outside of type definitions (such as \function{max})
%always exist in the global scope.

\subsection{Type scope}

A type consists of a number of members, these can be either functions or constants.
Consider the following type definition:

\codesample{typescope1.junta}

In the first line a type, \type{MyType}, is defined, with a constructor taking one
argument, \variable{a}. \type{MyType} has three members, a constant (\constant{aConstant})
and two methods (\constant{aMethodA} and \constant{aMethodB}). Unlike constants defined
outside of types, which exist in the global scope, type members can only be accessed by
using the dot-operator on an instance of that type.

When constructing the type in the last line (\texttt{\type{MyType}[\literal{5}]})
an instance is returned, in which the value $5$ is assigned to \variable{a}.
At construction all constants are evaluated, meaning that in the example
the constant \constant{aConstant} is evaluated to $15 / \variable{a} = 15 / 5 = 3$.

Using the dot-operator, the method \constant{aMethodB} is called on the instance of
\type{MyType} with $2$ as an argument. In the definition of this method, we use the
\keyword{this}-keyword to refer to the current instance, in order to call another
method, \constant{aMethodA}. This is in fact unnecessary, since the
\constant{aMethodB}-method is freely accessible within the type scope anyway. The
\keyword{this}-keyword has other more useful uses, such as passing the current
instance to other functions or constructors. In the \constant{aMethodA}-method
the sum of the constant \constant{aConstant} (this time accessed without the
\keyword{this}-keyword), the constructor parameter \variable{a}, and the function
parameter \variable{b} is evaluated and returned.

Each type method can be thought of as having a reference to the type scope of an
instance, so that when the method is called, it has access to the constructor 
parameters along with type members.

\subsubsection{Inheritance}

\productname{} supports single inheritance between types. An inheriting type, will
inherit all the members of its super type(s) (if \type{C} extends \type{B}, which
extends \type{A}, then \type{C} inherits all members from both \type{B} and \type{A}).
This also introduces the \keyword{super}-keyword, which can be used to access members
in superclasses. The following example shows how inheritance works:

\codesample{typescope2.junta}

\productname{} doesn't have an \keyword{abstract}-keyword for type definitions, only
for constant/function definitions. A type is implicitly marked as abstract if it has
unimplemented abstract members. In the above example the type \type{A} is abstract,
since because it's member, \constant{constantA}, is abstract. Likewise, the type
\type{B} is abstract because it extends \type{A}, but doesn't implement the abstract
member of \type{A}. The type \type{C} on the other hand is not abstract, since it
implements the abstract constant. Abstract types can't be constructed, albeit the
constructor for an abstract type has to be used when extending the type (after the
\keyword{extends}-keyword).

\subsection{Lambda expression scope}

When a lambda expression is created, a reference to the scope it was created in
is saved with it. This is known as a closure, and means that a lambda
expression may access variables outside of its own scope. The accessible
variables are the variables that were available at the time of the creation of
the lambda expression.

Consider the following example:

\codesample{closuredef.garry}

A function \function{getAdder}, which takes one argument (\variable{a}) and
returns a lambda expression, is defined. Notice how \variable{a} is used within
the lambda expression. This means that when the lambda expression is created, it
must remember the value of the variables that exist in the scope, in which it is
created. The use of the \function{getAdder}-function could look like this:

\codesample{closureuse.garry}

In the first line \function{getAdder} is called with the argument, $25$. A new scope, $A$,
is created, in which the variable \variable{a} is assigned the value $25$. Then the function
expression is evaluated, which results in a new lambda expression (with a reference to scope $A$).
This is returned and assigned to \variable{adder} in line 1 of the above example.

In the second line the \variable{adder} is called as a function, which means that a new scope, $B$,
is created, in which the variable \variable{b} is assigned the value $5$. The important part is
that $B$'s parent scope is set to $A$ (which is saved with the lambda expression). The expression
(the right side of the lambda expression) is then evaluated. First the \variable{a}-variable is
encountered. The interpreter first searches the $B$-scope for \variable{a}, and when unsuccessful,
searches the parent-scope, $A$, for \variable{a}. In $A$ the variable \variable{a} holds the value
$25$, and this is returned. Then the $B$-scope is searched for the \variable{b}-variable, and the value
$5$ is returned. The two integers are added, and the final return-value of the lambda-expression
ends up being $30$.

\subsection{Let-expressions}

\productname{} only supports \emph{single assignment}. Single assignment is not assignment
in the traditional imperative sense, but rather a way of binding a value to a symbol in a
certain scope. This is done using \emph{let-expressions}. Using a let-expression creates a
new scope in which the declared variables are accessible. When leaving the scope the
variables are destroyed.

The basic format of a let-expression is:

\texttt{let VARIABLE1 = EXPRESSION1 in EXPRESSION2}

In the example, the value of \texttt{EXPRESSION1} is assigned to \texttt{VARIABLE1}, which
is available in \texttt{EXPRESSION2}. Another example could be:

\texttt{let VARIABLE1 = EXPRESSION1, VARIABLE2 = EXPRESSION2 in EXPRESSION3}

In this example, the value of \texttt{EXPRESSION1} is assigned to \texttt{VARIABLE1}, and
the value of \texttt{EXPRESSION2} is assigned to \texttt{VARIABLE2}. Both \texttt{VARIABLE1}
and \texttt{VARIABLE2} are available in \texttt{EXPRESSION3} and only in \texttt{EXPRESSION3}.

Destructive assignments are not possible \productname{}, meaning that it isn't possible to 
reassign a variable. It is however possible to hide a variable.
Consider the following expression:

\codesample{lethiding.garry}

The value of this expression is $13$. This is because within the \texttt{\variable{x} + 2}-expression
the \variable{x}-variable evaluates to $6$. But in the outer expression \variable{x} evaluates to
$5$.

Nested \emph{let}-scopes are possible. Consider for instance:

\codesample{nestedlet.garry}

In the inner scope, both \variable{x} and \variable{y} are available. This is of course equivalent
to:

\codesample{nestedlet2.garry}


