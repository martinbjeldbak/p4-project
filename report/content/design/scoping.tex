\section{Scoping}
\label{sec:scoping}

A scope is the context in which one or more variables exist.
There are three types of scopes in \productname{}. Function scopes, lambda scopes and
``let''-scopes.

\subsection{Function scope}
Consider a function definition such as:

\codesample{functiondef.garry}

The variables \variable{a} and \variable{b} only exist within the function \function{max}.
When calling the function:

\codesample{functioncall.garry}

A new scope will be created and the values $5$ and $23$
are assigned to \variable{a} and \variable{b}, respectively.

\subsection{Lambda expression scope}

\subsection{Let-expressions}


Outline:

\begin{dlist}
\item What is scoping?
\item Examples of static/lexical versus dynamic scoping
\item Why do we want to use dynamic scoping
\item What does that mean for \productname?
\end{dlist}
