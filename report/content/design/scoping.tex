\section{Scoping}
\label{sec:scoping}

A scope is the context in which one or more variables exist.
There are three types of scopes in \productname{}. Function scopes, lambda scopes and
``let''-scopes.

\subsection{Function scope}
Consider a function definition such as:

\codesample{functiondef.garry}

The variables \variable{a} and \variable{b} only exist within the function \function{max}.
When calling the function:

\codesample{functioncall.garry}

A new scope will be created and the values $5$ and $23$
are assigned to \variable{a} and \variable{b}, respectively.

Named functions (such as \function{max}) always exist in the global scope.

\subsection{Lambda expression scope}

When a lambda expression is created, a reference to the scope it was created
in is saved with it. This is known as a closure, and means that a lambda
expression may access variables outside of its own scope. The accessible
variables are the variables that were available at the time of the creation
of the lambda expression.

Consider the following example:

\subsection{Let-expressions}


Outline:

\begin{dlist}
\item What is scoping?
\item Examples of static/lexical versus dynamic scoping
\item Why do we want to use dynamic scoping
\item What does that mean for \productname?
\end{dlist}
