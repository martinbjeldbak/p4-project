\section{Grammar}
\label{sec:grammar}

This section presents our grammar with all the consisting parts of it. Firstly, we present our notational conventions followed by sections that present our character classes, the reserved words, the literals and the identifiers in our programming language. After this we present the actual structure of a program written in \productname{} followed by the expressions and patterns which the structure consists of.

%Following this section the abstract node types section (\secref{sec:ant}) begins where we present the different node types for the program structure presented in this section.

\subsection{Notational Conventions}
We use a variant of Extended Backus-Naur Form to express the context-free grammar of
our programming language.

Each production rule assigns an expression of terminals, non-terminals and operations
to a non-terminal. E.g. in the following example the non-terminal $decimal$ is assigned
the possible terminals of $\gter{0}$ up to and including $\gter{9}$.

\begin{ebnf}
\grule{decimal}{\gter{0} \gor \gter{1} \gor \grange \gor \gter{9}}
\end{ebnf}

The following operations are used throughout this section to describe the grammar of the programming language:

\begin{center}
\begin{tabular}{r l}
  $\gopt{pattern}$ & an optional pattern \\
  $\grep{pattern}$ & zero or more repititions of pattern \\
  $\ggrp{pattern}$ & a group \\
  $pattern_1 \gor pattern_2$ & a selection \\
  $\gter{0} \gor \grange \gor \gter{9}$ & a range of terminals \\
  $pattern_1 \gex pattern_2$ & matched by $pattern_1$ but not by $pattern_2$\\
  $pattern_1 \gcat pattern_2$ & concatenation of $pattern_1$ and $pattern_2$ \\
  $\gter{test}$ & a terminal \\
  $\gtsq$ & a terminal single quotation mark \\
  $\gtdq$ & a terminal double quotation mark \\
  $\gtbs$ & a terminal backslash character \\
\end{tabular}
\end{center}

\subsection{Character Classes}
To be able to describe which characters or symbols a non-terminal can consist of in a concise manner, we need to define some sets of symbols to specific names which we can use in the description of our grammar.

The following classes of characters will be used throughout this section:

\begin{ebnf}
\grule{decimal}{\gter{0} \gor \gter{1} \gor \grange \gor \gter{9}}
\grule{lowercase}{\gter{a} \gor \gter{b} \gor \grange \gor \gter{z}}
\grule{uppercase}{\gter{A} \gor \gter{B} \gor \grange \gor \gter{Z}}
\grule{anycase}{lowercase \gor uppercase}
\grule{quotebs}{\gtdq \gor \gtbs}
\grule{unichar}{\gcomment{any unicode character}}
\grule{strchar}{unichar \gex quotebs}
\end{ebnf}

\subsection{Reserved Tokens}
Our programming language needs to be able to distinguish between words (a sequence of characters or symbols) that are reserved for the language to function properly and words which a user has constructed to be used as e.g. constant names.

The following is a list of reserved words which \productname{} will use to function properly and they will be refered to throughout this section:

\begin{ebnf}
%reserved
\grule{shared\_operator}{\gter{*} \gor \gter{+}}
\grule{normal\_operator}{\gter{-} \gor \gter{!=} \gor \gter{==} \gor \gter{<=} \gor \gter{>=} \gor \gter{<} \gor \gter{>}}
\grule{operator}{shared\_operator \gor normal\_operator \gor \gter{/}}
\grule{pattern\_operator}{\gter{?}}
\grule{pattern\_not}{\gter{!}}
\grule{pattern\_or}{\gter{|}}
\grule{pattern\_keyword}{\gter{friend} \gor \gter{foe} \gor \gter{empty}}
\end{ebnf}

\subsection{Literals}
\productname{} consists of the following four literals which are used throughout this section:

\begin{ebnf}
%Literals
\grule{integer}{decimal \gcat \grep{decimal}}
\grule{direction}{\gter{n} \gor \gter{s} \gor \gter{e} \gor \gter{w} \gor \gter{ne} \gor \gter{nw}}
\galt{\gter{se} \gor \gter{sw}}
\grule{coordinate}{upperccase \gcat \grep{uppercase} \gcat decimal \gcat \grep{decimal}}
\grule{string}{\gtdq \gcat \grep{strchar \gor \gtbs \gcat unichar} \gcat \gtdq}
\end{ebnf}

Non of the above literals can be empty (by empty we mean assigned to null-like value). The following list briefly describeds the literals in plain English:

\begin{dlist}
\item The integer literal must consist of at least one decimal
\item The direction literal can consist of either one of the eight coordinate-like letters 
\begin{dlist}
\item By coordinate-like we mean north, south, west, etc. and must not be confused with the coordinate literal
\end{dlist}
\item The coordinate literal must consist of at least one uppercase letter followed by at least one decimal value
\begin{dlist}
\item Coordinates will be described in greater detail in \secref{sec:coordinatetype}
\end{dlist}
\item The string literal can consist of a sequence of strchar's or a blackslash followed by any unicode character.
\begin{dlist}
\item A string literal can be empty and just contain nothing between the quotation marks, but this is not the same as been assign a null-like value
\end{dlist}
\end{dlist}

\subsection{Identifiers}
The following three identifiers are used throughout this section:

\begin{ebnf}
%Identifiers
\grule{constant}{lowercase \gcat \grep{anycase}}
\grule{type}{uppercase \gcat \grep{anycase}}
\grule{variable}{\gter{\$} \gcat anycase \gcat \grep{anycase}}
\end{ebnf}

This description means that every constant must start with a lowercase character and not uppercase. If an uppercase character begins a sequence of characters, the parser will interpret this as an type which is quite different from a constant. Furthermore, all variables begin with a dollar symbol followed by at least one character.

\subsection{Program Structure}
The following grammar describes the actual construction of any program written in \productname{}:

\begin{ebnf}
%Program structure
\grule{program}{\grep{definition}}
\grule{definition}{constant\_def}
\galt{type\_def}
\grule{constant\_def}{\gter{define} \gcat constant \gcat \gopt{varlist} \gcat \gter{=} \gcat \gcat expression}
\grule{type\_def}{\gter{type} \gcat type \gcat varlist \gcat \gopt{\gter{extends} \gcat type \gcat list} \gopt{type\_body}}
\grule{type\_body}{\gter{\{} \gcat \grep{member\_def} \gcat \gter{\}}}
\grule{member\_def}{abstract\_def}
\galt{constant\_def}
\grule{abstract\_def}{\gter{define} \gcat \gter{abstract} \gcat constant \gcat \gopt{varlist}}
\grule{varlist}{\gter{[} \gcat \gopt{variable \gcat \grep{\gter{,} \gcat variable} \gcat \gopt{\gter{,} \gcat vars} \gor vars} \gcat \gter{]}}
\grule{vars}{\gter{...} \gcat variable}
\end{ebnf}

The grammar describes that any program written in \productname{} consists of constants and types. Within these types and constants the programmer can for example define expressions and variable lists.

\subsection{Expressions}
The following grammar describes the expressions of the programming language:

\begin{ebnf}
%Expressions
\grule{expression}{assignment}
\galt{if\_expr}
\galt{lambda\_expr}
\galt{\gter{not} \gcat expression}
\galt{lo\_sequence}
\grule{lo\_sequence}{eq\_sequence \gcat \grep{\ggrp{\gter{and} \gor \gter{or}} \gcat eq\_sequence}}
\grule{eq\_sequence}{cm\_sequence \gcat \grep{\ggrp{\gter{==} \gor \gter{!=}} \gcat cm\_sequence}}
\grule{cm\_sequence}{as\_sequence \gcat \grep{\ggrp{\gter{<} \gor \gter{>} \gor \gter{<=} \gor \gter{>=}} \gcat as\_sequence}}
\grule{as\_sequence}{md\_sequence \gcat \grep{\ggrp{\gter{+} \gor \gter{-}} \gcat md\_sequence}}
\grule{md\_sequence}{negation \gcat \grep{\ggrp{\gter{*} \gor \gter{/} \gor{\%}} \gcat negation}}
\grule{negation}{element}
\galt{\gter{-} \gcat negation}
\grule{element}{call\_sequence \gcat \grep{member\_access}}
\grule{member\_access}{\gter{.} \gcat constant \gcat \grep{list}}
\grule{call\_sequence}{atomic \gcat \grep{list}}
\grule{atomic}{\gter{(} \gcat expression \gcat \gter{)}}
\galt{variable}
\galt{list}
\galt{\gter{/} \gcat pattern \gcat \gter{/}}
\galt{\gter{this}}
\galt{\gter{super}}
\galt{direction}
\galt{coordinate}
\galt{integer}
\galt{string}
\galt{type}
\galt{constant}
\grule{assignment}{\gter{let} \gcat variable \gcat \gter{=} \gcat expression \gcat \grep{\gter{,} \gcat variable \gcat \gter{=} \gcat expression} \gcat \gter{in} \gcat expression}
\grule{if\_expr}{\gter{if} \gcat expression \gcat \gter{then} \gcat expression \gcat \gter{else} \gcat expression}
\grule{lambda\_expr}{\gter{\#} \gcat varlist \gcat \gter{=>} \gcat expression}
\grule{list}{\gter{[} \gcat \gopt{expression \gcat \grep{\gter{,} \gcat expression}} \gcat \gter{]}}
\end{ebnf}

The expression-expression and atomic-expression are the two main expressions which are repeatedly used to form different expressions such as assignments, if expressions, lambda expressions, patterns, and lists.

\subsection{Patterns}
The following patterns are constructed from the atomic-expression:

\begin{ebnf}
%Patterns
\grule{pattern}{pattern\_expr \gcat \grep{pattern\_expr}}
\grule{pattern\_expr}{pattern\_val \gcat \gopt{\gter{*} \gor \gter{?} \gor \gter{+}}}
\galt{pattern\_val \gcat \gter{|} \gcat pattern\_expr}
\grule{pattern\_val}{direction}
\galt{variable}
\galt{pattern\_check}
\galt{\gter{!} \gcat pattern\_check}
\galt{\gter{(} \gcat pattern \gcat \gter{)} \gcat \gopt{integer}}
\grule{pattern\_check}{pattern\_keyword}
\galt{\gter{this}}
\galt{type}
\end{ebnf}


