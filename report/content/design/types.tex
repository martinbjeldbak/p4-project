\section{Types}
\label{sec:types}

\productname{} has support for the following types: Integer, String, Direction, Coordinate,
Function, Pattern, List and Action.

There is no $null$-type or $null$-value, since all expressions must have a value. This
is also evident in the definition of the $if$-expression in \secref{sec:grammar}, in that
all $if$-expressions must have the $else$-branch.

\subsection{Integer}
The integer-type in \productname{} represents a 32-bit integer.

An integer-value can be created using an integer-literal, such as \texttt{2155} or \texttt{0}.

\subsection{String}
The string-type represents a UTF-8 encoded string.

A string-value is created using a string-literal, such as \texttt{"Hello, World"} or \texttt{""}.

\subsection{Direction}
The direction-type represents a direction on a game board.

\fig{direction_n}{The \texttt{n}-direction.}

\subsection{Coordinate}

\subsection{Function}

\subsection{Pattern}

\subsection{List}

\subsection{Action}

\begin{description}[noitemsep]
\item[Integer] a 32-bit signed integer.
\item[String] a UTF-8 encoded string.
\item[Direction] a directional value.
\item[Coordinate] a 2-dimensional vector consisting of an $x$-integer value and
a $y$-integer value, representing a position in a grid.
\item[Function] a function reference (could be anonymous, e.g. lambda expression).
\item[Pattern] a pattern used for matching patterns on the board.
\item[List] a list of values (can be empty).
\item[Action]
\end{description}


Dynamic typing.


