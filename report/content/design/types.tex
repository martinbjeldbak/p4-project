\section{Types}
\label{sec:types}

\productname{} has support for the following types: Integer, String, Direction, Coordinate,
Function, Pattern, List and Action.

There is no $null$-type or $null$-value, since all expressions must have a value. This
is also evident in the definition of the $if$-expression in \secref{sec:grammar}, in that
all $if$-expressions must have the $else$-branch.

\subsection{Integer}
The integer-type in \productname{} represents a 32-bit integer.

An integer-value can be created using an integer-literal, such as \texttt{2155} or \texttt{0}.

\subsection{String}
The string-type represents a UTF-8 encoded string.

A string-value is created using a string-literal, such as \texttt{"Hello, World!"} or \texttt{""}.

\subsection{Direction}
The direction-type represents a direction on a game board. It works like a vector
in the sense that they can be combined to compute new directions. The basic directions
are \texttt{n}, \texttt{e}, \texttt{s} and \texttt{w} (north, east, south and west).
On a 2-dimensional grid (such as for chess) north is up, east is right, south is down and 
west is left.

The directions \texttt{ne}, \texttt{nw},
\texttt{se} and \texttt{sw} are also available, although these could also be produced
by combining the basic directions (e.g. \texttt{n + e = ne}). An example of a direction combination is the
expression \texttt{n + n + e} which produces
the direction shown in \figref{fig:direction_nne}. This direction could also be produced by
\texttt{n + ne} or \texttt{ne + n}.

\fig{direction_n}{The \texttt{n}-direction.}

\fig{direction_nne}{The \texttt{n + n + e}-direction.}

\subsection{Coordinate}
This type represents a position in a grid, i.e. on the game board. It is created with
a coordinate-literal, e.g. \texttt{A1} or \texttt{AH23}. The first part (the alphabetical part)
represents the column (or x-value), i.e. \texttt{A} means column $1$ and \texttt{AH} means
column $1 \cdot 26 + 8 = 34$. The second part (the numeric part) represents the row (or y-value).

\fig{coordinates}{All coordinates on an $8 \times 8$ board.}

\subsection{Function}
Functions in \productname{} are first-class citizens, meaning that they can be used as any
other value. A function name without a list of arguments results in a reference to that
function. Function references can be passed as arguments to other functions or as a return
value.

Another way to create function references, is to use anonymous functions in the form
of lambda expressions. A lambda expression is created by combining a list of input-variables
with an expression, like so:

\codesample{lambdaexpression.garry}

In the example above, a lambda expression is assigned to to \variable{max}-variable, before
being called as a function in line 2. The \texttt{\#}-symbol is used to mark the beginning
of a lambda expression. The scope rules of lambda expressions are further described in \secref{sec:scoping}
while the declaration of named functions is described in \secref{sec:functions}.

\subsection{Pattern}
A unique feature of \productname{} is patterns.
a pattern used for matching patterns on the board.
\subsection{List}
A list is an ordered collection of values. The same value may occur more than once. 

\subsection{Identifier}

\subsection{Action}

Something about monads here...
