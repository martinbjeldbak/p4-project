\section{Types}
\label{sec:types}

\productname{} has support for the following types: Integer, String, Direction, Coordinate,
Function, Pattern, List and Action.

There is no $null$-type or $null$-value, since all expressions must have a value. This
is also evident in the definition of the $if$-expression in \secref{sec:grammar}, in that
all $if$-expressions must have the $else$-branch.

\subsection{Integer}
The \type{Integer}-type in \productname{} represents a 32-bit integer.

An \type{Integer}-value can be created using an integer-literal, such as \texttt{2155} or \texttt{0}.

\subsection{String}
The \type{String}-type represents a UTF-8 encoded string.

A \type{String}-value is created using a string-literal, such as \texttt{"Hello, World!"} or \texttt{""}.

\subsection{Direction}
The \type{Direction}-type represents a direction on a game board. It works like a vector
in the sense that they can be combined to compute new directions. The basic directions
are \texttt{n}, \texttt{e}, \texttt{s} and \texttt{w} (north, east, south and west).
On a 2-dimensional grid (such as for chess) north is up, east is right, south is down and 
west is left. The \texttt{n}-direction is shown on \figref{fig:direction_n}.

The directions \texttt{ne}, \texttt{nw},
\texttt{se} and \texttt{sw} are also available, although these could also be produced
by combining the basic directions (e.g. \texttt{n + e = ne}). An example of a direction combination is the
expression \texttt{n + n + e} which produces
the direction shown in \figref{fig:direction_nne}. This direction could also be produced by
\texttt{n + ne} or \texttt{ne + n}.

\fig{direction_n}{The \texttt{n}-direction.}

\fig{direction_nne}{The \texttt{n + n + e}-direction.}

\subsection{Coordinate}
\label{sec:coordinatetype}

This type represents a position in a grid, i.e. on the game board. It is created with
a coordinate-literal, e.g. \texttt{A1} or \texttt{AH23}. The first part (the alphabetical part)
represents the column (or x-value), i.e. \texttt{A} means column $1$ and \texttt{AH} means
column $1 \cdot 26 + 8 = 34$. The second part (the numeric part) represents the row (or y-value).
In \figref{fig:coordinates} all available coordinates on an $8 \times 8 $ board are shown.

\fig{coordinates}{All coordinates on an $8 \times 8$ board.}

\subsection{Function}
\label{sec:functiontype}
Functions in \productname{} are first-class citizens, meaning that they can be used as any
other value. A function name without a list of arguments results in a reference to that
function, a value of type \type{Function}. Function references can be passed as arguments 
to other functions or as a return value.

Another way to create function references, is to use anonymous functions in the form
of lambda expressions. A lambda expression is created by combining a list of input-variables
with an expression, like so:

\codesample{lambdaexpression.garry}

In the example above, a lambda expression is assigned to to \variable{max}-variable, before
being called as a function in line 2. The \texttt{\#}-symbol is used to mark the beginning
of a lambda expression. The scope rules of lambda expressions are further described in \secref{sec:scoping}
while the declaration of named functions is described in \secref{sec:functions}.

\subsection{Boolean}
Although there are no boolean constants (``true'' and ``false'') in \productname,
the type, \type{Boolean}, still exists, since boolean values can be returned by
some expressions. Boolean values are also required in the condition-part of
if-expressions.

\subsection{Pattern}
The \type{Pattern}-type represents a special kind of expression used for
matching certain board configurations in \productname{}.

\subsection{List}
The \type{List}-type represents an ordered collection of values. The same value
may occur more than once. Lists are created using square brackets containing a
comma separated list of elements. The empty list can be created with
$\texttt{[]}$. Lists are important, and necessary when calling functions.

List values may be accessed by calling the list as if it was a function.
The following expression will return the element at offset 1 in the list
($5$):

\codesample{listaccess1.junta}

Ranges of elements can also be returned. For instance in the following
expression a new list is returned containing elements from offset 1 up
to and including offset 2 (the list $\texttt{["is", "a"]}$):

\codesample{listaccess2.junta}

An offset can be negative, which means that the offset is dependent on
the size of the list. For example in order to return the last two elements
of a list one could use the offsets, $-2$ and $-1$:

\codesample{listaccess3.junta}

Some other examples are:

\codesample{listaccess4.junta}

\subsection{Type}
The \type{Type}-type is a kind of metatype, which every other type (including itself)
is an instance of. As with functions,
types in \productname{} are first-class citizens, and can be referred to and passed
around as any other type. Types are defined in the global scope using the
\keyword{type}-keyword, and their names must always start with an uppercase letter.

All built-in types, as well as user-defined types, can be referred in expressions using
their name. For instance in the following expression the \type{Integer}-value (\literal{25}) is compared
to the \type{Type}-value (\type{Integer}) using the \texttt{is}-operator:

\codesample{intisint.junta}

\subsection{Action}

The \type{Action}-type represent state changes in a board game. A state change could for instance be the movement
of a piece on the board. Action-values are returned from certain built-in action functions. An example
of such a function could be:

\codesample{actions1.junta}

To someone familiar with imperative programming languages, this might look like a function that
instantly moves a piece from square $A1$ to square $C3$, but this isn't the case. Instead the
function returns an action-value that encapsulates the change in state.

Multiple actions can be combined in a sequence such as:

\codesample{actions2.junta}

This returns a new action, that describes a change in state where, first a piece on square $A1$
is moved to $C3$, then a piece on square $A2$ is moved to $C4$ and last a piece on square $A3$
is moved to $C5$.



