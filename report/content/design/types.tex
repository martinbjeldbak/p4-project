\section{Types}
\label{sec:types}

\productname{} has support for the following types: Integer, String, Direction, Coordinate,
Function, Pattern, List and Action.

There is no $null$-type or $null$-value, since all expressions must have a value. This
is also evident in the definition of the $if$-expression in \secref{sec:grammar}, in that
all $if$-expressions must have the $else$-branch.

\subsection{Integer}
The integer-type in \productname{} represents a 32-bit integer.

An integer-value can be created using an integer-literal, such as \texttt{2155} or \texttt{0}.

\subsection{String}
The string-type represents a UTF-8 encoded string.

A string-value is created using a string-literal, such as \texttt{"Hello, World"} or \texttt{""}.

\subsection{Direction}
The direction-type represents a direction on a game board. It works like a vector
in the sense that they can be combined to compute new directions. The basic directions
are \texttt{n}, \texttt{e}, \texttt{s} and \texttt{w} (north, east, south and west).
On a 2-dimensional grid (such as for chess) north is up, east is right, south is down and 
west is left.

The directions \texttt{ne}, \texttt{nw},
\texttt{se} and \texttt{sw} are also available, although these could also be produced
by combining the basic directions (e.g. \texttt{n + e = ne}). An example of a direction combination is the
expression \texttt{n + n + e} which produces
the direction shown in \figref{fig:direction_nne}. This direction could also be produced by
\texttt{n + ne} or \texttt{ne + n}.

\fig{direction_n}{The \texttt{n}-direction.}

\fig{direction_nne}{The \texttt{n + n + e}-direction.}

\subsection{Coordinate}
This type represents a position in a grid, i.e. on the game board. It is created with
a direction-literal, e.g. \texttt{A1} or \texttt{AH23}. The first part (the alphabetical part)
represents the column (or x-value), i.e. \texttt{A} means column $1$ and \texttt{AH} means
column $1 \cdot 26 + 8 = 34$. The second part (the numeric part) represents the row (or y-value).

\fig{coordinates}{All coordinates on an $8 \times 8$ board.}

\subsection{Function}
a function reference (could be anonymous, e.g. lambda expression).
\subsection{Pattern}
a pattern used for matching patterns on the board.
\subsection{List}
a list of values (can be empty).
\subsection{Action}


