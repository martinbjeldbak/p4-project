\section{Operators}

\productname{} supports a number of built-in operators. In this section the operators
of \productname{} are described using the format:

\operator[LeftOperandType]{operator}{RightOperandType}{ReturnType}

The available types are described in \secref{sec:types}.
Operations that are not described in this section can be considered invalid.

\subsection{Boolean operators}

These operators only accept boolean operands and only return boolean values.

\begin{dlist}
  \item \operator[Boolean]{and}{Boolean}{Boolean}\\
    Returns true when both operands are true and false otherwise. 
  \item \operator[Boolean]{or}{Boolean}{Boolean}\\
    Returns true when at least one of the operands are true and false otherwise.
  \item \operator{not}{Boolean}{Boolean}\\
    Returns true if the single operand is false and false otherwise.
\end{dlist}

\subsection{Comparison operators}

These operators are used when comparing other values, they will always return
boolean values.

\begin{dlist}
  \item \operator[Integer]{<}{Integer}{Boolean}\\
    Returns true if the left operand is less than the right one.
  \item \operator[Integer]{>}{Integer}{Boolean}\\
    Returns true if the left operand is greater than the right one.
  \item \operator[Integer]{<=}{Integer}{Boolean}\\
    Returns true if the left operand is less than or equal to the right one.
  \item \operator[Integer]{>=}{Integer}{Boolean}\\
    Returns true if the left operand is greater than or equal to the right one.
  \item \operator[Integer]{==}{Integer}{Boolean}\\
    \operator[String]{==}{String}{Boolean}\\
    \operator[Boolean]{==}{Boolean}{Boolean}\\
    \operator[Coordinate]{==}{Coordinate}{Boolean}\\
    \operator[Direction]{==}{Direction}{Boolean}\\
    Returns true if the left operand is equal to the right one.
  \item \operator[Integer]{!=}{Integer}{Boolean}\\
    \operator[String]{!=}{String}{Boolean}\\
    \operator[Boolean]{!=}{Boolean}{Boolean}\\
    \operator[Coordinate]{!=}{Coordinate}{Boolean}\\
    \operator[Direction]{!=}{Direction}{Boolean}\\
    Returns true if the left operand is not equal to the right one.
\end{dlist}

\subsection{Integer operators}

The following operations are possible on integers:

\begin{dlist}
  \item \operator[Integer]{+}{Integer}{Integer} \\
    Integer addition.
  \item \operator[Integer]{-}{Integer}{Integer} \\
    Integer subtraction.
  \item \operator[Integer]{*}{Integer}{Integer} \\
    Integer multiplication.
  \item \operator[Integer]{/}{Integer}{Integer} \\
    Integer division.
\end{dlist}

\subsection{String operators}

It is possible to concatenate strings:

\begin{dlist}
  \item \operator[String]{+}{String}{String} \\
    Returns the concatenation of two strings.
  \item \operator[String]{+}{Integer}{String} \\
   \operator[String]{+}{Boolean}{String} \\
   \operator[String]{+}{Coordinate}{String} \\
   \operator[String]{+}{Direction}{String} \\
   \operator[Integer]{+}{String}{String} \\
   \operator[Boolean]{+}{String}{String} \\
   \operator[Coordinate]{+}{String}{String} \\
   \operator[Direction]{+}{String}{String} \\
    Returns the concatenation of a string and the string-representation of another type
\end{dlist}

\subsection{Direction operators}

It is possible to combine directions using the following operator:

\begin{dlist}
  \item \operator[Direction]{+}{Direction}{Direction} \\
    Combine two directions.
\end{dlist}

For instance adding the directions \texttt{n} and \texttt{e} produces a direction
equivalent with the direction \texttt{ne}. More information about the direction-type
is available in \todo{section ????}.


