\section{Patterns}
\label{sec:patterns}

This section covers how to use patterns and what to use them for. The operators of a pattern looks like and behaves a little like regular expressions. This EBNF-grammar shows how a pattern is constructed:

\begin{ebnf}
\grule{pattern}{pattern\_expr \gcat \grep{pattern\_expr}}
\grule{pattern\_expr}{pattern\_val \gcat \gopt{\gter{*} \gor \gter{?} \gor \gter{+}}}
\galt{pattern\_val \gcat \gter{|} \gcat pattern\_expr}
\grule{pattern\_val}{direction}
\galt{variable}
\galt{pattern\_check}
\galt{\gter{!} \gcat pattern\_check}
\galt{\gter{(} \gcat pattern \gcat \gter{)} \gcat \gopt{integer}}
\grule{pattern\_check}{\gter{friend}}
\galt{\gter{foe}}
\galt{\gter{empty}}
\galt{\gter{this}}
\galt{type}
\end{ebnf}


A pattern is checked on a particular square, and returns either true or false. An example of a pattern is \text{/n n e empty/}. This pattern can be checked on the board seen in \figref{patternboard} on the field \textbf{A1}. The pattern says ``go one square north, go one square north, go one square east, check if square is empty''. This means that the square \textbf{B3} will be checked for emptiness. Since the square is occupied by a piece, the pattern will return false if checked on \textbf{A1}. However, the same pattern checked on \textbf{C1} will return true since the square at \textbf{D3} is empty.
\fig[scale=2]{patternboard}{A simple 4 X 4 board with 3 pieces}

With this basic introduction to a simple pattern check, the table (insert ref and convert list below to a table) describes briefly how the different pattern operators work. For each operator, an example of the use in a context are given in \secref{sec:patternexamples}. Note that the description of patterns assumes a minor understanding of regular expressions, see (ref http://www.regular-expressions.info/).

n : north \\
e : east \\
s : south \\
w : west \\
* : zero-to-many times\\
+ : one-to-many times\\
? : zero-or-one time\\
| : or\\
! : not\\
( \textit{pattern} ) : encapsulation\\
empty : square contains no pieces\\
friend :  square contains at least one friendly piece of the current player\\
foe : square contains at least one enemy piece of the current player\\
this : square contains the piece for which the check is being performed\\ 

\subsection{Pattern examples}
\label{sec:patternexamples}
All these examples of pattern checks are performed on the board and pieces seen in \figref{patternboard}. For each operator, two examples of a pattern check on a particular square is given. The first check returns true and the second check false.

on A1, the pattern \text{/n empty/} returns true because A2 is empty\\
on B1, the pattern \text{/n empty/} returns false because B2 is not empty\\
on C3, the pattern \text{/e empty/} returns true because C4 is empty\\
on C4, the pattern \text{/e empty/} returns false because C5 is out of board\\
on C3, the pattern \text{/n* n e empty/} returns true because moving north 2 times then north and east hits an empty square on B4\\

Notice that the *-operator causes many branches to be made. In the previous pattern-example, \text{/n* n e empty/}, the branch north zero times, north, east empty is a false branch. If just one of the branches returns true, the entire pattern does as well. The same counts for the +, ? and | operator. When a branch moves out of board it returns false immediately.

on C3, the pattern \text{/n* s s !empty/} returns false because neither A1 nor A2 contains a piece\\
on A1, the pattern \text{/n+ e empty/} returns true only because B4 is empty\\
on A1, the pattern \text{/n+ e empty/} returns true only because B4 is empty\\
