\section{Big-step semantics}
There are two kinds of operational semantics. we will only use and describe the
big-step semantics for \productname{}. The reason for not describing small-step
semantics is taht we do not find i necessary. The big-step semantics will
capture the operational semantics of programs written in \productname{}.

We have divided the following subsections into expressions, lists, variable
lists, and definitions. In each section we describe the construction of the
big-step semantics by showing the transition rules for each construct.

\subsection{Expressions}

\begin{table}[ht]
  \begin{center}
    \begin{tabular*}{\textwidth}{lc}
      $[\mbox{PLUS}_{\mbox{INT}}]$ & \infrule{env_{V}, sto \vdash E_{1} \ra
      v_{1} \quad env_{V}, sto \vdash E_{2} \ra
      v_{2}}{env_{V}, sto \vdash E_{1} + E_{2} \ra v} \\ 
      & where $v=v_{1}+v_{2}$ \\
      & \\
      $[\mbox{PLUS}_{\mbox{STR}}]$ & \infrule{s \vdash E_{1} \ra v_{1} \: s \vdash E_{2} \ra
      v_{2}}{s \vdash E_{1} + E_{2} \ra v} \\
      & where $v=v_{1}+v_{2}$ \\
      & \\
    \end{tabular*}
  \end{center}
\end{table}



%      $[\mbox{MINUS}]$ & \infrule{s \vdash E_{1} \ra v_{1} \: s \vdash E_{2} \ra
%      v_{2}}{s \vdash E_{1} - E_{2} \ra v} & where $v=v_{1}-v_{2}$ \\
 %     $[\mbox{MULTIPLICATION}]$ & \infrule{s \vdash E_{1} \ra v_{1} \: s \vdash E_{2} \ra
 %     v_{2}}{s \vdash E_{1} * E_{2} \ra v} & where $v=v_{1}*v_{2}$ \\
 %     $[\mbox{DIVISION}]$ & \infrule{s \vdash E_{1} \ra v_{1} \: s \vdash E_{2} \ra
 %     v_{2}}{s \vdash E_{1} / E_{2} \ra v} & where $v=v_{1}/v_{2}$ \\
 %     $[\mbox{MODULO}]$ & \infrule{s \vdash E_{1} \ra v_{1} \: s \vdash E_{2} \ra
 %     v_{2}}{s \vdash E_{1} \% E_{2} \ra v} & where $ $ \\

\begin{table}[ht]
  \begin{center}
    \begin{tabular*}{\textwidth}{lc}
      $[\mbox{RULE}]$ & \infrule{\lag Something, Something \rag \ra
      something}{\lag something, something, something \rag \ra something } \\
    \end{tabular*}
  \end{center}
\end{table}


\subsection{Lists}

\begin{table}[ht]
  \begin{center}
    \begin{tabular*}{\textwidth}{lc}
      $[\mbox{RULE}]$ & \infrule{\lag Something, Something \rag \ra
      something}{\lag something, something, something \rag \ra something } \\
    \end{tabular*}
  \end{center}
\end{table}


\subsection{Variable lists}

\begin{table}[ht]
  \begin{center}
    \begin{tabular*}{\textwidth}{lc}
      $[\mbox{RULE}]$ & \infrule{\lag Something, Something \rag \ra
      something}{\lag something, something, something \rag \ra something } \\
    \end{tabular*}
  \end{center}
\end{table}


\subsection{Definitions}

\begin{table}[ht]
  \begin{center}
    \begin{tabular*}{\textwidth}{lc}
      $[\mbox{RULE}]$ & \infrule{\lag Something, Something \rag \ra
      something}{\lag something, something, something \rag \ra something } \\
    \end{tabular*}
  \end{center}
\end{table}

