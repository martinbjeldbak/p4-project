\section{Definitions}
\label{sec:definitions}
In this section we present how programs written in \productname{} can be
structured with definitions of constants and types.

We begin by presenting what a program can consist of and then we further specify
how these different definitions are built. We present constant definitions
followed by type definitions. We provide big-step semantics for type definitions
in \secref{sec:typedefinitions}.

\subsection{Program structure}

The outermost layer of a \productname{} program is a list of definitions:

\begin{ebnf}
\grule{program}{\grep{definition}}
\grule{definition}{constant\_def}
\galt{type\_def}
\end{ebnf}

A definition is either a constant definition or a type definition. So, when a
program has been run, we are left with a symbol table full of types and
constants.

As the above grammar shows, an empty program is valid in \productname{}, because
it is possible to have zero, one, or more definitions in the outermost layer of
the structuring of programs.

\subsection{Constant definitions}
\label{sec:constantdefinitions}

Functions are also constants and are defined with the following definition:

\begin{ebnf}
\grule{constant\_def}{\gter{define} \gcat constant \gcat \gopt{varlist} \gcat
\gter{=} \gcat \gcat expression}
\end{ebnf}

A function definition needs a list of formal parameters. Formal parameters which
are represented as $varlists$ are described with the following grammar:

\begin{ebnf}
\grule{varlist}{\gter{[} \gcat \gopt{variable \gcat \grep{\gter{,} \gcat
variable} \gcat \gopt{\gter{,} \gcat vars} \gor vars} \gcat \gter{]}}
\grule{vars}{\gter{...} \gcat variable}  
\end{ebnf}

Creating constants and functions outside of type definitions adds them to the
global scope, meaning that they are essentially accessible from anywhere in the
program (provided that they are not hidden by a constant within a type).

An example of a global function is as the following implementation of a function for
computing the greatest of two numbers, presented in \csref{functiondef.garry}.

\codesample{functiondef.garry}

Essentially this creates a constant in the global scope named \constant{max},
which when used, returns a value of type \type{Function}. Since function values
can also be created with lambda expressions (as described in
\secref{sec:lambdaexpressions}), the following constant definition is equivalent
to \csref{functiondef.garry}:

\codesample{functiondef.junta}

This equivalence only holds for global constants, since type constants/methods
are a bit more special as described in \secref{sec:typedefinitions}.

Constants and functions are useful for putting frequently used expressions or
values in one place.

The non-terminal $vars$ is used for variadic functions, another feature of
\productname{}. Variadic functions are functions with indefinite arity, meaning
they will accept any number of actual parameters. In \productname{} this is
supported for both lambda expressions and functions. The \csref{variadic.junta} 
is an example of two variadic functions.

\codesample{variadic.junta}

The use of the \texttt{"..."} terminal marks that the following variable
represents a list that holds all additional parameters passed to the function.
In \csref{variadic.junta}, calling the function \constant{last} with no parameters,
is possible and results in the formal parameter \variable{args} holding the
value of an empty list, \texttt{[]}. Additionally, \constant{last} can be called
with any number of parameters, which will then be appended to the list in
\variable{args}. In the second function, \constant{myMap}, at least one
actual parameter must be provided (since the variadic parameter is the second
one). But other than that, the parameter passing works in the same way as with
\constant{last}, in that the function accepts any number of parameters greater
than or equal to $1$. Some uses and results of the two functions are shown in
\csref{variadicuse2.junta} and \csref{variadicuse1.junta}.

\codesample{variadicuse2.junta}
\codesample{variadicuse1.junta}

One limit is that the variadic parameter (the one after \texttt{"..."}) must be
the last one in the list of formal parameters, and there can only be one. This
is expressed in the grammar.

\subsubsection{Scope rules for functions and constants}

Consider the \constant{max} function in \csref{functiondef.garry}. Its formal
parameters \variable{a} and \variable{b} only exist and are only available
within that function. The example in \csref{functioncall.garry} shows a call of the
\constant{max} function, which was presented in \csref{functiondef.junta}.

\codesample{functioncall.garry}

When called with the actual parameters $5$ and $23$, a new scope is created and
the actual parameters are assigned to the formal parameters \variable{a} and
\variable{b}, respectively. The body of the function (the if expression) is then
evaluated and the result is returned. When returning, the variables
\variable{a} and \variable{b} cease to exist. 

\subsection{Type definitions}
\label{sec:typedefinitions}

As described in the introduction to this chapter, types are a central part of
\productname{} and being able to define custom user types is essential when
creating board games.
The following grammar rules present how type definitions work in
\productname{} followed by the associated definitions which can be used within
a type definition.

\begin{ebnf}
\grule{type\_def}{\gter{type} \gcat type \gcat varlist \gcat
\gopt{\gter{extends} \gcat type \gcat list} \gopt{type\_body}}
\grule{type\_body}{\gter{\{} \gcat \grep{member\_def} \gcat \gter{\}}}
\grule{member\_def}{abstract\_def}
\galt{constant\_def}
\galt{data\_def}
\grule{abstract\_def}{\gter{define} \gcat \gter{abstract} \gcat constant \gcat
\gopt{varlist}} \grule{data\_def}{\gter{data} \gcat variable \gcat \gter{=}
\gcat expression}
\end{ebnf}

The simplest type that can be created is a type such as the example in
\csref{simplesttype.junta}.

\codesample{simplesttype.junta}

The type \type{A} is a type without any data, constructor parameters, constants,
or parent type. This type is truly useless, since it has no identity or
behaviour. It can however be instantiated, and instances of it can be compared
using the \texttt{==}, \texttt{!=}, and \texttt{is} operators. But since the
type has no identity in any way, all instances will be equal as presented in
\csref{simplesttypeuse.junta}.

\codesample{simplesttypeuse.junta}

\subsubsection{The constructor}

One way to add identity to objects, is with the constructor. In the previous
\csref{simplesttype.junta} of a very simple type, the constructor takes no parameters (the empty
parameter list \texttt{[]} after the type name). If we were to add some formal
parameters to the type definition, it could look like the
\csref{typedef1.junta}.

\codesample{typedef1.junta}

Now in order to instantiate \type{A}, we must provide the constructor with two
parameters as presented in \csref{typedef1use.junta}. 

\codesample{typedef1use.junta}

Notice how in the first line, the two objects are equal to each other, while in
the second line they are not. This means that we have successfully added
identity to the \type{A} type.

\subsubsection{The constants}

Constants within types are defined in the same way as constants outside of
types. The difference is that constants defined within a type can only be
accessed within that type (and inheriting types) or by using the dot-notation
outside of the type. This is illustrated in \csref{typedef2.junta}.

\codesample{typedef2.junta}

In \csref{typedef2.junta}, two constants are defined within type \type{A}:
\constant{b} and \constant{calculate}. \constant{b} is a simple
constant holding the value of \variable{b}, the constructor parameter, plus one.
It can be seen as a getter, since it makes the value of \variable{b} visible to
the outside. \constant{calculate} is a method that returns the sum of some
numbers. In order to call the constants contained within a type, we use the dot-notation 
on an object of that type illustrated in \csref{typedef2use.junta}.

\codesample{typedef2use.junta}

The variable \variable{obj} is assigned an instance of the \type{A} type. Using
the dot-notation, the method \constant{calculate} is called on the object.

\subsubsection{The data}

Another way to add identity to objects, is to add data fields to the type.
Unlike constants, data fields contain private semi-mutable data. In the
strictest sense, the data is still immutable, but by using the
\keyword{set} keyword, it is possible to clone the current object, and return a
new one with the selected data fields set to new values. The example in
\csref{typedef3.junta} shows a new version of type \type{A}, with a data field.

\codesample{typedef3.junta}

In this example, we define the data field \variable{att} with the default value
of $15$. Since data fields are not accessible from outside of the type, we must
define a getter, the constant \constant{att}, in order to make the value
visible. Using the \keyword{set} keyword, we can also define at setter, the
\constant{setAtt} method, which returns a new instance of \type{A} with
\variable{att} set to whatever parameter \constant{setAtt} was called with.

The example in \csref{typedef3use.junta} shows the use of this setter, to create a clone of an
instance of \type{A}.

\codesample{typedef3use.junta}

This time we create an instance of \type{A}, and then call \constant{setAtt}
on that instance, in order to get a new instance of \type{A} with \variable{att}
set to $2$. After that, the values are accessed using the getter. Note that
again the two objects are not equal, since the value of \variable{obj2} is
different than \variable{obj1}.

\subsubsection{The super type}

In \productname{} single inheritance is possible using the
\keyword{extends} keyword. When extending another type, that type's constructor
must be invoked with some parameters. The following complicated example
illustrated in \csref{typescope2.junta} shows how single inheritance can be used.

\codesample{typescope2.junta}

\productname{} doesn't have an \keyword{abstract} keyword for type definitions, only
for constant/function definitions. A type is implicitly marked as abstract if it has
unimplemented abstract members. In example presented in \csref{typescope2.junta}, 
the type \type{A} is abstract
since its member \constant{constantA} is abstract. Likewise, the type
\type{B} is abstract because it extends \type{A}, but doesn't implement the abstract
member of \type{A}. The type \type{C} on the other hand is not abstract, since it
implements the abstract constant. Abstract types can't be constructed, albeit the
constructor for an abstract type has to be used when extending the type (after the
\keyword{extends} keyword).

When instantiating type \type{C}, several things happen. First \type{C}'s parent
is instantiated using the actual parameters $2$ and $4$. Then \type{B}'s parent
is instantiated using the value of $\variable{b}\; \texttt{+}\; \variable{c}$. It is worth
noting that, when calling the parent constructor, it is not possible to access
constants or data from within the type. In \figref{fig:scopes} a
visualisation of the scopes of a type definition is shown.

\fig[width=8cm]{scopes}{The scopes of a type definition. Each scope is marked with a
  rectangle. A scope contained within another scope has access to the variables
of the parent scope.}

\subsubsection{Overriding}

\productname{} supports overriding and abstract constants/methods. Any constant
in the parents can be overridden in the subtype. It is however required that the
overriding constant has the same signature as the original. Since
\productname{} is dynamically typed, the only constraints are:
\begin{nlist}
\item If the original member is a constant (no formal parameter list after the name),
  then the overriding member must be a constant as well
\item If the original member is a method with $n$ formal parameters, then the
  overriding method must have $n$ formal parameters as well
\end{nlist}

So, unlike in the global scope, there is a small difference between constants and
functions (methods). For abstract members, the same constraints hold.

\subsubsection{Big-step semantics}

The semantics presented in \tableref{semantic:typedef} are the transition rules
for type definitions.  These type definitions have some optional arguments which
correspond with the written grammar for these definitions, and this is why there
are four transition rules described.

\input{billeder/tikz/typedef}




In the premises of the rules we present a $5$-tuple, a
\textbf{TypeValue} where $env_{T}$ is updated according to the rule. In
three of the four $5$-tuples we include the symbol $\varepsilon$, which
denotes that the given position of the symbol is an empty slot. This is
again due to the fact that we have some optional arguments.

The $5$-tuple is ordered as follows:

\begin{nlist}
\item $\mathbf{T_{1}}$ -- current type
  \item $\mathbf{X}$ -- current type's formal parameters
  \item $\mathbf{D_{M}}$ -- member definitions
  \item $\mathbf{L}$ -- super type's parameters
  \item $\mathbf{T_{2}}$ -- super type
\end{nlist}

Throughout the transition rules we use the $5$-tuple to update the type
environment, $env_T$.
