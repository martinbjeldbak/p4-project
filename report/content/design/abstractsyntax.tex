\section{Abstract syntax}
\label{sec:abstractsyntax}
Before we can describe the behaviour of programs written in \productname{} and
their lexical structure, we must first present the syntax of programs. At this
point, we are only interested in a notion of abstract syntax because we do not
need to concern ourselves with operator precedence and so forth.

\subsubsection{Syntactic categories}
The abstract syntax of programming languages is defined as follows\cite[p.
27]{tt-hh}:

\begin{dlist}
  \item A collection of syntactic categories
  \item For each syntactic category we have a finite set of formation rules
    that define how the elements of the given category can be built and combined
\end{dlist}

Table \ref{table:syn-cat} presents the syntactic categories of \productname{}.

\begin{table}[ht]
  \begin{center}
    \begin{tabular}{rl}
      \hline
      $n \in$ & \textbf{Integer}         \\
      $x \in$ & \textbf{Variable}        \\
      $s \in$ & \textbf{String}          \\
      $e \in$ & \textbf{Expression}      \\
      $p \in$ & \textbf{Pattern}         \\
      $i \in$ & \textbf{List}            \\
      $g \in$ & \textbf{VarList}         \\
      $y \in$ & \textbf{Coordinate}      \\
      $z \in$ & \textbf{Direction}       \\
      $C \in$ & \textbf{ConstantNames}   \\
      $T \in$ & \textbf{TypeNames}       \\
      $F \in$ & \textbf{FunctionConstants} \\
      $D_{G} \in$ & \textbf{GlobalDef}   \\
      $D_{M} \in$ & \textbf{MemberDef}   \\
      \hline
    \end{tabular}  
    \capt{The syntactic categories of \hspace{0.02cm} \productname{}.}
    \label{table:syn-cat}
  \end{center}
\end{table}

In \tableref{table:syn-cat} we have letters that define one syntactic
category. For instance we have $n \in \mathbf{Integer}$, which means that when
we see a $n$ in the formation rules this is actually an integer value. There
are two types of lists: \textbf{VarList} and \textbf{List}. These differ in that
a variable list is used as formal parameters for types and the list is the
sequence of parameters for a super type. 

Furthermore, we have defined two definitions: \textbf{GlobalDef} and
\textbf{MemberDef}. These are the set of definitions in their respective scopes.
E.g.\ in the global scope we can define different constants and types. In the
member scope we can also define different constants, and also the abstract
definitions and the notion of data definitions. This will become very apparent
in \secref{sec:definitions}, where we present the grammar of program structures.

\subsubsection{Definitions}
\label{sec:abstractdefinitions}

In this section we present the definitions which will be used throughout the
construction of semantics for \productname{}. In the following definitions we
use the syntactic categories presented in \tableref{table:syn-cat}. For some of
the definitions we define arbitrary members which will be used in the semantics
of the constructs of the language.

\textbf{Definition} (Type environment) \hspace{0.5cm} The set of type environments is the set of
partial functions from type names to type values:

\[
  \mathbf{EnvT} = \mathbf{TypeNames} \rightharpoonup \mathbf{TypeValues}
\]

An arbitrary member is defined as $env_{T} \in \mathbf{EnvT}$.


\textbf{Definition} (Constant environment) \hspace{0.5cm} The set of constant environments is the set
of partial functions from constant names to expressions and variable lists:

\[
  \mathbf{EnvC} = \mathbf{ConstantNames} \rightharpoonup \mathbf{Expression}
  \times \mathbf{VarList}
\]

An arbitrary member is defined as $env_{C} \in \mathbf{EnvC}$.

\textbf{Definition} (Variable environment) \hspace{0.5cm} The set of variable environments is the set
of partial functions from variables to values:

\[
  \mathbf{EnvV} = \mathbf{Variable} \rightharpoonup \mathbf{Values}
\]


An arbitrary member is defined as $env_{V} \in \mathbf{EnvV}$.

\textbf{Definition} (Values) \hspace{0.5cm} The set of values can contain many
different values, and is defined as follows:

\begin{align*}
 \mathbf{Values} = \mathbf{Integers} \cup \mathbf{Strings} \cup \mathbf{Lists}
 \cup \mathbf{Patterns} \cup \mathbf{Coordinates} \cup \mathbf{Directions}\\ 
 \cup\; \mathbf{TypeValues} \cup \mathbf{FunctionValues} \cup \mathbf{ObjectValues} 
 \cup \mathbf{Booleans}
\end{align*}

\textbf{Definition} (List values) \hspace{0.5cm} The values of lists is
defined as follows:

\[
  \mathbf{ListValues} = \mathbf{Integers} \times \mathbf{Elements}
\]

The length of a list is an arbitrary member of $l \in \mathbf{Integer}$.

\textbf{Definition} (List elements) \hspace{0.5cm} The set of elements in
lists is the set of partial functions from integers to values:

\[
  \mathbf{Elements} = \mathbf{Integers} \rightharpoonup \mathbf{Values}
\]

An arbitrary member is defined as $elem \in \mathbf{Elements}$.

\textbf{Definition} (Function values) \hspace{0.5cm} The set of function values
is defined as follows:
 
\[
  \mathbf{FunctionValues} = \mathbf{VarLists} \times \mathbf{Expressions} \times
  \mathbf{EnvV} \times \mathbf{EnvC}
\]

A function value consists of a variable list, an expression and the set of variable
and constant environments.

\textbf{Definition} (Type values) \hspace{0.5cm} The set of type values is
defined as follows:

\[
  \mathbf{TypeValues} = \mathbf{TypeNames} \times \mathbf{VarLists} \times
  \mathbf{D_{M}} \times \mathbf{List} \times \mathbf{TypeValues }
\]

An arbitrary member is defined as $t \in \mathbf{TypeValues}$.

A type value consists of a type name, a variable list, the member definitions,
the super type's paramters and the super type's type value. A type value is recursively defined since
$\mathbf{TypeValues}$ is defined in terms of itself. This is not a problematic
definition because these must be considered as values for two different types.
The $\mathbf{TypeValues}$ on the right-hand side is in fact the super type's set
of values whereas the $\mathbf{TypeValues}$ on the left-hand side is the current
type's set of values.  The list in the definition is in fact the super type's
parameters whereas the variable list is the current type's formal parameters.

\textbf{Definition} (Object values) \hspace{0.5cm} The set of object values (an instantiated $\mathbf{TypeValue}$) is
defined as follows:

\[
  \mathbf{ObjectValues} = \mathbf{TypeValues} \times \mathbf{EnvC} \times
  \mathbf{EnvV} \times \mathbf{ObjectValues}
\]

An object value consists of the set of type values, the set of constant
environment, the set of variable environments and the set of object values. This
is yet another recursively defined definition, because an object value can have a parent 
object value, this is the case when an object value extends another object value. 


\subsubsection{Formation rules}
Each syntactic category is used in one or more of the formation rules
presented in \figref{fig:form-rules}. The formation rules define the structure
of the members of the syntactic categories. 

Not all of the constituents of the formation rules are syntactic categories. We
for instance see different parentheses, forward slashes, different operators, and
words like $\texttt{this}$ and $\texttt{define}$. These are part of the
construction of the given formation rule. If they are omitted from a rule then
it is not valid in \productname{}.

\begin{figure}[ht]
  \begin{center}
    \begin{tabular}[ht]{r l}
      $e\; \mathbf{::=}$ & $n$ \for $x$ \for $s$ \for $y$ \for $z$
      \for $T$ \for $C$ \for $i$ \for $\texttt{-} e$ \for $\left(\;
      e\; \right)$ \for $\texttt{/}\; p\; \texttt{/}$ \for \\ & $e\;
      i$ \for $e\texttt{.}C$ \for $\texttt{\#}\; g\; \texttt{=>}\; e$
      \for $\texttt{if}\; e_0\; \texttt{then}\; e_1\; \texttt{else}\;
      e_2$ \for $\texttt{not}\; e$ \for $e_1\; F\; e_2$ \for \\ &
      $\texttt{let}\; x_{1}\; \texttt{=}\; e_{1},\; \ldots,\; x_{k}\;
      \texttt{=}\; e_{k}\; \texttt{in}\; e_{k+1}$ \for $\texttt{set}\;
      x_{1}\; \texttt{=}\; e_{1},\; \ldots,\; x_{k}\; \texttt{=}\;
      e_{k}$ \for $\texttt{this}$ \for $\texttt{super}$ \\

      $F\; \mathbf{::=}$ & \texttt{is} \for \texttt{and} \for
      \texttt{or} \for \texttt{==} \for $\texttt{!=}$ \for $\texttt{<}$ \for
      $\texttt{>}$ \for $\texttt{<=}$ \for $\texttt{>=}$ \for $\texttt{+}$
      \for $\texttt{-}$ \for $\texttt{*}$ \for \\ & $\texttt{/}$ \for
      $\texttt{\%}$ \\
      
      $i\; ::=$ & $\texttt{[} e_{1},\; \cdots, e_{k} \texttt{]}$ \\
      
      $g\; ::=$ & $\texttt{[} x_{1},\; \cdots,\; x_{k} \texttt{]}$ \for
      $\texttt{[} x_{1},\; \cdots,\; \dots x_{k} \texttt{]}$ \for $\texttt{[}
      \dots x \texttt{]}$ \\
      
      $D_{G}\; ::=$ & $\texttt{define}\; C\; \texttt{=}\; e\; D_{G}$ \for
      $\texttt{define}\; C\; g\; \texttt{=}\; e\; D_{G}$ \for $\texttt{type}\;
      T\; g\; D_{G}$ \for \\ 
      & $\texttt{type}\; T\; g\; \texttt{extends}\; T\; i\; D_{G}$ \for
      $\texttt{type}\; T\; g\; \left\{D_{M}\right\}\; D_{G}$ \for \\
      & $\texttt{type}\; T\; g\; \texttt{extends}\; T\; i\;
      \left\{D_{M}\right\} D_{G}$ \for $\varepsilon$ \\

      $D_{M} ::=$ & $\texttt{define}\; C\; \texttt{=}\; e\; D_{M}$ \for
      $\texttt{define}\; C\; g\; \texttt{=}\; e\; D_{M}$ \for $\texttt{define}\;
      \texttt{abstract}\; C\; D_{M}$ \for \\ 
      & $\texttt{define}\; \texttt{abstract}\; C\; g\; D_{M}$ \for
      $\texttt{data}\; x\; \texttt{=}\; e\; D_{M}$ \for $\varepsilon$ \\

    \end{tabular}  
    \capt{The formation rules for the syntactic categories of \hspace{0.02cm}
    \productname{}.}
    \label{fig:form-rules} 
  \end{center}
\end{figure}

The $\mathbf{::=}$ means that the left-hand side of the rule can be any one of
the $\for$-separated right-hand sides. Furthermore, we use ``$\cdots$'' to
illustrate a repetition of some element in the rule. We have also used the
slightly different ``$\dots$'' to illustrate the three dots that precede a
variable argument (vars) which we present in \secref{sec:constantdefinitions}.
It should be clear from the context which of the two is being used.

Epsilon denoted by $\varepsilon$ represents an empty definition.
