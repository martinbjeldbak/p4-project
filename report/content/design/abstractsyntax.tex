\subsection{Abstract syntax}
Before we can describe the behaviour of programs written in
\productname{}, we must first present the syntax of programs. At this point, we 
are only interested in a notion of abstract syntax because we do not need to
concern ourselves with operator precedence and so forth.

\subsubsection{Syntactic categories}
The abstract syntax of programming languages is defined as
follows\cite[pg. 27]{tt-hh}:

\begin{dlist}
  \item A collection of syntactic categories
  \item For each syntactic category we have a finite set of formulation rules
    that define how the elements of the given category can be built and
    combined
\end{dlist}

Table \ref{table:syn-cat} shows the syntactic cateogries for
\productname{}.

\begin{table}[ht]
  \begin{center}
    \begin{tabular}{rl}
      \hline
      $n \in$ & $\mathbf{Integer}$       \\
      $x \in$ & $\mathbf{Variable}$      \\
      $s \in$ & $\mathbf{String}$        \\
      $E \in$ & $\mathbf{Expression}$    \\
      $P \in$ & $\mathbf{Pattern}$       \\
      $L \in$ & $\mathbf{List}$          \\
      $X \in$ & $\mathbf{Variable list}$ \\
      $Y \in$ & $\mathbf{Coordinate}$    \\
      $Z \in$ & $\mathbf{Direction}$     \\
      $C \in$ & $\mathbf{Constant}$      \\
      $T \in$ & $\mathbf{Type}$          \\
      $D \in$ & $\mathbf{Definitions}$   \\
      \hline
    \end{tabular}  
    \capt{The syntactic categories of \productname{}.}
    \label{table:syn-cat}
  \end{center}
\end{table}

In \tableref{table:syn-cat} we have one letter that defines one syntactic
category. For instance we have $n \in \mathbf{Integer}$, which means that when
we see a $n$ in the formulation rules this is actually an integer value.

\subsubsection{Formulation rules}
Each syntactic category is used in one or more of the formulation rules
presented in \figref{fig:form-rules}. The $\mathbf{::=}$ means that the
left-hand side of the rule can be any one of the $\for$-separated right-hand
sides.

\begin{figure}[ht]
  \begin{center}
    \begin{tabular}[ht]{r p{10cm}}
      $E\; \mathbf{::=}$ & $n$ \for $x$ \for $s$ \for $Y$ \for $Z$ \for $T$ \for
      $C$ \for $L$ \for $\texttt{-} E$ \for $\texttt{(}\; E\; \texttt{)}$ \for
      $\texttt{/}\; P\; \texttt{/}$ \for $E_{1}\; L$ \for $E_{1}\texttt{.}C$
      \for $\texttt{\#}\; X\; \texttt{=>}\; E$ \for $\texttt{if}\; E_{1}\;
      \texttt{then}\; E_{2}\; \texttt{else}\; E3$ \for $E_{1}\; \texttt{is}\;
      E_{2}$ \for $\texttt{not}\; E$ \for $E_{1}\; \texttt{and}\; E_{2}$ \for
      $E_{1} \;\texttt{or}\; E_{2}$ \for $E_{1}\; \texttt{==}\; E_{2}$ \for
      $E_{1}\; \texttt{!=}\; E_{2}$ \for $E_{1}\; \texttt{<}\; E_{2}$ \for
      $E_{1}\; \texttt{>}\; E_{2}$ \for $E_{1}\; \texttt{<=}\; E_{2}$ \for
      $E_{1}\; \texttt{>=}\; E_{2}$ \for $E_{1}\; \texttt{+}\; E_{2}$ \for
      $E_{1}\; \texttt{-}\; E_{2}$ \for $E_{1}\; \texttt{*}\; E_{2}$ \for
      $E_{1}\; \texttt{/}\; E_{2}$ \for $E_{1}\; \texttt{\%}\; E_{2}$ \for
      \texttt{this} \for \texttt{super} \for $\texttt{let}\; x_{1}\;
      \texttt{=}\; E_{1},\; x_{2}\; \texttt{=}\; E_{2},\; \cdots,\; x_{k}\;
      \texttt{=}\; E_{k}\; \texttt{in}\; E_{k+1}$ \\
      $L\; ::=$ & $\texttt{[} E_{1},\; \cdots, E_{k} \texttt{]}$ \\
      $X\; ::=$ & $\texttt{[} x_{1},\; \cdots,\; x_{k} \texttt{]}$ \for
      $\texttt{[} x_{1},\; \cdots,\; \dots x_{k} \texttt{]}$ \for $\texttt{[}
      \dots x \texttt{]}$ \\
      $D\; ::=$ & $\texttt{define}\; C\; \texttt{=}\; E$ \for $\texttt{define}\;
      C\; X\; \texttt{=}\; E$ \for $\texttt{type}\; T\; X$ \for
      $\texttt{define}\; \texttt{abstract}\; C\; X$ \for $\texttt{define}\;
      \texttt{abstract}\; T\;  X\; \texttt{extends}\; T\;  L$ \for
      $\texttt{define}\; \texttt{abstract}\; T\;  X\; \texttt{extends}\; T\; L\;
      D$ \for $\texttt{define}\; \texttt{abstract}\; C$ 
    \end{tabular}  
    \capt{The formulation rules for the syntactic categories of \productname{}.}
    \label{fig:form-rules} 
  \end{center}
\end{figure}

In \figref{fig:form-rules} we have used ``$\cdots$'' to illustrate a repetition
of some element in the rule. We have also used the slightly different
$``\dots''$ to illustrate the three dots that precede a variable argument (vars)
which we present in \secref{sec:grammar}. It should be clear from the context which of the
two is beeing used.

