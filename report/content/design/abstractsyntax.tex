\section{Abstract syntax}

The abstract syntax is the interpreter or compiler's internal representation of a program. It is represented
as an abstract syntax tree.

This section should cover all aspects of our abstract syntax tree, and how it differs from the
parse tree.

\subsection{Program}
\begin{figure}[H]
\begin{center}
\begin{tikzpicture}[level/.style={sibling distance=40mm/#1}]
\node [square] {Program}
  child {node [square] (a) {Function definition} edge from parent[dashed];}
  child {node [square] (b) {Function definition} edge from parent[dashed];}
  child {node [square] {Game declaration}};
\path (a)--(b) node [midway] {$\cdots$};
\end{tikzpicture}
\end{center}
\capt{The abstract syntax for the program node.}
\label{ast:program}
\end{figure}

\subsection{Variable list}
\begin{figure}[H]\begin{center}\begin{tikzpicture}[level/.style={sibling distance=30mm/#1}]
\node [square] {Variable list}
  child {node [square] (a) {Variable} edge from parent[dashed];}
  child {node [square] (b) {Variable} edge from parent[dashed];};
\path (a)--(b) node [midway] {$\cdots$};
\end{tikzpicture}
\end{center}
\capt{The abstract syntax for the variable list node.}
\label{ast:variablelist}
\end{figure}

\subsection{Function definition}
\begin{figure}[H]\begin{center}\begin{tikzpicture}[level/.style={sibling distance=30mm/#1}]
\node [square] {Function definition}
  child {node [square] {Function}}
  child {node [square] {Variable list}}
  child {node [ellipse,draw] {\textit{Expression}}};
\end{tikzpicture}
\end{center}
\capt{The abstract syntax for the function definition node.}
\label{ast:funcdef}
\end{figure}

\subsection{Game declaration}
\begin{figure}[H]\begin{center}\begin{tikzpicture}[level/.style={sibling distance=30mm/#1}]
\node [square] {Game declaration}
  child {node [square] {Declaration struct}};
\end{tikzpicture}
\end{center}
\capt{The abstract syntax for the game declaration node.}
\label{ast:gamedecl}
\end{figure}

\subsection{declaration struct}
\begin{figure}[H]\begin{center}\begin{tikzpicture}[level/.style={sibling distance=30mm/#1}]
\node [square] {Declaration struct}
  child {node [square]  {Declaration}}
  child {node [square] (b) {Declaration} edge from parent[dashed];}
  child {node [square] (c) {Declaration} edge from parent[dashed];};
  
\path (b)--(c) node [midway] {$\cdots$};
\end{tikzpicture}
\end{center}
\capt{The abstract syntax for the declaration structure node.}
\label{ast:declstruct}
\end{figure}

\subsection{declaration}
\begin{figure}[H]\begin{center}\begin{tikzpicture}[level/.style={sibling distance=30mm/#1}]
\node [square] {Declaration}
  child {node [ellipse split,draw] {Keyword \nodepart{lower} Identifier}}
  child {node [ellipse,draw] {\textit{Structure}}};
\end{tikzpicture}
\end{center}
\capt{The abstract syntax for the declaration node.}
\label{ast:decl}
\end{figure}

\subsection{Assignment}
\begin{figure}[H]\begin{center}\begin{tikzpicture}
[level/.style={sibling distance=40mm},
level 1/.style={sibling distance = 39mm},
level 2/.style={sibling distance = 20mm}]

\node [square] (z) {Assignment}
  child {node [square,left of=b,xshift=-4cm] (a) {Variable}}
  child {node [ellipse,draw,left of=c,xshift=-4.5cm] (b) {\textit{Expression}}}
  child {node [square] (c) {Assignment} edge from parent[dashed]
  	child {node [square,xshift=-1cm] (x) {Variable} edge from parent[solid]}
  	child {node [ellipse,draw,solid,xshift=-1cm] (y) {\textit{Expression}} edge from parent[solid]}
  }
  child {node [square,xshift=-1cm] (d) {Assignment} edge from parent[dashed]
  	child {node [square,xshift=1cm] (o) {Variable} edge from parent[solid]}
  	child {node [ellipse,draw,solid,xshift=1cm] (p) {\textit{Expression}} edge from parent[solid]}
  }
  child {node [ellipse,draw,right of=d,xshift=1.5cm](e) {\textit{Expression}}};

\path (c)--(d) node [midway] {$\cdots$};
\end{tikzpicture}
\end{center}
\capt{The abstract syntax for the assignment node.}
\label{ast:assignment}
\end{figure}

\subsection{If expression}
\begin{figure}[H]\begin{center}\begin{tikzpicture}[level/.style={sibling distance=30mm/#1}]
\node [square] {If expression}
  child {node [ellipse,draw] {\textit{Expression}}}
  child {node [ellipse,draw] {\textit{Expression}}}
  child {node [ellipse,draw] {\textit{Expression}}};
\end{tikzpicture}
\end{center}
\capt{The abstract syntax for the if expression node.}
\label{ast:ifexpr}
\end{figure}

\subsection{Lambda expression}
\begin{figure}[H]\begin{center}\begin{tikzpicture}[level/.style={sibling distance=30mm/#1}]
\node [square] {Lambda expression}
  child {node [square] {Variable list}}
  child {node [ellipse,draw] {\textit{Expression}}};
\end{tikzpicture}
\end{center}
\capt{The abstract syntax for the lambda expression node.}
\label{ast:lambdaexpr}
\end{figure}

\subsection{List}
\begin{figure}[H]\begin{center}\begin{tikzpicture}[level/.style={sibling distance=30mm/#1}]
\node [square] {List}
  child {node [ellipse,draw] (a) {Element} edge from parent[dashed]}
  child {node [ellipse,draw] (b) {Element} edge from parent[dashed]};

\path (a)--(b) node [midway] {$\cdots$};
\end{tikzpicture}
\end{center}
\capt{The abstract syntax for the list node.}
\label{ast:list}
\end{figure}

\subsection{Pattern}
\begin{figure}[H]\begin{center}\begin{tikzpicture}[level/.style={sibling distance=40mm/#1}]
\node [square] {Pattern}
  child {node [square] {Pattern expression}}
  child {node [square] (b) {Pattern expression} edge from parent[dashed]}
  child {node [square] (c) {Pattern expression} edge from parent[dashed]};

\path (b)--(c) node [midway] {$\cdots$};
\end{tikzpicture}
\end{center}
\capt{The abstract syntax for the pattern node.}
\label{ast:pattern}
\end{figure}

\subsection{Pattern or-operator}
\begin{figure}[H]\begin{center}\begin{tikzpicture}[level/.style={sibling distance=40mm/#1}]
\node [square] {Pattern, or-operator}
  child {node [square] {Pattern value}}
  child {node [square] {Pattern expression}};
\end{tikzpicture}
\end{center}
\capt{The abstract syntax for the pattern or-operator node.}
\label{ast:pattern-or}
\end{figure}

\subsection{Pattern multiplier-operator}
\begin{figure}[H]\begin{center}\begin{tikzpicture}[level/.style={sibling distance=40mm/#1}]
\node [square] {Pattern, multiplier-operator}
  child {node [square] {Pattern value}};
\end{tikzpicture}
\end{center}
\capt{The abstract syntax for the pattern multiplier-operator node.}
\label{ast:patter-mult}
\end{figure}

\subsection{Pattern not-operator}
\begin{figure}[H]\begin{center}\begin{tikzpicture}[level/.style={sibling distance=40mm/#1}]
\node [square] {Pattern, not-operator}
  child {node [ellipse,draw] {\textit{Pattern check}}};
\end{tikzpicture}
\end{center}
\capt{The abstract syntax for the pattern not-operator node.}
\label{ast:pattern-not}
\end{figure}

\subsection{Not-operator}
\begin{figure}[ht]\begin{center}\begin{tikzpicture}[level/.style={sibling distance=40mm/#1}]
\node [square] {Not-operator}
  child {node [ellipse, draw] {\textit{Expression}}};
\end{tikzpicture}
\end{center}
\capt{The abstract syntax for the not-operator node.}
\label{ast:not}
\end{figure}