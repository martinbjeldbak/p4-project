\section{Predefined types and constants}
\label{sec:predefined}

In order to write programs in a programming language, it is often necessary to use a number of built-in
functions and types. \productname{} provides a number of built-in functions and constants, as well as 
a number of simple types for representing values such as integers and strings. \productname{} also provides
a type hierarchy designed for implementing and expressing board games. The built-ins will in many cases make
the lives of \productname{} programmers easier since they won't have to
implement the functionality the built-ins
provide for themselves.

Since \productname{} doesn't have a module, package or name space system, the distinction between a standard-
and game environment doesn't actually exist in the language, and all types and constants exist in the same
global name space. The distinction between the two is merely formal and based on the sort of types and functionality
that each provide.
 
\subsection{Standard environment}
\label{sec:standardenvironment}

We now introduce the standard environment of \productname{}. The standard environment provides the simple types, such as
integers, strings, boolean values, etc.\ and their related functions and constants for working with the these.

The following global constants are available:

\begin{dlist}
  \item \constdef{typeOf}{[\farg{value}{\opstar}]}{Type}\\
    A function that returns the type of any value.
  \item \constdef{union}{[\farg{list}{List}, ... \farg{lists}{List}]}{List}\\
    A function that returns the union of a number of lists.
  \item \constdef{true}{}{Boolean}\\
    The boolean true value.
  \item \constdef{false}{}{Boolean}\\
    The boolean false value.
\end{dlist}

\subsubsection{Integer}

\begin{dlist}
  \item \type{Integer}[\variable{integer} : \type{Integer}]\\
    The standard environment provides the \type{Integer} type, which is implemented as Java's primitive data type, integer. That is
it's a 32-bit signed two's complement integer. When the interpreter detects a numeral it returns an integer value object. If
for instance the numeral exceeds the highest possible value a \classref{TypeError} is thrown.
\end{dlist}

\subsubsection{Boolean}

\begin{dlist}
  \item \type{Boolean}[\variable{boolean} : \type{Boolean}]\\
    The standard environment provides the \type{Boolean} type, which is implemented as Java's primitive data type, boolean. That is, it only has two possible values: true and false. Even though the data type represents only one bit of information, according to the Java documentation, the ``size'' isn't precisely defined. 
\end{dlist}

\subsubsection{String}
\begin{dlist}
  \item \type{String}[\variable{string} : \type{String}]\\
    The standard environment provides the \type{String} type, which is implemented as Java's data type, \classref{String}. That is, it may contain any unicode (UTF-16) characters. Though it is not possible to write unicode characters of the form ``\textbackslash{}uXXXX'' as in Java (for instance ``\textbackslash{}u0108'', which is the capital C with circumflex, Ĉ). The \type{String} type contains one built-in constant:  
  \begin{dlist}
  \item \constant{size} : \type{Integer}\\
  The size constant returns the number of characters in the string, which is an integer value. For example ``test\_string''.size = $11$
  \end{dlist}
\end{dlist}
\subsubsection{List}

\begin{dlist}
  \item \type{List}[\variable{list} : \type{List}]\\
    The standard environment provides the \type{List} type. A list object can contain a mix of any types: strings, integers, other lists, game objects etc.
  This has both advantages and disadvantages. It increases the orthogonality of the programming language but it increases the risk of getting
  errors, which don't show until at run time. The List type is similar to the \classref{ArrayList} of Java and it's resizeable, which means that types can be added to the List. The type comes with a number of built-in constants and functions. 
  \item \constant{size} : \type{Integer}\\
  The size constant returns the number of elements in the list, which is an integer value. For example [``hi'', 2, 4].size = 3.
  \item \constant{sort}[\variable{comparator} : \type{Function}] : \type{List} \\
    The sort function sorts a list using a function that must take two parameters as input and return an integer value. For example \texttt{[1, 6, 2, 5, 4, 3].sort[\#[\$a, \$b] $=>$ if \$a > \$b then 1 else if \$a == \$b then 0 else -1]}. Will sort the list in ascending order. That is \texttt{[1, 2, 3, 4, 5, 6]}.
  \item \constant{map}[\variable{mapper} : \type{Function}] : \type{List} \\
    The map function maps each element of the list with a function of style \texttt{\#[\$a] => \$a}. The function must take one parameter. For example \texttt{[1, 2, 3, 4, 5, 6].map[\#[\$a] => \$a + 1]} will return the list: \texttt{[2, 3, 4, 5, 6, 7]}.
  \item \constant{filter}[\variable{filter} : \type{Function}] : \type{List} \\
    The filter function filters a list by feeding it with a function of style \texttt{\#[\$a] => \$a >= 5}, and returns a list with only the elements which comply with the function. The function fed to the filter function must take one parameter and return a boolean value. For example \texttt{[1, 2, 3, 4, 5, 6].filter[\#[\$a] => \$a >= 5]} will return \texttt{[5, 6]}. 
\end{dlist}

\subsubsection{Direction}
\begin{dlist}
  \item \type{Direction}[\variable{direction} : \type{Direction}]\\
  The standard environment provides the Direction type, which can be compared to a vector. There are eight different directions: n (north), s (south), w (west), e (east), nw, ne, sw, se. The type consist of an $x$ value and a $y$ value. For example n has value $y = 1$ and $x = 0$, s has value $y = -1$ and $x = 0$, w has value $y = 0$ and $x = -1$, etc.\ The Direction type is meant as a practical tool for use in patterns. 
\end{dlist}

\subsubsection{Coordinate}
\begin{dlist}
  \item \type{Coordinate}[\variable{coordinate} : \type{Coordinate}]\\
    The standard environment provides the Coordinate type. The Coordinate type is closely related to the Direction type in the way that it also consist of a $x$ value and an $y$ value. When the interpreter detects a number of capital letters followed by a number a numerals it returns a coordinate value object. Examples of coordinate values are \literal{A1}, \literal{Z99} and \literal{ABCD1234}. The coordinate value \literal{A1} corresponds to the $x$ value $1$ and $y$ value $1$, which is the top-left square on a board. The coordinate type is means as a practical tool to specify squares on a grid-formed board. Coordinate values must be positive, as negative $x$ and $y$ values make no sense representing coordinates off of the board.
\end{dlist}
\subsubsection{Type}
\begin{dlist}
  \item \type{Type}[\variable{type} : \type{Type}]\\
\end{dlist}
\subsubsection{Function}

\begin{dlist}
  \item \constdef{call}{[\farg{parameters}{List}]}{\opstar}\\
    Calls the function with the specified parameter list. 
\end{dlist}

\subsubsection{Pattern}

\subsection{Game environment}
\label{sec:gameenvironment}

The game environment provides a class hierarchy for describing a board game in an object-oriented manner.
In the game environment the following global functions are available:

\begin{dlist}
  \item \constant{addAction}[\variable{piece} : \type{Piece}, \variable{squares} : \type{List}] : \type{List}\\
    A function that returns a list of \type{AddAction}s to where it's possible to add a piece (\variable{piece}. The functions
    takes two parameters. The first parameter contains information on which type of piece the actions applies to. The second parameter is
    the list of squares where the type of piece can be added to.
    
    In the code example in \secref{codesample} in the beginning of the chapter, \function{addAction} is used in the following
    way: \\
    \begin{center}
    {addAction}[\variable{pieceType}[\keyword{this}], \variable{gameState}.\constant{board}.\constant{emptySquares}]
    \end{center}
    
    Here \constant{addAction} returns a list of empty squares to where it's possible to add a piece of the type \keyword{this}, which in this case was
    either a crosses piece or noughts piece depending whose turn it is.
    
  \item \constant{moveAction}[\variable{piece} : \type{Piece}, \variable{squares} : \type{List}] : \type{List}\\
  \constant{moveAction} works like \constant{addAction}, but instead of returning a \type{List} of \type{AddAction}s it returns a \type{List} of \type{MoveAction}s.
  	
    
\end{dlist}



\subsubsection{Game}
The \type{Game} type contains all information to describe a board game at a specific point at time.

\begin{dlist}
  \item \type{Game}[\variable{title} : \type{String}]\\
  Creates a instance of the \type{Game} type with a Game title of \variable{title}, \constant{board} set to \constant{initialBoard} and \constant{currentPlayer} set to \constant{turnOrder}[0].
  
  \item \constant{players} : \type{List}\\
  List of all \type{Player}s which are a part of this game.
  
  \item \constant{currentPlayer} : \type{Player}\\
  The \type{Player} from \constant{players} which currently have the turn.
  
  \item \constant{turnOrder} : \type{List}\\
  The order of \type{Player}s which determines in which order each \type{Player} from \constant{players} has their turn.
  
  \item \constant{initialBoard} : \type{Board}\\
  The value of \constant{board} at the beginning of each game.
  
  \item \constant{board} : \type{Board}\\
  The current state of a \type{Board} for this game.
  
  \item \constant{title} : \type{String}\\
  The title of the game which users can identify the game with.
  
  \item \constant{description} : \type{String}\\
  An short explanation of the game and/or its rules.
  
  \item \constant{matchSquare}[ \variable{position} : \type{Coordinate}, \variable{pattern} : \type{Pattern} ] : \type{Boolean}\\
  Is true if \variable{pattern} is valid for \variable{position}.
  
  \item \constant{matchSquares}[ \variable{positions} : \type{List}, \variable{pattern} : \type{Pattern} ] : \type{Boolean}\\
  Is true if and only if all \type{Coordinate}s in \variable{positions} is true for \constant{matchSquare} with \variable{pattern}.
  
  \item \constant{findSquares}[ \variable{pattern} : \type{Pattern} ] : \type{List}\\
  \type{List} of all \type{Square}s where its \constant{position} matches \variable{pattern}.
  
  \item \constant{findSquaresIn}[ \variable{positions} : \type{List}, \variable{pattern} : \type{Pattern} ] : \type{List}\\
  \type{List} of \type{Square}s where its \constant{position} matches \variable{pattern}, but only \type{Squares} which \type{Coordinate} exists in \variable{positions}.
  
  \item \constant{history} : \type{List}\\
  \type{List} of all applied \type{Action}s.
  
  \item \constant{applyAction}[ \variable{action} : \type{Action} ] : \type{Game}\\
  A \type{Game} where \constant{board} have been updated according to \variable{action} and where \variable{action} is appended to \constant{history}.
  
  \item \constant{undoAction}[ \variable{action} : \type{Action} ] : \type{Game}\\
  A \type{Game} where \constant{board} have been reset to its state before \type{Action} was applied and with \constant{history} updated accordantly.
  
  \item \constant{setHistory}[ \variable{history} : \type{List} ] : \type{Game}\\
  A \type{Game} where \constant{history} is equal to \variable{history}.
  
  \item \constant{setBoard}[ \variable{board} : \type{GridBoard} ] : \type{Game}\\
  A \type{Game} where \constant{board} is equal to \variable{board}.
  
  \item \constant{setCurrentPlayer}[ \variable{i} : \type{Integer} ] : \type{Game}\\
  A \type{Game} where \constant{currentPlayer} is \constant{turnOrder}[\variable{i}].
  
  \item \constant{nextTurn}[] : \type{Game}\\
  The \type{Player} which has the turn after \constant{currentPlayer}.
\end{dlist}

\subsubsection{Board}
\begin{dlist}
  \item \type{Board}[]\\
  A \type{Board} with no \type{Piece}s.
  
  \item \constant{pieces} : \type{List}\\
  A \type{List} containing all \type{Piece}s associated with the \type{Board}.
  
  \item \constant{setPieces}[ \variable{pieces} : \type{List} ] : \type{Board}\\
  A \type{Board} where \constant{pieces} is equal to \variable{pieces}.
\end{dlist}

\subsubsection{GridBoard}
\type{GridBoard} \keyword{extends} \type{Board} to provide an easy way to describe rectangular \type{Board}s.

\begin{dlist}
  \item \type{GridBoard}[ \variable{width} : \type{Integer}, \variable{height} : \type{Integer} ]\\
  A \type{GridBoard} with \constant{width} and \constant{height} being \variable{width} and \variable{height} respectively.
  
  \item \constant{width} : \type{Integer}\\
  The width of the rectangular \type{Board}.
  
  \item \constant{height} : \type{Integer}\\
  The height of the rectangular \type{Board}.
  
  \item \constant{squares} : \type{List}\\
  A \type{List} of all associated \type{Square}s.
  
  \item \constant{setSqaures}[ \variable{squares} : \type{List} ] : \type{GridBoard}\\
  A \type{GridBoard} where \constant{squares} is equal to \variable{squares}.
  
  \item \constant{addPiece}[ \variable{piece} : \type{Piece}, \variable{position} : \type{Coordinate} ] : \type{GridBoard}\\
  A \type{GridBoard} where \variable{piece} is appended to \constant{pieces} and added to the \type{Square} at \variable{position}.
  
  \item \constant{addPieces}[ \variable{piece} : \type{Piece}, \variable{positions} : \type{List} ] : \type{GridBoard}\\
  A \type{GridBoard} where \variable{piece} is appended to \constant{pieces} and added to all the \type{Square}s at any of \variable{positions}.
  
  \item \constant{removePiece}[ \variable{piece} : \type{Piece} ] : \type{GridBoard}\\
  A \type{GridBoard} where \variable{piece} is off-board.
  
  \item \constant{movePiece}[ \variable{piece} : \type{Piece}, \variable{position} : \type{Coordinate} ] : \type{GridBoard}\\
  A \type{GridBoard} where \variable{piece} (which is already contained in \constant{pieces}) is \constant{onBoard} and is only included in one \type{Square}'s \constant{pieces}.
  
  \item \constant{squareAt}[ \variable{position} : \type{Coordinate} ] : \type{Square}\\
  The \type{Square} at \variable{position} in the rectangular grid of \type{GridBoard}.
  
  \item \constant{setSqauresAt}[ \variable{square} : \type{Square}, \variable{position} : \type{List} ] : \type{Square}\\
  A \type{GridBoard} where \constant{squareAt}[ \variable{position} ] is equal to \variable{square}.
  
  \item \constant{isFull} : \type{Boolean}\\
  Is true if \constant{emptySquares}.size is 0.
  
  \item \constant{emptySquares} : \type{List}\\
  A \type{List} with \type{Square}s from \constant{squares} where \constant{isEmpty} is false.
  
  \item \constant{squareTypes} : \type{List}\\
  A \type{List} with default \type{Square}s which will be used to create a checkered pattern of \type{Square}s in the grid of \type{Square}s.
\end{dlist}

\subsubsection{Square}
\type{Square} describes a position on the \type{Board} where zero-to-many \type{Piece}s can be placed.

\begin{dlist}
	\item \type{Square}[]\\
	\type{Square} with no \type{Piece}s.
	
	\item \constant{position} : \type{Coordinate}\\
	\type{Coordinate} describing the position on a \type{GridBoard}.
	
	\item \constant{pieces} : \type{List}\\
	A \type{List} with \type{Piece}s located on this \type{Square}.
	
	\item \constant{addPiece}[ \variable{piece} : \type{Piece} ] : \type{Square}\\
	A \type{Square} where \variable{piece} is appended to \constant{pieces}.
	
	\item \constant{removePiece}[ \variable{piece} : \type{Piece} ] : \type{Square}\\
	A \type{Square} where \variable{piece} is not contained in \constant{pieces}.
	
	\item \constant{setPieces}[ \variable{pieces} : \type{List} ] : \type{Square}\\
	A \type{Square} where \constant{pieces} is equal to \variable{pieces}.
	
	\item \constant{image} : \type{String}\\
	Path to an image file used for visualizing the \type{Square}.
	
	\item \constant{isOccupied} : \type{Boolean}\\
	Is true if \constant{pieces}.\constant{size} is larger than 0.
	
	\item \constant{isEmpty} : \type{Boolean}\\
	Is true if \constant{pieces}.\constant{size} is 0.
	
	\item \constant{setPosition}[ \variable{position} : \type{Coordinate} ] : \type{Square}\\
	A \type{Square} where \constant{position} is equal to \variable{position}.
\end{dlist}

\subsubsection{Piece}
\type{Piece} describes an item associated to a \type{Player} which the \type{Player} can manipulate in order to progress the game.

\begin{dlist}
  \item \type{Piece}[ \variable{owner} : \type{Player} ]\\
  \type{Piece} with \constant{owner} set to \variable{owner}.
  
  \item \constant{owner} : \type{Player}\\
  \type{Player} which owns this \type{Piece}.
  
  \item \constant{image} : \type{String}\\
  Path to an image file used for visualizing the \type{Piece}.
  
  \item \constant{position} : \type{Coordinate}\\
  \type{Coordinate} for the \type{Square} this \type{Piece} is located on.
  
  \item \constant{move}[ \variable{position} : \type{Coordinate} ] : \type{Piece}\\
  A \type{Piece} with \constant{position} set to \variable{position} and \constant{onBoard} set to true.
  
  \item \constant{remove}[] : \type{Piece}\\
  A \type{Piece} where \constant{position} is invalid and \constant{onBoard} is false.
  
  \item \constant{onBoard} : \type{Boolean}\\
  Is true if \type{Piece} is on the \type{GridBoard}.
  
  \item \constant{actions}[ \variable{game} : \type{Game} ] : \type{List}\\
  A \type{List} of possible \type{Action}s the \type{Piece} can make on its \constant{owner}'s turn.
\end{dlist}

\subsubsection{Player}
\begin{dlist}
  \item \type{Player}[ \variable{name} : \type{String} ]\\
  \type{Player} with \constant{name} set to \variable{name}
  
  \item \constant{name} : \type{String}\\
  The name of the \type{Player}.
  
  \item \constant{winCondition}[ \variable{game} : \type{Game} ] : \type{Boolean}\\
  Is true if the \type{Player} has won at the ending of this turn.
  
  \item \constant{tieCondition}[ \variable{game} : \type{Game} ] : \type{Boolean}\\
  Is true if the game ended without a winner.
  
  \item \constant{actions}[ \variable{game} : \type{Game} ] : \type{List}\\
  A \type{List} of \type{Action}s that the \type{Player} can do during his turn.
\end{dlist}

\subsubsection{Action}
\begin{dlist}
  \item \type{Action}[]\\
  Empty \type{Action}.
\end{dlist}

\subsubsection{UnitAction}
\type{UnitAction} \keyword{extends} \type{Action} to provide a basic change to be performed on \type{Game}.

\begin{dlist}
  \item \type{UnitAction}[ \variable{piece} : \type{Piece} ]\\
  A \type{UnitAction} with \constant{piece} set to \variable{piece}.
  
  \item \constant{piece} : \type{Piece}\\
  The \type{Piece} this \type{UnitAction} affects.
\end{dlist}

\subsubsection{AddAction}
\type{UnitAction} \keyword{extends} \type{Action} to add a \type{Piece} to a \type{Game}.

\begin{dlist}
  \item \type{AddAction}[ \variable{piece} : \type{Piece}, \variable{to} : \type{Square} ]\\
  An \type{AddAction} which adds \variable{piece} to \variable{to}.
  
  \item \constant{to} : \type{Square}\\
  \type{Square} to add \constant{piece} to.
\end{dlist}

\subsubsection{RemoveAction}
\type{UnitAction} \keyword{extends} \type{Action} to remove a \type{Piece} from a \type{Game}.

\begin{dlist}
  \item \type{RemoveAction}[ \variable{piece} : \type{Piece} ]\\
  A \type{RemoveAction} which removes \variable{piece}.
\end{dlist}

\subsubsection{MoveAction}
\type{UnitAction} \keyword{extends} \type{Action} to move a \type{Piece} to another \type{Square}.

\begin{dlist}
  \item \type{MoveAction}[ \variable{piece} : \type{Piece}, \variable{to} : \type{Square} ]\\
  A \type{MoveAction} which moves \variable{piece} to \variable{to}.
  
  \item \constant{to} : \type{Square}\\
  \type{Square} to add \constant{piece} to.
\end{dlist}

\subsubsection{ActionSequence}
\type{ActionSequence} \keyword{extends} \type{Action} to provide a sequence of \type{UnitAction}s to be performed in order.

\begin{dlist}
  \item \type{ActionSeqence}[ \ldots \variable{actions} : \type{UnitAction} ]\\
  \type{ActionSequence} with \constant{actions} set to [ \ldots \variable{actions} ].
  
  \item \constant{actions} : \type{List}\\
  A \type{List} of \type{UnitAction} to be performed in order.
  
  \item \constant{addAction}[ \variable{action} : \type{UnitAction} ] : \type{ActionSequence}\\
  A \type{ActionSequence} where \variable{action} is appended to constant{actions}.
\end{dlist}

\subsubsection{TestCase}
An abstract type for unit testing.
