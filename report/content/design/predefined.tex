\section{Predefined types and constants}
\label{sec:predefined}

In order to write programs in a programming language, it is often neccessary to use a number of built-in
functions and types. \productname{} provides a number of built-in functions and constants, as well as 
a number of simple types for representing values such as integers and strings. \productname{} also provides
a type-hierachy designed for implementing and expressing board games.

Since \productname{} doesn't have a module, package or namespace system, the distinction between a standard-
and game environment doesn't actually exist in the language, and all types and constants exist in the same
global namespace.

\subsection{Standard environment}

The standard environment of \productname{} provides the simple types, and related methods, functions and
constants for working with the simple types.

The following global constants are available:

\begin{dlist}
  \item \constdef{typeOf}{[\farg{value}{\opstar}]}{Type}\\
    A function that returns the type of any value.
  \item \constdef{union}{[\farg{list}{List}, ... \farg{lists}{List}]}{List}\\
    A function that returns the union of a number of lists.
  \item \constdef{true}{}{Boolean}\\
    The boolean true value.
  \item \constdef{false}{}{Boolean}\\
    The boolean false value.
\end{dlist}

\subsubsection{Integer}

\subsubsection{Boolean}

\subsubsection{String}

\subsubsection{List}

\subsubsection{Direction}

\subsubsection{Coordinate}

\subsubsection{Type}

\subsubsection{Function}

\begin{dlist}
  \item \constdef{call}{[\farg{parameters}{List}]}{\opstar}\\
    Calls the function with the specified parameter list. 
\end{dlist}

\subsubsection{Pattern}

\subsection{Game environment}

The game environment provides a class hierachy for describing a board game in an object-oriented manner.

\subsubsection{Game}

\subsubsection{Board}

\subsubsection{GridBoard}

\subsubsection{Square}

\subsubsection{Piece}

\subsubsection{Player}

\subsubsection{Action}

\subsubsection{ActionSequence}

\subsubsection{UnitAction}

\subsubsection{AddAction}

\subsubsection{MoveAction}

\subsubsection{RemoveAction}

\subsubsection{TestCase}
