\section{Abstract syntax tree}

The abstract syntax is the interpreter's or compiler's internal representation of a program. It is represented as an abstract syntax tree which is a series of nodes and leaves connected forming a so-called tree.

This section covers all aspects of our abstract syntax tree (AST), \todo{and how it differs from the parse tree(?)}. When a piece of code is parsed by a parser that understands the specific programming language, the output of the parser will be a parse tree which corresponds to the abstract syntax of each program part. Each program part is represented with an abstract syntax tree within its respective subsection within this section.

\subsection{Program}
Every program written in \productname{} begins with an abstraction which we call ``program'' that consists of either zero or more function definitions followed by a single game declaration. The production for this rule is a follows:

\begin{ebnf}
\grule{program}{\grep{function\_def} \gcat game\_decl}
\end{ebnf}

It is from this production each and every program is derived from. The AST for this abstraction is illustrated in \figref{ast:program}.


\begin{figure}[ht]
\begin{center}
\begin{tikzpicture}[level/.style={sibling distance=40mm/#1}]
\node [square] {Program}
  child {node [square,xshift=0.5cm] (a) {\textit{Definition}} edge from parent[dashed];}
  child {node [square,xshift=-0.5cm] (b) {\textit{Definition}} edge from parent[dashed];};
\path (a)--(b) node [midway] {$\cdots$};
\end{tikzpicture}
\end{center}
\capt{The abstract syntax tree for the program node.}
\label{ast:program}
\end{figure}


Figure \ref{ast:program} consists of one root which is called ``Program'' and this root has three children, two called ``Function definition'' and one called ``Game declaration''. The nodes named ``Function definition'' are optional because the production says that a program can begin with either zero or more of these abstractions. The AST illustrates this by making the connecting lines dashed. Within these two nodes there are three dots ($\cdots$) which illustrate that it is possible to have more of these abstractions following each other.

This means that a program can begin with either a function or a game declaration. The following section will illustrate the abstraction of a function definition which is part of the program abstraction.

\subsection{Function definition}

\begin{ebnf}
\grule{function\_def}{\gter{define} \gcat function \gcat varlist \gcat expression}
\end{ebnf}


\begin{figure}
\begin{center}
\begin{tikzpicture}[level/.style={sibling distance=30mm/#1}]
\node [square] {Function definition}
  child {node [square,xshift=0.9cm] {Function}}
  child {node [square] {Variable list}}
  child {node [ellipse,draw,xshift=-0.5cm] {\textit{Expression}}};
\end{tikzpicture}
\end{center}
\capt{The abstract syntax tree for the function definition node.}
\label{ast:funcdef}
\end{figure}


\subsection{Game declaration}

\begin{ebnf}
\grule{game\_decl}{\gter{game} \gcat declaration\_struct}
\end{ebnf}

\begin{figure}[ht]
\begin{center}
\begin{tikzpicture}[level/.style={sibling distance=30mm/#1}]
\node [square] {Game declaration}
  child {node [square] {Declaration struct}};
\end{tikzpicture}
\end{center}
\capt{The abstract syntax tree for the game declaration node.}
\label{ast:gamedecl}
\end{figure}


\subsection{Declaration struct}

\begin{ebnf}
\grule{declaration\_struct}{\gter{\{} \gcat declaration \gcat \grep{declaration} \gter{\}}}
\end{ebnf}


\begin{figure}
\begin{center}
\begin{tikzpicture}[level/.style={sibling distance=30mm/#1}]
\node [square] {Declaration struct}
  child {node [square, xshift=0.5cm]  {Declaration}}
  child {node [square] (b) {Declaration} edge from parent[dashed];}
  child {node [square] (c) {Declaration} edge from parent[dashed];};
  
\path (b)--(c) node [midway] {$\cdots$};
\end{tikzpicture}
\end{center}
\capt{The abstract syntax tree for the declaration structure node.}
\label{ast:declstruct}
\end{figure}


\subsection{Declaration}


\begin{figure}
\begin{center}
\begin{tikzpicture}[level/.style={sibling distance=30mm/#1}]
\node [square] {Declaration}
  child {node [ellipse split,draw] {Keyword \nodepart{lower} Identifier}}
  child {node [ellipse,draw] {\textit{Structure}}};
\end{tikzpicture}
\end{center}
\capt{The abstract syntax tree for the declaration node.}
\label{ast:decl}
\end{figure}


\subsection{Variable list}

\begin{ebnf}
\grule{varlist}{\gter{[} \gcat \gopt{variable \gcat \grep{\gter{,} \gcat variable}} \gcat \gter{]}}
\end{ebnf}


\begin{figure}
\begin{center}
\begin{tikzpicture}[level/.style={sibling distance=30mm/#1}]
\node [square] {Variable list}
  child {node [square, xshift=0.5cm] (a) {Variable} edge from parent[dashed];}
  child {node [square] (b) {Variable} edge from parent[dashed];}
  child {node [square, xshift=-0.5cm] {Variable arguments} edge from parent[dashed];};
\path (a)--(b) node [midway] {$\cdots$};
\end{tikzpicture}
\end{center}
\capt{The abstract syntax tree for the variable list node.}
\label{ast:variablelist}
\end{figure}


\subsection{Assignment}

\begin{ebnf}
\grule{assignment}{\gter{let} \gcat variable \gcat \gter{=} \gcat expression \gcat \grep{\gter{,} \gcat variable \gcat \gter{=} \gcat expression} \gcat \gter{in} \gcat expression}
\end{ebnf}


\begin{figure}
\begin{center}
\begin{tikzpicture}
[level/.style={sibling distance=40mm},
level 1/.style={sibling distance = 39mm},
level 2/.style={sibling distance = 20mm}]

\node [square] (z) {Assignment}
  child {node [square,left of=b,xshift=-4cm] (a) {Variable}}
  child {node [ellipse,draw,left of=c,xshift=-4.5cm] (b) {\textit{Expression}}}
  child {node [square] (c) {Assignment} edge from parent[dashed]
  	child {node [square,xshift=-1cm] (x) {Variable} edge from parent[solid]}
  	child {node [ellipse,draw,solid,xshift=-1cm] (y) {\textit{Expression}} edge from parent[solid]}
  }
  child {node [square,xshift=-1cm] (d) {Assignment} edge from parent[dashed]
  	child {node [square,xshift=1cm] (o) {Variable} edge from parent[solid]}
  	child {node [ellipse,draw,solid,xshift=1cm] (p) {\textit{Expression}} edge from parent[solid]}
  }
  child {node [ellipse,draw,right of=d,xshift=1.5cm](e) {\textit{Expression}}};

\path (c)--(d) node [midway] {$\cdots$};
\end{tikzpicture}
\end{center}
\capt{The abstract syntax for the assignment node.}
\label{ast:assignment}
\end{figure}


\subsection{If expression}

\begin{ebnf}
\grule{if\_expr}{\gter{if} \gcat expression \gcat \gter{then} \gcat expression \gcat \gter{else} \gcat expression}
\end{ebnf}


\begin{figure}
\begin{center}
\begin{tikzpicture}[level/.style={sibling distance=30mm/#1}]
\node [square] {If expression}
  child {node [label={[xshift=1.2cm, yshift=0.2cm]\textbf{if}}] [ellipse,draw] {\textit{Expression}}}
  child {node [label={[xshift=-0.38cm, yshift=0.2cm]\textbf{then}}] [ellipse,draw] {\textit{Expression}}}
  child {node [label={[xshift=-2.2cm, yshift=0.2cm]\textbf{else}}] [ellipse,draw] {\textit{Expression}}};
\end{tikzpicture}
\end{center}
\capt{The abstract syntax tree for the if expression node.}
\label{ast:ifexpr}
\end{figure}


\subsection{Lambda expression}

\begin{ebnf}
\grule{lambda\_expr}{\gter{\#} \gcat varlist \gcat \gter{=>} \gcat expression}
\end{ebnf}

\begin{table}[ht]
  \begin{tabular*}{\textwidth}{l l c}
    \hline \\
    \hspace{0.5cm} $[\mbox{LAMBDA}]$ & $env_{T}, env_{C}, env_{V} \vdash \lag
    \texttt{\#} \; g\;
    \texttt{=>}\; e \rag \ra v$ & \hspace{1cm} where $v = \left(g, e, env_{V}, env_{C}\right)$ \\
    & & \\
    \hline
  \end{tabular*}
  \capt{Transition rules for lambda expressions.}
  \label{semantic:lambda}
\end{table}



\subsection{List}

\begin{ebnf}
\grule{list}{\gter{[} \gcat \gopt{expression \gcat \grep{\gter{,} \gcat expression}} \gcat \gter{]}}
\end{ebnf}


\begin{figure}
\begin{center}
\begin{tikzpicture}[level/.style={sibling distance=30mm/#1}]
\node [square] {List}
  child {node [ellipse,draw] (a) {\textit{Element}} edge from parent[dashed]}
  child {node [ellipse,draw] (b) {\textit{Element}} edge from parent[dashed]};

\path (a)--(b) node [midway] {$\cdots$};
\end{tikzpicture}
\end{center}
\capt{The abstract syntax for the list node.}
\label{ast:list}
\end{figure}


\subsection{Pattern}

\begin{ebnf}
\grule{pattern}{pattern\_expr \gcat \grep{pattern\_expr}}
\end{ebnf}


\begin{figure}[ht]
\begin{center}
\begin{tikzpicture}[level/.style={sibling distance=40mm/#1}]
\node [square] {Pattern}
  child {node [square,xshift=0.5cm] {\textit{Pattern expression}}}
  child {node [square] (b) {\textit{Pattern expression}} edge from parent[dashed]}
  child {node [square,xshift=-0.5cm] (c) {\textit{Pattern expression}} edge from parent[dashed]};

\path (b)--(c) node [midway] {$\cdots$};
\end{tikzpicture}
\end{center}
\capt{The abstract syntax tree for the pattern node.}
\label{ast:pattern}
\end{figure}


\subsection{Pattern or-operator}


\begin{figure}[ht]
\begin{center}
\begin{tikzpicture}[level/.style={sibling distance=40mm/#1}]
\node [square] {Pattern, or-operator}
  child {node [ellipse,draw,xshift=0.2cm] {\textit{Pattern value}}}
  child {node [ellipse,draw,xshift=-0.2cm] {\textit{Pattern expression}}};
\end{tikzpicture}
\end{center}
\capt{The abstract syntax tree for the pattern or-operator node.}
\label{ast:pattern-or}
\end{figure}


\subsection{Pattern multiplier-operator}


\begin{figure}
\begin{center}
\begin{tikzpicture}[level/.style={sibling distance=40mm/#1}]
\node [square] {Pattern, multiplier-operator}
  child {node [square] {Pattern value}};
\end{tikzpicture}
\end{center}
\capt{The abstract syntax for the pattern multiplier-operator node.}
\label{ast:patter-mult}
\end{figure}


\subsection{Pattern not-operator}


\begin{figure}[ht]
\begin{center}
\begin{tikzpicture}[level/.style={sibling distance=40mm/#1}]
\node [square] {Pattern, not-operator}
  child {node [ellipse,draw] {\textit{Pattern check}}};
\end{tikzpicture}
\end{center}
\capt{The abstract syntax tree for the pattern not-operator node.}
\label{ast:pattern-not}
\end{figure}


\subsection{Not-operator}


\begin{figure}
\begin{center}
\begin{tikzpicture}[level/.style={sibling distance=40mm/#1}]
\node [square] {Not-operator}
  child {node [ellipse, draw] {\textit{Expression}}};
\end{tikzpicture}
\end{center}
\capt{The abstract syntax for the not-operator node.}
\label{ast:not}
\end{figure}

\subsection{Operator}


\begin{figure}[ht]
\begin{center}
\begin{tikzpicture}[level/.style={sibling distance=25mm/#1}]
\node [square] {Operator}
  child {node [ellipse,draw] {\textit{Element}}}
  child {node [ellipse,draw] {\textit{Expression}}};
\end{tikzpicture}
\end{center}
\capt{The abstract syntax tree for the operator node.}
\label{ast:operator}
\end{figure}

