\section{Expressions}
\label{sec:expressions}

Throughout this section we will present all the expressions that are included in
\productname{}. We have provided big-step semantics for six central expressions
in this section. These are the ones that we have deemed most necessary and they
are not as any other construct. Which sections that include semantics will be
explained in the following.

The smallest parts of expressions are presented in
\secref{sec:atomicexpressions}. In \secref{sec:lists} the notion of
lists is presented and explained. Here we present big-step semantics. Then we
present the let expressions in \secref{sec:letexpressions}. Also here we present
big-step semantics. Afterwards, we present the conditional expressions in 
\secref{sec:conditionalexpressions} followed by a lambda expressions in
\secref{sec:lambdaexpressions}. We provide big-step semantics for lambda
expressions. Furthermore, we present the concept of a set expression in
\secref{sec:setexpressions}. We also supply big-step semantics for set
expressions. Lastly, we present the different operators and calls in
\secref{sec:operatorsandcalls}. Here we provide big-step semantics for function
calls and for member access.

We begin here by listing the main expressions below this
paragraph:

\begin{ebnf}
%Expressions
\grule{expression}{let\_expr}
\galt{if\_expr}
\galt{set\_expr}
\galt{lambda\_expr}
\galt{\gter{not} \gcat expression}
\galt{lo\_sequence}
\end{ebnf}

Two statements hold about expressions in \productname{}:

\begin{nlist}
\item An expression \textbf{always} has a value.
\item An expression \textbf{cannot} have side effects.
\end{nlist}

\subsection{Atomic expressions}
\label{sec:atomicexpressions}

These are the smallest possible parts of expressions in \productname{}, defined by
the rule:

\begin{ebnf}
\grule{atomic}{\gter{(} \gcat expression \gcat \gter{)}}
\galt{constant}
\galt{type}
\galt{variable}
\galt{\gter{this}}
\galt{\gter{super}}
\galt{integer}
\galt{string}
\galt{direction}
\galt{coordinate}
\galt{\gter{/} \gcat pattern \gcat \gter{/}}
\galt{list}
\end{ebnf}

The first atomic expression is $\gter{(} \gcat expression \gcat \gter{)}$, which means
that it is possible to embed expressions within other expressions, and manually control
the precedence of operations. Consider the following two expressions:

\codesample{parentheses1.junta}
\codesample{parentheses2.junta}

In \csref{parentheses2.junta} the parentheses are in fact unnecessary because the
\texttt{*} operator has precedence over the \texttt{+} operator (see
\secref{sec:operatorsandcalls}).

Names (constants, types, and variables) are also atomic expressions, and are evaluated
to whatever value they are associated with, based on the current scope. The keywords
($\gter{this}$ and $\gter{super}$) are atomic, but only applicable within type
definitions, where $\gter{this}$ refers to the current object and $\gter{super}$
refers to the current object casted to its parent type (if it has one).

The literals ($integer$, $string$, $direction$, and $coordinate$) are evaluated to their
respective values, while patterns are evaluated to \type{Pattern} values according to
the grammar in \secref{sec:patterns} and lists are evaluated to \type{List} values
according to the grammar in \secref{sec:lists}.

\subsection{Lists}
\label{sec:lists}

The \type{List} type is one of the basic types in \productname{}. A \type{List} value is
created using the following syntax:

\begin{ebnf}
\grule{list}{\gter{[} \gcat \gopt{expression \gcat \grep{\gter{,} \gcat
expression}} \gcat \gter{]}}
\end{ebnf}

Essentially this means that a list is created from zero, one, or more expressions (separated
by commas). The following statements can be made about lists:

\begin{nlist}
\item A list can be empty: \texttt{[]}
\item Lists begin at offset $0$
\item Lists are ordered ($\texttt{[1, 2]} \ne \texttt{[2, 1]}$)
\item Lists are immutable
\end{nlist}

When a list is evaluated, each expression is evaluated to a value (the order of
evaluation does not matter, since no side-effects are possible, lazy-evaluation
of expressions could even be a possibility), and all values (in the same order
as the expressions) are added to the resulting \type{List} value. When a
\type{List} value is created, it can't be altered further (because of the
no-side-effects condition). All operations on that \type{List} value will create
new \type{List} values, and leave the original value intact.

Values within lists can be accessed using the \texttt{[]} operation (same as
with function calls and type instantiation). Because lists begin at offset $0$, in
order to access the second element of a list, one write as illustrated in
\csref{listaccess1.junta}.

\codesample{listaccess1.junta}

Ranges of elements can also be returned. For instance in
\csref{listaccess2.junta} a new list is returned containing elements from offset 1 up
to and including offset 2 (the list $\texttt{["is", "a"]}$).

\codesample{listaccess2.junta}

An offset can be negative, which means that the offset is dependent on
the size of the list. This way offset $-1$ will always refer to the last element
of the list, because the length of the list minus $1$ will give the offset of
the last element. For example in order to return the last two elements of a list one
could use the offsets, $-2$ and $-1$:

\codesample{listaccess3.junta}

Some other examples are presented in \csref{listaccess4.junta}:

\codesample{listaccess4.junta}

\subsubsection{Big-step semantics}

The semantics presented in \tableref{semantic:lists} are the transition rules
for lists.

\begin{table}[ht]
  \begin{center}
    \begin{tabular*}{\textwidth}{l p{\textwidth}}
      \hline \\
      \hspace{0.5cm} $[\mbox{LIST}_{\mbox{ACCESS-1}}]$ & \infrule{env_{T},
      env_{C}, env_{V} \vdash \lag e\; \rag \ra v_2  \qquad env_{T}, env_{C},
      env_{V} \vdash \lag i \rag \ra v_3}
      {env_{T}, env_{C}, env_{V} \vdash \lag e\; i \rag \ra v_1} \\
       & where $v_2 = \left(l_1, elem_1\right)$ \\
       & and $v_3 = (l_2,elem_2)$ \\
       & and $l_2 = 1$ \\
       & and $d = elem_2\; 0$ \vspace{0.1cm} \\
       & and $v_1 = \left\{
	 \begin{array}{l l}
           elem_1\; d         & \quad \text{if $d \geq 0$}\\
           elem_1\; (l_1 + d) & \quad \text{if $d < 0$}
	 \end{array} \right.$ \\
	 
      \hspace{0.5cm} $[\mbox{LIST}_{\mbox{ACCESS-2}}]$ & \infrule{env_{T},
      env_{C}, env_{V} \vdash \lag e \rag \ra v_2  \qquad env_{T}, env_{C},
      env_{V} \vdash \lag i \rag \ra v_3}
      {env_{T}, env_{C}, env_{V} \vdash \lag e\; i \rag \ra v_1} \\
       & where $v_2 = \left(l_1, elem_1\right)$ \\
       & and $v_3 = (l_2,elem_2)$ \\
       & and $l_2 = 2$ \vspace{0.1cm} \\
       & and $d = \left\{
	 \begin{array}{l l}
           elem_2\; 0         & \quad \text{if $elem_2\; 0 \geq 0$}\\
           l_1 + elem_2\; 0   & \quad \text{if $elem_2\; 0 < 0$}
	 \end{array} \right.$ \vspace{0.1cm} \\
       & and $j = \left\{
	 \begin{array}{l l}
           elem_2\; 1         & \quad \text{if $elem_2\; 1 \geq 0$}\\
           l_1 + elem_2\; 1   & \quad \text{if $elem_2\; 1 < 0$}
	 \end{array} \right.$ \vspace{0.1cm} \\
       & and $elem_3\; z = \left\{
	 \begin{array}{l l}
           elem_1\; d+1       & \quad \text{if $z = 0$}\\
	   \hspace{1cm} \vdots &   \\
           elem_1\; d+n-1     & \quad \text{if $z = n-1$}
	 \end{array} \right.$ \vspace{0.1cm} \\
       & $n=j-d+1,\; elem_3$ \\	 
       & and $v_1 = (n)$ \\
       & \\
       \hline
    \end{tabular*}
    \capt{Transition rules for accessing lists.}
    \label{semantic:lists}
  \end{center}
\end{table}



List access requires two rules, since two cases are possible. The first case, is
when just one offset is requested, then that element should be returned. In the
second case, two offsets are requested, and a range of elements should be
returned, also in the form of a list.

\subsection{Let expressions}
\label{sec:letexpressions}

Variables in \productname{} are assigned using let expressions:

\begin{ebnf}
\grule{expression}{\grange \gor let\_expr}
\grule{let\_expr}{\gter{let} \gcat variable \gcat \gter{=} \gcat expression
\gcat \grep{\gter{,} \gcat variable \gcat \gter{=} \gcat expression} \gnl
\gcat \gter{in} \gcat expression}
\end{ebnf}

A let expression consists of one or more assignments and an expression. Each
assignment assigns the value of an expression to a variable name. These
variables can then be used in the expression after the \texttt{in} keyword.
After the evaluation of the let expression, the variables cease to exist.

\subsubsection{Informal scope rules for let expressions}

Destructive assignments are not possible in \productname{}, meaning that it isn't
possible to reassign a variable. It is however possible to hide a variable.
Consider the expression presented in \csref{lethiding.garry}.

\codesample{lethiding.garry}

The value of this expression is $13$. This is because within the
\texttt{\variable{x} + \numeral{2}} expression the \variable{x} variable evaluates to $6$.
But in the outer expression \variable{x} evaluates to $5$.

Nested let scopes are possible. Consider for instance \csref{nestedlet.garry}.

\codesample{nestedlet.garry}

In the inner scope, both \variable{x} and \variable{y} are available. This is of
course equivalent to \csref{nestedlet2.garry}.

\codesample{nestedlet2.garry}

\subsubsection{Big-step semantics}

The semantics presented in \tableref{semantic:let} are the transition rules for
the let expression. This transition rule is defined recursively to best
illustrate the functionality of the expression.

\begin{table}[ht]
  \begin{tabular*}{\textwidth}{l l}
    \hline \\
    \hspace{1.2cm} $[\mbox{LET-1}]$ & $env_{T}, env_{C}, env_{V}
    \vdash e_{1} \ra v_{1} \vspace{-0.3cm}$ \\
    & \infrule{env_{T}, env_{C}, env_{V}[x_{1} \mapsto v_{1}] \vdash \lag
    \texttt{let}\; x_{2} = e_{2}, \cdots,\; x_{k} = e_{k}\; \texttt{in}\;
    e_{k+1} \rag \ra v_{k+1}}
    {env_{T}, env_{C}, env_{V} \vdash \lag \texttt{let}\; x_{1} = e_{1},\; x_{2}
    = e_{2}, \cdots, x_{k} = e_{k}\; \texttt{in}\; e_{k+1} \rag \ra v_{k+1}} \\
    & where $k \geq 2$\\
    & \\
    
    \hspace{1.2cm} $[\mbox{LET-2}]$ & \hspace{0.4cm} $env_{T}, env_{C}, env_{V}
    \vdash e_{1} \ra v_{1}$ \vspace{-0.3cm} \\
    & \infrule{env_{T}, env_{C}, env_{V}[x_{1} \mapsto v_{1}] \vdash \lag
    e_{2}\rag \ra v_{2}} {env_{T}, env_{C}, env_{V} \vdash \lag \texttt{let}\;
    x_{1} = e_{1}\; \texttt{in}\; e_{2} \rag \ra v_{2}} \\
    \hline \\
  \end{tabular*}
  \capt{Transition rules for let expressions.}
  \label{semantic:let}
\end{table}


The transition rules for $[\mbox{LET-1}]$ are recursive because we must
evaluate each expression $(x_{1}=E_{1})$ individually before we move on to the next one. This
is a must because of the fact that the next expressions can in fact make use of
the previous expressions value. As an example take a look at the following code
sample:

\codesample{letbigstep.junta}

So, each call where there is more than one expression to be evaluated, 
the transition rule $[\mbox{LET-1}]$ where $k \geq 2$ is used. Here the expression first
in line to be evaluated will be evaluated before a new call to one of the two
transition rules is made. When we reach a let expression with only one
expression, we call the transition rule $[\mbox{LET-2}]$ where $k < 2$.

\subsection{Conditional expressions}
\label{sec:conditionalexpressions}

It is often desirable to base the result of an expression on some sort of condition.
In \productname{} this is achieved bye using \emph{if} expressions, as defined by the
following syntax:

\begin{ebnf}
\grule{expression}{\grange \gor if\_expr}
\grule{if\_expr}{\gter{if} \gcat expression \gcat \gter{then} \gcat expression
\gcat \gter{else} \gcat expression}
\end{ebnf}

Unlike in most imperative languages, the conditional construct in \productname{}
is not a statement (\productname{} doesn't have statements) but an expression.
Since all expressions must have a value, the \emph{else} part of en
if expression is compulsory.

The if expression first evaluates the condition (the first expression). The
resulting value must be of type \type{Boolean}. If the value is equal to the
boolean true value, the \emph{then} expression is evaluated, and the result is
returned. If the value is false, then the \emph{else} expression is evaluated,
and the result returned.

\subsection{Lambda expressions}
\label{sec:lambdaexpressions}

Lambda expressions are expressions that evaluate to anonymous functions. In
\productname{} they are defined as:

\begin{ebnf}
\grule{expression}{\grange \gor lambda\_expr}
\grule{lambda\_expr}{\gter{\#} \gcat varlist \gcat \gter{=>} \gcat expression}
\end{ebnf}

The non-terminal $varlist$ represents a list of formal parameters, and is further
explained in \secref{sec:constantdefinitions}.

When a lambda expression is created, a reference to the scope it was created in
is saved with it. This is known as a closure, and means that a lambda
expression may access variables outside of its own scope. The accessible
variables are the variables that were available at the time of the creation of
the lambda expression.

Consider the example in \csref{closuredef.garry}.

\codesample{closuredef.garry}

The function \function{getAdder} takes one argument (\variable{a}) and
returns a lambda expression. Notice how \variable{a} is used within
the lambda expression. This means that when the lambda expression is created, it
must remember the value of the variables that exist in the scope, in which it is
created. The use of the \function{getAdder} function could be as illustrated in
\csref{closureuse.garry}.

\codesample{closureuse.garry}

In the first line \function{getAdder} is called with the argument $25$. A new
scope, A, is created in which the variable \variable{a} is assigned the value
$25$. Then the function expression is evaluated, which results in a new lambda
expression (with a reference to scope $A$).  This is returned and assigned to
\variable{adder} in line 1 of \csref{closureuse.garry}.

In the second line, \variable{adder} is called as a function, meaning
that a new scope, $B$, is created where the variable \variable{b} is
assigned the value $5$. The important part is that $B$'s parent scope is set to
$A$ (which is saved with the lambda expression). The expression (the right side
of the lambda expression) is then evaluated. First the \variable{a} variable is
encountered. The interpreter first searches the $B$ scope for \variable{a}, and
when unsuccessful, searches the parent scope, $A$, for \variable{a}. In $A$ the
variable \variable{a} holds the value $25$, and this is returned. Then the
$B$ scope is searched for the \variable{b} variable, and the value $5$ is
returned. The two integers are added, and the final return value of the
lambda expression ends up being $30$.

\subsubsection{Big-step semantics}

The semantics presented in \tableref{semantic:lambda} is the transition rule for
lambda expressions.

\begin{table}[ht]
  \begin{tabular*}{\textwidth}{l l c}
    \hline \\
    \hspace{0.5cm} $[\mbox{LAMBDA}]$ & $env_{T}, env_{C}, env_{V} \vdash \lag
    \texttt{\#} \; g\;
    \texttt{=>}\; e \rag \ra v$ & \hspace{1cm} where $v = \left(g, e, env_{V}, env_{C}\right)$ \\
    & & \\
    \hline
  \end{tabular*}
  \capt{Transition rules for lambda expressions.}
  \label{semantic:lambda}
\end{table}



The three environments ($env_{T}, env_{C}, env_{V}$) must be known before it is
possible to execute a lambda expression. We need to know which types, constants,
and different variables are given in the specific scope.

The lambda expressions evaluates to a value $v$. The side condition
of the transition rule explains that $v$ is assigned the $4$-tuple, a
function value.

\subsection{Set expressions}
\label{sec:setexpressions}

Set expressions look a bit like let expressions, but are only applicable within
type definitions:

\begin{ebnf}
\grule{expression}{\grange \gor set\_expr}
\grule{set\_expr}{\gter{set} \gcat variable \gcat \gter{=} \gcat expression
\gcat \grep{\gter{,} \gcat variable \gcat \gter{=} \gcat expression}}
\end{ebnf}

Set expressions are used to ``modify'' the value of data members in objects (see
\secref{sec:typedefinitions} for an explanation of data). Each variable in the
set expression must exist in the current type as data members. Since modifying a
data member would be a side-effect, which is not allowed, the set expression
instead returns a clone of the object, with the specified data members set to
their respective values. This is useful for making setters (or something that
looks like setters). Consider for example presented in \csref{setget.junta}.

\codesample{setget.junta}

In \csref{setget.junta} the method \constant{setMyData} returns a new instance of
\type{MyType}, with the data member \variable{myData} set to something else. The
\csref{setget2.junta} shows the use of a getter and a setter.

\codesample{setget2.junta}

The call to \constant{setMyData} does not change the state of the original
instance of \type{MyType}, instead it returns a new instance.

\subsubsection{Big-step semantics}

The semantics presented in \tableref{semantic:set} is the transition rule for
set expressions.

\begin{table}[ht]
  \begin{tabular*}{\textwidth}{l l}
    \hline \\
    \hspace{1.5cm} $[\mbox{SET}]$ & \infrule{env_C, env_V, env_T \vdash \lag E_1
      \rag\ra u_1 \quad
    \ldots \quad env_C, env_V, env_T \vdash \lag E_k \rag \ra u_k}
    {env_C, env_V, env_T \vdash \lag \texttt{set}\; x_1 = E_1, \ldots, x_k =
    E_k \rag \ra v_1} \\
    & where $env_v\; \texttt{this} = \left(t, env'_c, env'_v, v_2 \right)$ \\
    & and $v_1 = \left( t, env'_c, env'_v, v_2\right)$ \\
    & and $env''_v = env'_v \left[ x_1 \mapsto u_1, \ldots, x_k \mapsto u_k
    \right]$ \\
    & \\
    \hline
  \end{tabular*}
  \capt{Transition rules for set expressions.}
  \label{semantic:set}
\end{table}


The transition rule assumes that the current constant environment $(env_C)$
contains a pointer to the current object, $\texttt{this}$. It then returns a
copy of that object with a new variable environment, containing the new data.

\subsection{Operators and calls}
\label{sec:operatorsandcalls}

Operators are useful for doing calculations, and \productname{} supports
basic mathematical operators and precedence. In order to prevent
left-recursion, which makes it possible for us to construct an LL-parser
(this was discussed in \secref{subsec:llparsersandlrparsers}), but
preserve left-associativity we have created a hierarchy of operators,
taking advantage of LL-parsing by putting the operators with highest
precedence the lowest in any parse tree that includes them. The grammar
for the operators of \productname{} are described using operator
sequences. A sequence is essentially just a list of operations on that
particular precedence level. In this way all the precedence levels of
\productname{} are described formally:

\begin{ebnf}
\grule{expression}{\grange \gor \gter{not} \gcat expression \gor lo\_sequence}
\grule{lo\_sequence}{eq\_sequence \gcat \grep{\ggrp{\gter{and} \gor \gter{or}}
\gcat eq\_sequence}}
\grule{eq\_sequence}{cm\_sequence \gcat \grep{\ggrp{\gter{==} \gor \gter{!=}
\gor \gter{is}} \gcat cm\_sequence}}
\grule{cm\_sequence}{as\_sequence \gcat \grep{\ggrp{\gter{<} \gor \gter{>} \gor
\gter{<=} \gor \gter{>=}} \gcat as\_sequence}}
\grule{as\_sequence}{md\_sequence \gcat \grep{\ggrp{\gter{+} \gor \gter{-}}
\gcat md\_sequence}}
\grule{md\_sequence}{negation \gcat \grep{\ggrp{\gter{*} \gor \gter{/} \gor
\gter{\%}} \gcat negation}}
\grule{negation}{element}
\galt{\gter{-} \gcat negation}
\grule{element}{call\_sequence \gcat \grep{member\_access}}
\grule{member\_access}{\gter{.} \gcat constant \gcat \grep{list}}
\grule{call\_sequence}{atomic \gcat \grep{list}}
\end{ebnf}

In order to completely understand this grammar, we must first take a look at the list of
operators ordered by precedence. The precedence of operators is presented in
\tableref{table:operatorPrecedence}.

\tab[\textwidth]{operatorPrecedence}{2}{The precedence of operators in \productname{}.}
                  {Operator precedence}
           {Level}{Operator & Description}{
    \tabrow{1}{\texttt{f[]} & Function/constructor invocation and list access}
    \tabrow{2}{\texttt{r.m r.m[]} & Record member access and member invocation}
    \tabrow{3}{\texttt{-} & Unary negation operation}
    \tabrow{4}{\texttt{* / \%} & Multiplication, division, and modulo}
    \tabrow{5}{\texttt{+ -} & Addition and subtraction}
    \tabrow{6}{\texttt{< > <= >=} & Comparison operators}
    \tabrow{7}{\texttt{== != is} & Equality operators and type checking}
    \tabrow{8}{\texttt{and or} & Logical $and$ and $or$}
    \tabrow{9}{\texttt{not} & Logical $not$}
    \tabrow{10}{\texttt{if let set \#} & if-, let-, set-, and lambda-expressions}
}

Each precedence level will correspond to a certain rule in the grammar. For
instance the fifth precedence level for addition and subtraction is expressed
using the $as\_sequence$ rule. Combined with some multiplication an expression
making use of the $as\_sequence$ and $md\_sequence$ rules could look like
\csref{asandmd.junta}:

\codesample{asandmd.junta}

The resulting parse tree, using the grammar of \productname{}, could look
somewhat like the tree in \figref{fig:parsetreesequences}.

\begin{figure}[ht]
  \begin{center}
      \begin{tikzpicture}[]
	%the nodes
     	\node[square,xshift=4em]      			    (as)     {as\_sequence};
     	\node[circle,draw,yshift=-3em] 		    (plus1)  {$+$};
     	\node[circle,draw,yshift=-3em,xshift=8em]   (minus1) {$-$};
     	\node[circle,draw,yshift=-3em,xshift=16em]  (plus2)  {$+$};
     	\node[square,yshift=-3em,xshift=-8em]       (int1)   {integer $\left(2\right)$};
     	\node[square,yshift=-6em]      		    (md1)    {md\_sequence};
	\node[square,yshift=-6em,xshift=8em]        (int2)   {integer $\left(3\right)$};
     	\node[square,yshift=-6em,xshift=16em]       (md2)    {md\_sequence};
     	\node[square,yshift=-9em,xshift=-4em]      (int3)   {integer $\left(3\right)$};
     	\node[circle,draw,yshift=-9em,xshift=4em]  (mult1)  {$*$};
     	\node[square,yshift=-9em,xshift=12em]      (int4)   {integer $\left(2\right)$};
     	\node[circle,draw,yshift=-9em,xshift=20em] (div1)   {$/$};
     	\node[square,yshift=-12em,xshift=4em]       (int5)   {integer $\left(5\right)$};
     	\node[square,yshift=-12em,xshift=20em]      (int6)   {integer $\left(3\right)$};

	%the solution
	\node[rectangle,yshift=-14.5em,xshift=-8em] (a) {$2$};
     	\node[rectangle,yshift=-14.5em,xshift=-6em]     {$+$};
     	\node[rectangle,yshift=-14.5em,xshift=-4em] (b) {$3$};
     	\node[rectangle,yshift=-14.5em]                 {$*$};
     	\node[rectangle,yshift=-14.5em,xshift=4em]  (c) {$5$};
     	\node[rectangle,yshift=-14.5em,xshift=6em]      {$-$};
	\node[rectangle,yshift=-14.5em,xshift=8em]  (d) {$3$};
     	\node[rectangle,yshift=-14.5em,xshift=10em]     {$+$};
     	\node[rectangle,yshift=-14.5em,xshift=12em] (e) {$2$};
     	\node[rectangle,yshift=-14.5em,xshift=16em]     {$/$};
	\node[rectangle,yshift=-14.5em,xshift=20em] (f) {$3$};

	%first level
	\draw[-,-|,-,thin,] (as.south) |-+(0,-0.75em)-| (int1.north);
	\draw[-,-|,-,thin,] (as.south) |-+(0,-0.75em)-| (plus1.north);
	\draw[-,-|,-,thin,] (as.south) |-+(0,-0.75em)-| (minus1.north);
	\draw[-,-|,-,thin,] (as.south) |-+(0,-0.75em)-| (plus2.north);

	%second level
	\draw[-,thin,] (plus1.south) -- (md1.north);
	\draw[-,thin,] (minus1.south) -- (int2.north);
	\draw[-,thin,] (plus2.south) -- (md2.north);

	%third level
	\draw[-,-|,-,thin,] (md1.south) |-+(0,-0.75em)-| (int3.north);
	\draw[-,thin,] (md1.south) |-+(0,-0.75em)-| (mult1.north);
	\draw[-,-|,-,thin,] (md2.south) |-+(0,-0.75em)-| (int4.north);
	\draw[-,thin,] (md2.south) |-+(0,-0.75em)-| (div1.north);

	%fourth level
	\draw[-,thin,] (mult1.south) -- (int5.north);
	\draw[-,thin,] (div1.south) -- (int6.north);

	%solution level
	\draw[-,dashed,] (int1) -- (a);
	\draw[-,dashed,] (int2) -- (d);
	\draw[-,dashed,] (int3) -- (b);
	\draw[-,dashed,] (int4) -- (e);
	\draw[-,dashed,] (int5) -- (c);
	\draw[-,dashed,] (int6) -- (f);

    \end{tikzpicture}
  \end{center}
  \capt{A parse tree for the expression $2 + 3 * 5 - 3 + 2 / 3$.}
  \label{fig:parsetreesequences}
\end{figure}


The figure clearly shows the precedence, because the lower nodes will be
evaluated before the nodes that are higher in the parse tree. E.g.\ all of the
multiplication and division nodes will be calculated before the additions and
subtraction. The circular nodes contain information about each operation in a
sequence, i.e.\ all children of a sequence, except the first one, must have
information about the operation. How exactly this is done, is up to the parser.
It could also be an attribute on each node, decreasing the number of nodes. The
extra information is needed, since for instance addition and subtraction must be
on the same precedence level. Making two separate sequences for addition and
subtraction would give one operator precedence over the other.

\subsubsection{Big-step semantics}

The semantics presented in \tableref{semantic:callfun} is the transition
rule for function calls. Big-step semantics for the others are left out,
as they are mostly trivial.

\begin{table}[ht]
  \begin{tabular*}{\textwidth}{l l}
    \hline \\
    \hspace{3cm} $[\mbox{CALL}_{\mbox{FUN}}]$ & \hspace{0.1cm} $env_C, env_V,
    env_T \vdash \lag E \rag \ra v_2$ \\
    & \hspace{0.1cm} $env_C, env_V, env_T \vdash \lag L \rag \ra v_3$
    \vspace{-0.3cm} \\
    & \infrule{env'_C, env''_V, env_T \vdash \lag E' \rag \ra v_1}{env_C, env_V,
    env_T \vdash \lag E\; L\; \rag \ra v_1} \\
    & where $v_2 = \left(X, E', env'_V, env'_C\right)$ \\
    & and $v_3 = \left(l, elem\right)$ \\
    & and $env''_V = \left[x_1 \mapsto elem\; 1, \ldots, x_n \mapsto elem\; n \right]$ \\
    & \\
    \hline
  \end{tabular*}
  \capt{Transition rules for function calls.}
  \label{semantic:callfun}
\end{table}



In this rule the expression $e$ is evaluated to a function value $v_2$, which
has its own variable and constant environments, as the static scope rules.
The variable environment is then updated with the actual parameters assigned to
the formal parameters, after which the expression, contained within the function
value, is evaluated.

The semantics presented in \tableref{semantic:memaccess} is the transition rule
for member access, also known as dot-notation.

\begin{table}[ht]
  \begin{tabular*}{\textwidth}{l l}
    \hline \\
    \hspace{3cm} $[\mbox{MEMBER}_{\mbox{ACCESS}}]$ & \infrule{env_C, env_V, env_T
    \vdash \lag e \rag \ra v_1}{env_C, env_V, env_T \vdash \lag e\texttt{.}C
  \rag \ra v_3} \\
     & where $v_2 = \left(t, env'_C, env'_V, v_2 \right)$ \\
     & and $env'_C\; C = v_3$ \\
     & \\
     \hline
  \end{tabular*}
  \capt{Transition rules for member access.}
  \label{semantic:memaccess}
\end{table}



A member access is as simple as evaluating the left-side of the object
first (the constant), and then accessing the evaluated constant in that
objects constant environment.

\subsubsection{Valid operands}
\label{sec:validoperands}
In the following paragraphs we present the valid operands for the operators of
\productname{}. These will be grouped in the following categories: boolean,
comparison, integer, string, list, and direction and coordinate operators.

\paragraph{Boolean operators}

These operators only accept boolean operands, and only return boolean values:

\begin{dlist}
  \item \operator[Boolean]{and}{Boolean}{Boolean}\\
    Returns true when both operands are true and false otherwise. 
  \item \operator[Boolean]{or}{Boolean}{Boolean}\\
    Returns true when at least one of the operands are true and false otherwise.
  \item \operator{not}{Boolean}{Boolean}\\
    Returns true if the single operand is false and false otherwise.
\end{dlist}

\paragraph{Comparison operators}

These operators are used to compare two values, and always returns a boolean value:

\begin{dlist}
  \item \operator[Integer]{<}{Integer}{Boolean}\\
    Returns true if the left operand is less than the right one.
  \item \operator[Integer]{>}{Integer}{Boolean}\\
    Returns true if the left operand is greater than the right one.
  \item \operator[Integer]{<=}{Integer}{Boolean}\\
    Returns true if the left operand is less than or equal to the right one.
  \item \operator[Integer]{>=}{Integer}{Boolean}\\
    Returns true if the left operand is greater than or equal to the right one.
  \item \operator[\opstar]{==}{\opstar}{Boolean}\\
    Returns true if the left operand is equal to the right one.
  \item \operator[\opstar]{!=}{\opstar}{Boolean}\\
    Returns true if the left operand is not equal to the right one.
  \item \operator[\opstar]{is}{Type}{Boolean}\\
    Returns true if the type of the first operand is equal to or inherits from
    the type operand.
\end{dlist}

\paragraph{Integer operators}

The following operations are possible on integers:

\begin{dlist}
  \item \operator{-}{Integer}{Integer} \\
    Integer negation.
  \item \operator[Integer]{+}{Integer}{Integer} \\
    Integer addition.
  \item \operator[Integer]{-}{Integer}{Integer} \\
    Integer subtraction.
  \item \operator[Integer]{*}{Integer}{Integer} \\
    Integer multiplication.
  \item \operator[Integer]{/}{Integer}{Integer} \\
    Integer division.
  \item \operator[Integer]{\%}{Integer}{Integer} \\
    Integer modulo operation.
\end{dlist}

\paragraph{String operators}

It is possible to concatenate strings:

\begin{dlist}
  \item \operator[String]{+}{String}{String} \\
    Returns the concatenation of two strings.
  \item \operator[String]{+}{\opstar}{String} \\
   \operator[\opstar]{+}{String}{String} \\
    Returns the concatenation of a string and the string-representation of another type
\end{dlist}

\paragraph{List operators}

Some operators are available for list values as well:

\begin{dlist}
\item \operator[List]{+}{List}{List} \\
  Returns a list containing all elements from the first list followed
  by all elements from the second list.
\item \operator[List]{-}{List}{List} \\
  Returns a list containing all the elements from the first list that
  do not exist in the second list.
\item \operator[List]{+}{\opstar}{List} \\
  Appends any element on to the end of a list, and returns the resulting list.
\item \operator[List]{-}{\opstar} \\
  Returns a list containing the elements that do not match the right operand.
\item \operator[\opstar]{+}{List}{List} \\
  Prepends any element on to the start of a list, and returns the resulting list.
\end{dlist}

\paragraph{Direction and coordinate operators}

The following operators can manipulate directions and coordinates:

\begin{dlist}
  \item \operator[Direction]{+}{Direction}{Direction} \\
    Add a direction (vector) to another direction.
  \item \operator[Direction]{-}{Direction}{Direction} \\
    Subtract a direction from another direction.
  \item \operator[Direction]{+}{Coordinate}{Coordinate} \\
    Add a coordinate to a direction.
  \item \operator{-}{Direction}{Direction} \\
    Negate a direction.
  \item \operator[Coordinate]{-}{Coordinate}{Direction} \\
    Returns the distance between two coordinates as a direction.
  \item \operator[Coordinate]{+}{Direction}{Coordinate} \\
    Add a direction to a coordinate. 
  \item \operator[Coordinate]{-}{Direction}{Coordinate} \\
    Subtract a direction from a coordinate.
\end{dlist}

For instance adding the directions \texttt{n} and \texttt{e} produces a
direction equivalent with the direction \texttt{ne}. Adding a coordinate and
direction (and vice versa) gives a coordinate. As an example, $\texttt{A2} \;
\verb!+! \; \texttt{e}$ gives \texttt{B2}. More information about the coordinate
and direction types is available in \secref{sec:standardenvironment}.

