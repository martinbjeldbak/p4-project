\section{Expressions}
\label{sec:expressions}

Expressions in \productname{} is the 

\begin{ebnf}
%Expressions
\grule{expression}{let\_expr}
\galt{if\_expr}
\galt{set\_expr}
\galt{lambda\_expr}
\galt{\gter{not} \gcat expression}
\galt{lo\_sequence}
\end{ebnf}

Two statements hold about expressions in \productname{}:

\begin{nlist}
\item An expression \textbf{always} has a value.
\item An expression \textbf{cannot} have side effects.
\end{nlist}

\subsection{Atomic expressions}

These are the smallest possible parts of expressions in \productname{}, defined by
the rule:

\begin{ebnf}
\grule{atomic}{\gter{(} \gcat expression \gcat \gter{)}}
\galt{constant}
\galt{type}
\galt{variable}
\galt{\gter{this}}
\galt{\gter{super}}
\galt{integer}
\galt{string}
\galt{direction}
\galt{coordinate}
\galt{\gter{/} \gcat pattern \gcat \gter{/}}
\galt{list}
\end{ebnf}

The first atomic expression is $\gter{(} \gcat expression \gcat \gter{)}$, which means
that it is possible to embed expressions within other expressions, and manually control
the precedence of operations. Consider the following two expressions:

\codesample{parentheses1.junta}
\codesample{parentheses2.junta}

In \csref{parentheses2.junta} the parentheses are in fact unnecessary because the
\texttt{*}-operator has precedence over the \texttt{+}-operator (see
\secref{sec:operatorsandcalls}).

Names (constants, types, and variables) are also atomic expressions, and are evaluated
to whatever value they are associated with, based on the current scope. The keywords
($\gter{this}$ and $\gter{super}$) are atomic, but only applicable within type
definitions, where $\gter{this}$ refers to the current object and $\gter{super}$
refers to the current object casted to its parent type (if it has one). \todo{casting??}

The literals ($integer$, $string$, $direction$, and $coordinate$) are evaluated to their
respective values, while patterns are evaluated to \type{Pattern}-values according to
the grammar in \secref{sec:patterns} and lists are evaluated to \type{List}-values
according to the grammar in \secref{sec:lists}.

\subsection{Lists}
\label{sec:lists}

The \type{List}-type is one of the basic types in \productname{}. A \type{List}-value is
created using the following syntax:

\begin{ebnf}
\grule{list}{\gter{[} \gcat \gopt{expression \gcat \grep{\gter{,} \gcat expression}} \gcat \gter{]}}
\end{ebnf}

Essentially this means that a list is created from zero or more expressions (separated
by commas). The following statements can be made about lists:

\begin{nlist}
\item A list can be empty: \texttt{[]}
\item Lists are ordered ($\texttt{[1, 2]} \ne \texttt{[2, 1]}$).
\item Lists are immutable.
\end{nlist}

When a list is evaluated, each expressions is evaluated to a value (the order
of evaluation does not matter, since no side-effects are possible, lazy-evaluation of expressions
could even be a possibility), and all values (in the same order as the expressions) are
added to the resulting \type{List}-value. When a \type{List}-value is created, it can't be
altered further (because of the no-side-effects condition). All operations on that \type{List}-value
will create new \type{List}-values, and leave the original value intact.

\subsection{Let expressions}

Variables in \productname{} are assigned using let expressions:

\begin{ebnf}
\grule{expression}{\grange \gor let\_expr}
\grule{let\_expr}{\gter{let} \gcat variable \gcat \gter{=} \gcat expression \gcat \grep{\gter{,} \gcat variable \gcat \gter{=} \gcat expression} \gnl
\gcat \gter{in} \gcat expression}
\end{ebnf}

A let expression consists of one or more assignments and an expression. Each assignments assigns
the value of an expression to a variable name. These variables can then be used in the expression
after the \texttt{in}-keyword. After the evaluation of the let-expression, the variables cease to
exist.

\todo{maybe the scope stuff about let expressions here}

\subsection{Conditional expressions}

It is often desirable to base the result of an expression on some sort of condition.
In \productname{} this is achievable using \emph{if}-expressions, as defined by the
following syntax:

\begin{ebnf}
\grule{expression}{\grange \gor if\_expr}
\grule{if\_expr}{\gter{if} \gcat expression \gcat \gter{then} \gcat expression \gcat \gter{else} \gcat expression}
\end{ebnf}

\subsection{Lambda expressions}

\begin{ebnf}
\grule{expression}{\grange \gor lambda\_expr}
\grule{lambda\_expr}{\gter{\#} \gcat varlist \gcat \gter{=>} \gcat expression}
\end{ebnf}

\subsection{Set expressions}

\begin{ebnf}
\grule{expression}{\grange \gor set\_expr}
\grule{set\_expr}{\gter{set} \gcat variable \gcat \gter{=} \gcat expression \gcat \grep{\gter{,} \gcat variable \gcat \gter{=} \gcat expression}}
\end{ebnf}

\subsection{Operators and calls}
\label{sec:operatorsandcalls}

\begin{ebnf}
\grule{expression}{\grange \gor \gter{not} \gcat expression \gor lo\_sequence}
\grule{lo\_sequence}{eq\_sequence \gcat \grep{\ggrp{\gter{and} \gor \gter{or}} \gcat eq\_sequence}}
\grule{eq\_sequence}{cm\_sequence \gcat \grep{\ggrp{\gter{==} \gor \gter{!=} \gor \gter{is}} \gcat cm\_sequence}}
\grule{cm\_sequence}{as\_sequence \gcat \grep{\ggrp{\gter{<} \gor \gter{>} \gor \gter{<=} \gor \gter{>=}} \gcat as\_sequence}}
\grule{as\_sequence}{md\_sequence \gcat \grep{\ggrp{\gter{+} \gor \gter{-}} \gcat md\_sequence}}
\grule{md\_sequence}{negation \gcat \grep{\ggrp{\gter{*} \gor \gter{/} \gor{\%}} \gcat negation}}
\grule{negation}{element}
\galt{\gter{-} \gcat negation}
\grule{element}{call\_sequence \gcat \grep{member\_access}}
\grule{member\_access}{\gter{.} \gcat constant \gcat \grep{list}}
\grule{call\_sequence}{atomic \gcat \grep{list}}
\end{ebnf}

\subsubsection{Operator precedence}

\tab[\textwidth]{operatorPrecedence}{2}{The precedence of operators in \productname{}.}
         {Operator precedence}
  {Level}{Operator & Description}{
    \tabrow{1}{\texttt{f[]} & Function/constructor invocation and list access}
    \tabrow{2}{\texttt{r.m r.m[]} & Record member access and member invocation}
    \tabrow{3}{\texttt{-} & Unary negation operation}
    \tabrow{4}{\texttt{* / \%} & Multiplication, division, and modulo}
    \tabrow{5}{\texttt{+ -} & Addition and subtraction}
    \tabrow{6}{\texttt{< > <= >=} & Comparison operators}
    \tabrow{7}{\texttt{== != is} & Equality operators and type checking}
    \tabrow{8}{\texttt{and or} & Logical $and$ and $or$}
    \tabrow{9}{\texttt{not} & Logical $not$}
    \tabrow{10}{\texttt{if let \#} & if-, let-, and lambda-expressions}
}
