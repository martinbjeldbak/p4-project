\chapter{Analysis}
\label{chap:analysis}

This chapter focuses on the preliminary thoughts made before designing
the language. Here research is shown and decisions on basic design
principles are made. In \secref{sec:board-game-analysis} we begin by
defining what a board game actually is in. What is the definition
of it? Afterwards in \secref{sec:chessandkalah} we analyse two
board games, namely Chess and Kalah, to gain a better understanding
of which elements and artefacts they contain. After the analysis
of board games, we begin by analysing the four main programming
paradigms in \secref{sec:paradigms} and how the different paradigms
are used and what they are good for. After this we take a look at
how syntactic analysis is done in \secref{sec:syntacticanalysis}
and which methods we can use to analyse our own syntax. In this
section we for instance look at context-free grammars and different
approaches to parsing and how a parser can be constructed. After
this we look at which contextual constraints context-free grammars
have in \secref{sec:contextualconstraints}. Furthermore, we take a
look at code generation and interpretation of a given source code in
\secref{sec:codegenerationandinterpretation}. Here we investigate the
translation process. We also describe when an intermediate language can
be used and what it is good for. In this section, we also define what a
compiler and an interpreter is. In \secref{sec:simulator} we investigate
and describe how a game simulator can be used to visualise the produced
game in our programming language. Here we also describe which possible
features we can add to make the simulator a good environment for the
user.

This chapter must make it possible for us to make informed
design decisions for our programming language designed in
\chapref{chap:design}. Lastly in \secref{sec:summaryofdecisions},
we present our major decisions which have been made throughout this
chapter.

\section{Board game analysis}
\label{sec:board-game-analysis}

One may wonder what a board game really is. Could it just be any game
containing some kind of a board? If so, would Trivial Pursuit be a board
game and what about the game Twister, where you have to place your hands
and/or feet on a spot marked with a particular color on a sheet - or
board, as you could call it. Most people have a mental model of a board
game that does not include games like Twister. Here is one definition of
a board game \cite{def-board-game}:

\begin{quote}
  ``A board game is a game played across a board by two or
  more players. The board may have markings and designated spaces, and the
  board game may have tokens, stones, dice, cards, or other pieces that
  are used in specific ways throughout the game.''
\end{quote}

The definition above is very broad and will to some extent allow a game
like Twister to be categorized as a board game. All kinds of things
like cards and dice can be part of a board game, but one board game
designer may also be able to invent a new and yet unseen widget, which
he wants to include in his board game. A programming language that makes
it possible to describe any board will cover a very broad category of
games. You could argue that it would actually cover all games that can
be made, since even a first person shooter could technically be played across a
board. With such a broad definition, a programming language that aims
to make the programming of board game easier, will likely have to be a general purpose programming language. If a programming
language is aimed to make the programming of only a specific kind of
board games easier, there might be many things that can be expressed easy in that language compared 
to how it would have been done in existing general purpose programming languages.

To define what elements that is to be included in our programming language \productname{}, we think it is essential to look at some existing board games. For that reason we have analysed two well known board games. We have investigated the game elements that might be clumsy or not straightforward to implement in common general purpose programming languages. After the analysis of the games, a list of elements is served that respects all of the elements from the games. The aim of \productname{} is to ease the programming of these game elements

\todo{finish this section}

\section{Analysis of Chess and Kalah}
\label{sec:chessandkalah}

In the following sections we will analyse the two games: Chess and Kalah. The
reason for picking these two games in particular is because they are
universially known games which we think contain some very fundamental board game
elements that we need to know about to gain a better understanding of which
features are needed in \productname{}.

\subsection{Chess}
Chess is a board game of two oppenent players. It's a turn-based game which means one player makes a move, 
then the other player makes a move, then the first player makes a move and so on. Chess is played on a board of $8 \times 8$ squares. The squares are typically black and white, but can be any two colors (see figure \ref{fig:chess}). The squares can only contain one piece at a time, unlike games like Mancala and Backgammon. Each player has a total of 16 pieces: 8 pawns, 2 knights, 2 bishops, 2 rooks, a queen and a king. Each type of piece has unique ways to move. For instance a pawn can move only one square vertically forward or one square diagonal when capturing an enemy piece. A rook can move unlimited squares either forward or backward (vertical movement), or to the right or to the left (horisontal movement). This separates chess from a lot of other common board games where all pieces have the same abilities, like Naughts and Crosses, Mancala, Ludo, Backgammon.  

Cut to the bone Chess goes as follow: When a game starts the pieces are in their starting positions as seen in figure \ref{fig:chess}. The player with the white pieces always makes the first move, and after that the players shifts in turn in which clever moves are beeing taken and pieces are beeing captured until one player has checkmated the other - and the game is over. The checkmate condition is obtained when the king piece is in a position to be captured and cannot escape from capture in the next move. \cite{chessrules}. Therefore it's nessecary to look one move ahead to control if the checkmate condition is optained.

Special moves. In chess there are numerus special moves which doesn't follow the normal pattern of chess. Earlier we mentioned that a pawn can move only one square vertically forward or one square diagonal when capturing an enemy piece. But this is not always true. If the pawn is in it's respective starting position it can move either one or two squares vertically forward. After that it can only move one square forward or one square vertically the rest of the game. Another special move is the move called ``castling''. This move allows a player to move two pieces in one turn (the king and one of the rooks). But to do the move several conditions needs to be met. First: the move has to be the very first move of the king and the rook, second: there can't be any pieces standing between the king and the rook and third: there can't be any opposing pieces that could capture the king in his original square, the squares he moves through or the square he end up in \cite{chessrules}. There exits two more special moves which are called ''En Passant´´ and ''Promotion´´. These are not going to described here, but information about them can be found in \cite{chessrules}. So what is the problem with these special moves? The problem is the fact that they don't follow the regular pattern of the game and this has to be taken into consideration when designing \productname{}.

\fig[scale=0.1]{chess}{The board game chess with the pieces in start position.}

From the above analysis here is a list of interesting game elements we found in chess:
\begin{dlist}
\item Pieces has different movement abilities.
\item A squared board with a number of squares in it.
\item A winning condition - when the king has been checkmated.
\item A starting state - how the pieces are placed on the board before the game's very first move.
\item Special moves like ``castling'', ``Promotion'' and ``En Passant.
\item Constraints that disallows a piece to move if some condition is true after the move has been made (a move that sets your king in check).
\item A piece can be ``captured'' by another piece, which causes the piece to be removed from the board.
\end{dlist}

\subsection{Kalah}

Kalah is like chess, a turn-based game consisting of two opposing
players. But Kalah pieces, called seeds, are very different from the
Chess pieces. They do not have specific moves but rather functions,
as their name also suggest, as seeds. The board is not like the Chess
board either. It consists of $14$ squares, sometimes referred to as houses
\cite{kalahrules}, with two of the houses separating themselves from the
rest by being the houses or bases of each of the players. Furthermore,
each player has six houses belonging to them (see \figref{fig:kalah}).
Each house (including the players' houses) can contain an arbitrary
number of seeds, unlike in chess where the squares can only contain one
piece.

Cut to the bone, Kalah goes as follows: When the game starts, each
of the $12$ houses contain $4$ seeds (in some versions of the game
each house contains $5$ or $6$ seeds), and the player bases are empty.
Now the players take turn to pick up piles of seeds and deal them out
to the $12$ houses, including their own base. The dealing of seeds
works by a player picking up a pile of seeds from one of his six houses
and dropping one seed down in each of the following houses, moving
counter-clockwise. If, when dealing out seeds, he lands in a house
belonging to himself (not including his own base), that is not empty
and does not belong to the opponent, the player can pick up the pile of
seeds in the house and start dealing these out. A player's turn ends
if, when dealing out seeds, he lands in an empty house or one of the
opponent's houses. For a more detailed description of the rules, we
refer to \cite{kalahrules}. The game is over once one of the players'
six houses is empty and the winner player with most seeds in their base.

\fig[width=0.25\textwidth]{kalah}{The board game Kalah.}

\subsubsection{Special moves}
Like in Chess there are some special moves in Kalah that don't follow the
regular pattern of the Kalah game. We are not going to describe them here, but
they will be present in the list of interesting game elements and can be found
in a detailed description in \cite{kalahrules}.

Here is a list of some interesting game elements we found in Kalah. For
simplicity we are going to refer to houses and bases as squares and refer to
seeds as pieces:

\begin{dlist}
  \item Squares can contain an arbitrary number of pieces
  \item Making a move can be considered as simple as choosing a square
  \item The number of pieces on a square determines how long a move you can make
  \item A turn may contain more than one move
    \begin{dlist}
      \item If the last piece is dropped in a non-empty square, the player can
	make another move
    \end{dlist}
  \item A square can belong to a player
  \item Squares can be related to other squares
    \begin{dlist}
      \item You place pieces on squares counter-clockwise
      \item If you place the last piece on an empty square, the square across
	the board belonging to the opponent is emptied over to your own square
    \end{dlist}
  \item An end game condition - when all of the squares belonging to a player are
    empty
  \item A winning condition - the player with the most pieces in their square
    wins when the end game condition has been met
  \item Only one type of piece
\end{dlist}
  
\subsection{Summary}

% Conclusion: Dynamic typing makes it hard to do typechecking


\section{Overview of the four main programming paradigms}
\label{sec:paradigms}

A programming paradigm describes a method and style of computer programming.
Some of the primary paradigms are imperative, object-oriented, functional and
declarative programming. While some programming languages strictly follow one
paradigm, there are many so-called multi-paradigm languages, that implement
several paradigms and therefore allow multiple styles of programming. Examples
of multi-paradigm languages include C\# and Java. It is essential for us to be
aware of these different paradigms, since it may help us to find a good style of
programming for the design of \productname{}.

\subsection{Overview}
The four main paradigms are described as follows:\cite{fourparadigms}
\begin{description}
\item[Imperative programming] describes computation in terms of statements that
  change the program state. Primary characteristics are assignments, procedures,
  data structures, control structures. Imperative programming can be seen as a
  direct abstraction of how most computers work, and many imperative languages
  are just abstractions of assembly language. Typical examples of imperative
  languages are C and Fortran.
\item[Object-oriented programming] describes computation in terms of objects
  described by attributes manipulated through methods. Primary characteristics
  are objects, classes, methods, encapsulation, polymorphism, inheritance. An
  example of a pure object-oriented languages is Smalltalk, while many other
  languages are either primarily designed for object-oriented programming (such
  as Java and C\#) or have support for object-oriented programming (such as PHP
  and Perl).
\item[Declarative programming] describes computational logic without describing
  control flow, i.e. describing {\em what} a program does rather than {\em how}
  it does it. Many domain-specific languages such as SQL, HTML and CSS are
  declarative. Logic programming, such as Prolog, is a subset of declarative
  programming.
\item[Functional programming] describes computation in terms of mathematical
  functions and seeks to avoid program state and mutable data. Purely functional
  functions have no side effects, and the result is constant in relation to the
  parameters (e.g. $add(2, 4)$ always returns $6$). An example of a purely
  function programming languages is Haskell. Other examples of languages
  designed for functional programming are Erlang, F\# and Lisp, while it is
  possible to apply functional programming concepts to many other languages.
\end{description}


While general-purpose languages, such as C\# and Java, generally tend to lean
towards the imperative and object-oriented paradigms, a domain-specific
language, with very specific goals in design, may benefit from other paradigms,
e.g. declarative programming. We want to be able to have complete control
over state changes in board games, so following the functional paradigm
and preventing side-effects, will make it easier to manage state changes
and the movement history of a game. Also, the object-oriented paradigm,
has proved to be very useful for describing many real-word systems. Allowing
the use of classes and objects, may make it easier to create board games
(e.g. an instance of class \emph{Board} consists of many \emph{Squares} 
containing \emph{Pieces} belonging to \emph{Players}). Combining these two
paradigms, we will get a functional object-oriented programming language
for describing board games.


\section{Syntactic analysis}
All languages whether natural or artificial is a set of strings of characters over some alphabet. There are rules for how the strings can look that are in a language and how they can be combined. The lexemes described how the strings can look and now the tokens are useful when analysis how the lexemes can be combined. The rules can be specified formally to describe the syntax of a language\cite[p. 135]{sebesta2013}. A common way to describe a language's syntax is by a formal language-generation mechanism (also called grammars or context free grammars). By describing a grammar that can generate all possible strings in a language, the language has also been formally described. Backus-Naur Form is such a mechanism which in the 1950's became the most widely used method for describing programming language syntax\cite[p. 137]{sebesta2013}.

\subsection{Lexical analysis}
The lowest level syntactic units of a language is called lexemes. A language's formal description does not often include these. They are instead described by a lexical specification, regular expressions i.e., separated from the syntactic specification\cite[p. 135]{sebesta2013}. Typical lexemes for a programming language includes integer literals, operators and special keywords like \textit{if} and \textit{while}. If both \textit{\$a} and \textit{\$b} are lexemes describing a variable and \textit{102} and \textit{42} are lexemes describing an integer, then \textit{\$a} and \textit{\$b} or \textit{102} and \textit{42} can typically be used interchangeably and still give a meaningful program. Therefore the lexemes are grouped into tokens. The name of a variable or the value of an integer is preserved when tokenising. The tokens are an abstraction that makes it easier to analyse if correct syntax of the language. An example of the grouping of lexemes into tokens can be seen by \tableref{table:lexandtokens}. After the lexical analysis an input stream of characters has been converted to an output stream of tokens.

\tab[4cm]{lexandtokens}{1}{Lexemes and their corresponding token group.}
		    {               }
{Lexemes   }{\textbf{Tokens}}{
\tabrow{\$a}{var(a) 		}
\tabrow{=  }{assign 		}
\tabrow{3  }{int(3) 		}
\tabrow{\$b}{var(b) 		}
\tabrow{+  }{plus   		}
\tabrow{4  }{int(4) 		}
\tabrow{\$a}{var(a) 		}
}

\subsection{Context-free grammars}
Grammars are defined using Backus-Naur Form (BNF). \todo{Finish\ldots}

BNF contains a set of terminals and a set of non-terminals. The terminals are the tokens from the lexical analysis. The non-terminals all have a set of productions, from which a mix of terminals and non-terminals can be derived from. A start production specifies a single non-terminal, from where all syntactically valid strings that are in the language can be derived from by using the production rules until only a sequence of terminals (the tokens) are left. The syntax analysis takes a sequence of tokens as input and tries to create a set of derivations from the start symbol that creates the given sequence of tokens. If success, the input has been parsed and the parse tree is kept for later analysis. The parse tree is the information concerning how the start symbol was derived into the sequence of tokens, which yields a tree structure. This tree is called an abstract syntax tree.

\begin{ebnf}
%Expressions
\grule{program}{\gter{print} \gcat expr}
\grule{expr}{\gter{(} \gcat term \gcat \gter{)} \gcat operator \gcat \gter{(} \gcat term \gcat \gter{)}}
\grule{operator}{\gter{=}}
\galt{\gter{>}}
\galt{\gter{<}}
\grule{term}{number}
\galt{expr}
\grule{number}{\textbf{any number}}
\end{ebnf}

% This subsection is massive and from the previous structure,
% hence it is inputted
\subsection{LL- and LR-parsers}
\label{subsec:llparsersandlrparsers}
As mentioned in the introduction to the section, the LL-parsers derive from the
top-down parsing approach. In terms of grammars, this means the LL-parsers
attempt to parse a string by starting at the start symbol of the grammar and
through a series of left-most derivations match the input string. On the
opposite, the LR-parsers derive from the bottom-up parsing approach. Here the
LR-parsers attempt to parse by starting with the input string and through a
series of reductions get back to the start symbol.

The LL-parsers have two
actions; predict and match. The predict action is used when the parser is trying
to guess the next production to apply in order to get closer to the input
string. While the match action eats the next unconsumed input symbol if it
corresponds to the left-most predicted terminal. These two actions are
continuously called until the entire input string has been eaten and thereby has
been matched. An example of a LL(1)-parser can be seen in
\tableref{table:LL1}. In the example the parser is based on the simple grammar: 

\begin{centering}
\begin{ebnf}
  \grule{S}{E}
  \grule{E}{T \gcat + \gcat E}
  \galt{T}
  \grule{T}{int}
\end{ebnf}
\end{centering}

\tab[11cm]{LL1}{3}{A LL(1) parser seen in action parsing the string ``int + int''.}
	      {The process                                          }
{Step  	 }{Production & Input       & Action                        }{
\tabrow{1}{$S$        & $int + int$ & Predict $S \rightarrow E$     }
\tabrow{2}{$E$	      & $int + int$ & Predict $E \rightarrow T + E$ }
\tabrow{3}{$T+E$      & $int + int$ & Predict $T \rightarrow int$   }
\tabrow{4}{$int+E$    & $int + int$ & Match $int$  		    }
\tabrow{5}{$+E$       & $+\; int$   & Match $+$		    	    }
\tabrow{6}{$E$ 	      & $int$ 	    & Predict $E \rightarrow T$     }
\tabrow{7}{$T$ 	      & $int$ 	    & Predict $T \rightarrow int$   }
\tabrow{8}{$int$      & $int$       & Match $int$   		    }
\tabrow{ }{           &             & Accept			    }
}

$S$, $E$ and $T$ are non-terminals, and $+$ and $int$ are terminals. 

The LR-parsers
also have two actions; shift and reduce. The shift action adds the next input
symbol of the input string into a buffer for consideration. The reduce action
reduces a collection of non-terminals and terminals into a non-terminal by
reversing a production. These two actions are continuously called until the
input string is reduced to the start symbol.
\cite{LL(1)andLR(2)inaction} 
An example of a LR(2)-parser in action is illustrated in \tableref{table:LR2}.

\tab[11cm]{LR2}{3}{A LR(2) parser seen in action parsing the string ``int + int''.}
	  {The process	    					 }
{Step  	 }{Production & Input       & Action                     }{
\tabrow{1}{           & $int + int$ & Shift   			 }
\tabrow{2}{$int$      & $+\; int$   & Reduce $T \rightarrow int$ }
\tabrow{3}{$T$        & $+\; int$   & Shift     		 }
\tabrow{4}{$T+$       & $int$ 	    & Shift			 }
\tabrow{5}{$T+int$    & 	    & Reduce $T \rightarrow int$ }
\tabrow{6}{$T+T$      &             & Reduce $E \rightarrow T$   }
\tabrow{7}{$T+E$      &      	    & Reduce $E \rightarrow T+E$ }
\tabrow{8}{$E$        &             & Reduce $S \rightarrow E$   }
\tabrow{ }{$S$        &             & Accept			 }
}

\subsubsection{Comparison of the parsers}
Compared to the LL-parsers, the LR-parsers are more complex and they are
generally harder to construct,\cite[p. 193]{sebesta2013} thus by the use of
automated generator tools this might not be the case. We take a loot at how to 
construct a parser in \secref{subsec:constructingaparser}. 

The LR-parsers are more powerful than the LL-parsers, because they accept a
bigger variety of grammars. For instance LL-parsers can't handle grammars with
left-recursion, while LR-parsers can. The ``power'' and complexity of a parser
is very dependent on the number of lookahead tokens, $k$, which the parser makes
use of. The bigger $k$ is, the more complex and difficult the parser is to
contruct, but the bigger variety of grammars the parser also accepts. As
illustrated in \figref{fig:LL-parserandLR-parser} the LL-parser is a proper
subset of the LR-parsers.

\fig[width=0.75\textwidth]{LL-parserandLR-parser}{The set of grammars accepted
by different parsers. As illustrated LL(k)-parsers are a subsets of
LR(k)-parsers for different number of lookahead tokens, $k$. The figure is
modified from slides presented in the ``Languages and Compilers''
course from Aalborg University in the spring of 2013.}


\subsection{Constructing a parser}
\label{sec:ana-parsers}
The principle of generating parsers is very systematic and therefore there are different automated tools to generate parsers for a specific grammar that meets some standards. A grammar must for instance not be ambiguous otherwise the tools cannot make a distinct parser for the grammar. A grammar is ambiguous if a string can be generated with more than one parse tree. We start by taking a look at the constructing of a handwritten parser. Then we take a look a two different parser generators that produce different parsers. Finally we sum up the pros and cons of the analysis on handwritten and generated parsers. 

\subsection{Handwritten parsers}
\label{sec:handparser}
Why would you write a parser on your own when you have automated tools for this job? If we were to construct our own handwritten parser, and not use the tools alreadt built for this, it would be so that we would gain a greater understanding of how these parsers work. How are they constructed? What kind of errors could occur when trying to develop a parser? One of the best ways to learn is to fail - learn from the mistakes and correct them. But this is also a pain in the neck if a lot of errors are popping up and the person trying to get the job done does not have the expertise to resolve the errors and find a solution.

So we would gain experience and probably learn a lot by writing our own parser - but what are the pitfalls of taking on this task? First of all we could be stumbling upon many errors in the code. These errors must be solved before the parser can be finished. Therefore the construction of the parser will be time-consuming. So we will be gaining knowledge about the process but it will take a lot of time compared to an automatic generator.

Programming languages evolve and their grammar can change. When this is the case the parsers must be maintained so that they still output the correct result. When we have generated a parser by hand this task will be time-consuming because we must search through the code of the parser and tweak it so it will be correct again. Whenever we work on tweaking existing code we are most likely going to run into new errors that need to be resolved. 

This brings us to the topic of how reliable our handwritten parsers are compared to the generated parsers. So far we have discussed that the produced code for a handwritten parser will be error-prone so this will naturally bring us to conclude that this code must be less reliable than the automated generators produced code. This is a big con because we have to be able to rely on the output of the parser. To check the validity of a parser, it can be given a set of inputs that are almost in its language but contains a small error, which must cause the parser to reject the inputs. Additionally, a set of inputs can also be constructed, which is \textbf{in} the language, hence the parser must build a correct abstract syntax tree. For each input, the AST outputted by the handwritten parser can be compared to the one outputted by the parse generator tool. If the parse generator has not been constructed yet, one can choose to manually write the expected AST in XML, which most languages / programming frameworks support, using the grammar as reference.

\subsection{Generated parsers}
\label{sec:ana-genparser}
There are quite a few automated parser generaters (compiler compilers). To construct af parser with a generator the developer must input the grammar into the generator, and it will output a parser for that specific grammar. This can be a bit different from software to software but the grammar is often expressed using the Extended Backus Naur Form (EBNF) and the output parser will be produced in the language the generator is meant to output. We take a short look at SableCC and JavaCC in the following two section that both produce Java source code.

What differences are there between a handwritten and a generated parser? We need a grammar in both methods so what is the advantages of a generated parser? First of all, it takes less time to construct the parser because once we have a grammar we can input it in the generator and it will automatically generate the parser for us. Before we can use the software we have to figure out how to use it but this is not a complicated process. For every software there is some kind of tutorial on how to use the software.

The software that generate the parsers have been under development for quite a while and therefore the developers are using efficient algorithms to implement the parsers. This means that the parser we construct by hand will not be as fast and reliable as the ones generated by the software - unless the programmer is very experienced with a wide knowledge base about this subject. So the generated parser is most-likely more efficient.

Maintenance of the parser is also much easier because everything is machine-produced and can easily be changed to correspond witht he new grammar if the grammar has been changed.

There are different methods for constructing parsers. We have top-down parsers, where the parse trees are built from the root (the top) to the bottom, and bottom-up parsers, where the parse trees are built from the bottom to the root. Different grammars have different limitations and  the different types of parsers work on specific grammars. We will shortly discuss this in the following sections.

\subsubsection{SableCC, a bottom-up parser}
\label{sec:ana-sablecc}
SableCC is a bottom-up parser generator that generates LALR(1) parsers. This generator runs on the Java-platform and produces object-oriented code with clearly seperated machine-generated code and handwritten code. This contributes to the simplification and ease of maintaining the code.\cite[pp. 11]{sableccdoc}

The following is a list of advantages that LR parsers have:\cite[pp. 193]{sebesta2013} \todo{kontroller sidetal! afsnit 4.5.3}


The only disadvantage a LR parser has is that it is verey difficult to produce by hand. We have the automated generators to solve this disadvantage.

\subsubsection{JavaCC, a top-down parser}
\label{sec:ana-javacc}
JavaCC is a top-down parser generator that generetes LL(k) parsers. As the name of the parser generator it also produces the output code as Java source code.\cite{wiki-javacc}

\subsection{Summary}
\label{sec:ana-parsersum}
By reading the above section about handwritten parser we can conclude the following advantages by handwriting a parser:

\begin{dlist}
\item Gain experience in constructing parsers
\item Gain a better understanding of how parsers work
\end{dlist}

We will be gaining a lot of experience by writing a parser by hand and solving the problems that arise along the way. But there are quite a few disadvantages accompanied with handwriting a parser, and they are as follows.

\begin{dlist}
\item Can be error-prone
\item Can be time-consuming to construct
\item Time-consuming to maintain
\item Less reliable than generated parsers
\item Slower than generated parsers
\end{dlist}

We've summed up the advantages and disadvantages of handwritten parsers. Know we take a short look at the advantages of generated parsers. By reading the above section about the automatically generated parsers we can conclude the following advantages:

\begin{dlist}
\item Efficient
\item Reliable
\item Fast
\item Easy to maintain
\end{dlist}

These parsers will most-likely be more efficient and faster than handwritten parsers because they include efficient algorithms developed and maintained throughout the lifetime of the software. This also makes them more reliable than handwritten parsers because there will be much less errors in the process of constructing the parser.

This brings us towards a conclusion on handwritten and generated parsers. It is very clear that there are more benefits in using an automated generator to construct a parser. It is quite obvious that it will be much easier to reach our goal of constructing a parser. But we believe that it is very important that we try to gain experience and therefore we will be both handwriting a parser and using a generator as well.

\section{consideration of scope and type systems}

In the following sections we take a quick look at the consequences of scope and
type systems for \productname{}. We begin by introducing scope rules with a
short discussion of the consequences of static and dynamic scoping. Afterwards,
we introduce and discuss the consequences of static and dynamic type systems.

\subsection{Which scope rules are we considering?}

With static scoping the scopes of variables can be determined prior to
execution. This means that a compiler can easily determine the type of every
variable in the program by just examining the source code. So, when a variable
is referenced in a statically scoped language the value and type of the variable 
is the on it had at the time of declaration.


Static scope rules provide subprograms with access to nonlocal identifiers and
this type of scoping works very well with compilers because the scope can be
determined at compile time.

A disadvantage of static scoping is that it can give too much access and might
need restrictions. But programs are dynamic and are often restructured which
can lead to destruction of initial restrictions.

With dynamic scoping the scopes of variables can only be determine at run time
because it is based on the calling sequence of subprograms. When a variable is
referenced in a language with dynamic scoping then the value and type of the 
variable is what it had on the time of the call to it.
\cite[p. 227]{sebesta2013}
%5.5.6, side 227 (sebesta)

It is not possible to determine scopes statically because the calling sequence
of subprograms is not always known. When a method A calls a method B then B has 
access to variables in that were declared in A. As a result dynamically scoped
languages are much more difficult to read and understand and this results in
them being less reliable.


Comparing two similar programs with different scoping, then the statically
scoped program will be much easier to read, more reliable, and it will execute
much faster than the program written with dynamic scoping.
\cite[p. 229]{sebesta2013}
%5.5.7, side 229 (sebesta)

\subsection{dynamic and static type systems}

The type system is an essential part of the feel and touch of a programming language. There exits two main kinds of type systems: the dynamic and the static type system. Wether to have one over another, is today a hotly discussed topic. In the following section we list some advantages and disadvantages of each. 

When it comes to detecting errors the static type system can make it easier to detect programming errors in an earlier stage than the dynamic type systems. This is due to the fact that it's already possible to check type errors during compile-time rather than at run-time. The readability can also be improved by the static type system because of the precense of type names. This can make it easier for the programmer to get an idea of what a certain sub-program is ment to do. The forced precense of types is also what decreases the writeability of static type systems, since the programmer has to write down the types at all times and, when declaring variable, spend time considering wether a variable should be an integer type, a floating points or another type. This can take valuable programming time. So the dynamic type system is faster to writ and it's more flexible but the static type system is easier to read and more reliable at run-time since type checking is done on compile-time. There is a whole list of other advantages and disadvantages of each type system, which are provocatively explained in \cite{staticvsdynamictypesystem}, where it is argued that perhaps a middle solution between the two should be considered.

But which type system should we then go for? We are aiming to create a programming language in which it's easy and fast to create board games. It should be possible to create a board game within fewest possibles line of code and for that purpose the dynamic type system is the more relevant. 


\section{Contextual constraints}
\label{sec:staticsemantics}

The \classref{ScopeChecker} is the class responsible for enforcing some of the
static semantic rules of \productname{} (described in \chapref{chap:design}) at
compile time. This section aims to explain how these static semantic checks
are performed by the \classref{ScopeChecker} in simple sequential steps. Any
error detected by the \classref{ScopeChecker} will cause a \classref{ScopeError}
exception to be thrown. This error contains helpful information about the type
of error and where in the input program the error is located. The checking of
static semantics as well as the interpretation of the code both use the visitor
pattern which is a commonly used approach for both purposes.

After the AST has been created, the visitor pattern allows us to traverse the
AST and execute encapsulated pieces of code for each specific type of
\classref{AstNode}. 

\subsection{TypeVisitor}
The first visitor used is the \classref{TypeVisitor}. This visitor traverses the
AST, finds all type definitions in the input program and for each type
definition an object of class \classref{TypeSymbolInfo} is instantiated. After
running the \classref{TypeVisitor}, each type definition in the input program
has an associated object of class \classref{TypeSymbolInfo} which makes it easy
to get information about any declared type in the input program by accessing the
following members contained in the \classref{TypeSymbolInfo} object:

\begin{dlist}
  \item name (\classref{String}): The type's name
  \item parentName (\classref{String}): The name of its super type (null if
    not a derived type)  
  \item args (\classref{Integer}): The number of arguments in the type's
    constructor
  \item parentArgs (\classref{Integer}): The number of arguments given in the
    call to its parent constructor
  \item data (\classref{List of Data}): Each data defined in the type body has a
    corresponding \classref{Data} object describing its name and position in the
    input program
  \begin{dlist}
    \item The input program position is stored as a line and an offset and makes
      it possible to produce error messages with information about where in the
      input program an error was found
  \end{dlist}
  \item members (\classref{List of Member}): Each constant and function defined
    in the type body has a corresponding \classref{Member} object describing its
    name, argument count, an abstract flag, a \classref{TypeSymbolInfo}
    reference to the type defining the \classref{Member} and a pointer to the line
    and offset in the code
  \item node (\classref{AstNode}): A reference to the
    TYPE\_DEF-\classref{AstNode} which defines the type
  \begin{dlist}
    \item This reference is used to get the input program position where the
      type was defined (for generating useful error messages), but is also used
      for marking abstract type definitions (described in detail in
      \secref{sec:abstractTypeMarker}) 
  \end{dlist}
  \item parent (\classref{TypeSymbolInfo}): This will contain an object
    reference to its parent type if it has one
  \begin{dlist}
    \item This is however a null pointer until running the
      \classref{TypeTableCleaner} described in \secref{sec:typetablecleaner}
  \end{dlist} 
\end{dlist}

All the \classref{TypeSymbolInfo} objects are kept in an object of class
\classref{TypeTable}. The \classref{TypeTable} class is a layer of abstraction
which provides easy and fast access information about the types contained in the
input program. The underlying implementation is a hashmap from the type's name
as a \classref{String} to its object reference, which provides quick lookup on
type names, and a list of \classref{TypeSymbolInfo}, which makes iteration over the
\classref{TypeSymbolInfo}s convenient and makes it possible to sort the types
with a purpose described in \secref{sec:typetablecleaner}. For convenience, we
say that a type is added to a type table which means that a
\classref{TypeSymbolInfo} object representing the type is added to the
\classref{TypeTable} object representing the type table.

When a type is added to the type table, it is checked that no other types with
the same name exist.

\subsection{TypeParentRefMaker}
\label{sec:TypeParentRefMaker}
The \classref{TypeSymbolInfo} objects only contain the name of their super type
as a \classref{String} or a null value if there is no super type. By making a
lookup in the \classref{TypeTable} on the parent name, the real object
references can be found and stored for faster and more convenient parent lookups
which is used greatly by the visitor described in
\secref{sec:usesaredeclaredvisitor}.

\subsection{TypeMemberPropagator}
\label{sec:TypeMemberPropagator}
Some of the later checks that will be performed requires us to determine whether
or not a type member (constant or function) with a specific name is visible in a
given type. This requires searching in the given type and recursively in all
super types for the member. When this kind of lookup is done many times on the
same member, the traversal of the same long chains of parent references become
inefficient which results in clumsy code. To simplify and speed up this process
the \classref{TypeMemberPropagator} ensures that all members of a type A are
also present in a type C, if A is a super type of C. With this approach,
checking if a member is visible in type C, only requires looking in C instead of
following the chain parent types.

This propagation of members is done by first doing a topological sort on the
\classref{TypeSymbolInfo} objects, such that when iterating over the type table,
any type yielded will always appear before all of its subtypes. This makes the
afterwards propagation of members possible in linear time by iterating over the
topologically sorted types. If a type C is met, the members in its parent type B
are put in C as well. If B has a parent A, we know that B has already the
members from A due to the topological sorting.

The topological sorting is done using the algorithm that goes by the same name, from the book Introduction
to Algorithms \cite[p. 612]{ad} working on a graph $G = (V, E)$. Each type
represents a vertex $v$. An edge $e$ exist from $v_1$ to $v_2$ if the type $v_2$
is a parent of the type $v_1$. However this results in a topological order where
a type always appears before its super types. This problem is quickly solved
simply by reversing the list.

Given the graph in \figref{fig:topological}, the following sequences are examples
of correct and wrong topologically sorted orders after the list has been reversed:

\begin{align*}
 Correct &: \texttt{a, d, b, e, c, f} \\
 Correct &: \texttt{a, b, e, c, f, d} \\
 Wrong &: \texttt{a, b, c, \textbf{f}, \textbf{e}, d} \\
 Wrong &: \texttt{\textbf{b}, c, e, \textbf{a}, f, d}
\end{align*}


\begin{figure}[ht]
  \begin{center}
    \begin{tikzpicture}[level/.style={sibling distance=30mm/#1}]      
      \node [square] (a) {A};
      \node [square, yshift=-4em, xshift=-2.5em] (b) {B};
      \node [square, yshift=-4em, xshift=2.5em] (d) {D};
      \node [square, yshift=-8em, xshift=-5em] (c) {C};
      \node [square, yshift=0em, xshift=12em] (e) {E};
      \node [square, yshift=-4em, xshift=12em] (f) {F};

      \draw[<-, thick,] (a) -- (b);
      \draw[<-, thick,] (b) -- (c);
      \draw[<-, thick,] (a) -- (d);
      \draw[<-, thick,] (e) -- (f);
    \end{tikzpicture}
  \end{center}
  \capt{Example of topologically sorted types.}
  \label{fig:topological}
\end{figure}



Another great advantage from topological sorting is the fact that it reveals
cycles in the graph. A cycle in the graph means an extend cycle between types
exists, e.g: A extends B, B extends C and C extends A, which is not accepted.

\subsection{AbstractTypeMarker}
\label{sec:abstractTypeMarker}
The interpreter needs to know whether or not a given type contains any abstract
members. Such a type is an abstract type and should not be allowed to be
instantiated. Marking these abstract types is now smooth. Due to the propagated
members it can just be checked whether or not any abstract members are present in the
given type. This check is done by the \classref{AbstractTypeMarker}. Any type
described by a \classref{TypeSymbolInfo} has a reference to the
\classref{AstNode} it was defined from. If the type is found to be an abstract
type, the type of the \classref{AstNode} is changed from TYPE\_DEF to
ABSTRACT\_TYPE\_DEF. This is the only way to make information visible to the
interpreter since the \classref{Interpreter} does not use the same
\classref{TypeTable} class used by the \classref{ScopeChecker}.

\subsection{TypeSuperCallChecker}
This checker ensures that any type that extends another type provides the right
amount of arguments when calling the parent's constructor. A constructor can have
$x$ arguments and may or may not contain a variable amount of additional
arguments. Consider the type constructor \texttt{Type A[\$var1, \$var2, \dots
\$varargs]}. When calling the constructor from another type, e.g. \texttt{Type
B[] extends A[5, 2, 7, 4]}, it must be checked that the type B provides \textit{at
least} the number of arguments in A's constructor (not counting the variable
amount of extra arguments). If A does not have a variable amount of additional
arguments, the argument count must match exactly. The implemented code for doing
this check can be seen in \lstref{lst:tscc}.

\lstinputlisting[caption={\emph{How the TypeSuperCallChecker is implemented.}},
label=lst:tscc, language=Java]{listings/typeSuperCallChecker.java}

\subsection{UsesAreDeclaredVisitor}
\label{sec:usesaredeclaredvisitor}
This visitor ensures that any use of a variable, constant, function, data
member, or type can be bound to a declaration. The visitor uses a variable
(\classref{TypeSymbolInfo} \varref{currentType}) which updates upon visiting a
TYPE\_DEF or an ABSTRACT\_TYPE\_DEF \classref{AstNode}, to keep track of which
type it is currently visiting inside. If the visitor is not traversing inside a
type (\classref{TypeSymbolInfo} \varref{currentType}) references a special type
called \varref{globalType}, which is used only to contain the standard- and game
environment as well as the global constants and functions declared in a
\productname{} game. 

It is important to realise that \varref{globalType} is not a super type of all
other types, it is a stand alone type that no type can derive from. Its name
contains an invalid character for a type name to ensure that no type can derive
from it. This becomes handy when checking if constants and functions used can be
bound to a declaration.

\subsubsection{Constants and functions}
When a constant or a function is referenced it is necessary to know two things
about the context in which it was referenced:

\begin{nlist}
  \item In what type did the reference occur?
  \item Is the reference a member access?
\end{nlist}

The first thing is easy to check since we have the \varref{currentType}
variable. This variable may however point to the global type. The structure of
the AST makes it easy to determine if it was a member access, since we would
have been visiting a MEMBER\_ACCESS \classref{AstNode} prior to the referenced
constant or function. In the expression: \texttt{A[].B.C[2,3]}, both B and C are
member accesses, but A is not. Given this information, a different check can be
done regarding to the context of the reference:

\begin{nlist}
  \item Type was global and a member access
  \begin{dlist}
    \item Must be visible in at least one type
  \end{dlist} 
  \item Type was global but not a member access
  \begin{dlist}
    \item Must be visible in the global scope
  \end{dlist}
  \item Type was A and a member access
  \begin{dlist}
    \item If prefixed by this, it must be visible in A or any super type of A
    \item If prefixed by super, it must be visible in any super type of A
    \item If prefixed by a variable name, it must be visible in at least one type
  \end{dlist}
  \item Type was A but not a member access: Must be visible in A, a super type
    of A or global scope
\end{nlist}

One may wonder why an accessed member is accepted if the accessed member is
visible in at least one type. Consider the member access \texttt{randomType.B}.
Here it is unknown in what type we shall look for the member B. The constant
\texttt{randomType} could literally return a random type, or the type returned
could be determined by an arbitrary complex algorithm. Therefore, we can only
enforce the rule that the member \texttt{B} must exist in at least one type.

%Skal nedenstående  afsnit med?

%One may think that it is also nice to know if a referenced constant or function
%has a number of parameters along with it and whether the actual number of
%arguments correctly matches the formal number of arguments. This is however
%quite hard to determine. In the example, if \texttt{randomConstant} was declared
%as a constant, the expression \texttt{randomConstant[2]} would still make sense if
%the constant returned a list, in which 2 was an index. This is however
%something the scope checker cannot look into. Given a function declared as
%\texttt{randomFunction[\$a, \$b] = \dots} the expression \texttt{let \$var =
%randomFunction in \dots} would also be correct, in which case \texttt{\$var} is just
%a reference to \texttt{randomFunction} in the \texttt{in}-scope. So it is valid to
%use a constant followed by a parameter list as well as it is valid to not apply
%a parameter list behind a function. 

\subsubsection{Variables}
For any variable, a declaration must always exist before it is used. A variable
can only declared in four ways:

\begin{dlist}
  \item As a type constructor
  \item As a formal parameter in a function declaration
  \item In a lambda expression
  \item In a \texttt{let-in} expression
\end{dlist}

In all cases the \productname{} semantics require that a new scope is opened, in
which the declared variable is known while the body of the expression is
executed. When the scope closes the declared variables are removed. The body of
an expression can also contain new variable declarations, e.g. a \texttt{let-in}
expression in the body of a \texttt{let-in} expression. 

The scope checker uses a \classref{SymbolTable} class which is basically a
symbol table with a list of variable names and a reference to a parent symbol
table.  The reference to the parent symbol table is exactly how the scopes
inside other scopes are implemented. 

\codesample{openscopeexpressions.junta}

Notice how the four code samples in in the above codesample all result in the same
scope checking routine, which can be seen in \figref{fig:scope1}; First, a new
symbol table is instantiated in which the variables \$a and \$b are put in. The
symbol table's parent reference is updated so it points to the current symbol
table, which is referred by \classref{SymbolTable} \varref{currentST}. Next, the
current symbol table is updated to the newly created symbol table, and the body
(the triple dots) are executed. Lastly, the current scope is closed, which
updates the current symbol table reference to point to the parent symbol table
of the current symbol table.  Notice that the symbol tables maintain a
stack-like structure, where opening a scope pushes a symbol table on the stack
and closing a scope pops one. The variable \classref{SymbolTable}
\varref{currentST} points to the element on top of the stack.

When a variable is used, it is checked that the variable exists in any of the
symbol tables by first looking in the current symbol table and recursively
following the parent reference until a null reference is found. If a variable
declaration with the same name as the used variable cannot be found, an error is
generated.

\fig[height=5em]{scope1}{Four different expressions that all result in the scope
action depicted.}

It is important to realise the reason for maintaining the stack-like structure
of symbol tables. It might seem like a single symbol table would be enough and
that all variable declarations could just be put in there. This is indeed wrong,
since the scope checker must also check for double declarations. A double
declaration exists if a symbol table contains the same variable twice. Notice
how \figref{fig:scope2} contains two symbol tables, each containing a
declaration of \$a. 

\fig[height=5em]{scope2}{The variable \$a declared in two different scopes.}

This is completely valid and is caused by the following codesample. If only a
single symbol table was used, an incorrect double declaration would be detected.

\codesample{scope2.junta}

\subsubsection{Data members}
When visiting a type body a new scope is opened and the data members of
\classref{TypeSymbolInfo} \varref{currentType} are immediately inserted into that
scope. The children of the type body is then visited and the scope is closed.
This ensures that the data members of a type can be used anywhere in the type
body but in that type body only. When exiting that type body and closing the
scope the symbol table containing the data members are no longer visible.

\subsection{Evaluation}
Many different static semantic checks are implemented in the scope checker.
Though many other checks could have been included as well, the scope of the
static semantics has been limited due to a few constraints. First of all, there
is a deadline for this project, and with an almost endless set of semantic
checks one can keep developing these checks. Furthermore, with new techniques
being discovered once in a while, a compiler or interpreter can simply not
include them all. A big set of the checks not included in \productname{}
requires type checking, which is cumbersome in a dynamic programming language.
However, it is generally possible to use type inference to find at least some of
the types and errors associated with the use of them. It is important to realise
that everything cannot always be inferred, for instance an algorithm could be so
complex that it would need to be executed to determine all possible outcomes.
Running the algorithm is not possible since you cannot know if the algorithm
will ever halt.
\cite[p. 173]{itttoc}

\section{Code generation \& Interpretation}
\todo{Merge with compilersandinterpreters.tex}
The following section presents a brief overview of the phases of translators (compilers and interpreters). The typical translator takes as input some given source code written in a language with a high level of abstraction and translates it into a language with lower abstraction e.g. machine code which can be executed directly by a computer \cite[p. 44]{sebesta2013}. Some translators work differently though. They translate the source code into another high-level language or into machine code for virtual machines, which can provide portability. The translation process is typically not a simple task, therefore it is often split into different phases, which is shown in \figref{fig:compileroverview}. The process can be split into more or less phases though, depending on how detailed one wished to describe the process. In this section we describe the following phases: the lexical analysis, the syntax analysis, the semantic analysis, the code generation and the interpretation.

\begin{figure}
	\begin{center}
		\scalebox{0.85}{
		\begin{tikzpicture}
  			[node distance=.6cm, start chain=going below,]
     		\node[punktchain, join] (scode) {Source code};
     		\node[punktchain, join] (leana) {Lexical analyser};
     		\node[punktchain, join] (syana) {Syntax analyser};
     		\node[punktchain, join] (seana) {Semantic analyser};
     		\node[punktchain, join] (cgen)  {Code generator};
     		\node[punktchain, join] (mach)  {Machine};
     
  			\draw[tuborg, decoration={brace}] let 
  				\p1=(leana.south), \p2=(syana.north) in
    			($(2, \y1)$) -- ($(2, \y2)$) node[tubnode] {Tokens};
  
  			\draw[tuborg, decoration={brace}] let 
  				\p1=(syana.south), \p2=(seana.north) in
    			($(2, \y1)$) -- ($(2, \y2)$) node[tubnode] {Parse tree};
  
  			\draw[tuborg, decoration={brace}] let 
  				\p1=(seana.south), \p2=(cgen.north) in
    			($(2, \y1)$) -- ($(2, \y2)$) node[tubnode] {Intermediate code};
  
 			\draw[tuborg, decoration={brace}] let 		
 				\p1=(cgen.south), \p2=(mach.north) in
    			($(2, \y1)$) -- ($(2, \y2)$) node[tubnode] {Machine language};
		\end{tikzpicture}}
	\end{center}
	\capt{The different phases of a compiler. Based on Sebesta \textit{et al.}\cite{sebesta2013} p. 46, Figure 1.3}
	\label{fig:compileroverview}
\end{figure}


\subsection{Intermediate language}

\subsection{Security}

\subsection{Just in Time compilation}

\subsection{Interpretation}
A pure interpretation of a program lies at the opposite end (from compilation) regarding to methods of implementation. With this approach, which can be see, on \figref{fig:compileroverviewinterpretation}, no translation is performed at all. An interpreter is interpreting a program written in the targeted language. It acts like a virtual machine which instructions are statements of high level language. By purely using interpretation, a source code debugger can easily be implemented. Various errors that might occur can once they are detected easily refer to which place in the source code that caused the error. The debugging is eased because the interpreter works like a software implementation of a virtual machine, thus the state of the machine and the value of a specific variable can be outputted at any time when requested. This will of course lead to the disadvantage that an interpreter uses more space than a compiler. Further more, the execution speed of an interpreter is usually 10 to 100 times slower than that of a compiler \cite[p. 48]{sebesta2013}.

\begin{figure}
	\begin{center}
		\scalebox{0.85}{
			\begin{tikzpicture}
  				[node distance=.8cm, start chain=going below,]
  				\node[punktchain, join,] (sprog) {Source program};
  				\node[punktchain, join,] (interp) {Interpreter};
  				\begin{scope}[start branch=venstre, every join/.style={->, thick, shorten <=1pt}, ]
  					\node[punktchain, on chain=going right, join=by {<-}] (indat) {Input data};
  				\end{scope}
  				\node[punktchain, join,] (res) {Result};
			\end{tikzpicture}}
		\end{center}
	\capt{The different phases of an interpreter. Based on Sebesta \textit{et al.}\cite{sebesta2013} p. 48, Figure 1.4}
	\label{fig:compileroverviewinterpretation}
\end{figure}


\subsection{Hybrid compilation \& interpretation}

The compiling or interpreting approach can be combined to form a hybrid implementation system. This method is illustrated in \figref{fig:compileroverviewhybrid}, where a program is compiled into an intermediate code which is then interpreted. By using this approach, errors in a program can be detected before interpretation which can save much time for a programmer. A great portability can also be achieved when using hybrid system. The initial implementation of Java was hybrid and allowed Java to be compiled to an intermediate code that could run on any platform which had an implementation of Java Virtual Machine\cite[p. 50]{sebesta2013}. 

\begin{figure}[ht]
  \begin{center}
    \scalebox{0.85}{
      \begin{tikzpicture}[node distance=.8cm, start chain=going below,]
        \node[punktchain, join,] (sprog) {Source program};
        \node[punktchain, join] (leana) {Lexical analyser};
        \node[punktchain, join] (syana) {Syntax analyser};
	\node[punktchain, join] (seana) {Intermediate code generator};
	\node[punktchain, join,] (interp) {Interpreter};
	
	\begin{scope}[start branch=venstre, every join/.style={->, thick, shorten <=1pt}, ]
	  \node[punktchain, on chain=going left, join=by {<-}] (indat) {Input data};
      	\end{scope}
  				
  	\node[punktchain, join,] (res) {Result};
  				
  	\draw[tuborg, decoration={brace}] let 
  	      \p1=(leana.south), \p2=(syana.north) in
    	      ($(1.2, \y1)$) -- ($(1.2, \y2)$) node[tubnode] {Token list};
  
  	\draw[tuborg, decoration={brace}] let 
  	      \p1=(syana.south), \p2=(seana.north) in
    	      ($(1.2, \y1)$) -- ($(1.2, \y2)$) node[tubnode] {Abstract syntax
	      tree};
  
  	\draw[tuborg, decoration={brace}] let 
  	      \p1=(seana.south), \p2=(interp.north) in
    	      ($(1.2, \y1)$) -- ($(1.2, \y2)$) node[tubnode] {Intermediate code};
      \end{tikzpicture}}
  \end{center}
  \capt{The different phases of a hybrid implementation systems. Based on
    Sebesta et al.\cite[p. 49, figure 1.5]{sebesta2013}}
  \label{fig:compileroverviewhybrid}
\end{figure}


\subsection{Summary}

\section{Simulator}
\label{sec:simulator-impl}

In this section, we begin by giving an overview of the relations of classes in
the Simulator in \secref{sec:overview}. Afterwards, we present
\classref{Widget}s and their properties in \secref{sec:widget}. We also explain
the different actions that are available in \secref{sec:actions}. Furthermore,
we explain how the different \classref{Widget}s communicate with each other in
\secref{sec:communication}. Lastly, we present how we have made the game
interactive in \secref{sec:interaction}, and how everything is connected in
\secref{sec:connection}.

%Overview
%Widget
%Propagated actions
%Communication between widgets
%Making games interactive
%Connecting everything

The Simulator has been constructed to provide a visual interface to the API
provided by the Game Abstraction Layer (GAL). The interface should give a visual
representation of the board and pieces, similar to how the board game would look
in real life. Furthermore, it should provide an easy way to interact with the
game. This interaction should be sufficient enough to make it possible to play
the game.

The result is a Java application with a graphical user interface, which takes a
\productname{} code file and makes it playable by the use of GAL.

\subsection{Overview}
\label{sec:overview}

The implementation is based around the class \classref{Widget}, which simplifies the
process of distributing drawings and handling input. Some \classref{Widget}s
manage other \classref{Widget}s, while others provide visual and interactive
content. The game \classref{Widget}s provide visual and interactive intefaces to
GAL.

To work with graphics and input, the game framework slick2d is used\cite{slick2d}.
\classref{SimulatedGame} is used to bind slick2d, \classref{Widget}s, and GAL
together to provide a complete system.

Figure \ref{fig:simulator-overview} shows all these components and their most
important relationships.

\fig[width=0.7\textwidth]{simulator-overview}{Generalised overview of classes in
the Simulator and their relations. Triangles means inheriance, while arrows means that one class makes use of the one it points to.}

\subsection{Widget}
\label{sec:widget}

\classref{Widget} specifies an object which can have a size and a position, can be
drawn, take mouse input, and send messages to each other. The most important
property of \classref{Widget} however is that a \classref{Widget} can contain
several sub-\classref{Widget}s which each can contain futher sub-\classref{Widget}s. We use this tree-structure to control how drawing and mouse input is handled.

\subsubsection{Placement}

A \classref{Widget} has a position specified with a $(x, y)$ coordinate. Its
position is relative to its parent, so if a \classref{Widget} has position
$(7, 10)$ and its parent has $(23, 50)$, its absolute position is $(30, 60)$.

Furthermore, a \classref{Widget} has a size specified with a width and height,
but it also contains an allowed range for each dimension. This allows us to
specify that a \classref{Widget} might be dynamic in size and can be adjusted if
wanted.

\subsubsection{Automatic placement and sizing}

Instead of setting sizes and positions manually, we create container classes
that manage the position and size of their sub-\classref{Widget}s. By using
\classref{Widget}s which are dynamic in size, we can create a layout which works
independent of the window and board size.

\classref{ScaleContainer} is such a container \classref{Widget} and positions
\classref{Widget}s along an axis. For \classref{Widget}s whose size is dynamic,
the remaining available space is distributed evenly amoung them. An example is
shown in \figref{fig:ScaleContainer}. The top-level \classref{Widget} is a
\classref{ScaleContainer} set to position \classref{Widget}s vertically. It does
not affect its own size, only its sub-\classref{Widget}s. The second
\classref{ScaleContainer} (containing two buttons, to be positioned
horizontally) is thus resized by the first \classref{ScaleContainer}. The
ordering of the sub-\classref{Widget}s determines the sequence they are
positioned in the \classref{ScaleContainer}.

\fig[width=0.7\textwidth]{ScaleContainer}{How \classref{ScaleContainers} (marked
with color) affects the positioning and resizing of sub-\classref{Widget}s.}

Creating a layout is now only a matter of building a hierarchy of
\classref{Widget}s, not deciding the exact position and size of each and every
single \classref{Widget}.

\subsection{Propagated actions}
\label{sec:actions}

Drawing a \classref{Widget} should not only draw the \classref{Widget}, but also
all its sub-\classref{Widget}s and their sub-\classref{Widget}s. To do this,
\classref{Widget} has two methods, \texttt{draw()} and \texttt{handleDraw()}.
\texttt{handleDraw()} needs to be overridden in sub-\classref{Widget}s that 
wants to provide a custom drawing method. \texttt{draw()} handles all the logic
for drawing sub-\classref{Widget}s, so the inherited class only needs to worry
about itself. An overview of \texttt{draw()} is given in
\lstref{lst:simulatorDraw}.

\lstinputlisting[caption={\emph{Pseudocode for the \texttt{draw()} method.}},
label={lst:simulatorDraw}, language=Java]{listings/simulatorDraw.java} 

\todo{looks wrong, draw() will never be called on the top-level
  \classref{Widget}.  Secondly, draw() should be called before handleDraw() }

To further ease development, the coordinate system is translated so
\texttt{handleDraw()} will be done using local coordinates instead of absolute
coordinates.  Furthermore, we apply clipping, so that any drawing outside the
\classref{Widget} will be clipped and not displayed. This way we can ensure that
\classref{Widget}s can't mess with other \classref{Widget}s.

We enforce this by restricting the \texttt{draw()} method with Java's final
keyword so it can't be overwridden, and \texttt{handleDraw()} is protected, so
the calling class can't call \texttt{handleDraw()} instead of \texttt{draw()} by
accident.

\subsubsection{Mouse input}

The same pattern is used for mouse input, but here we use it to determine which
\classref{Widget} is responsible for handling it. An overview is given in
\lstref{lst:simulatorMouseClicked}.

\lstinputlisting[caption={\emph{Pseudocode for the \texttt{mouseClicked()}
method.}}, label={lst:simulatorMouseClicked},
language=Java]{listings/simulatorMouseClicked.java}

Like with drawing, we translate coordinates into local coordinates, however
notice that it returns a boolean which is used to determine if the event was
handled, and it will stop as soon as any callee returns true. A second
difference is that in constrast to drawing, input is handled bottom up. The
reasoning is that the lower we get in the hierarchy, the more specific the
behaviour of each \classref{Widget} is. Thus, we try to see if the more specific
\classref{Widget}s will handle the input and if not, less and less specific
\classref{Widget}s are tried.

When mouse buttons are pushed and released, it will only try \classref{Widget}s
which contain the position at which the mouse is currently pointioned at. For
mouse dragging the situation is different, it will try any \classref{Widget}
which has initiated a drag, even if the mouse has moved outside it. If this was
not the case, a scrollbar for example would only move if we kept the mouse
exactly on top of it, which usually is tricky as they are long and slim.

\subsection{Communication between Widgets}
\label{sec:communication}

Consider the case where a \classref{Widget} represent a button. The user might
click on it, but the button by itself is not interested in what this should
signify. Thus, we need some way of notifying \classref{Widget}s that some events
have happened inside other \classref{Widget}s. For this, the Observer design
pattern is used in \classref{Widget}.

\subsection{Making games interactive}
\label{sec:interaction}

Two "game \classref{Widget}s" which interact with GAL are used to present the
game to the user. They are \classref{GameInfoWidget}s which provide information
like move history, and \classref{BoardWidget} which displays an interactive
board with pieces based on GAL.

\subsubsection{BoardWidget}

For interaction, \classref{BoardWidget} supports selecting \classref{Action}s by
the use of either Drag\&Drop or Click\&Select. While Drag\&Drop only allows
you to move a \classref{Piece} from one \classref{Square} to another, Click\&Select 
will work on any two \classref{Square}s whether or not it contains any
\classref{Piece}s. It will go through all available \classref{Action}s and find
the ones which are related to those Squares. To help ease this process, usable
\classref{Square}s are hinted as shown in \figref{fig:multiple-moves}.

\fig[width=0.7\textwidth]{multiple-moves}{The Square at E2 was selected and shows the 4 pieces which can move there.}

In reality we have two \classref{Widget}s: \classref{BoardWidget} and
\classref{GridBoardWidget} which are specialisations of \classref{BoardWidget}.
While it is not necessary at this point, it is an attempt to generalise
\classref{GirdBoard} so that a future addition with new \classref{Board} types
will be easier to implement.

\subsection{Binding everything together}
\label{sec:connection}

The class \classref{SimulatedGame} has the responsibility to connect slick2d
with the \classref{Widget} structure, and GAL with the game \classref{Widget}s.

\classref{SimulatedGame} contains one \classref{ScaleContainer}, which it resizes
to fit the whole window and tells it to adjust the sizes of its
sub-\classref{Widgets}. Secondly, it sends all mouse-events to this
\classref{Widget} and draws it whenever slick2d wants to be redrawn.

On construction of \classref{SimulatedGame}, it reads the \productname{} code file
and attempts to load it through GAL. It then creates the game
\classref{Widget}s, but it does not pass a reference to the game directly. As
the game object changes each time an \classref{Action} is applied, which
requires us to update it in every game object each time. Instead we pass a
reference to this instance of \classref{SimulatedGame} and the game
\classref{Widget}s must
then access the game object through its accessor methods directly, without
caching it.

One final task of \classref{SimulatedGame} is to handle any exceptions in GAL or
the \classref{Simulator} and show them to the user, without the application
crashing.

\section{Summary of essential decisions}
\label{sec:summaryofdecisions}

Based on this chapter we have made the following decisions which our continued
work will focus on.

When talking about crafting a parser we have chosen to craft an LL($1$)-parser
by hand and also generate a LALR($1$)-parser with SableCC.  The reason for this
decision is the fact that this project is meant as a learning experience and
therefore knowing how the parsing phase works is important. By both crafting a
parser by hand and by the use of a generator tool we will get to know how a
well-known parser generator works as well as construct our own customised
parser.

We also presented the four main paradigms of programming languages and we will
focus on combining the functional and object-oriented programming paradigms.
These paradigms make great sense for a programming language that focus on board
game programming. The reason for this is specified in \secref{sec:paradigms}.

We will focus on crafting an interpreter rather than a compiler. This makes
sense especially because we have decided to make a game simulator. By having an
interpreter rather than a compiler, the game programmer will be able to make
changes in the code and see them visualised in the simulator right away without
having to recompile the whole program.

The decision of including a simulator is based on the assumption that board game
programmers wish to have their games visualised and quickly be able to play and
test them.

This chapter has set the foundation for a list of requirements for our
programming language, which can be seen in the following chapter.

