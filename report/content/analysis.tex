\chapter{Analysis}
\label{chap:analysis}
\todo{Rewrite so it fits the new structure of the analysis chapter}

%In the following chapter we analyse some different board games, namely kalah and chess. This is done to gain a better understanding of which elements and components they contain, which will be useful knowledge when designing our programming language. The analysis of the games can be found in \secref{sec:chessandkalah}. Furthermore, we investigate programming paradigms in \secref{sec:paradigms}. We need to know about the existing paradigms to be able to make a good decision, on which paradigm(s) our own language should be based on. In the chapter we also analyse the phases of compilation and interpretation. We give an overview of their phases, which is seen in \secref{sec:compileroverview}, and after that we dig deeper into the details of the parsing phase where we describe what a context free grammar is, in \secref{sec:cfg}, and we look at the advantages and disadvantages of different kind of parsers and different ways of constructing parsers in \secref{sec:parsers}. In the end of the chapter a list of requirements to our programming language is presented together with a problem statements. These are the results of the chapter and will work as the foundation for the further work of the project.

\section{What is a board game?}
\label{sec:board-game-analysis}

One may wonder what a board game really is. Could it just be any game containing
some kind of a board? If so, would Trivial Pursuit be a board game and what
about the game Twister, where you have to place your hands and/or feet on a spot
marked with a particular colour on a sheet -- or board, as you could call it. Most
people have a mental model of a board game that does not include games like
Twister. Here is one definition of a board game:

\begin{quote}
  ``A board game is a game played across a board by two or more players. The
  board may have markings and designated spaces, and the board game may have
  tokens, stones, dice, cards, or other pieces that are used in specific ways
  throughout the game.''\cite{def-board-game}
\end{quote}

The definition above is very broad and will to some extent allow a game like
Twister to be categorized as a board game. All kinds of artefacts like cards and
dice can be part of a board game, but one board game designer may also be able
to invent a new and yet unseen widget he wants to include in his board
game. A programming language that makes it possible to describe any board will
cover a very broad category of games. You could argue that it would actually
cover all games that can be made, since even a first-person shooter could
technically be played across a board. With such a broad definition a
programming language that aims to make the programming of board games easier,
will likely have to be a general-purpose programming language. If a programming
language is aimed to make the programming of only a specific kind of board games
easier, there might be many things that can be expressed easily in that language
compared to how it would have been done in existing general-purpose programming
languages.

To define what elements are to be included in our programming language
(\productname{}), we believe it is essential to look at some existing board
games.  For this reason we have analysed two well-known board games. We have
investigated the game elements that might cause trouble, or be clumsy, or not
straightforward to implement in common general-purpose programming languages. In
the analysis of the games a list of game elements is presented that respects all
of the elements which we found interesting from the games. The aim of
\productname{} is to ease the programming of these game elements.


\section{Analysis of Chess and Kalah}
\label{sec:chessandkalah}

In the following sections we will analyse the two games: Chess and Kalah. The
reason for picking these two games in particular is because they are
universially known games which we think contain some very fundamental board game
elements that we need to know about to gain a better understanding of which
features are needed in \productname{}.

\subsection{Chess}
Chess is a board game of two oppenent players. It's a turn-based game which means one player makes a move, 
then the other player makes a move, then the first player makes a move and so on. Chess is played on a board of $8 \times 8$ squares. The squares are typically black and white, but can be any two colors (see figure \ref{fig:chess}). The squares can only contain one piece at a time, unlike games like Mancala and Backgammon. Each player has a total of 16 pieces: 8 pawns, 2 knights, 2 bishops, 2 rooks, a queen and a king. Each type of piece has unique ways to move. For instance a pawn can move only one square vertically forward or one square diagonal when capturing an enemy piece. A rook can move unlimited squares either forward or backward (vertical movement), or to the right or to the left (horisontal movement). This separates chess from a lot of other common board games where all pieces have the same abilities, like Naughts and Crosses, Mancala, Ludo, Backgammon.  

Cut to the bone Chess goes as follow: When a game starts the pieces are in their starting positions as seen in figure \ref{fig:chess}. The player with the white pieces always makes the first move, and after that the players shifts in turn in which clever moves are beeing taken and pieces are beeing captured until one player has checkmated the other - and the game is over. The checkmate condition is obtained when the king piece is in a position to be captured and cannot escape from capture in the next move. \cite{chessrules}. Therefore it's nessecary to look one move ahead to control if the checkmate condition is optained.

Special moves. In chess there are numerus special moves which doesn't follow the normal pattern of chess. Earlier we mentioned that a pawn can move only one square vertically forward or one square diagonal when capturing an enemy piece. But this is not always true. If the pawn is in it's respective starting position it can move either one or two squares vertically forward. After that it can only move one square forward or one square vertically the rest of the game. Another special move is the move called ``castling''. This move allows a player to move two pieces in one turn (the king and one of the rooks). But to do the move several conditions needs to be met. First: the move has to be the very first move of the king and the rook, second: there can't be any pieces standing between the king and the rook and third: there can't be any opposing pieces that could capture the king in his original square, the squares he moves through or the square he end up in \cite{chessrules}. There exits two more special moves which are called ''En Passant´´ and ''Promotion´´. These are not going to described here, but information about them can be found in \cite{chessrules}. So what is the problem with these special moves? The problem is the fact that they don't follow the regular pattern of the game and this has to be taken into consideration when designing \productname{}.

\fig[scale=0.1]{chess}{The board game chess with the pieces in start position.}

From the above analysis here is a list of interesting game elements we found in chess:
\begin{itemize}[noitemsep]
\item Pieces has different movement abilities.
\item A squared board with a number of squares in it.
\item A winning condition - when the king has been checkmated.
\item A starting state - how the pieces are placed on the board before the game's very first move.
\item Special moves like ``castling'', ``Promotion'' and ``En Passant.
\item Constraints that disallows a piece to move if some condition is true after the move has been made (a move that sets your king in check).
\item A piece can be ``captured'' by another piece, which causes the piece to be removed from the board.
\end{itemize}

\subsection{Kalah}

Kalah is like chess, a turn-based game of two opposing players. The Kalah pieces,
called seeds, are very different from the Chess pieces. They do not have
specific moves but rather functions, as their name also suggest, as seeds. The
board is not like the Chess board either. It consists of 14 squares, sometimes
referred to as houses,\cite{kalahrules} with two of the houses separating
themselves from the rest by being the houses or bases of each of the players.
Furthermore, each player has six houses belonging to them (see
\figref{fig:kalah}). Each house (including the players' houses) can contain an
arbitrary number of seeds, unlike in chess where the squares can only contain
one piece.  

Cut to the bone, Kalah goes as follow; When the game starts, each of the 12
houses contain 4 seeds (in some versions of the game each house contains 5 or 6
seeds) and the player bases are empty. Now the players take turn to pick up
piles of seeds and deal them out to the 12 houses and their own base. The
dealing of seeds works by a player picking up a pile of seeds from one of his
six houses and dropping one seed down in each of the following houses moving
counter-clockwise. If, when dealing out seeds, the player lands in a house
belonging to himself (not including his own base), which is not empty, the
player can pick up the pile of seeds in the house and start dealing these out.
The turn shifts once a player, when dealing out seeds, lands in an empty house
or in one of the opponent's houses. For a more detailed description of the
rules, we refer to \cite{kalahrules}. The game is over once one of the players'
six houses are empty and the winner is the one who has the most seeds in their
base.

\fig[width=0.25\textwidth]{kalah}{The board game Kalah.}

\subsubsection{Special moves}
Like in Chess there are some special moves in Kalah which don't follow the
regular pattern of the Kalah game. We are not going to describe them here, but
they will be present in the list of interesting game elements and can be found
in a detailed description in \cite{kalahrules}.

Here is a list of some interesting game elements we found in Kalah. For
simplicity we are going to refer to houses and bases as squares and refer to
seeds as pieces:

\begin{dlist}
  \item Squares can contain an arbitrary number of pieces
  \item Making a move can be considered as simple as choosing a square
  \item The number of pieces on a square determines how long a move you can make
  \item A turn may contain more than one move
    \begin{dlist}
      \item If the last piece is dropped in a non-empty square, the player can
	make another move
    \end{dlist}
  \item A square can belong to a player
  \item Squares can be related to other squares
    \begin{dlist}
      \item You place pieces on squares counter-clockwise
      \item If you place the last piece on an empty square, the square across
	the board belonging to the opponent is emptied over to your own square
    \end{dlist}
  \item An end game condition - when all of the squares belonging to a player are
    empty
  \item A winning condition - the player with the most pieces in their square
    wins when the end game condition has been met
  \item Only one type of piece
\end{dlist}
  

\section{Summary}
From looking at the 3 board games, Chess, Kalah and Naughts \& Crosses, many different game elements have been recognised. For a programming language that allows all of the 3 games to be described, all of the game elements in each of those must could be described. Some of the elements are very similar, in which case a more generalised description has been provided. For example, the rook piece and the bishop piece in chess do not have equal moves but their moves have been generalised to just \textit{movement by patterns}. The general game elements in the earlier mentioned board games can bee seen here:
\begin{itemize}
\item The game has one single board.
\item The board contains squares in a 2-dimensional grid.
\item The game contains one or more types of pieces.
\item The game has an initial setup.
\item A piece can be even on the board or off the board. If a piece is on the board it is also on one specific square.
\item A piece can belong to a player.
\item A pattern may consider the position of occupied squares, empty squares and other pieces on the board in relation to a specific square or a specific piece on the board.
\item A piece has a set of moves which might be an empty set. The possible moves may be based on a pattern.
\item At any given time, just one player has the turn.
\item If a piece is owned by a player, only that player can move it.
\item A turn may consist of more than one move.
\item A player can win if the game is in one or more specific states, which can consider a pattern.
\item A game can be a tie if the game is in one or more specific states, which can consider a pattern.
\end{itemize}


\section{Paradigms}

A programming paradigm describes a method and style of computer programming.
Some of the primary paradigms are imperative, object-oriented, functional and declarative programming. While some programming languages strictly follow one paradigm, there are many so-called multi-paradigm languages, that implement several paradigms and therfore allow multiple styles of programming. Examples of multi-paradigm languages include C\# and Java.

\subsection{Overview}
The four main paradigms are described as follows:
\begin{description}
\item[Imperative programming] describes computation in terms of statements that change the program state. Primary characteristics are assignments, procedures, data structures, control structures. Imperative programming can be seen as a direct abstraction of how most computers work, and many imperative lnaguages are just abstractions of assembly language. Typical examples of imperative languages are C and Fortran.
\item[Object-oriented programming] describes computation in terms of objects described by attributes manipulated through methods. Primary charcteristics are objects, classes, methods, encapsulation, polymorphism, inheritance. An example of a pure object-oriented languages is Smalltalk, while many other languages are either primarily designed for object-oriented programming (such as Java and C\#) or have support for object-oriented programming (such as PHP and Perl).
\item[Declarative programming] describes computational logic without describing control flow, i.e. describing {\em what} a program does rather than {\em how} it does it. Many domain-specific languages such as SQL, HTML and CSS are declarative.  
\item[Functional programming] describes computation in terms of mathematical functions and seeks to avoid program state and mutable data. Purely functional functions have no side effects, and the result is constant in relation to the parameters (e.g. $add 2 4$ always returns $6$). An example of a purely function programming languages is Haskell. Other examples of languages designed for functional programming are Erlang, F# and Lisp, while it is possibly to apply functional programming concepts to many other languages.
\end{description}

\subsection{Summary}

While general-purpose languages, such as C\# and Java, usually tend to lean towards the imperative and object-oriented paradigms, a domain-specific language, with very specific goals in design, may benefit from other paradigms, e.g. declarative programming.
Blabla

%\section{Context-Free Grammars}
\label{sec:cfg}
Context-free grammars (CFGs) are used to specify the syntactical structure of programming languages. Throughout the report we will be using context-free grammars to represent our programming language. In this section we define what a context-free grammar is. 

A context-free grammar is a collection of substitution rules (or productions) constituted by nonterminals and terminals that describe the construction of a language. Any language that can be generated by a CFG is called a context-free langauge. A CFG is a 4-tuple $(V, \Sigma, R, S)$:\cite[p. 100]{itttoc}

\begin{itemize}[noitemsep]
\item $V$ is a finite set called the nonterminals
\begin{itemize}[noitemsep]
\item To create a string, we think of the nonterminals as variables
\end{itemize}
\item $\Sigma$ is a finite set, disjoint from V, called terminals
\begin{itemize}[noitemsep]
\item The terminals cannot be the same as the nonterminals
\item This set can be thought of as the alphabet
\end{itemize}
\item $R$ is a finite set of productions
\begin{itemize}[noitemsep]
\item Each production consist of a nonterminal and a string of nonterminals and terminals
\item The nonterminal and the string is seperated by an arrow or a |
\end{itemize}
\item $S \in V$ is the start symbol (a nonterminal)
\begin{itemize}[noitemsep]
\item This is the first nonterminal at the left-most top of the grammar
\end{itemize}
\end{itemize} 

In the following section we give an example of a CFG.

\subsection{Production}
It is easier to understand what a CFG really is by showing an example. The following is an example of a CFG, which we call \textit{acb}:

\begin{center}
	\begin{ebnf}
		\grule{A}{aAb}
		\galt{B}
		\grule{B}{c}
	\end{ebnf}
\end{center}

In the example we have two nonterminals; A and B, and three terminals; a, b, and c. The production for nonterminal A states that A can derive the string ``aAb'' or the string ``B'', and the nonterminal B can derive the string ``c''. When a nonterminal is present in a string they are substituted with their own production. For instance the string \textit{aaacbbb} can be derived from the CFG we called \textit{acb}. It is not really possible to create a lot of different strings with \textit{acb}. It is only possible to create strings with an equal amount of a's and b's with a c in between them.

\subsection{Derivation}
The sequence of substitutions needed to obtain a string from the CFG is called a derivation.\cite[p. 100]{itttoc} A derivation of the above given string \textit{aaacbbb} is:

\[ 
A \Rightarrow aAb \Rightarrow aaAbb \Rightarrow aaaAbbb \Rightarrow aaaBbbb \Rightarrow aaacbbb 
\]

The derivation starts with the start symbol (which is the left hand side of the production) which is substituted with its substitution rule (which is the right hand side of the production). The nonterminals are substituted until there are no more left. When the string only contains terminals the derivation is complete.

Now that we have an understanding for CFGs we can go on to the next topic. The following sections will cover the topic about parsers; what they are and how they are generated.
%\section{Parser}
\label{sec:parserimplementation}

In this section we present the handwritten parser of \productname{}'s.
We have written a top-down recursive descent parser, which is within the class
of LL(1) parsers. The grammar for \productname{} is suited for this because
e.g.\ it does not have left-recursive productions. In the end of the section we present our work 
with SableCC and the reason why we chose not to continue working with this tool.

\subsection{Constructing the parser}
%structured as the grammar
The parser was very simple to implement, because it is structured exactly the same way as the
grammar which can be found in \chapref{ap:CFG}.  For instance if the grammar
expresses that the next set of terminals must begin with a left bracket (`['),
  then the parser will expect the next token to be a \tokenref{LBRACKET} which
  is the token name for a left bracket. If the grammar then expects a
  non-terminal, then the parser simply calls the method for that non-terminal,
  allowing it to finish, possibly calling more non-terminals and expecting
  terminals, before continuing parsing the next part of the rule.

%discuss the if expression
In \lstref{lst:ifexpr} we give an example of how this structure looks like in
our handwritten parser. The production rule for an if-expression is presented in
\secref{sec:conditionalexpressions}.

%\begin{ebnf}
%\grule{if\_expr}{\gter{if} \gcat expression \gcat \gter{then} \gcat expression
%\gcat \gter{else} \gcat expression}
%\end{ebnf}

The production for if-expression says that every expression of this type
must start with the combination of the two symbols which spell the word
\gter{if}. When the parser meets this word in an expression, it knows
that it has to parse an if-expression.

\lstinputlisting[caption="How if-expressions are parsed using top-down parsing in Java.",
label=lst:ifexpr, language=Java]{listings/ifexpr.java}

\subsection{Building an abstract syntax tree}
%astNode()
In \lstref{lst:ifexpr}, the parser initialises the node for the
expected if-expression. The parser starts by calling the method
\methodref{astNode} to create a node for the Abstract Syntax Tree (AST).
We call the method with information about what type of expression this
is (\tokenref{IF\_EXPR}). The method calls the \methodref{expect} method
to verify that the next token is what we are expecting. If the two
tokens do not match, the parser throws a syntax error with information
about the error. If everything is syntactically correct the parser
constructs a node for the AST for the given expression. The first child
of the node is the boolean expression, and the next two siblings of that
child are the expression branches of the if-expression.

\subsubsection{Terminal and nonterminals}
Every grammar has a finite set of nonterminals and terminals that
constitute the productions of the grammar. We have defined tokens in the
parser for every nonterminal in our grammar. The if-expression has the
token name of \tokenref{IF\_EXPR}.

In the production for the if-expression, we have three terminals: the \gter{if},
\gter{then}, and the \gter{else}. These are all required in the method for any
if-expression. When the parser finishes reading a terminal, it knows that the
following token will be an expression, and therefore a new child for the node is
made with a call to the \methodref{expression} method wherein we parse
expressions. Finally the method returns the node containing every child for the
whole if expression.

\subsection{Looking ahead in the input}
%lookAhead methods - atomic
We mentioned earlier that the parser is an LL(1) parser, which means that the
parser is able to look ahead in the sequence of tokens. We have shown the
\methodref{lookAhead} method to determine if the next token is part of an
atomic expression. The production for the atomic expression is presented in
\secref{sec:atomicexpressions}.

%\begin{ebnf}
%\grule{atomic}{\gter{(} \gcat expression \gcat \gter{)}}
%\galt{variable}
%\galt{list}
%\galt{\gter{/} \gcat pattern \gcat \gter{/}}
%\galt{\gter{this}}
%\galt{\gter{super}}
%\galt{direction}
%\galt{coordinate}
%\galt{integer}
%\galt{string}
%\galt{type}
%\galt{constant}
%\end{ebnf}

An atomic expression can derive quite a few productions. This is why we have
constructed a specific method to determine whether the next token is part of an
atomic expression. This method is shown in \lstref{lst:lookaheadatomic}.

\lstinputlisting[caption="The lookAhead method to determine if the next
  expression is an atomic type.", label=lst:lookaheadatomic,
language=Java]{listings/method_lookAheadAtomic.java}

The method \methodref{lookAheadAtomic} makes use of two methods to figure out if
the next token is part of an atomic expression. The first method is the
\methodref{lookAhead} method that takes a token as an argument and figures out
if the next token in the sequence of tokens are equal to each other. The second
method is the \methodref{lookAheadLiteral} method which is similar to the method 
in \lstref{lst:lookaheadatomic} but instead of checking for atomic expressions it 
checks for literals. All these methods return true or false.

%example of lookAheadAtomic
%LL(1)
In \lstref{lst:examplelookahead} we show an example of how the
\methodref{lookAhead} method is used in the parser. The example is taken from
the \methodref{expression} method. The productions for expressions are presented
in \secref{sec:expressions}.
The production of an expression is reflected in the code of the parser. An
example of this is given in \lstref{lst:examplelookahead}.

%\begin{ebnf}
%\grule{expression}{assignment}
%\galt{if\_expr}
%\galt{lambda\_expr}
%\galt{\gter{not} \gcat expression}
%\galt{operation}
%\end{ebnf}

\lstinputlisting[caption="Use of the \methodref{lookAhead}-method. This example
is from the \methodref{expression}-method.", label=lst:examplelookahead,
language=Java]{listings/example_lookAheadAtomic.java}

The code presented in \lstref{lst:examplelookahead} is a small section of the
\methodref{expression} method. We have removed code from the section which is
not relevant for the example we are trying to give. The removed code is
presented as \{\ldots\}. In \lstref{lst:examplelookahead} we wish to present how
the \methodref{lookAhead} methods are used.

An assignment begins with the reserved word \gter{let} and the first
\methodref{lookAhead} method peeks for exactly that token to determine
if the next production is an assignment. If the method returns true then
the next token is in fact the \gter{let} word, and the parser enters a
new method, namely the \methodref{assignment} method which checks to
determine if the rest of the production is correctly written. The same
is done for the if expression, lambda expression and operations which
begins with the ``loSequence()'' (logical operators).

The operation production is a bit different, because it needs two lookAhead
methods to determine if the next production is an operation. An operation can
begin with either an atomic value or a minus operator. So the code uses a
\methodref{lookAheadAtomic} and a regular \methodref{lookAhead} with the
specific token as a parameter to check if the next production is an operation.

The methods return nodes which are connected with each other to form
a complete AST\@. When the parser has parsed every token of the input,
it can produce an AST that corresponds to the program written in
\productname{}. This shows that the parser is built systematically
according to the grammar, producing a parse tree consisting of AST
nodes.

\subsection{SableCC}
We have also implemented a scanner and parser using a
compiler/interpreter generator, or a compiler compiler
known as SableCC\cite{sableccdoc}. As described in
\secref{subsec:generatedparsers}, it is an automated scanner and
LALR($1$) parser generator written in Java, with support for making
compilers and interpreters. We have implemented an early version of
\productname{} in SableCC to evaluate the capabilities of such a tool.

\subsubsection{Choice of SableCC}
We chose SableCC instead of various other popular tools such as
ANTLR\cite{antlr} and JavaCC\cite{javacc}. Though ANTLR and JavaCC are far more
well-documented than SableCC, we still chose this tool because it's parser
generates a LALR($1$) parser, whereas ANTLR and JavaCC create simpler
LL($k$) parsers, which our hand-written implementation already takes advantage of.
Therefore, we felt LALR($1$) parsing to be more interesting, since it
also supports more powerful grammars (such as left-recursion).

SableCC outputs an abstract syntax node type fokokr each alternation in
every rule in the grammar specifications file. It's then possible to
iterate over these nodes via extending the visitor pattern SableCC also
supplies, generating code or directly interpreting a syntactically
correct program. This is all done in classes separate from the grammar
specifications, which is also desirable and different from ANTLR
and JavaCC, where action code is injected directly in the grammar
specification. This is all done in Java, which is also desirable, since
it would work well with the rest of the project (also written in Java).

\subsubsection{Experience with SableCC}
Our experience with the tool has been rather cumbersome, in that it took
quite a while to read the documentation before and during writing the
specifications, as it simply isn't just copy/pasting EBNF grammar
into a file. An example of the if-expression rule is seen here

\begin{lstlisting}[caption={Part of the grammar specifications file of SableCC, with focus on if-expressions.}]
Tokens
  else          = 'else';
  then          = 'then';
  if            = 'if';

Productions
  expression   = {elopexp} element operator expression
               | {assign} assignment
               | {if} if_expr
               | {lambda} lambda_expr
               | {el} element list?
               | {not} not expression;
  if_expr      = if [left]:expression then [mid]:expression else [right]:expression;
\end{lstlisting}

% Pros
This example brings out the strengths of SableCC, as it looks very
similar to the EBNF for if-expressions, with a few additions. As long as
you know the special syntax and how helpers, tokens, and productions works, it
is possible to create scanners and parsers very quickly. This was not
our case, as no one had experience with any form of compiler-compilers.
Given that it generates a LALR parser, it grants the ability to have
more powerful grammar than what we had designed. Lastly, the fact that
there's a clear and clean separation between automated, generated code
and user code makes the grammar and compilation/interpretation parts
easier to maintain. When adding new features to the language, you simply
have to update the specifications file and generate a new scanner/parser
combination. On the other hand when adding new features to the language while
using a handwritten scanner and parser many lines of codes needs to be change
in order to implement the new feature. 

% Cons
Even though SableCC looks like a prime candidate to continue
interpretation with, we chose not to use the tool. This is because it
took an unreasonable amount of time to figure out how to precisely
define the grammar to keep it from being ambiguous. And with poor
documentation, it took even longer. Also, it offers less control and
customizability, compared to writing our own from scratch. As an
example, the tool offers an application-specific interface to tree
walking the AST nodes with the visitor pattern, requiring knowledge of
how SableCC implements it. SableCC also generates around 17000 lines of
Java code, even for our simple grammar, which seems superfluous compared
to the handwritten code, consisting of around 1500 lines of code.
%  - Don't learn as much about different parser techniques
%  - Old project, not as active anymore

% WHAT TO CALL THIS SUBSUBSECTION?
\subsubsection{Discussion}
We chose not to continue using SableCC on our updated grammar, due
to the weight of cons against pros, and the fact that the time spent
working on implementing SableCC was also spent making the handwritten
scanner and parser and making them work exactly the way we want them to. It might
not be as easy to modify our language with this solution, but the time
spend on modifying and adding features to our language with the use of a
handwritten scanner and parser is not wasted time, but learning time, which
gives us better understanding of their underlying functionality.  


%\section{Generating a parser}
The principle of generating parsers is very systematic and therefore there are different automated tools to generate parsers for a specific grammar that meets some standards. A grammar must for instance not be ambiguous otherwise the tools cannot make a distinct parser for the grammar. 

\subsection{Handwritten parser}
Recursive descent parser (LL).

\subsection{Generated parsers}
There are quite a few automated parser generaters (compiler compilers). Theoretically, the only thing the developer must do is input the grammar into the generater, and it will output a parser for that specific grammar.

\subsubsection{SableCC}
Generates a LALR(1) parser.

\subsubsection{JavaCC}
Generates a LL(k) parser.

\subsection{Summary}
pros and cons. 
%\section{Compiler and interpreter}
Along with the design of the programming language \productname{}, we also want
to make it possible for programmers who write games in \productname{} to
actually play them afterwards. 
There are a number of ways we can make that happen. It can be translated to
platform dependant machine instruction using a compiler, or it can be parsed and
executed on-the-fly using an interpreter.

\subsection{Compilation}
With compilation, an executable file would be created for a specific platform
which contains all the code required to play the game.  Since games have common
aspects, a game engine containing all the common aspects such as user interface,
AI and/or network play would most likely be written.  This engine would then be
included directly in the executable.

An obvious disadvantage is that the executable is platform dependant and it
would therefore be necessary to develop a new compiler for each platform we want
to support.  On the other hand knowing the specific platform makes it possible
to create optimized code which runs faster.

Instead of compiling to native machine code, it could be compiled to an
intermediate format such as Java bytecode which is supported on many platforms.
While Java bytecode is interpreted and therefore slower, modern interpreters
uses sophisticated methods such as Just-In-Time compilation (JIT) which at
runtime compiles intermediate code into native machine code.  This process of
course adds an overhead, however the speed differences are not that great
anymore.\cite{java-speed}

\subsection{Interpretation}
An interpreter takes the original source code directly to parse and execute it
in one step.  This separates the game code from the forehand mentioned engine
and requires the end user to get both the interpreter and the actual game.
Different games written in our language would then use the same copy of the
interpreter, instead of having a copy of the engine for each executable. This
separation will be further explored in \secref{subsec:engineseperation}.

The execution speed will however suffer and while techniques such as JIT exists
to improve this, it is beyond the scope of this project. \todo{can we justify
this?}

The execution speed is not critical however. 
Processors nowadays are fast and executing tasks such as calculating moves would
most likely finish so fast that one wouldn't be able to notice the difference
between a compiled and a interpreted version.
One place where speed does become important is in an artificial intelligence
(AI), which would be used to create an virtual opponent controlled by the
computer. 
The virtual opponent becomes harder to beat the more turns it can look ahead,
however the computational complexity is high and a doubling of speed is quite
noticeable when the time to make a move is reduced from 2 minutes to one.

An inherent consequence of interpretation is that the original source code is
available for everyone which obtains the game. This can discourage developers
which intends to sell their game, as it is easy for everyone obtaining a copy to
make derivatives of it. Others will find it as an advantage as they can fix
errors, add new gameplay elements or use it as a base for a completely new game.

\subsection{Separation of game and engine}
\label{subsec:engineseperation}
Keeping the game and the engine separated opens up for the possibility of
changing the game engine while still being able to use the same game. 

One major advantage is that it is possible to update the engine and in result,
update all your games.  An update which improves the graphics or add new
features such as network support would work with older games instantly, without
having to wait for the developer to update it.  If the developer no longer
maintains the game, a updated version might never come out.

The disadvantage with this is however that the responsibility for maintaining
compatibility is moved from the developer of the game to the developers of the
engine. A game developer can simply change his program so it works with a new
engine, however the engine developers would have to support games written for
every version released.

\subsubsection{Compiled plug-in}
It is possible to achieve this using compilation too.  The game could be
compiled to a plug-in, which the engine loads dynamically.

\subsection{Security}
When compiling to a native instruction set, we have access to every instruction
on that platform, also potentially unsafe instructions. Even though our
language does not include features which makes use of those instructions, it is
possible to use code injection on the compiled code to make it execute any code.
This way you could create a trojan horse, which appears to be a normal game but
might do malicious actions in the background.  For example, it could randomly
delete files from the users document folder each time he won.

With interpretation we define ourselves which instructions exists and therefore
can chose only to include instructions which we know are safe.  Even if we
decide to allow certain questionable actions, since it is not executed directly
on the CPU and instead goes through our interpreter, we can provide a sand-boxed
environment which restricts the actions to only allow a safe subset.  For
example, we might provide access to the file system, but only allow file
deletion in the games own directory.

\subsection{Intermediate format}
A middle step between compiling and interpretation is to compile to a
intermediate format which is then interpreted. The intermediate language could
be more low-level which would make it possible to optimize the code for higher
efficiency.

The intermediate format could be stored as an archive file which contains not
only the code, but also sounds, images and other resources required to play the
game. This would allow for easy distribution of a game in \productname{}. The
source code would not be available like with a compiled game, however a package
format could allow to optionally include the original source if the developer
wants to share.

Using an intermediate format however means that you need to create a compiler,
an interpreter and the intermediate language, which in turn is a significant
larger amount of work.

\subsection{Summary}
A compiler can make faster code, however an interpreter allows us to fine-tune
security considerations.  Separating the game engine and game code gives a
significant improvement for both methods.  Creating an intermediate language can
give some advantages over a purely interpreted language, however it also
requires more work to develop.

%\section{Code examples}




\section{Syntactic analysis}

In the following section we describe and analyse the syntatic analysis phase of compilers. The syntatic analysis is performed by the component or algorithm called a parser. A parser checks whether or not the source code of a given application is set up syntactically correct according to the programming language it is written in. All programming languages have rules for how it's tokens can be combined (we will describe these in \secref{sec:lexicalanalysis}). 
A common way to describe these rules is by a formal language-generation mechanism called grammars or context-free grammars. Context-free grammers are described in this section in \secref{sec:context-freegrammars}. There exists two main approaches to performing syntatic analysis, one is top-down parsing from which the family of LL(k) parsers derives from (the k defines the number of lookahead tokens). The other is bottom-up parsing from which the family of LR(k) parser derives from. In this section we analyse how the LL-parsers and LR-parsers work and we look at the advantages and the disadvantages of each by comparing them. 
furthermore we look at different methods for making parsers, more specifically we analyse the pros and cons against writing the parser by hand versus using parser generator tools to generate it.

\subsection{Lexical analysis}
\label{sec:lexicalanalysis}
Before the syntax analysis can be performed a scanner must perform a lexical
analysis. The scanner's job is basically to check a given source code for lexical
errors and translating the input stream of characters from the source code into
a stream of tokens which the parser can work with, this
is done by identifying every lexeme in the source and attaching a potential
token to it.
\cite[p. 57]{fischer2009}

Lexemes are strings of characters described by regular expressions.  Typical
examples of lexemes in a programming language are: variable names, integer
literals, operators and special keywords etc. A variable name lexeme could be
defined by the following regular expression: $[a-z, A-Z, "\_"][a-z, A-Z, 0-9,
"\_"]^*$. Which means a variable name can start with either an uppercase letter,
lowercase letter or an underscore followed by zero or more lowercase letters,
uppercase letters, numbers or underscores. In \tableref{table:lexandtokens} we
give examples of lexemes and the tokens they have been paired with. If both
\textit{a} and \textit{b} are lexemes describing variable names and \textit{102}
and \textit{42} are lexemes describing integer literals, then \textit{a} and
\textit{b} or \textit{102} and \textit{42} can typically be used interchangeably
and still give a syntatic meaningful program.

\tab[10cm]{lexandtokens}{1}{Lexemes and their corresponding token group.}
		             {               }
       {Lexemes             }{\textbf{Tokens} } {
\tabrow{ x                  }{VAR\_NAME       } 
\tabrow{ random\_var\_name  }{VAR\_NAME       }
\tabrow{ RANdom\_var\_name2 }{VAR\_NAME       }
\tabrow{ 1 		    }{INT\_LITERAL    }
\tabrow{ 342 		    }{INT\_LITERAL    }
\tabrow{ 52890 		    }{INT\_LITERAL    }
\tabrow{ +		    }{PLUS\_OPERATOR  }
\tabrow{ - 		    }{MINUS\_OPERATOR }
\tabrow{ * 		    }{MULT\_OPERATOR  }
\tabrow{ if		    }{KEYWORD         }
\tabrow{ while 		    }{KEYWORD  	      }
\tabrow{ switch 	    }{KEYWORD  	      }
}

A scanner is a relatively simple component which can be constructed by writing
it by hand or using a scanner-generating tool such as Lex, which generates an
executable scanner by feeding it with a set of regular expressions. When
implementing the scanner for \productname{}, we would likely benefit more from
crafting the scanner by hand than by using a scanner-generating tool. By
crafting it by hand we will know excatly how it is implemented. There might be
some advantages of using a scanner-generating tool such as the fact that it is
more reliable, it is easier to maintain and it is faster to implement if one
already knows how it works, if not, a handwritten scanner might be just as fast. 

\subsection{Context-free grammars}
\label{sec:context-freegrammars}
In the previous section we described how to transfrom an input stream of
characters into an output stream of tokens. We will now describe
Grammars are defined using Backus-Naur Form (BNF).

BNF contains a set of terminals and a set of non-terminals. The terminals are
the tokens from the lexical analysis. The non-terminals all have a set of
productions, from which a mix of terminals and non-terminals can be derived
from. A start production specifies a single non-terminal, from where all
syntactically valid strings that are in the language can be derived from by
using the production rules until only a sequence of terminals (the tokens) are
left. The syntax analysis takes a sequence of tokens as input and tries to
create a set of derivations from the start symbol that creates the given
sequence of tokens. If success, the input has been parsed and the parse tree is
kept for later analysis. The parse tree is the information concerning how the
start symbol was derived into the sequence of tokens, which yields a tree
structure. This tree is called an abstract syntax tree.

\begin{ebnf}
%Expressions
\grule{program}{\gter{print} \gcat expr}
\grule{expr}{\gter{(} \gcat term \gcat \gter{)} \gcat operator \gcat \gter{(}
\gcat term \gcat \gter{)}}
\grule{operator}{\gter{=}}
\galt{\gter{>}}
\galt{\gter{<}}
\grule{term}{number}
\galt{expr}
\grule{number}{\textbf{any number}}
\end{ebnf}

\section{Parser}
\label{sec:parserimplementation}

In this section we present the handwritten parser of \productname{}'s.
We have written a top-down recursive descent parser, which is within the class
of LL(1) parsers. The grammar for \productname{} is suited for this because
e.g.\ it does not have left-recursive productions. In the end of the section we present our work 
with SableCC and the reason why we chose not to continue working with this tool.

\subsection{Constructing the parser}
%structured as the grammar
The parser was very simple to implement, because it is structured exactly the same way as the
grammar which can be found in \chapref{ap:CFG}.  For instance if the grammar
expresses that the next set of terminals must begin with a left bracket (`['),
  then the parser will expect the next token to be a \tokenref{LBRACKET} which
  is the token name for a left bracket. If the grammar then expects a
  non-terminal, then the parser simply calls the method for that non-terminal,
  allowing it to finish, possibly calling more non-terminals and expecting
  terminals, before continuing parsing the next part of the rule.

%discuss the if expression
In \lstref{lst:ifexpr} we give an example of how this structure looks like in
our handwritten parser. The production rule for an if-expression is presented in
\secref{sec:conditionalexpressions}.

%\begin{ebnf}
%\grule{if\_expr}{\gter{if} \gcat expression \gcat \gter{then} \gcat expression
%\gcat \gter{else} \gcat expression}
%\end{ebnf}

The production for if-expression says that every expression of this type
must start with the combination of the two symbols which spell the word
\gter{if}. When the parser meets this word in an expression, it knows
that it has to parse an if-expression.

\lstinputlisting[caption="How if-expressions are parsed using top-down parsing in Java.",
label=lst:ifexpr, language=Java]{listings/ifexpr.java}

\subsection{Building an abstract syntax tree}
%astNode()
In \lstref{lst:ifexpr}, the parser initialises the node for the
expected if-expression. The parser starts by calling the method
\methodref{astNode} to create a node for the Abstract Syntax Tree (AST).
We call the method with information about what type of expression this
is (\tokenref{IF\_EXPR}). The method calls the \methodref{expect} method
to verify that the next token is what we are expecting. If the two
tokens do not match, the parser throws a syntax error with information
about the error. If everything is syntactically correct the parser
constructs a node for the AST for the given expression. The first child
of the node is the boolean expression, and the next two siblings of that
child are the expression branches of the if-expression.

\subsubsection{Terminal and nonterminals}
Every grammar has a finite set of nonterminals and terminals that
constitute the productions of the grammar. We have defined tokens in the
parser for every nonterminal in our grammar. The if-expression has the
token name of \tokenref{IF\_EXPR}.

In the production for the if-expression, we have three terminals: the \gter{if},
\gter{then}, and the \gter{else}. These are all required in the method for any
if-expression. When the parser finishes reading a terminal, it knows that the
following token will be an expression, and therefore a new child for the node is
made with a call to the \methodref{expression} method wherein we parse
expressions. Finally the method returns the node containing every child for the
whole if expression.

\subsection{Looking ahead in the input}
%lookAhead methods - atomic
We mentioned earlier that the parser is an LL(1) parser, which means that the
parser is able to look ahead in the sequence of tokens. We have shown the
\methodref{lookAhead} method to determine if the next token is part of an
atomic expression. The production for the atomic expression is presented in
\secref{sec:atomicexpressions}.

%\begin{ebnf}
%\grule{atomic}{\gter{(} \gcat expression \gcat \gter{)}}
%\galt{variable}
%\galt{list}
%\galt{\gter{/} \gcat pattern \gcat \gter{/}}
%\galt{\gter{this}}
%\galt{\gter{super}}
%\galt{direction}
%\galt{coordinate}
%\galt{integer}
%\galt{string}
%\galt{type}
%\galt{constant}
%\end{ebnf}

An atomic expression can derive quite a few productions. This is why we have
constructed a specific method to determine whether the next token is part of an
atomic expression. This method is shown in \lstref{lst:lookaheadatomic}.

\lstinputlisting[caption="The lookAhead method to determine if the next
  expression is an atomic type.", label=lst:lookaheadatomic,
language=Java]{listings/method_lookAheadAtomic.java}

The method \methodref{lookAheadAtomic} makes use of two methods to figure out if
the next token is part of an atomic expression. The first method is the
\methodref{lookAhead} method that takes a token as an argument and figures out
if the next token in the sequence of tokens are equal to each other. The second
method is the \methodref{lookAheadLiteral} method which is similar to the method 
in \lstref{lst:lookaheadatomic} but instead of checking for atomic expressions it 
checks for literals. All these methods return true or false.

%example of lookAheadAtomic
%LL(1)
In \lstref{lst:examplelookahead} we show an example of how the
\methodref{lookAhead} method is used in the parser. The example is taken from
the \methodref{expression} method. The productions for expressions are presented
in \secref{sec:expressions}.
The production of an expression is reflected in the code of the parser. An
example of this is given in \lstref{lst:examplelookahead}.

%\begin{ebnf}
%\grule{expression}{assignment}
%\galt{if\_expr}
%\galt{lambda\_expr}
%\galt{\gter{not} \gcat expression}
%\galt{operation}
%\end{ebnf}

\lstinputlisting[caption="Use of the \methodref{lookAhead}-method. This example
is from the \methodref{expression}-method.", label=lst:examplelookahead,
language=Java]{listings/example_lookAheadAtomic.java}

The code presented in \lstref{lst:examplelookahead} is a small section of the
\methodref{expression} method. We have removed code from the section which is
not relevant for the example we are trying to give. The removed code is
presented as \{\ldots\}. In \lstref{lst:examplelookahead} we wish to present how
the \methodref{lookAhead} methods are used.

An assignment begins with the reserved word \gter{let} and the first
\methodref{lookAhead} method peeks for exactly that token to determine
if the next production is an assignment. If the method returns true then
the next token is in fact the \gter{let} word, and the parser enters a
new method, namely the \methodref{assignment} method which checks to
determine if the rest of the production is correctly written. The same
is done for the if expression, lambda expression and operations which
begins with the ``loSequence()'' (logical operators).

The operation production is a bit different, because it needs two lookAhead
methods to determine if the next production is an operation. An operation can
begin with either an atomic value or a minus operator. So the code uses a
\methodref{lookAheadAtomic} and a regular \methodref{lookAhead} with the
specific token as a parameter to check if the next production is an operation.

The methods return nodes which are connected with each other to form
a complete AST\@. When the parser has parsed every token of the input,
it can produce an AST that corresponds to the program written in
\productname{}. This shows that the parser is built systematically
according to the grammar, producing a parse tree consisting of AST
nodes.

\subsection{SableCC}
We have also implemented a scanner and parser using a
compiler/interpreter generator, or a compiler compiler
known as SableCC\cite{sableccdoc}. As described in
\secref{subsec:generatedparsers}, it is an automated scanner and
LALR($1$) parser generator written in Java, with support for making
compilers and interpreters. We have implemented an early version of
\productname{} in SableCC to evaluate the capabilities of such a tool.

\subsubsection{Choice of SableCC}
We chose SableCC instead of various other popular tools such as
ANTLR\cite{antlr} and JavaCC\cite{javacc}. Though ANTLR and JavaCC are far more
well-documented than SableCC, we still chose this tool because it's parser
generates a LALR($1$) parser, whereas ANTLR and JavaCC create simpler
LL($k$) parsers, which our hand-written implementation already takes advantage of.
Therefore, we felt LALR($1$) parsing to be more interesting, since it
also supports more powerful grammars (such as left-recursion).

SableCC outputs an abstract syntax node type fokokr each alternation in
every rule in the grammar specifications file. It's then possible to
iterate over these nodes via extending the visitor pattern SableCC also
supplies, generating code or directly interpreting a syntactically
correct program. This is all done in classes separate from the grammar
specifications, which is also desirable and different from ANTLR
and JavaCC, where action code is injected directly in the grammar
specification. This is all done in Java, which is also desirable, since
it would work well with the rest of the project (also written in Java).

\subsubsection{Experience with SableCC}
Our experience with the tool has been rather cumbersome, in that it took
quite a while to read the documentation before and during writing the
specifications, as it simply isn't just copy/pasting EBNF grammar
into a file. An example of the if-expression rule is seen here

\begin{lstlisting}[caption={Part of the grammar specifications file of SableCC, with focus on if-expressions.}]
Tokens
  else          = 'else';
  then          = 'then';
  if            = 'if';

Productions
  expression   = {elopexp} element operator expression
               | {assign} assignment
               | {if} if_expr
               | {lambda} lambda_expr
               | {el} element list?
               | {not} not expression;
  if_expr      = if [left]:expression then [mid]:expression else [right]:expression;
\end{lstlisting}

% Pros
This example brings out the strengths of SableCC, as it looks very
similar to the EBNF for if-expressions, with a few additions. As long as
you know the special syntax and how helpers, tokens, and productions works, it
is possible to create scanners and parsers very quickly. This was not
our case, as no one had experience with any form of compiler-compilers.
Given that it generates a LALR parser, it grants the ability to have
more powerful grammar than what we had designed. Lastly, the fact that
there's a clear and clean separation between automated, generated code
and user code makes the grammar and compilation/interpretation parts
easier to maintain. When adding new features to the language, you simply
have to update the specifications file and generate a new scanner/parser
combination. On the other hand when adding new features to the language while
using a handwritten scanner and parser many lines of codes needs to be change
in order to implement the new feature. 

% Cons
Even though SableCC looks like a prime candidate to continue
interpretation with, we chose not to use the tool. This is because it
took an unreasonable amount of time to figure out how to precisely
define the grammar to keep it from being ambiguous. And with poor
documentation, it took even longer. Also, it offers less control and
customizability, compared to writing our own from scratch. As an
example, the tool offers an application-specific interface to tree
walking the AST nodes with the visitor pattern, requiring knowledge of
how SableCC implements it. SableCC also generates around 17000 lines of
Java code, even for our simple grammar, which seems superfluous compared
to the handwritten code, consisting of around 1500 lines of code.
%  - Don't learn as much about different parser techniques
%  - Old project, not as active anymore

% WHAT TO CALL THIS SUBSUBSECTION?
\subsubsection{Discussion}
We chose not to continue using SableCC on our updated grammar, due
to the weight of cons against pros, and the fact that the time spent
working on implementing SableCC was also spent making the handwritten
scanner and parser and making them work exactly the way we want them to. It might
not be as easy to modify our language with this solution, but the time
spend on modifying and adding features to our language with the use of a
handwritten scanner and parser is not wasted time, but learning time, which
gives us better understanding of their underlying functionality.  


\section{Generating a parser}
The principle of generating parsers is very systematic and therefore there are different automated tools to generate parsers for a specific grammar that meets some standards. A grammar must for instance not be ambiguous otherwise the tools cannot make a distinct parser for the grammar. 

\subsection{Handwritten parser}
Recursive descent parser (LL).

\subsection{Generated parsers}
There are quite a few automated parser generaters (compiler compilers). Theoretically, the only thing the developer must do is input the grammar into the generater, and it will output a parser for that specific grammar.

\subsubsection{SableCC}
Generates a LALR(1) parser.

\subsubsection{JavaCC}
Generates a LL(k) parser.

\subsection{Summary}
pros and cons. 


\section{Contextual constraints}
\label{sec:staticsemantics}

The \classref{ScopeChecker} is the class responsible for enforcing some of the
static semantic rules of \productname{} (described in \chapref{chap:design}) at
compile time. This section aims to explain how these static semantic checks
are performed by the \classref{ScopeChecker} in simple sequential steps. Any
error detected by the \classref{ScopeChecker} will cause a \classref{ScopeError}
exception to be thrown. This error contains helpful information about the type
of error and where in the input program the error is located. The checking of
static semantics as well as the interpretation of the code both use the visitor
pattern which is a commonly used approach for both purposes.

After the AST has been created, the visitor pattern allows us to traverse the
AST and execute encapsulated pieces of code for each specific type of
\classref{AstNode}. 

\subsection{TypeVisitor}
The first visitor used is the \classref{TypeVisitor}. This visitor traverses the
AST, finds all type definitions in the input program and for each type
definition an object of class \classref{TypeSymbolInfo} is instantiated. After
running the \classref{TypeVisitor}, each type definition in the input program
has an associated object of class \classref{TypeSymbolInfo} which makes it easy
to get information about any declared type in the input program by accessing the
following members contained in the \classref{TypeSymbolInfo} object:

\begin{dlist}
  \item name (\classref{String}): The type's name
  \item parentName (\classref{String}): The name of its super type (null if
    not a derived type)  
  \item args (\classref{Integer}): The number of arguments in the type's
    constructor
  \item parentArgs (\classref{Integer}): The number of arguments given in the
    call to its parent constructor
  \item data (\classref{List of Data}): Each data defined in the type body has a
    corresponding \classref{Data} object describing its name and position in the
    input program
  \begin{dlist}
    \item The input program position is stored as a line and an offset and makes
      it possible to produce error messages with information about where in the
      input program an error was found
  \end{dlist}
  \item members (\classref{List of Member}): Each constant and function defined
    in the type body has a corresponding \classref{Member} object describing its
    name, argument count, an abstract flag, a \classref{TypeSymbolInfo}
    reference to the type defining the \classref{Member} and a pointer to the line
    and offset in the code
  \item node (\classref{AstNode}): A reference to the
    TYPE\_DEF-\classref{AstNode} which defines the type
  \begin{dlist}
    \item This reference is used to get the input program position where the
      type was defined (for generating useful error messages), but is also used
      for marking abstract type definitions (described in detail in
      \secref{sec:abstractTypeMarker}) 
  \end{dlist}
  \item parent (\classref{TypeSymbolInfo}): This will contain an object
    reference to its parent type if it has one
  \begin{dlist}
    \item This is however a null pointer until running the
      \classref{TypeParentRefMaker} described in \secref{sec:TypeParentRefMaker}
  \end{dlist} 
\end{dlist}

All the \classref{TypeSymbolInfo} objects are kept in an object of class
\classref{TypeTable}. The \classref{TypeTable} class is a layer of abstraction
which provides easy and fast access information about the types contained in the
input program. The underlying implementation is a hashmap from the type's name
as a \classref{String} to its object reference, which provides a quick way to map a type's name
to the \classref{TypeSymbolInfo} object that represents the type. One use of this mapping is 
described in \secref{sec:TypeParentRefMaker}. Another feature of \classref{TypeTable} is a 
convenient way to iterate over all the \classref{TypeSymbolInfo}s. This is for instance used
to topological sort the types, which is described in \secref{sec:TypeMemberPropagator}.
For convenience, we say that a type is added to a type table which means that a
\classref{TypeSymbolInfo} object representing the type is added to the
\classref{TypeTable} object representing the type table.

When a type is added to the type table, it is checked that no other types with
the same name exist.

\subsection{TypeParentRefMaker}
\label{sec:TypeParentRefMaker}
The \classref{TypeSymbolInfo} objects only contain the name of their super type
as a \classref{String} or a null value if there is no super type. By making a
lookup in the \classref{TypeTable} on the parent name, the real object
references can be found and stored for faster and more convenient parent lookups
which is used greatly by the visitor described in 
\secref{sec:usesaredeclaredvisitor}.

\subsection{TypeMemberPropagator}
\label{sec:TypeMemberPropagator}
Some of the later checks that will be performed requires us to determine whether
or not a type member (constant or function) with a specific name is visible in a
given type. This requires searching in the given type and recursively in all
super types for the member. When this kind of lookup is done many times on the
same member, the traversal of the same long chains of parent references become
inefficient which results in clumsy code. To simplify and speed up this process
the \classref{TypeMemberPropagator} ensures that all members of a type A are
also present in a type C, if A is a super type of C. With this approach,
checking if a member is visible in type C, only requires looking in C instead of
following the chain parent types.

This propagation of members is done by first doing a topological sort on the
\classref{TypeSymbolInfo} objects, such that when iterating over the type table,
any type yielded will always appear before all of its subtypes. This makes the
afterwards propagation of members possible in linear time by iterating over the
topologically sorted types. If a type C is met, the members in its parent type B
are put in C as well. If B has a parent A, we know that B has already the
members from A due to the topological sorting.

The topological sorting is done using the algorithm that goes by the same name, from the book Introduction
to Algorithms \cite[p. 612]{ad} working on a graph $G = (V, E)$. Each type
represents a vertex $v$. An edge $e$ exist from $v_1$ to $v_2$ if the type $v_2$
is a parent of the type $v_1$. However this results in a topological order where
a type always appears before its super types. This problem is quickly solved
simply by reversing the list.

Given the graph in \figref{fig:topological}, the following sequences are examples
of correct and wrong topologically sorted orders after the list has been reversed:

\begin{align*}
 Correct &: \texttt{a, d, b, e, c, f} \\
 Correct &: \texttt{a, b, e, c, f, d} \\
 Wrong &: \texttt{a, b, c, \textbf{f}, \textbf{e}, d} \\
 Wrong &: \texttt{\textbf{b}, c, e, \textbf{a}, f, d}
\end{align*}


\begin{figure}[ht]
  \begin{center}
    \begin{tikzpicture}[level/.style={sibling distance=30mm/#1}]
      \node [square] (c) {C};
      \node [square, yshift=-4em, xshift=2.5em] (b) {B};
      \node [square, yshift=-4em, xshift=7.5em] (d) {D};
      \node [square, yshift=-8em, xshift=5em] (a) {A};
      \node [square, yshift=-2em, xshift=12em] (f) {F};
      \node [square, yshift=-6em, xshift=12em] (e) {E};

      \draw[->, thick,] (c) -- (b);
      \draw[->, thick,] (b) -- (a);
      \draw[->, thick,] (d) -- (a);
      \draw[->, thick,] (f) -- (e);
    \end{tikzpicture}
  \end{center}
  \capt{Example of topologically sorted types. An edge goes from type X to type Y if X is a subtype of Y.}
  \label{fig:topological}
\end{figure}



Another great advantage from topological sorting is the fact that it reveals
cycles in the graph. A cycle in the graph means an extend cycle between types
exists, e.g: A extends B, B extends C and C extends A, which is not accepted.

\subsection{AbstractTypeMarker}
\label{sec:abstractTypeMarker}
The interpreter needs to know whether or not a given type contains any abstract
members. Such a type is an abstract type and should not be allowed to be
instantiated. Marking these abstract types is now smooth. Due to the propagated
members it can just be checked whether or not any abstract members are present in the
given type. This check is done by the \classref{AbstractTypeMarker}. Any type
described by a \classref{TypeSymbolInfo} has a reference to the
\classref{AstNode} it was defined from. If the type is found to be an abstract
type, the type of the \classref{AstNode} is changed from TYPE\_DEF to
ABSTRACT\_TYPE\_DEF. This is the only way to make information visible to the
interpreter since the \classref{Interpreter} does not use the same
\classref{TypeTable} class used by the \classref{ScopeChecker}.

\subsection{TypeSuperCallChecker}
This checker ensures that any type that extends another type provides the right
amount of arguments when calling the parent's constructor. A constructor can have
$x$ arguments and may or may not contain a variable amount of additional
arguments. Consider the type constructor \texttt{Type A[\$var1, \$var2, \dots
\$varargs]}. When calling the constructor from another type, e.g. \texttt{Type
B[] extends A[5, 2, 7, 4]}, it must be checked that the type B provides \textit{at
least} the number of arguments in A's constructor (not counting the variable
amount of extra arguments). If A does not have a variable amount of additional
arguments, the argument count must match exactly. The implemented code for doing
this check can be seen in \lstref{lst:tscc}.

\lstinputlisting[caption={\emph{How the TypeSuperCallChecker is implemented.}},
label=lst:tscc, language=Java]{listings/typeSuperCallChecker.java}

\subsection{UsesAreDeclaredVisitor}
\label{sec:usesaredeclaredvisitor}
This visitor ensures that any use of a variable, constant, function, data
member, or type can be bound to a declaration. The visitor uses a variable
(\classref{TypeSymbolInfo} \varref{currentType}) which updates upon visiting a
TYPE\_DEF or an ABSTRACT\_TYPE\_DEF \classref{AstNode}, to keep track of which
type it is currently visiting inside. If the visitor is not traversing inside a
type (\classref{TypeSymbolInfo} \varref{currentType}) references a special type
called \varref{globalType}, which is used only to contain the standard- and game
environment as well as the global constants and functions declared in a
\productname{} game. 

It is important to realise that \varref{globalType} is not a super type of all
other types, it is a stand alone type that no type can derive from. Its name
contains an invalid character for a type name to ensure that no type can derive
from it. This becomes handy when checking if constants and functions used can be
bound to a declaration.

\subsubsection{Constants and functions}
When a constant or a function is referenced it is necessary to know two things
about the context in which it was referenced:

\begin{nlist}
  \item In what type did the reference occur?
  \item Is the reference a member access?
\end{nlist}

The first thing is easy to check since we have the \varref{currentType}
variable. This variable may however point to the global type. The structure of
the AST makes it easy to determine if it was a member access, since we would
have been visiting a MEMBER\_ACCESS \classref{AstNode} prior to the referenced
constant or function. In the expression: \texttt{A[].B.C[2,3]}, both B and C are
member accesses, but A is not. Given this information, a different check can be
done regarding to the context of the reference:

\begin{nlist}
  \item Type was global and a member access
  \begin{dlist}
    \item Must be visible in at least one type
  \end{dlist} 
  \item Type was global but not a member access
  \begin{dlist}
    \item Must be visible in the global scope
  \end{dlist}
  \item Type was A and a member access
  \begin{dlist}
    \item If prefixed by this, it must be visible in A or any super type of A
    \item If prefixed by super, it must be visible in any super type of A
    \item If prefixed by a variable name, it must be visible in at least one type
  \end{dlist}
  \item Type was A but not a member access: Must be visible in A, a super type
    of A or global scope
\end{nlist}

One may wonder why an accessed member is accepted if the accessed member is
visible in at least one type. Consider the member access \texttt{randomType.B}.
Here it is unknown in what type we shall look for the member B. The constant
\texttt{randomType} could literally return a random type, or the type returned
could be determined by an arbitrary complex algorithm. Therefore, we can only
enforce the rule that the member \texttt{B} must exist in at least one type.

%Skal nedenstående  afsnit med?

%One may think that it is also nice to know if a referenced constant or function
%has a number of parameters along with it and whether the actual number of
%arguments correctly matches the formal number of arguments. This is however
%quite hard to determine. In the example, if \texttt{randomConstant} was declared
%as a constant, the expression \texttt{randomConstant[2]} would still make sense if
%the constant returned a list, in which 2 was an index. This is however
%something the scope checker cannot look into. Given a function declared as
%\texttt{randomFunction[\$a, \$b] = \dots} the expression \texttt{let \$var =
%randomFunction in \dots} would also be correct, in which case \texttt{\$var} is just
%a reference to \texttt{randomFunction} in the \texttt{in}-scope. So it is valid to
%use a constant followed by a parameter list as well as it is valid to not apply
%a parameter list behind a function. 

\subsubsection{Variables}
For any variable, a declaration must always exist before it is used. A variable
can only declared in four ways:

\begin{dlist}
  \item As a type constructor
  \item As a formal parameter in a function declaration
  \item In a lambda expression
  \item In a \texttt{let-in} expression
\end{dlist}

In all cases the \productname{} semantics require that a new scope is opened, in
which the declared variable is known while the body of the expression is
executed. When the scope closes the declared variables are removed. The body of
an expression can also contain new variable declarations, e.g. a \texttt{let-in}
expression in the body of a \texttt{let-in} expression. 

The scope checker uses a \classref{SymbolTable} class which is basically a
symbol table with a list of variable names and a reference to a parent symbol
table.  The reference to the parent symbol table is exactly how the scopes
inside other scopes are implemented. 

\codesample{openscopeexpressions.junta}

Notice how the four code samples in in the above code sample all result in the same
scope checking routine, which can be seen in \figref{fig:scope1}; First, a new
symbol table is instantiated in which the variables \$a and \$b are put in. The
symbol table's parent reference is updated so it points to the current symbol
table, which is referred by \classref{SymbolTable} \varref{currentST}. Next, the
current symbol table is updated to the newly created symbol table, and the body
(the triple dots) are executed. Lastly, the current scope is closed, which
updates the current symbol table reference to point to the parent symbol table
of the current symbol table.  Notice that the symbol tables maintain a
stack-like structure, where opening a scope pushes a symbol table on the stack
and closing a scope pops one. The variable \classref{SymbolTable}
\varref{currentST} points to the element on top of the stack.

When a variable is used, it is checked that the variable exists in any of the
symbol tables by first looking in the current symbol table and recursively
following the parent reference until a null reference is found. If a variable
declaration with the same name as the used variable cannot be found, an error is
generated.

\fig[height=5em]{scope1}{Four different expressions that all result in the scope
action depicted.}

It is important to realise the reason for maintaining the stack-like structure
of symbol tables. It might seem like a single symbol table would be enough and
that all variable declarations could just be put in there. This is indeed wrong,
since the scope checker must also check for double declarations. A double
declaration exists if a symbol table contains the same variable twice. Notice
how \figref{fig:scope2} contains two symbol tables, each containing a
declaration of \$a. 

\fig[height=5em]{scope2}{The variable \$a declared in two different scopes.}

This is completely valid and is caused by the following code sample. If only a
single symbol table was used, an incorrect double declaration would be detected.

\codesample{scope2.junta}

\subsubsection{Data members}
When visiting a type body a new scope is opened and the data members of
\classref{TypeSymbolInfo} \varref{currentType} are immediately inserted into that
scope. The children of the type body is then visited and the scope is closed.
This ensures that the data members of a type can be used anywhere in the type
body but in that type body only. When exiting that type body and closing the
scope the symbol table containing the data members are no longer visible.

\subsection{Summary of scope checking}
Many different static semantic checks are implemented in the scope checker.
Though many other checks could have been included as well, the scope of the
static semantics has been limited due to a few constraints. First of all, there
is a deadline for this project, and with an almost endless set of semantic
checks one can keep developing these checks. Furthermore, with new techniques
being discovered once in a while, a compiler or interpreter can simply not
include them all. A big set of the checks not included in \productname{}
requires type checking, which is cumbersome in a dynamic programming language.
However, it is generally possible to use type inference to find at least some of
the types and errors associated with the use of them. It is important to realise
that everything cannot always be inferred, for instance an algorithm could be so
complex that it would need to be executed to determine all possible outcomes.
Running the algorithm is not possible since you cannot know if the algorithm
will ever halt.
\cite[p. 173]{itttoc}

\section{The phase of interpretion and generation of code}
\label{sec:codegenerationandinterpretation}

In this section we present a brief overview of the phases of translators. Along
with the design of the programming language \productname{}, we also want to make
it possible for programmers who write games in \productname{} to actually play
them afterwards. There are a number of ways we can make that happen. These are
presented in the folllowing sections. 

We begin by presenting the translation process followed by a presentation of
what an intermediate language (IL) is and what it is good for. In the subsequent
sections we present compilation and interpretation and their differences. Then
we present a hybrid solution based upon compilation and interpretation combined.
Lastly, we present what it will mean to have the game and engine seperate,
followed by a short summary.

%\subsection{The translation process}
%\label{sec:translationprocess}
%The typical translator takes as input some given source code written in a
%language with a high level of abstraction and translates it into a language with
%lower abstraction e.g. machine code which can be executed directly by a
%computer.
%\cite[p. 44]{sebesta2013} 

%Some translators work differently though. They translate the source code into
%another high-level language or into machine code for virtual machines, which can
%provide portability. The translation process is typically not a simple task,
%therefore it is often split into different phases, which is shown in
%\figref{fig:compileroverview}. The process can be split into more or less phases
%though, depending on how detailed one wished to describe the process. In this
%section we describe the following phases: the lexical analysis, the syntax
%analysis, the semantic analysis, the code generation and the interpretation.

\subsection{Intermediate language}
\label{sec:intermediatelanguage}
A middle step between compiling and interpretation is to translate source code
to an IL which is then interpreted further. The IL could be more low-level than
the initial source code which would make it possible to optimize the code for
higher efficiency.

The intermediate format could be stored as an archive file which contains not
only the code, but also sounds, images and other resources required to play a game. 
This for instance could make it easier to distribute games in \productname{}. The
source code would not be available like with a compiled game, however a package
format could allow to optionally include the original source if the developer
wanted to share.

Using an intermediate format however means that you need to create a compiler,
an interpreter, and the IL, which in turn requires a significant larger amount of work.

\subsection{Compilation}
\label{sec:compilation}
With compilation an executable file is created for a specific platform which
contains all the code required to play the game. Since games have common
aspects, a game engine containing all the common aspects such as user interface,
AI and/or network connection would most likely be written. This engine would
then be included directly in the executable.

The translation process is typically not a simple task, therefore it is often
split into different phases, which is shown in \figref{fig:compileroverview}.
The process can be split into more or less phases though, depending on how
detailed one wished to describe the process. In \figref{fig:compileroverview}
the lexical and syntax analyser make a lookup in a symbol table. Then the
semantic analyser and the code generator use the symbol table to generate the
correct code. Optimisation is optional in the phase of semantic analysis.
\cite[p. 46]{sebesta2013}

\section{Compiler overview}
\label{sec:compileroverview}
The following section presents a brief overview of the phases of translators (compilers and interpreters). The typical translator takes as input some given source code written in a language with a high level of abstraction and traslates it into a language with lower abstraction e.g. machine code which can be executed directly by a computer \cite[p. 44]{sebesta2013}. Some translators work differently though. They translate the source code into another high-level language or into machine code for virtual machines. The translation process is typically not a simple task, therefore it is often split into different phases, which is shown in \figref{fig:compileroverview}. The process can be split into more or less phases though, depending on how detailed one wished to describe the process. In this section we describe the following phases: the lexical analysis, the syntax analysis, the semantic analysis, the code generation and the interpretation.

\begin{figure}
	\begin{center}
		\scalebox{0.85}{
		\begin{tikzpicture}
  			[node distance=.6cm, start chain=going below,]
     		\node[punktchain, join] (scode) {Source code};
     		\node[punktchain, join] (leana) {Lexical analyser};
     		\node[punktchain, join] (syana) {Syntax analyser};
     		\node[punktchain, join] (seana) {Semantic analyser};
     		\node[punktchain, join] (cgen)  {Code generator};
     		\node[punktchain, join] (mach)  {Machine};
     
  			\draw[tuborg, decoration={brace}] let 
  				\p1=(leana.south), \p2=(syana.north) in
    			($(2, \y1)$) -- ($(2, \y2)$) node[tubnode] {Tokens};
  
  			\draw[tuborg, decoration={brace}] let 
  				\p1=(syana.south), \p2=(seana.north) in
    			($(2, \y1)$) -- ($(2, \y2)$) node[tubnode] {Parse tree};
  
  			\draw[tuborg, decoration={brace}] let 
  				\p1=(seana.south), \p2=(cgen.north) in
    			($(2, \y1)$) -- ($(2, \y2)$) node[tubnode] {Intermediate code};
  
 			\draw[tuborg, decoration={brace}] let 		
 				\p1=(cgen.south), \p2=(mach.north) in
    			($(2, \y1)$) -- ($(2, \y2)$) node[tubnode] {Machine language};
		\end{tikzpicture}}
	\end{center}
	\capt{The different phases of a compiler. Based on Sebesta \textit{et al.}\cite{sebesta2013} p. 46, Figure 1.3}
	\label{fig:compileroverview}
\end{figure}

\subsection{Lexical analysis}
The lowest level syntactic units of a language is called lexemes. A language's formal description does not often include these. They are instead described by a lexical specification, regular expressions i.e., separated from the syntactic specification\cite[p. 135]{sebesta2013}. Typical lexemes for a programming language includes integer literals, operators and special keywords like \textit{if} and \textit{while}. If both \textit{\$a} and \textit{\$b} are lexemes describing a variable and \textit{102} and \textit{42} are lexemes describing an integer, then \textit{\$a} and \textit{\$b} or \textit{102} and \textit{42} can typically be used interchangeably and still give a meaningful program. Therefore the lexemes are grouped into tokens. The name of a variable or the value of an integer is preserved when tokenising. The tokens are an abstraction that makes it easier to analyse if correct syntax of the language. An example of the grouping of lexemes into tokens can be seen by \tableref{table:lexandtokens}. After the lexical analysis an input stream of characters has been converted to an output stream of tokens.

%\tab[11cm]{lexandtokens}{7}{Lexemes and their corresponding token group.}
%		               {Input stream}
%       {Lexemes        }{\$a  & = & 3 & \$b & + & 4 & \$a                       }{
%\tabrow{\textbf{Tokens}}{var & assign & int(3) & var(b) & plus & int(4) & var(a)}
%}

\tab[4cm]{lexandtokens}{1}{Lexemes and their corresponding token group.}
		    {               }
{Lexemes   }{\textbf{Tokens}}{
\tabrow{\$a}{var(a) 		}
\tabrow{=  }{assign 		}
\tabrow{3  }{int(3) 		}
\tabrow{\$b}{var(b) 		}
\tabrow{+  }{plus   		}
\tabrow{4  }{int(4) 		}
\tabrow{\$a}{var(a) 		}
}

\subsection{Syntax analysis}
All languages whether natural or artificial is a set of strings of characters over some alphabet. There are rules for how the strings can look that are in a language and how they can be combined. The lexemes described how the strings can look and now the tokens are useful when analysis how the lexemes can be combined. The rules can be specified formally to describe the syntax of a language\cite[p. 135]{sebesta2013}. A common way to describe a language's syntax is by a formal language-generation mechanism (also called grammars or context free grammars). By describing a grammar that can generate all possible strings in a language, the language has also been formally described. Backus-Naur Form is such a mechanism which in the 1950's became the most widely used method for describing programming language syntax\cite[p. 137]{sebesta2013}.
The BNF contains a set of terminals and a set of non-terminals. The terminals are the tokens from the lexical analysis. The non-terminals all have a set of productions, from which a mix of terminals and non-terminals can be derived from. A start production specifies a single non-terminal, from where all syntactically valid strings that are in the language can be derived from by using the production rules until only a sequence of terminals (the tokens) are left. The syntax analysis takes a sequence of tokens as input and tries to create a set of derivations from the start symbol that creates the given sequence of tokens. If success, the input has been parsed and the parse tree is kept for later analysis. The parse tree is the information concerning how the start symbol was derived into the sequence of tokens, which yields a tree structure. This tree is called an abstract syntax tree.

\begin{ebnf}
%Expressions
\grule{program}{\gter{print} \gcat expr}
\grule{expr}{\gter{(} \gcat term \gcat \gter{)} \gcat operator \gcat \gter{(} \gcat term \gcat \gter{)}}
\grule{operator}{\gter{=}}
\galt{\gter{>}}
\galt{\gter{<}}
\grule{term}{number}
\galt{expr}
\grule{number}{\textbf{any number}}
\end{ebnf}


\subsection{Semantic analysis}
Not all characteristics of programming languages are easy to describe with a BNF and some even cannot be described using a BNF. If a programming language allows a floating-point value to be assigned to an integer variable but not the opposite, this \textit{can} be expressed with a BNF but if all such rules should be specified in the BNF, it would increase the size of it remarkably. With increased size, the formal description gets more clumsy to look at and also increases the risk that an error is contained in the BNF.
The rule that all variables must be declared before being used is impossible to express in a BNF. That would require the BNF to remember things, particularly those variables it had seen before, which it cannot. The problem of remembering things also shows up when we start to concern about scope rules. Typically, a variable declared in one scope cannot be used outside that scope. The BNF cannot describe such problems that we describe as static semantics rules. It is named static because the analysis required to check the specifications can be done at compile-time rather than runtime\cite[p. 153]{sebesta2013}.
In this semantic analysis phase, the compiler can check for type rules by starting to decorate the parse tree from the syntactic analysis with types. If the non-terminal \textit{expr} derives the terminal sequence \textit{int} \textit{plus} \textit{int} \textit{semicolon}, it can be decorated with the \textit{int}-type, and the analysis can proceed further up the tree and check that the type of the \textit{expr(int)} is legal. If the \textit{expr(int)} is derived from a \textit{expr} $\rightarrow$ \textit{expr(int)} +  \textit{expr(bool)} production, the static semantic rules must determine if the programs semantic is wrong or it the boolean value can be converted to the integer values zero or one.

\subsection{Code generation}
Every compiler must focus the translation on the capability of a particular machine architecture. The targeted architecture can be virtual such as the Java Virtual Machine. Generally speaking, the code generation phase translates the program into instructions that are carried out by a physical processor. Whether the architecture is virtual or real the program code must be mapped into the processors memory. Typically, the overall translation is broken into smaller pieces, where smaller subtrees of the abstract syntax tree are translated into executable form one at a time. However, there many things that must be considered when translating, i.e. \textbf{instruction selection}, which concerns how an intermediate code representation from the abstract syntax tree is to be implemented. There are many different ways to implement the same functionality, but some might be carried out faster than other. The code generation phase must also deal with problems concerning \textbf{register allocation} and \textbf{code scheduling}. Register allocation is concerned with effectively using the registers so moving the same variables between registers and memory is minimised\cite[p. 521]{fischer2009}. The code scheduling is an important aspect with pipelined processors. The aim is to produce instructions that executes in a way such that the pipelined execution will not have to stall unnecessary\cite[p. 551]{fischer2009}. Some of the problems associated with the pipelined execution is solved by move apart instructions that will interlock\cite[p. 552]{fischer2009}.

\subsection{Interpretation}
A pure interpretation of a program lies at the opposite end (from compilation) regarding to methods of implementation. With this approach, which can be see, on \figref{fig:compileroverviewinterpretation}, no translation is performed at all. An interpreter is interpreting a program written in the targeted language. It acts like a virtual machine which instructions are statements of high level language. By purely using interpretation, a source code debugger can easily be implemented. Various errors that might occur can once they are detected easily refer to which place in the source code that caused the error. The debugging is eased because the interpreter works like a software implementation of a virtual machine, thus the state of the machine and the value of a specific variable can be outputted at any time when requested. This will of course lead to the disadvantage that an interpreter uses more space than a compiler. Further more, the execution speed of an interpreter is usually 10 to 100 times slower than that of a compiler \cite[p. 48]{sebesta2013}.

\begin{figure}
	\begin{center}
		\scalebox{0.85}{
			\begin{tikzpicture}
  				[node distance=.8cm, start chain=going below,]
  				\node[punktchain, join,] (sprog) {Source program};
  				\node[punktchain, join,] (interp) {Interpreter};
  				\begin{scope}[start branch=venstre, every join/.style={->, thick, shorten <=1pt}, ]
  					\node[punktchain, on chain=going right, join=by {<-}] (indat) {Input data};
  				\end{scope}
  				\node[punktchain, join,] (res) {Result};
			\end{tikzpicture}}
		\end{center}
	\capt{The different phases of an interpreter. Based on Sebesta \textit{et al.}\cite{sebesta2013} p. 48, Figure 1.4}
	\label{fig:compileroverviewinterpretation}
\end{figure}

The compiling or interpreting approach can be combined to form a hybrid implementation system. This method is illustrated in \figref{fig:compileroverviewhybrid}, where a program is compiled into an intermediate code which is then interpreted. By using this approach, errors in a program can be detected before interpretation which can save much time for a programmer. A great portability can also be achieved when using hybrid system. The initial implementation of Java was hybrid and allowed Java to be compiled to an intermediate code that could run on any platform which had an implementation of Java Virtual Machine\cite[p. 50]{sebesta2013}. 

\begin{figure}
	\begin{center}
		\scalebox{0.85}{
			\begin{tikzpicture}
  				[node distance=.8cm, start chain=going below,]
  				\node[punktchain, join,] (sprog) {Source program};
  				
	     		\node[punktchain, join] (leana) {Lexical analyser};
    	 		\node[punktchain, join] (syana) {Syntax analyser};
     			\node[punktchain, join] (seana) {Semantic analyser};
  				\node[punktchain, join,] (interp) {Interpreter};
  	
  				\begin{scope}[start branch=venstre, every join/.style={->, thick, shorten <=1pt}, ]
        			\node[punktchain, on chain=going left, join=by {<-}] (indat) {Input data};
      			\end{scope}
  				
  				\node[punktchain, join,] (res) {Result};
  				
  				\draw[tuborg, decoration={brace}] let 
  				\p1=(leana.south), \p2=(syana.north) in
    			($(2, \y1)$) -- ($(2, \y2)$) node[tubnode] {Tokens};
  
  				\draw[tuborg, decoration={brace}] let 
  				\p1=(syana.south), \p2=(seana.north) in
    			($(2, \y1)$) -- ($(2, \y2)$) node[tubnode] {Parse tree};
  
  				\draw[tuborg, decoration={brace}] let 
  				\p1=(seana.south), \p2=(interp.north) in
    			($(2, \y1)$) -- ($(2, \y2)$) node[tubnode] {Intermediate code};
			\end{tikzpicture}}
		\end{center}
	\capt{The different phases of a hybrid implementation systems. Based on Sebesta \textit{et al.}\cite{sebesta2013} p. 49, Figure 1.5.}
	\label{fig:compileroverviewhybrid}
\end{figure}


An obvious disadvantage is that the executable is platform dependant and it
would therefore be necessary to develop a new compiler for each platform we want
to support. On the other hand knowing the specific platform makes it possible to
create optimized code which runs faster.

Instead of compiling to native machine code it could be compiled to an
intermediate format such as Java bytecode which is supported on many platforms.
While Java bytecode is interpreted and therefore slower, modern interpreters
use sophisticated methods such as Just-in-Time compilation (JIT) which at
run-time compiles intermediate code into native machine code. This process of
course adds an overhead, however the speed differences are not that great
anymore.
\cite{java-speed}

\todo{why is that?}

\subsection{Interpretation}
\label{sec:interpretation}
An interpreter takes the original source code and executes each instruction at
each translation. This means that a program will be parsed and executed
on-the-fly when using an interpreter. It is required that the end-user has the
interpreter. Different games written in our language would then use the same
copy of the interpreter instead of having a copy of the engine for each
executable. This separation will be further explored in
\secref{subsec:engineseperation}. The execution speed will however suffer and
while techniques such as JIT exists to improve this, it is beyond the scope of
this project.

A pure interpretation of a program lies at the opposite end from compilation in
regard to methods of implementation. With this approach, which is illustrated in
\figref{fig:compileroverviewinterpretation}, no translation is performed at all.

\begin{figure}
	\begin{center}
		\scalebox{0.85}{
			\begin{tikzpicture}
  				[node distance=.8cm, start chain=going below,]
  				\node[punktchain, join,] (sprog) {Source program};
  				\node[punktchain, join,] (interp) {Interpreter};
  				\begin{scope}[start branch=venstre, every join/.style={->, thick, shorten <=1pt}, ]
  					\node[punktchain, on chain=going right, join=by {<-}] (indat) {Input data};
  				\end{scope}
  				\node[punktchain, join,] (res) {Result};
			\end{tikzpicture}}
		\end{center}
	\capt{The different phases of an interpreter. Based on Sebesta \textit{et al.}\cite{sebesta2013} p. 48, Figure 1.4}
	\label{fig:compileroverviewinterpretation}
\end{figure}


An interpreter is interpreting a program written in the targeted language. It
acts like a virtual machine where instructions are statements of a high-level
language. By purely using interpretation a source code debugger can easily be
implemented. Various errors that might occur can once they are detected easily
refer to the location of faulty source code that caused the error. The debugging is
eased because the interpreter works like a software implementation of a virtual
machine, thus the state of the machine and the value of a specific variable can
be outputted at any time when requested. This will of course lead to the
disadvantage that an interpreter uses more space than a compiler. Furthermore,
the execution speed of an interpreter is usually 10 to 100 times slower than
that of a compiler.
\cite[p. 48]{sebesta2013}

\subsection{Hybrid compilation and interpretation}
The compiling or interpreting approach can be combined to form a hybrid
implementation system. This method is illustrated in
\figref{fig:compileroverviewhybrid}, where a program is compiled into an
intermediate code which is then interpreted. By using this approach errors in a
program can be detected before interpretation which can save much time for a
programmer. Great portability can also be achieved when using hybrid system.
The initial implementation of Java was hybrid and allowed Java to be compiled to
an intermediate code that could run on any platform which had an implementation
of Java Virtual Machine.
\cite[p. 50]{sebesta2013}

\begin{figure}
	\begin{center}
		\scalebox{0.85}{
			\begin{tikzpicture}
  				[node distance=.8cm, start chain=going below,]
  				\node[punktchain, join,] (sprog) {Source program};
  				
	     		\node[punktchain, join] (leana) {Lexical analyser};
    	 		\node[punktchain, join] (syana) {Syntax analyser};
     			\node[punktchain, join] (seana) {Semantic analyser};
  				\node[punktchain, join,] (interp) {Interpreter};
  	
  				\begin{scope}[start branch=venstre, every join/.style={->, thick, shorten <=1pt}, ]
        			\node[punktchain, on chain=going left, join=by {<-}] (indat) {Input data};
      			\end{scope}
  				
  				\node[punktchain, join,] (res) {Result};
  				
  				\draw[tuborg, decoration={brace}] let 
  				\p1=(leana.south), \p2=(syana.north) in
    			($(2, \y1)$) -- ($(2, \y2)$) node[tubnode] {Token list};
  
  				\draw[tuborg, decoration={brace}] let 
  				\p1=(syana.south), \p2=(seana.north) in
    			($(2, \y1)$) -- ($(2, \y2)$) node[tubnode] {Abstract syntax tree};
  
  				\draw[tuborg, decoration={brace}] let 
  				\p1=(seana.south), \p2=(interp.north) in
    			($(2, \y1)$) -- ($(2, \y2)$) node[tubnode] {Decorated abstract syntax tree};
			\end{tikzpicture}}
		\end{center}
	\capt{The different phases of a hybrid implementation systems. Based on Sebesta \textit{et al.}\cite{sebesta2013} p. 49, Figure 1.5.}
	\label{fig:compileroverviewhybrid}
\end{figure}


\subsection{Separation of game and engine}
\label{subsec:engineseperation}
Keeping the game and the engine separated opens up for the possibility of
changing the game engine while still being able to use the same game. 

One major advantage is that it is possible to update the engine and in result
update all your games. An update which improves the graphics or adds new
features such as network support would work with older games instantly without
having to wait for the developer to update it. If the developer no longer
maintains the game an updated version might never come out.

The disadvantage is however that the responsibility for maintaining
compatibility is moved from the developer of the game to the developers of the
engine. A game developer can simply change his program so it works with a new
engine, however the engine developers would have to support games written for
every version released.

\subsection{Summary of code generation and interpretation}
The advantage of compilation is that the outputted code will run faster because 
a complete list of instructions will be ready to be executed. Although,
a disadvantage is the time it takes to compile the code will take longer because
the complete source code must be translated.

The advantage of interpretation is that it is possible to begin executing the
program quickly because each instruction is interpreted on-the-fly which makes
it faster than compiling the complete code. A disadvantage of interpretation is
that it is usually 10 to 100 times slower than compilation.

If code is translated to an IL and then further translated this takes a lot of
work away when talking about generating compilers because compilers are platform
dependant. If we say that we have $n$ compilers and $m$ platforms, then when
compiling to an IL we only have to develop $n+m$ compilers instead of $n*m$
because when compiling to an IL every platform can compile to this language and
from the IL to the target platform.

It is possible to combine compilation and interpretation. The prorgam is
compiled to intermediate code which is then interpreted. By using this approach
errors in a program can be detected before interpretation which can save much
time for a programmer.

\section{A game simulator}
\label{sec:simulator}
Considering the fact that board games consist of physical entities in the real
world and rely purely on user-to-game and user-to-user interaction, we find it
necessary to analyse how we can emulate this behaviour in the most ``realistic''
way. To do this, we look at what a simulator is and could be, what we can use it
for, and set up some features an optimal simulator for our programming language
would include, ending with a final definition of the simulator for our language.

So what is a simulator and what does it consist of? A simulator can be seen as a
front end to an interpreter (or a compiler, though not as practical). It is the
glue between the user and code execution. A user interacts with the simulator,
which in turn interprets the user's input and does something with it, such as
updating a graphical user interface or supplying some other kind of feedback.

Examples of simulators are seen in various different contexts, such as the
Ruby\cite{rubyLang} programming language's interactive shell $irb$, which is run
from the command line and allows programmers to interact, experiment, and write
code with immediate response, calling Ruby's interpreter upon every command
entered. The $irb$ keeps track of all current code entered, allowing programmers
to write an entire program in $irb$. Another example could be various different
kinds of environmental simulators, such as physics simulators created by the
University of Colorado at Boulder\cite{colSim}. These simulators offer a
computerized environment that allows changing of different factors within a
simulated world, such as changing the pressure and gravity of an environment,
providing instant feedback.

\subsection{Usage}
We see the need for a simple simulator because board games consist of so much
interactivity between the players and the board, that we need to mimic it.
Nobody wants to sit and play noughts and crosses or chess in front of a
terminal; that'd be both awkward and impractical. Therefore, we see the
simulator playing a crucial role as the engine that drives the graphics and
gameplay of a written game - in part being a front end to everything in the
interpretation/compilation phases.

A board game designer could program his game in \productname{} and see it
displayed with the current implementation fully working and playable on the
screen in a matter of a few clicks. Another advantage with having such a
simulator is that it can be used to prototype games before they physically need
to be produced. Such a construction will allow quickly changing the game rules
and board layout, etc.\ and support experimentation with different set-ups. This
type of simulator could allow dynamically changing board game parameters, such
as the board size, the amount of players, how the pieces behave, etc.

Another, more simple version of the simulator directed at the end users can be
used to merely play the games. All they would have to do is open a game file in
the simulator or set the simulator as the default program for game files written
in our language. This is useful for games that don't necessarily need a physical
version or when the game designer wants to test it with a broad group of people
before putting it into production.

\subsection{Possible features}
We decide that creating a set of potential features for a simulator will also be
useful when it comes to designing the programming language itself, as these
features can influence the syntax and semantics of the \productname{} language.
Described below are some descriptions of possible features we have discussed and
deem important for the simulator to offer.

\begin{description}

  \item[Interactive design] As a board game designer, it could be possible to
    quickly change pieces around and edit some things directly from the
    interface. This could influence the written code, creating a new game based
    off of the old one, much like the physics simulators mentioned previously.
    An alternative option to this would be to dynamically reload the file used
    as input if it is changed from an external source, allowing quick feedback
    if you're just editing a few lines in the game's source code.

  \item[Loading pictures] Pieces and illustrations of various entities in the
    game can be automatically found and determined from their names definitions
    in a \productname{} file. This lets the designer think about writing a game
    and not how to load specific files from a directory and so on, easily
    influencing cluttered code.

  \item[AI] As long as the code and game rules are well defined, an automatic AI
    could be implemented as a module in the simulator to simulate other players
    following the exact same set of rules, allowing the designer to test his
    entire game or parts of it without constantly needing other people. This
    could be very interesting, but unfortunately is out of the scope of this
    project.

  \item[Multi-player] Multi-player support using the same computer or over a
    network. Each real player could take turns sitting at a physical computer,
    replacing non-existent players or computer-controlled players. As long as
    the simulator is implemented optimally, supporting multi-player games should
    be considerably simple, as the simulator needs to handle commands from a
    single player anyway. Scaling this up and handling multiple turns from
    multiple players shouldn't be too much of a challenge. A better, yet not
    always more practical solution is to allow players to play against each
    other across a network. Sending turn commands back and forth could be
    established via a simple protocol.

  \item[Tracking moves] The simulator could offer a simple turn list displaying
    all the previous moves in the board game. Then it'd be possible to go back
    to a specific turn to ``rewind'' the game to a previous state.
\end{description}

These features could easily influence the syntax of our programming language.
There could be specific reserved constructs to determine how the board and
players are defined, making the simulator's job at displaying things easier.

\subsection{Definition of a simulator}
We define a simulator as a package consisting of the language's
interpreter/compiler and a GUI that is in direct contact with the users of our
programming language. Whether these users are designers or players is
irrelevant, as different versions of the simulator could easily be written and
implemented. It can support many different features and could allow changes to
be made as the user notices something that needs to be changed. The simulator
sends commands to the interpreter/compiler and responds to the commands returned
from it, such as updating a score, changing the position of a piece, or
displaying an error message upon an attempting an illegal move.

An example of this could be that the user clicks and drags on a knight in an
implementation of chess, moving it to another position on the game board. The
simulator would send this behaviour to the interpreter or compiler (which
recompiles), which checks it against the game's source code to see if the move
itself is legal, and also any side-effects this move could have, such as
eliminating an opposing player's piece.

We see spending time on writing a simulator useful because it links all the
different stages together and will act as the final product containing all the
other parts of the project. That said, it'd be ideal to separate the
interpreter/compiler and simulator, allowing greater modularity if the
interpreter/compiler is to be used in another implementation of a simulator or
something entirely different.

Considering the fact that most board games are very visual and consist of
different kinds of pieces placed at various different locations on a board, we
conclude that we need a simulator. This simulator needs to be graphical and
support all the elements a normal gaming session would, such as a board, pieces,
rules for moving pieces, multiple players, and so on. Adding the ability to
dynamically change programmatic features from the user interface is not rated as
important, because this can simply already be done from the source code. It
would help make testing and playing games as authentic as possible.

