\subsection{Dynamic and static scope rules}

In this section we take a quick look at the advantages and disadvantages of
static and dynamic scoping and compare the two types of scoping.

With static scoping the scopes of variables can be determined prior to
execution. This means that a compiler can easily determine the type of every
variable in the program by just examining the source code. So, when a variable
is referenced in a statically scoped language the value and type of the variable 
is the on it had at the time of declaration.
\cite[5.5.1, p. X]{sebesta2013}
%5.5.1, side 219 (sebesta)

Static scope rules provide subprograms with access to nonlocal identifiers and
this type of scoping works very well with compilers because the scope can be
determined at compile time.
\cite[5.5.5, p. X]{sebesta2013}
%5.5.5, side 227 (sebesta)

A disadvantage of static scoping is that it can give too much access and might
need restrictions. But programs are dynamic and are often restructured which
can lead to destruction of initial restrictions.
\cite[5.5.5, p. X]{sebesta2013}
%5.5.5, side 227 (sebesta)

With dynamic scoping the scopes of variables can only be determine at run time
because it is based on the calling sequence of subprograms. When a variable is
referenced in a language with dynamic scoping then the value and type of the 
variable is what it had on the time of the call to it.
\cite[5.5.6, p. X]{sebesta2013}
%5.5.6, side 227 (sebesta)

It is not possible to determine scopes statically because the calling sequence
of subprograms is not always known. When a method A calls a method B then B has 
access to variables in that were declared in A. As a result dynamically scoped
languages are much more difficult to read and understand and this results in
them being less reliable.
\cite[5.5.7, p. X]{sebesta2013}
%5.5.7, side 229 (sebesta)

Comparing two similar programs with different scoping, then the statically
scoped program will be much easier to read, more reliable, and it will execute
much faster than the program written with dynamic scoping.
\cite[5.5.7, p. X]{sebesta2013}
%5.5.7, side 229 (sebesta)
