\subsection{dynamic and static type systems}

The type system is an essential part of the feel and touch of a programming language. There exits two main kinds of type systems: the dynamic and the static type system. Wether to have one over another, is today a hotly discussed topic. In the following section we list some advantages and disadvantages of each. 

When it comes to detecting errors the static type system can make it easier to detect programming errors in an earlier stage than the dynamic type systems. This is due to the fact that it's already possible to check type errors during compile-time rather than at run-time. The readability can also be improved by the static type system because of the precense of type names. This for instance can make it easier for the programmer to get an idea of what a certain sub-program is ment to do. The forced precense of types is also what decreases the writeability of static type systems, since the programmer has to write down the types at all times and, when declaring variable, spend time considering wether a variable should be an integer type, a floating points or another type. This can take valuable programming time. So the dynamic type system is faster to writ and it's more flexible but the static type system is easier to read and more reliable at run-time since type checking is done on compile-time. There is a whole list of other advantages and disadvantages of each type system, which are provocatively explained in \cite{staticvsdynamictypesystem}, where it is argued that perhaps a middle solution between the two should be considered.

But which type system should we then go for? We are aiming to create a programming language in which it's easy and fast to create board games. It should be possible to create a board game within fewest possibles line of code and for that purpose the dynamic type system is the more relevant. 