\section{Programming Paradigms}
\label{sec:paradigms}

A programming paradigm describes a method and style of computer programming.
Some of the primary paradigms are imperative, object-oriented, functional and
declarative programming. While some programming languages strictly follow one
paradigm, there are many so-called multi-paradigm languages, that implement
several paradigms and therefore allow multiple styles of programming. Examples
of multi-paradigm languages include C\# and Java. It is essential for us to be
aware of these different paradigms, since it may help us to find a good style of
programming for the design of \productname{}.

\subsection{Overview}
The four main paradigms are described as follows:\cite{fourparadigms}
\begin{description}
\item[Imperative programming] describes computation in terms of statements that
  change the program state. Primary characteristics are assignments, procedures,
  data structures, control structures. Imperative programming can be seen as a
  direct abstraction of how most computers work, and many imperative languages
  are just abstractions of assembly language. Typical examples of imperative
  languages are C and Fortran.
\item[Object-oriented programming] describes computation in terms of objects
  described by attributes manipulated through methods. Primary characteristics
  are objects, classes, methods, encapsulation, polymorphism, inheritance. An
  example of a pure object-oriented languages is Smalltalk, while many other
  languages are either primarily designed for object-oriented programming (such
  as Java and C\#) or have support for object-oriented programming (such as PHP
  and Perl).
\item[Declarative programming] describes computational logic without describing
  control flow, i.e. describing {\em what} a program does rather than {\em how}
  it does it. Many domain-specific languages such as SQL, HTML and CSS are
  declarative. Logic programming, such as Prolog, is a subset of declarative
  programming.
\item[Functional programming] describes computation in terms of mathematical
  functions and seeks to avoid program state and mutable data. Purely functional
  functions have no side effects, and the result is constant in relation to the
  parameters (e.g. $add(2, 4)$ always returns $6$). An example of a purely
  function programming languages is Haskell. Other examples of languages
  designed for functional programming are Erlang, F\# and Lisp, while it is
  possible to apply functional programming concepts to many other languages.
\end{description}

\todo{rewrite ending}
%While general-purpose languages, such as C\# and Java, generally tend to lean
%towards the imperative and object-oriented paradigms, a domain-specific
%language, with very specific goals in design, may benefit from other paradigms,
%e.g. declarative programming. Since our language is primarily meant for \emph{
%declaring} board games, it would likely benefit from being a declarative
%programming language. Based on this we have set up some requirements in the
%requirement list in \chapref{chap:requirements} which indicates that we want
%\productname{} to become a declarative programming language.  

