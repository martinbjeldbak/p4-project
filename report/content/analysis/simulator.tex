\section{A game simulator}
\label{sec:simulator}
Considering the fact that board games consist of physical entities in the real
world and rely purely on user-to-game and user-to-user interaction, we find it
necessary to analyse how we can emulate this behaviour in the most ``realistic''
way. To do this, we look at what a simulator is and could be, what we can use it
for, and set up some features an optimal simulator for our programming language
would include, ending with a final definition of the simulator for our language.

So what is a simulator and what does it consist of? A simulator can be seen as a
front end to an interpreter (or a compiler, though not as practical). It is the
glue between the user and code execution. A user interacts with the simulator,
which in turn interprets the user's input and does something with it, such as
updating a graphical user interface or supplying some other kind of feedback.

Examples of simulators are seen in various different contexts, such as the
Ruby\cite{rubyLang} programming language's interactive shell $irb$, which is run
from the command line and allows programmers to interact, experiment, and write
code with immediate response, calling Ruby's interpreter upon every command
entered. The $irb$ keeps track of all current code entered, allowing programmers
to write an entire program in $irb$. Another example could be various different
kinds of environmental simulators, such as physics simulators created by the
University of Colorado at Boulder\cite{colSim}. These simulators offer a
computerized environment that allows changing of different factors within a
simulated world, such as changing the pressure and gravity of an environment,
providing instant feedback.

\subsection{Usage}
We see the need for a simple simulator because board games consist of so much
interactivity between the players and the board, that we need to mimic it.
Nobody wants to sit and play noughts and crosses or chess in front of a
terminal; that'd be both awkward and impractical. Therefore, we see the
simulator playing a crucial role as the engine that drives the graphics and
gameplay of a written game - in part being a front end to everything in the
interpretation/compilation phases.

A board game designer could program his game in \productname{} and see it
displayed with the current implementation fully working and playable on the
screen in a matter of a few clicks. Another advantage with having such a
simulator is that it can be used to prototype games before they physically need
to be produced. Such a construction will allow quickly changing the game rules
and board layout, etc.\ and support experimentation with different set-ups. This
type of simulator could allow dynamically changing board game parameters, such
as the board size, the amount of players, how the pieces behave, etc.

Another, more simple version of the simulator directed at the end users can be
used to merely play the games. All they would have to do is open a game file in
the simulator or set the simulator as the default program for game files written
in our language. This is useful for games that don't necessarily need a physical
version or when the game designer wants to test it with a broad group of people
before putting it into production.

\subsection{Possible features}
We decide that creating a set of potential features for a simulator will also be
useful when it comes to designing the programming language itself, as these
features can influence the syntax and semantics of the \productname{} language.
Described below are some descriptions of possible features we have discussed and
deem important for the simulator to offer.

\begin{description}

  \item[Interactive design] As a board game designer, it could be possible to
    quickly change pieces around and edit some things directly from the
    interface. This could influence the written code, creating a new game based
    off of the old one, much like the physics simulators mentioned previously.
    An alternative option to this would be to dynamically reload the file used
    as input if it is changed from an external source, allowing quick feedback
    if you're just editing a few lines in the game's source code.

  \item[Loading pictures] Pieces and illustrations of various entities in the
    game can be automatically found and determined from their names definitions
    in a \productname{} file. This lets the designer think about writing a game
    and not how to load specific files from a directory and so on, easily
    influencing cluttered code.

  \item[AI] As long as the code and game rules are well defined, an automatic AI
    could be implemented as a module in the simulator to simulate other players
    following the exact same set of rules, allowing the designer to test his
    entire game or parts of it without constantly needing other people. This
    could be very interesting, but unfortunately is out of the scope of this
    project.

  \item[Multi-player] Multi-player support using the same computer or over a
    network. Each real player could take turns sitting at a physical computer,
    replacing non-existent players or computer-controlled players. As long as
    the simulator is implemented optimally, supporting multi-player games should
    be considerably simple, as the simulator needs to handle commands from a
    single player anyway. Scaling this up and handling multiple turns from
    multiple players shouldn't be too much of a challenge. A better, yet not
    always more practical solution is to allow players to play against each
    other across a network. Sending turn commands back and forth could be
    established via a simple protocol.

  \item[Tracking moves] The simulator could offer a simple turn list displaying
    all the previous moves in the board game. Then it'd be possible to go back
    to a specific turn to ``rewind'' the game to a previous state.
\end{description}

These features could easily influence the syntax of our programming language.
There could be specific reserved constructs to determine how the board and
players are defined, making the simulator's job at displaying things easier.

\subsection{Definition of a simulator}
We define a simulator as a package consisting of the language's
interpreter/compiler and a GUI that is in direct contact with the users of our
programming language. Whether these users are designers or players is
irrelevant, as different versions of the simulator could easily be written and
implemented. It can support many different features and could allow changes to
be made as the user notices something that needs to be changed. The simulator
sends commands to the interpreter/compiler and responds to the commands returned
from it, such as updating a score, changing the position of a piece, or
displaying an error message upon an attempting an illegal move.

An example of this could be that the user clicks and drags on a knight in an
implementation of chess, moving it to another position on the game board. The
simulator would send this behaviour to the interpreter or compiler (which
recompiles), which checks it against the game's source code to see if the move
itself is legal, and also any side-effects this move could have, such as
eliminating an opposing player's piece.

We see spending time on writing a simulator useful because it links all the
different stages together and will act as the final product containing all the
other parts of the project. That said, it'd be ideal to separate the
interpreter/compiler and simulator, allowing greater modularity if the
interpreter/compiler is to be used in another implementation of a simulator or
something entirely different.

Considering the fact that most board games are very visual and consist of
different kinds of pieces placed at various different locations on a board, we
conclude that we need a simulator. This simulator needs to be graphical and
support all the elements a normal gaming session would, such as a board, pieces,
rules for moving pieces, multiple players, and so on. Adding the ability to
dynamically change programmatic features from the user interface is not rated as
important, because this can simply already be done from the source code. It
would help make testing and playing games as authentic as possible.
