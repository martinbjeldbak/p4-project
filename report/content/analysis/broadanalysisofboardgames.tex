\section{Board game analysis}
\label{sec:board-game-analysis}

One may wonder what a board game really is. Could it just be any game
containing some kind of a board? If so, would Trivial Pursuit be a board
game and what about the game Twister, where you have to place your hands
and/or feet on a spot marked with a particular color on a sheet - or
board, as you could call it. Most people have a mental model of a board
game that does not include games like Twister. Here is one definition of
a board game \cite{def-board-game}:

\begin{quote}
  ``A board game is a game played across a board by two or
  more players. The board may have markings and designated spaces, and the
  board game may have tokens, stones, dice, cards, or other pieces that
  are used in specific ways throughout the game.''
\end{quote}

The definition above is very broad and will to some extent allow a game
like Twister to be categorized as a board game. All kinds of things
like cards and dice can be part of a board game, but one board game
designer may also be able to invent a new and yet unseen widget, which
he wants to include in his board game. A programming language that makes
it possible to describe any board will cover a very broad category of
games. You could argue that it would actually cover all games that can
be made, since even a first person shooter could technically be played across a
board. With such a broad definition, a programming language that aims
to make the programming of board game easier, will likely have to be a general purpose programming language. If a programming
language is aimed to make the programming of only a specific kind of
board games easier, there might be many things that can be expressed easy in that language compared 
to how it would have been done in existing general purpose programming languages.

To define what elements that is to be included in our programming language \productname{}, we think it is essential to look at some existing board games. For that reason we have analysed two well known board games. We have investigated the game elements that might be clumsy or not straightforward to implement in common general purpose programming languages. After the analysis of the games, a list of elements is served that respects all of the elements from the games. The aim of \productname{} is to ease the programming of these game elements

\todo{finish this section}
