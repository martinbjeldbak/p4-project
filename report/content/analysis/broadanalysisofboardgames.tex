\section{Board game analysis}
\label{sec:board-game-analysis}

One may wonder what a board game really is. Could it just be any game
containing some kind of a board? If so, would Trivial Pursuit be a board
game and what about the game Twister, where you have to place your hands
and/or feet on a spot marked with a particular color on a sheet - or
board, as you could call it. Most people have a mental model of a board
game that does not include games like Twister. Here is one definition of
a board game \cite{def-board-game}:

\begin{quote}
  ``A board game is a game played across a board by two or
  more players. The board may have markings and designated spaces, and the
  board game may have tokens, stones, dice, cards, or other pieces that
  are used in specific ways throughout the game.''
\end{quote}

The definition above is very broad and will to some extent allow a game
like Twister to be categorized as a board game. All kinds of things
like cards and dice can be part of a board game, but one board game
designer may also be able to invent a new and yet unseen widget, which
he wants to include in his board game. A programming language that makes
it possible to describe any board will cover a very broad category of
games. You could argue that it would actually cover all games that can
be made, since even a first person shooter could technically be played across a
board. With such a broad definition, a programming language that aims
to make the programming of board game easier, will be a programming
language that aims to make almost everything easier. If a programming
language aimed to make the programming of only a very specific kind of
board games easier, there might be many things that can be optimized
compared to existing general purpose programming languages.

Due to there being very different definitions of board games depending
on the source, we decide to make our own definition of board games that
our programming language should be able to describe. This is partly to
help further define and outline a specific genre or type of board games
that we find interesting and relevant. We also decide to do this to keep
the language as simple and straightforward as possible, while still
keeping the things we choose not to include in the back of our minds, to
make including these features possible in a later stage of development.

To create our own definition of board games, we look at three existing and popular, yet relatively different board games and sum up what features they have in common.
