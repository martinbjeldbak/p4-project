\section{Context-Free Grammars}
\label{sec:cfg}
Context-free grammars (CFGs) are used to specify the syntactical structure of programming languages. Throughout the report we will be using context-free grammars to represent our programming language. In this section we define what a context-free grammar is. 

A context-free grammar is a collection of substitution rules (or productions) constituted by nonterminals and terminals that describe the construction of a language. Any language that can be generated by a CFG is called a context-free langauge. A CFG is a 4-tuple $(V, \Sigma, R, S)$:\cite[p. 100]{itttoc}

\begin{itemize}[noitemsep]
\item $V$ is a finite set called the nonterminals
\begin{itemize}[noitemsep]
\item To create a string, we think of the nonterminals as variables
\end{itemize}
\item $\Sigma$ is a finite set, disjoint from V, called terminals
\begin{itemize}[noitemsep]
\item The terminals cannot be the same as the nonterminals
\item This set can be thought of as the alphabet
\end{itemize}
\item $R$ is a finite set of productions
\begin{itemize}[noitemsep]
\item Each production consist of a nonterminal and a string of nonterminals and terminals
\item The nonterminal and the string is seperated by an arrow or a |
\end{itemize}
\item $S \in V$ is the start symbol (a nonterminal)
\begin{itemize}[noitemsep]
\item This is the first nonterminal at the left-most top of the grammar
\end{itemize}
\end{itemize} 

In the following section we give an example of a CFG.

\subsection{Production}
It is easier to understand what a CFG really is by showing an example. The following is an example of a CFG, which we call \textit{acb}:

\begin{center}
	\begin{ebnf}
		\grule{A}{aAb}
		\galt{B}
		\grule{B}{c}
	\end{ebnf}
\end{center}

In the example we have two nonterminals; A and B, and three terminals; a, b, and c. The production for nonterminal A states that A can derive the string ``aAb'' or the string ``B'', and the nonterminal B can derive the string ``c''. When a nonterminal is present in a string they are substituted with their own production. For instance the string \textit{aaacbbb} can be derived from the CFG we called \textit{acb}. It is not really possible to create a lot of different strings with \textit{acb}. It is only possible to create strings with an equal amount of a's and b's with a c in between them.

\subsection{Derivation}
The sequence of substitutions needed to obtain a string from the CFG is called a derivation.\cite[p. 100]{itttoc} A derivation of the above given string \textit{aaacbbb} is:

\[ 
A \Rightarrow aAb \Rightarrow aaAbb \Rightarrow aaaAbbb \Rightarrow aaaBbbb \Rightarrow aaacbbb 
\]

The derivation starts with the start symbol (which is the left hand side of the production) which is substituted with its substitution rule (which is the right hand side of the production). The nonterminals are substituted until there are no more left. When the string only contains terminals the derivation is complete.

Now that we have an understanding for CFGs we can go on to the next topic. The following sections will cover the topic about parsers; what they are and how they are generated.