\section{Summary of decisions}
\label{sec:summaryofdecisions}

Based on the analysis chapter, we have made the following decisions, which our further project work will be focused toward:

We are going to craft a LL(1)-parser by hand and crafting a LALR(1)-parser with the SableCC parser generator
The reason for this decision, is the fact that this project is ment as a learning tool and therefore knowing how the parsing
phase works is important. By both crafting a parser by hand and by the use of a parser generator tool, we will get to know
how a well-known parser generator tool works and we will know how a parser works under the in details.

We are going to focus on the functional and object oriented programming paradigms. These paradigms makes great sense for a programming
language that focuses on board game programming. The reason for this is specified in the end of the programming paradigm \secref{sec:paradigms}.

We are going to focus on crafting an intepreter rather than a compiler. This makes sense especially because we have also decided to make a game simulator.
By having an interpreter rather than a compiler, the game programmer will be able to make changes in the code and see them visualised in the simulator right away, without having to re-compile the whole program.

The decision of including a simulator is based on the assumption that board game programmers wish to have their games visualised and be playable quickly.

The analysis chapter has too set the fundation for a list of requirements for our programming language, which can be seen in following \chapref{chap:requirements}.    