\section{Summary of essential decisions}
\label{sec:summaryofdecisions}

Based on this chapter we have made the following decisions which our continued
work will focus on.

When talking about crafting a parser we have chosen to craft an LL($1$)-parser
by hand and also generate a LALR($1$)-parser with SableCC.  The reason for this
decision is the fact that this project is meant as a learning experience and
therefore knowing how the parsing phase works is important. By both crafting a
parser by hand and by the use of a generator tool we will get to know how a
well-known parser generator works as well as construct our own customised
parser.

We also presented the four main paradigms of programming languages and we will
focus on combining the functional and object-oriented programming paradigms.
These paradigms make great sense for a programming language that focus on board
game programming. The reason for this is specified in \secref{sec:paradigms}.

We will focus on crafting an interpreter rather than a compiler. This makes
sense especially because we have decided to make a game simulator. By having an
interpreter rather than a compiler, the game programmer will be able to make
changes in the code and see them visualised in the simulator right away without
having to recompile the whole program.

The decision of including a simulator is based on the assumption that board game
programmers wish to have their games visualised and quickly be able to play and
test them.

This chapter has set the foundation for a list of requirements for our
programming language, which can be seen in the following chapter.
