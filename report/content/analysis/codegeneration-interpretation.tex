\section{The interpretation and code generation phases}
\label{sec:codegenerationandinterpretation}

In this section, we present a brief overview of the phases of
translators. Along with the design of our programming language
\productname{}, we also want to make it possible for programmers who
wish to write games in \productname{} to actually have the computer
understand what they write. There are a number of ways we can make that
happen. These are presented in the following subsections.

We begin by presenting the translation process followed by a
presentation of what intermediate languages (IL) are and what they can
be used for. In the subsequent section we present interpretation. Then
we present a hybrid solution based upon compilation and interpretation.
Lastly, we present what it will mean to have the game and engine
separate, followed by a short summary, where we take some decisions
based on the what we decide is important for our language.

%\subsection{The translation process}
%\label{sec:translationprocess}
%The typical translator takes as input some given source code written in a
%language with a high level of abstraction and translates it into a language with
%lower abstraction e.g. machine code which can be executed directly by a
%computer.
%\cite[p. 44]{sebesta2013} 

%Some translators work differently though. They translate the source code into
%another high-level language or into machine code for virtual machines, which can
%provide portability. The translation process is typically not a simple task,
%therefore it is often split into different phases, which is shown in
%\figref{fig:compileroverview}. The process can be split into more or less phases
%though, depending on how detailed one wished to describe the process. In this
%section we describe the following phases: the lexical analysis, the syntax
%analysis, the semantic analysis, the code generation and the interpretation.

\subsection{Compilation}
\label{sec:compilation}
With compilation, an executable file is created for a specific platform
that contains all the code required to play the game. Since games have
common aspects, a game engine containing all the common aspects such
as user interface, AI and/or network connection would most likely be
written. This engine would then be included directly in the executable
upon compilation.

The translation process is typically not a simple task,
therefore it is often split into different phases, shown in
\figref{fig:compileroverview}. The process can be split into many or
few stages, depending on how detailed a process is desired and if an
optimisation is done. In \figref{fig:compileroverview}, the lexical and
syntactical analysers make a lookup in a symbol table. Then the semantic
analyser and the code generator use the symbol table to generate
the correct code. Optimisation is optional in the phase of semantic
analysis\cite[p. 46]{sebesta2013}.

\section{Compiler overview}

A programming language can be implemented by writing a compiler that translates programs into machine language, which a computer can execute directly \cite[p. 44]{sebesta2013}.
If the program to be translated is not valid, the compiler should give an error message that best possible helps the programmer identify what error he has made. A good compiler will try to fix the error and continue compilation, so even further errors can be identified, providing the programmer a complete list of errors that can handled all at once instead of one per compilation. Translating one language to another is typically not a simple task, therefore it is often split into different phases, which is shown by \figref{fig:compileroverviewphases} and will be clarified in the following.

\begin{figure}
\centering
\includegraphics[scale=0.5]{pictures/compileroverview.png}
\capt{The different phases of a compiler. Based on Sebesta \textit{et al.}\cite{sebesta2013} p. 46, Figure 1.3}\label{fig:compileroverviewphases}
\end{figure}

\subsection{Lexical analysis}

Patterns of characters from alphabet -> tokens

\subsection{Syntax analysis}
All languages whether natural or artificial is a set of strings of characters over some alphabet. A language does not contain all possible strings over an alphabet. There are rules for how the strings can look that are in a language. Those rules can be specified formally to describe the syntax of a language\cite[p. 135]{sebesta2013}. A common way to describe a language's syntax is by a formal language-generation mechanism (also called grammars or context free grammars). By describing a grammar that can generate all possible strings in a language, the language has also been formally described. Backus-Naur Form is such a mechanism which in the 1950's became the most widely used method for describing programming language syntax\cite[p. 137]{sebesta2013}.

\begin{ebnf}
%Expressions
\grule{program}{\gter{print} expr}
\grule{expr}{\gter{(} \gcat term \gcat \gter{)} \gcat operator \gcat \gter{(} \gcat term \gcat \gter{)}}
\grule{operator}{\gter{=}}
\galt{\gter{>}}
\galt{\gter{<}}
\grule{term}{number}
\galt{expr}
\grule{number}{\textbf{any number}}
\end{ebnf}


\subsection{Semantic analysis}
\subsection{Code generation}
\subsection{Interpretation}

A pure interpretation of a program lies at the opposite end (from compilation) regarding to methods of implementation. With this approach, which can be see, on \figref{fig:compileroverviewinterpretation}, no translation is performed at all. An interpreter is interpreting a program written in the targeted language. It acts like a virtual machine which instructions are statements of high level language. By purely using interpretation, a source code debugger can easily be implemented. Various errors that might occur can once they are detected easily refer to which place in the source code that caused the error. The debugging is eased because the interpreter works like a software implementation of a virtual machine, thus the state of the machine and the value of a specific variable can be outputted at any time when requested. This will of course lead to the disadvantage that an interpreter uses more space than a compiler. Further more, the execution speed of an interpreter is usually 10 to 100 times slower than that of a compiler \cite[p. 48]{sebesta2013}.

\begin{figure}
\centering
\includegraphics[scale=0.5]{pictures/compileroverviewinterpretation.png}
\capt{The different phases of an interpreter. Based on Sebesta \textit{et al.}\cite{sebesta2013} p. 48, Figure 1.4}\label{fig:compileroverviewinterpretation}
\end{figure}

The compiling or interpreting approach can be combined to form a hybrid implementation system. This method is illustrated in \figref{fig:compileroverviewhybrid}, where a program is compiled into an intermediate code which is then interpreted. By using this approach, errors in a program can be detected before interpretation which can save much time for a programmer. A great portability can also be achieved when using hybrid system. The initial implementation of Java was hybrid and allowed Java to be compiled to an intermediate code that could run on any platform which had an implementation of Java Virtual Machine\cite[p. 50]{sebesta2013}. 

\begin{figure}
\centering
\includegraphics[scale=0.5]{pictures/compileroverviewhybrid.png}
\capt{The different phases of a hybrid implementation systems. Based on Sebesta \textit{et al.}\cite{sebesta2013} p. 49, Figure 1.5}\label{fig:compileroverviewhybrid}
\end{figure}

An obvious disadvantage is that the executable is platform dependant
and it would therefore be necessary to develop a new compiler for each
platform we want to support. On the other hand knowing the specific
platform makes it possible to create optimized code which runs faster.

\subsubsection{An intermediate language}
\label{sec:intermediatelanguage}
A middle step between compiling or interpretation and generating executable
machine code is to translate source code to an intermediate language (IL), which
is then interpreted or compiled further. IL are usually more low-level than the
initial source code, which would make it possible to optimize code for higher
efficiency in later stages, such as eliminating superfluous node types and dead
code.

Furthermore, one can compile to an intermediate language such as Java bytecode,
executable on every machine that supports Java by having a Java virtual machine
installed. This is very useful because the programmer, whom is developing a
compiler for a language, does not have to construct a compiler for every
platform; just one that translates to Java bytecode, that can then be translated
further to many supported platforms. This way the programmer must only construct a
single compiler. If one does not compile to an intermediate language, then the
programmer must construct a compiler for each specific platform, which will be a
lengthy and cost-heavy process. If you have $m$ languages and $n$ platforms,
then the programmer must construct $m*n$ compilers to be able to compile to
every platform. The difference between compiling directly to a platform or to an
intermediate language and then further translating is illustrated in
\figref{fig:mtimesn}.

\subsection{Compile to intermediate language, then translate further}
\label{sec:compiletointermediate}

In this project we decided to implement an interpreter for our programming
language. This decision was made on the basis of our analysis chapter
(\chapref{chap:analysis}) where we could conclude that this would be the optimal
approach for \productname{}, but it is possible to develop a compiler instead. In
this section we will explain what we should have done differently in terms of
the translation method. In \secref{sec:codegenerationandinterpretation} in the
analysis chapter we have analysed and discussed the differences between these
translation methods. Furthermore, in \secref{sec:intermediatelanguage} we
describe what an intermediate language is.

\subsubsection{Where are the differences between an interpreter and a compiler?}

In \secref{sec:compilation} we introduced and explained the phases in the
process of compilation, and with \figref{fig:compileroverview} we visualised
these phases.

The initial phases of a compiler are quite similar to the initial phases of an
interpreter. For instance a compiler must also begin by scanning the source code
followed by parsing the source code to get a stream of tokens and an abstract
syntax tree, respectively. These steps are the same as we have described in
\chapref{chap:implementation}. Furthermore, the step where we check if the
different variables are visible in their respective scopes (scope checking) is
also the same as described in \chapref{chap:implementation}. When the above
mentioned phases have been run then the compiler should compile the code to
machine language or to an intermediate language.

According to \figref{fig:compileroverview} the result of the semantic analyser
would be an intermediate language and then the code generator would generate
machine code which can then be executed. It is of course also possible to compile
directly to machine language. Although, it is not the responsible choice to make 
because it kills opportunities to e.g. optimise the code which would be possible 
with an intermediate language.

\subsubsection{What is the advantage of compiling to an intermediate language?}

When compiling to an intermediate language before compiling to the target
language (often the object code), it is possible to make optimisations on the
intermediate code. This is one advantage of compiling to an intermediate
language and results in more efficient executable code.

Furthermore, one can compile to an intermediate language such as Java bytecode
and the compiled code can be run on every machine that supports Java, which is
very useful because the programmer whom is developing a compiler for the
specific language does not have to construct a compiler for every platform; just
one for Jave bytecode and then translate further to the platform. This way the
programmer must only construct $m+n$ compilers. If one does not compile to an
intermediate language, then the programmer must construct a compiler for each
specific platform, which will be a lengthy process. If you have $m$ compilers
and $n$ platforms, then the programmer must construct $m*n$ compilers to be able
to compile to every platform.  Compiling to an intermediate language and then
further translating is illustrated in \figref{fig:mtimesn}.

\subsection{Compile to intermediate language, then translate further}
\label{sec:compiletointermediate}

In this project we decided to implement an interpreter for our programming
language. This decision was made on the basis of our analysis chapter
(\chapref{chap:analysis}) where we could conclude that this would be the optimal
approach for \productname{}, but it is possible to develop a compiler instead. In
this section we will explain what we should have done differently in terms of
the translation method. In \secref{sec:codegenerationandinterpretation} in the
analysis chapter we have analysed and discussed the differences between these
translation methods. Furthermore, in \secref{sec:intermediatelanguage} we
describe what an intermediate language is.

\subsubsection{Where are the differences between an interpreter and a compiler?}

In \secref{sec:compilation} we introduced and explained the phases in the
process of compilation, and with \figref{fig:compileroverview} we visualised
these phases.

The initial phases of a compiler are quite similar to the initial phases of an
interpreter. For instance a compiler must also begin by scanning the source code
followed by parsing the source code to get a stream of tokens and an abstract
syntax tree, respectively. These steps are the same as we have described in
\chapref{chap:implementation}. Furthermore, the step where we check if the
different variables are visible in their respective scopes (scope checking) is
also the same as described in \chapref{chap:implementation}. When the above
mentioned phases have been run then the compiler should compile the code to
machine language or to an intermediate language.

According to \figref{fig:compileroverview} the result of the semantic analyser
would be an intermediate language and then the code generator would generate
machine code which can then be executed. It is of course also possible to compile
directly to machine language. Although, it is not the responsible choice to make 
because it kills opportunities to e.g. optimise the code which would be possible 
with an intermediate language.

\subsubsection{What is the advantage of compiling to an intermediate language?}

When compiling to an intermediate language before compiling to the target
language (often the object code), it is possible to make optimisations on the
intermediate code. This is one advantage of compiling to an intermediate
language and results in more efficient executable code.

Furthermore, one can compile to an intermediate language such as Java bytecode
and the compiled code can be run on every machine that supports Java, which is
very useful because the programmer whom is developing a compiler for the
specific language does not have to construct a compiler for every platform; just
one for Jave bytecode and then translate further to the platform. This way the
programmer must only construct $m+n$ compilers. If one does not compile to an
intermediate language, then the programmer must construct a compiler for each
specific platform, which will be a lengthy process. If you have $m$ compilers
and $n$ platforms, then the programmer must construct $m*n$ compilers to be able
to compile to every platform.  Compiling to an intermediate language and then
further translating is illustrated in \figref{fig:mtimesn}.

\subsection{Compile to intermediate language, then translate further}
\label{sec:compiletointermediate}

In this project we decided to implement an interpreter for our programming
language. This decision was made on the basis of our analysis chapter
(\chapref{chap:analysis}) where we could conclude that this would be the optimal
approach for \productname{}, but it is possible to develop a compiler instead. In
this section we will explain what we should have done differently in terms of
the translation method. In \secref{sec:codegenerationandinterpretation} in the
analysis chapter we have analysed and discussed the differences between these
translation methods. Furthermore, in \secref{sec:intermediatelanguage} we
describe what an intermediate language is.

\subsubsection{Where are the differences between an interpreter and a compiler?}

In \secref{sec:compilation} we introduced and explained the phases in the
process of compilation, and with \figref{fig:compileroverview} we visualised
these phases.

The initial phases of a compiler are quite similar to the initial phases of an
interpreter. For instance a compiler must also begin by scanning the source code
followed by parsing the source code to get a stream of tokens and an abstract
syntax tree, respectively. These steps are the same as we have described in
\chapref{chap:implementation}. Furthermore, the step where we check if the
different variables are visible in their respective scopes (scope checking) is
also the same as described in \chapref{chap:implementation}. When the above
mentioned phases have been run then the compiler should compile the code to
machine language or to an intermediate language.

According to \figref{fig:compileroverview} the result of the semantic analyser
would be an intermediate language and then the code generator would generate
machine code which can then be executed. It is of course also possible to compile
directly to machine language. Although, it is not the responsible choice to make 
because it kills opportunities to e.g. optimise the code which would be possible 
with an intermediate language.

\subsubsection{What is the advantage of compiling to an intermediate language?}

When compiling to an intermediate language before compiling to the target
language (often the object code), it is possible to make optimisations on the
intermediate code. This is one advantage of compiling to an intermediate
language and results in more efficient executable code.

Furthermore, one can compile to an intermediate language such as Java bytecode
and the compiled code can be run on every machine that supports Java, which is
very useful because the programmer whom is developing a compiler for the
specific language does not have to construct a compiler for every platform; just
one for Jave bytecode and then translate further to the platform. This way the
programmer must only construct $m+n$ compilers. If one does not compile to an
intermediate language, then the programmer must construct a compiler for each
specific platform, which will be a lengthy process. If you have $m$ compilers
and $n$ platforms, then the programmer must construct $m*n$ compilers to be able
to compile to every platform.  Compiling to an intermediate language and then
further translating is illustrated in \figref{fig:mtimesn}.

\input{figures/compiletointermediate}

Now it is alsp possible to optimise the compiled source code before further
translation. So, we can combine these two advantages by developing an optimiser
for the intermediate language. This way all code that has been compiled can be
optimised, which gives better efficiency.


Now it is alsp possible to optimise the compiled source code before further
translation. So, we can combine these two advantages by developing an optimiser
for the intermediate language. This way all code that has been compiled can be
optimised, which gives better efficiency.


Now it is alsp possible to optimise the compiled source code before further
translation. So, we can combine these two advantages by developing an optimiser
for the intermediate language. This way all code that has been compiled can be
optimised, which gives better efficiency.


Now it is also possible to optimise the compiled source code before
further translation. This way all
code that has been compiled can be optimised, yielding better
efficiency by noticing common patterns. An intermediate language also
gives the possibility to support multiple platforms and architectures,
if you choose a popular and well supported IL. A well supported IL is a
language that already has a compiler/interpreter built for the target
platforms, saving the programmer from having to write them. Examples
of such ILs can be high-level languages such as $C$, or low-level
languages such as Microsoft's Common Intermediate Language bytecode or
Java bytecode, that abstract away from platform-specific instructions and
registers that other languages, such as assembly language use.

While Java bytecode is interpreted and therefore slower, modern interpreters
use sophisticated methods such as Just-in-Time compilation (JIT), which at
run time compiles intermediate code into native machine code. This process of
course adds an overhead, and the speed differences are not that great
anymore\cite{java-speed}.

\subsubsection{An intermediate format}
To take it a step further, it could also be possible to use an
intermediate format, which could be stored as an archive file that
contains not only the code, but also sounds, images and other resources
required to play a game. This for instance could make it easier to
distribute games in \productname{}. The source code would not be
available like with a compiled game, however a package format could
allow to optionally include the original source if the developer wanted
to share.

Using an non-existing intermediate format however means that you need to
create a compiler, an interpreter, and the IL, which in turn requires a
significant larger amount of work. Then this IL would need to further be
compiled or interpreted so the machine understands it.

\subsection{Interpretation}
\label{sec:interpretation}
An alternative option is to write an interpreter. Interpreters take the
original source code and execute each instruction at each translation.
This means that a program will be parsed and executed on-the-fly
when using an interpreter. It is required that the end-user has the
interpreter. Different games written in our language would then use
the same copy of the interpreter instead of having a copy of the
engine for each executable. This separation will be further explored
in \secref{subsec:engineseperation}. The execution speed will however
suffer and while techniques such as JIT exists to improve this, it is
beyond the scope of this project.

A pure interpretation of a program lies at the opposite end from compilation in
regards to methods of implementation. With this approach, which is illustrated in
\figref{fig:compileroverviewinterpretation}, no translation is performed at all.

\begin{figure}
	\begin{center}
		\scalebox{0.85}{
			\begin{tikzpicture}
  				[node distance=.8cm, start chain=going below,]
  				\node[punktchain, join,] (sprog) {Source program};
  				\node[punktchain, join,] (interp) {Interpreter};
  				\begin{scope}[start branch=venstre, every join/.style={->, thick, shorten <=1pt}, ]
  					\node[punktchain, on chain=going right, join=by {<-}] (indat) {Input data};
  				\end{scope}
  				\node[punktchain, join,] (res) {Result};
			\end{tikzpicture}}
		\end{center}
	\capt{The different phases of an interpreter. Based on Sebesta \textit{et al.}\cite{sebesta2013} p. 48, Figure 1.4}
	\label{fig:compileroverviewinterpretation}
\end{figure}


An interpreter literally ``interprets'' a program written in the targeted language. It
acts like a virtual machine where instructions are statements of a high-level
language. By purely using interpretation, a source code debugger can easily be
implemented. Various errors that might occur can once they are detected, easily
refer to the location of faulty source code that caused the error. Debugging is
eased because the interpreter works like a software implementation of a virtual
machine, thus the state of the machine and the value of a specific variable can
be outputted at any time when requested. This will of course lead to the
disadvantage that an interpreter uses more space than a compiler. Furthermore,
the execution speed of an interpreter is usually 10 to 100 times slower than
that of a compiler\cite[p. 48]{sebesta2013}.

\subsection{Hybrid compilation and interpretation}
The compiling or interpreting approach can be combined to form
a hybrid implementation system. This method is illustrated in
\figref{fig:compileroverviewhybrid}, where a program is compiled into
an intermediate code which is then interpreted. By using this approach,
errors in a program can be detected before interpretation, saving time
for the programmer, since the error will most likely ruin later stages
anyway. Great portability can also be achieved when using hybrid system.
The initial implementation of Java was hybrid and allowed Java to be
compiled to an intermediate code that could run on any platform which
had an implementation of Java Virtual Machine\cite[p. 50]{sebesta2013}.

\begin{figure}[ht]
  \begin{center}
    \scalebox{0.85}{
      \begin{tikzpicture}[node distance=.8cm, start chain=going below,]
        \node[punktchain, join,] (sprog) {Source program};
        \node[punktchain, join] (leana) {Lexical analyser};
        \node[punktchain, join] (syana) {Syntax analyser};
	\node[punktchain, join] (seana) {Intermediate code generator};
	\node[punktchain, join,] (interp) {Interpreter};
	
	\begin{scope}[start branch=venstre, every join/.style={->, thick, shorten <=1pt}, ]
	  \node[punktchain, on chain=going left, join=by {<-}] (indat) {Input data};
      	\end{scope}
  				
  	\node[punktchain, join,] (res) {Result};
  				
  	\draw[tuborg, decoration={brace}] let 
  	      \p1=(leana.south), \p2=(syana.north) in
    	      ($(1.2, \y1)$) -- ($(1.2, \y2)$) node[tubnode] {Token list};
  
  	\draw[tuborg, decoration={brace}] let 
  	      \p1=(syana.south), \p2=(seana.north) in
    	      ($(1.2, \y1)$) -- ($(1.2, \y2)$) node[tubnode] {Abstract syntax
	      tree};
  
  	\draw[tuborg, decoration={brace}] let 
  	      \p1=(seana.south), \p2=(interp.north) in
    	      ($(1.2, \y1)$) -- ($(1.2, \y2)$) node[tubnode] {Intermediate code};
      \end{tikzpicture}}
  \end{center}
  \capt{The different phases of a hybrid implementation systems. Based on
    Sebesta et al.\cite[p. 49, figure 1.5]{sebesta2013}}
  \label{fig:compileroverviewhybrid}
\end{figure}


\subsection{Separation of game and engine}
\label{subsec:engineseperation}
Keeping the game and the engine separated opens up for the possibility of
changing the game engine while still being able to use the same game. 

One major advantage is that it is possible to update the engine and in result
update all your games. An update which improves the graphics or adds new
features such as network support would work with older games instantly without
having to wait for the developer to update it. If the developer no longer
maintains the game an updated version might never come out.

The disadvantage is however that the responsibility for maintaining
compatibility is moved from the developer of the game to the developers of the
engine. A game developer can simply change his program so it works with a new
engine, however the engine developers would have to support games written for
every version released.

\subsection{Summary of code generation and interpretation}
The advantage of compilation is that the outputted code will run faster
because a complete list of instructions will be ready to be executed.
Although, a disadvantage is the time it takes to compile the code will
take longer because the complete source code must be translated, though
only once before executing the program can be done any number of times.

The advantage of interpretation is that it is possible to begin executing the
program quickly because each instruction is interpreted on-the-fly which makes
it faster than compiling the complete code. 

If code is translated to an IL and then further translated, a lot of
work away when talking about generating compilers to machine code
because these compilers are platform dependant.

It is possible to combine compilation and interpretation. The program is
compiled to intermediate code which is then interpreted. By using this approach,
errors in a program can be detected before interpretation which can save much
time for a programmer.

We chose to interpret, because speed of executing a game is not a
factor, and we want to be able to support the programmer as much as
possible, providing detailed error messages if any exist.
