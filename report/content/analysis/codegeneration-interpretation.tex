\section{The interpretation and code generation phases}
\label{sec:codegenerationandinterpretation}

In this section, we present a brief overview of the phases of
translators. Along with the design of our programming language
\productname{}, we also want to make it possible for programmers who
wish to write games in \productname{} to actually have the computer
understand what they write. There are a number of ways we can make that
happen. These are presented in the following sub-sections.

We begin by presenting the translation process followed by a
presentation of what intermediate languages (IL) are and what they can
be used for. In the subsequent section we present interpretation. Then
we present a hybrid solution based upon compilation and interpretation.
Lastly, we present what it will mean to have the game and engine
separate, followed by a short summary, where we take some decisions
based on the what we decide is important for our language.

%\subsection{The translation process}
%\label{sec:translationprocess}
%The typical translator takes as input some given source code written in a
%language with a high level of abstraction and translates it into a language with
%lower abstraction e.g. machine code which can be executed directly by a
%computer.
%\cite[p. 44]{sebesta2013} 

%Some translators work differently though. They translate the source code into
%another high-level language or into machine code for virtual machines, which can
%provide portability. The translation process is typically not a simple task,
%therefore it is often split into different phases, which is shown in
%\figref{fig:compileroverview}. The process can be split into more or less phases
%though, depending on how detailed one wished to describe the process. In this
%section we describe the following phases: the lexical analysis, the syntax
%analysis, the semantic analysis, the code generation and the interpretation.

\subsection{Compilation}
\label{sec:compilation}
With compilation, an executable file is created for a specific platform
that contains all the code required to play the game. Since games have
common aspects, a game engine containing all the common aspects such
as user interface, AI and/or network connection would most likely be
written. This engine would then be included directly in the executable
upon compilation.

The translation process is typically not a simple task,
therefore it is often split into different phases, shown in
\figref{fig:compileroverview}. The process can be split into many or
few stages, depending on how detailed a process is desired and if an
optimisation is done. In \figref{fig:compileroverview}, the lexical and
syntactical analysers make a lookup in a symbol table. Then the semantic
analyser and the code generator use the symbol table to generate
the correct code. Optimisation is optional in the phase of semantic
analysis. \cite[p. 46]{sebesta2013}

\section{Compiler overview}
\label{sec:compileroverview}
The following section presents a brief overview of the phases of translators (compilers and interpreters). The typical translator takes as input some given source code written in a language with a high level of abstraction and traslates it into a language with lower abstraction e.g. machine code which can be executed directly by a computer \cite[p. 44]{sebesta2013}. Some translators work differently though. They translate the source code into another high-level language or into machine code for virtual machines. The translation process is typically not a simple task, therefore it is often split into different phases, which is shown in \figref{fig:compileroverview}. The process can be split into more or less phases though, depending on how detailed one wished to describe the process. In this section we describe the following phases: the lexical analysis, the syntax analysis, the semantic analysis, the code generation and the interpretation.

\begin{figure}
	\begin{center}
		\scalebox{0.85}{
		\begin{tikzpicture}
  			[node distance=.6cm, start chain=going below,]
     		\node[punktchain, join] (scode) {Source code};
     		\node[punktchain, join] (leana) {Lexical analyser};
     		\node[punktchain, join] (syana) {Syntax analyser};
     		\node[punktchain, join] (seana) {Semantic analyser};
     		\node[punktchain, join] (cgen)  {Code generator};
     		\node[punktchain, join] (mach)  {Machine};
     
  			\draw[tuborg, decoration={brace}] let 
  				\p1=(leana.south), \p2=(syana.north) in
    			($(2, \y1)$) -- ($(2, \y2)$) node[tubnode] {Tokens};
  
  			\draw[tuborg, decoration={brace}] let 
  				\p1=(syana.south), \p2=(seana.north) in
    			($(2, \y1)$) -- ($(2, \y2)$) node[tubnode] {Parse tree};
  
  			\draw[tuborg, decoration={brace}] let 
  				\p1=(seana.south), \p2=(cgen.north) in
    			($(2, \y1)$) -- ($(2, \y2)$) node[tubnode] {Intermediate code};
  
 			\draw[tuborg, decoration={brace}] let 		
 				\p1=(cgen.south), \p2=(mach.north) in
    			($(2, \y1)$) -- ($(2, \y2)$) node[tubnode] {Machine language};
		\end{tikzpicture}}
	\end{center}
	\capt{The different phases of a compiler. Based on Sebesta \textit{et al.}\cite{sebesta2013} p. 46, Figure 1.3}
	\label{fig:compileroverview}
\end{figure}

\subsection{Lexical analysis}
The lowest level syntactic units of a language is called lexemes. A language's formal description does not often include these. They are instead described by a lexical specification, regular expressions i.e., separated from the syntactic specification\cite[p. 135]{sebesta2013}. Typical lexemes for a programming language includes integer literals, operators and special keywords like \textit{if} and \textit{while}. If both \textit{\$a} and \textit{\$b} are lexemes describing a variable and \textit{102} and \textit{42} are lexemes describing an integer, then \textit{\$a} and \textit{\$b} or \textit{102} and \textit{42} can typically be used interchangeably and still give a meaningful program. Therefore the lexemes are grouped into tokens. The name of a variable or the value of an integer is preserved when tokenising. The tokens are an abstraction that makes it easier to analyse if correct syntax of the language. An example of the grouping of lexemes into tokens can be seen by \tableref{table:lexandtokens}. After the lexical analysis an input stream of characters has been converted to an output stream of tokens.

%\tab[11cm]{lexandtokens}{7}{Lexemes and their corresponding token group.}
%		               {Input stream}
%       {Lexemes        }{\$a  & = & 3 & \$b & + & 4 & \$a                       }{
%\tabrow{\textbf{Tokens}}{var & assign & int(3) & var(b) & plus & int(4) & var(a)}
%}

\tab[4cm]{lexandtokens}{1}{Lexemes and their corresponding token group.}
		    {               }
{Lexemes   }{\textbf{Tokens}}{
\tabrow{\$a}{var(a) 		}
\tabrow{=  }{assign 		}
\tabrow{3  }{int(3) 		}
\tabrow{\$b}{var(b) 		}
\tabrow{+  }{plus   		}
\tabrow{4  }{int(4) 		}
\tabrow{\$a}{var(a) 		}
}

\subsection{Syntax analysis}
All languages whether natural or artificial is a set of strings of characters over some alphabet. There are rules for how the strings can look that are in a language and how they can be combined. The lexemes described how the strings can look and now the tokens are useful when analysis how the lexemes can be combined. The rules can be specified formally to describe the syntax of a language\cite[p. 135]{sebesta2013}. A common way to describe a language's syntax is by a formal language-generation mechanism (also called grammars or context free grammars). By describing a grammar that can generate all possible strings in a language, the language has also been formally described. Backus-Naur Form is such a mechanism which in the 1950's became the most widely used method for describing programming language syntax\cite[p. 137]{sebesta2013}.
The BNF contains a set of terminals and a set of non-terminals. The terminals are the tokens from the lexical analysis. The non-terminals all have a set of productions, from which a mix of terminals and non-terminals can be derived from. A start production specifies a single non-terminal, from where all syntactically valid strings that are in the language can be derived from by using the production rules until only a sequence of terminals (the tokens) are left. The syntax analysis takes a sequence of tokens as input and tries to create a set of derivations from the start symbol that creates the given sequence of tokens. If success, the input has been parsed and the parse tree is kept for later analysis. The parse tree is the information concerning how the start symbol was derived into the sequence of tokens, which yields a tree structure. This tree is called an abstract syntax tree.

\begin{ebnf}
%Expressions
\grule{program}{\gter{print} \gcat expr}
\grule{expr}{\gter{(} \gcat term \gcat \gter{)} \gcat operator \gcat \gter{(} \gcat term \gcat \gter{)}}
\grule{operator}{\gter{=}}
\galt{\gter{>}}
\galt{\gter{<}}
\grule{term}{number}
\galt{expr}
\grule{number}{\textbf{any number}}
\end{ebnf}


\subsection{Semantic analysis}
Not all characteristics of programming languages are easy to describe with a BNF and some even cannot be described using a BNF. If a programming language allows a floating-point value to be assigned to an integer variable but not the opposite, this \textit{can} be expressed with a BNF but if all such rules should be specified in the BNF, it would increase the size of it remarkably. With increased size, the formal description gets more clumsy to look at and also increases the risk that an error is contained in the BNF.
The rule that all variables must be declared before being used is impossible to express in a BNF. That would require the BNF to remember things, particularly those variables it had seen before, which it cannot. The problem of remembering things also shows up when we start to concern about scope rules. Typically, a variable declared in one scope cannot be used outside that scope. The BNF cannot describe such problems that we describe as static semantics rules. It is named static because the analysis required to check the specifications can be done at compile-time rather than runtime\cite[p. 153]{sebesta2013}.
In this semantic analysis phase, the compiler can check for type rules by starting to decorate the parse tree from the syntactic analysis with types. If the non-terminal \textit{expr} derives the terminal sequence \textit{int} \textit{plus} \textit{int} \textit{semicolon}, it can be decorated with the \textit{int}-type, and the analysis can proceed further up the tree and check that the type of the \textit{expr(int)} is legal. If the \textit{expr(int)} is derived from a \textit{expr} $\rightarrow$ \textit{expr(int)} +  \textit{expr(bool)} production, the static semantic rules must determine if the programs semantic is wrong or it the boolean value can be converted to the integer values zero or one.

\subsection{Code generation}
Every compiler must focus the translation on the capability of a particular machine architecture. The targeted architecture can be virtual such as the Java Virtual Machine. Generally speaking, the code generation phase translates the program into instructions that are carried out by a physical processor. Whether the architecture is virtual or real the program code must be mapped into the processors memory. Typically, the overall translation is broken into smaller pieces, where smaller subtrees of the abstract syntax tree are translated into executable form one at a time. However, there many things that must be considered when translating, i.e. \textbf{instruction selection}, which concerns how an intermediate code representation from the abstract syntax tree is to be implemented. There are many different ways to implement the same functionality, but some might be carried out faster than other. The code generation phase must also deal with problems concerning \textbf{register allocation} and \textbf{code scheduling}. Register allocation is concerned with effectively using the registers so moving the same variables between registers and memory is minimised\cite[p. 521]{fischer2009}. The code scheduling is an important aspect with pipelined processors. The aim is to produce instructions that executes in a way such that the pipelined execution will not have to stall unnecessary\cite[p. 551]{fischer2009}. Some of the problems associated with the pipelined execution is solved by move apart instructions that will interlock\cite[p. 552]{fischer2009}.

\subsection{Interpretation}
A pure interpretation of a program lies at the opposite end (from compilation) regarding to methods of implementation. With this approach, which can be see, on \figref{fig:compileroverviewinterpretation}, no translation is performed at all. An interpreter is interpreting a program written in the targeted language. It acts like a virtual machine which instructions are statements of high level language. By purely using interpretation, a source code debugger can easily be implemented. Various errors that might occur can once they are detected easily refer to which place in the source code that caused the error. The debugging is eased because the interpreter works like a software implementation of a virtual machine, thus the state of the machine and the value of a specific variable can be outputted at any time when requested. This will of course lead to the disadvantage that an interpreter uses more space than a compiler. Further more, the execution speed of an interpreter is usually 10 to 100 times slower than that of a compiler \cite[p. 48]{sebesta2013}.

\begin{figure}
	\begin{center}
		\scalebox{0.85}{
			\begin{tikzpicture}
  				[node distance=.8cm, start chain=going below,]
  				\node[punktchain, join,] (sprog) {Source program};
  				\node[punktchain, join,] (interp) {Interpreter};
  				\begin{scope}[start branch=venstre, every join/.style={->, thick, shorten <=1pt}, ]
  					\node[punktchain, on chain=going right, join=by {<-}] (indat) {Input data};
  				\end{scope}
  				\node[punktchain, join,] (res) {Result};
			\end{tikzpicture}}
		\end{center}
	\capt{The different phases of an interpreter. Based on Sebesta \textit{et al.}\cite{sebesta2013} p. 48, Figure 1.4}
	\label{fig:compileroverviewinterpretation}
\end{figure}

The compiling or interpreting approach can be combined to form a hybrid implementation system. This method is illustrated in \figref{fig:compileroverviewhybrid}, where a program is compiled into an intermediate code which is then interpreted. By using this approach, errors in a program can be detected before interpretation which can save much time for a programmer. A great portability can also be achieved when using hybrid system. The initial implementation of Java was hybrid and allowed Java to be compiled to an intermediate code that could run on any platform which had an implementation of Java Virtual Machine\cite[p. 50]{sebesta2013}. 

\begin{figure}
	\begin{center}
		\scalebox{0.85}{
			\begin{tikzpicture}
  				[node distance=.8cm, start chain=going below,]
  				\node[punktchain, join,] (sprog) {Source program};
  				
	     		\node[punktchain, join] (leana) {Lexical analyser};
    	 		\node[punktchain, join] (syana) {Syntax analyser};
     			\node[punktchain, join] (seana) {Semantic analyser};
  				\node[punktchain, join,] (interp) {Interpreter};
  	
  				\begin{scope}[start branch=venstre, every join/.style={->, thick, shorten <=1pt}, ]
        			\node[punktchain, on chain=going left, join=by {<-}] (indat) {Input data};
      			\end{scope}
  				
  				\node[punktchain, join,] (res) {Result};
  				
  				\draw[tuborg, decoration={brace}] let 
  				\p1=(leana.south), \p2=(syana.north) in
    			($(2, \y1)$) -- ($(2, \y2)$) node[tubnode] {Tokens};
  
  				\draw[tuborg, decoration={brace}] let 
  				\p1=(syana.south), \p2=(seana.north) in
    			($(2, \y1)$) -- ($(2, \y2)$) node[tubnode] {Parse tree};
  
  				\draw[tuborg, decoration={brace}] let 
  				\p1=(seana.south), \p2=(interp.north) in
    			($(2, \y1)$) -- ($(2, \y2)$) node[tubnode] {Intermediate code};
			\end{tikzpicture}}
		\end{center}
	\capt{The different phases of a hybrid implementation systems. Based on Sebesta \textit{et al.}\cite{sebesta2013} p. 49, Figure 1.5.}
	\label{fig:compileroverviewhybrid}
\end{figure}


An obvious disadvantage is that the executable is platform dependant
and it would therefore be necessary to develop a new compiler for each
platform we want to support. On the other hand knowing the specific
platform makes it possible to create optimized code which runs faster.

\subsubsection{An intermediate language}
\label{sec:intermediatelanguage}
A middle step between compiling or interpretation and generating
executable machine code is to translate source code to an intermediate
language (IL), which is then interpreted or compiled further. IL are
usually more low-level than the initial source code, which would make it
possible to optimize code for higher efficiency in later stages, such as
eliminating superfluous node types and dead code.

Furthermore, one can compile to an intermediate language such as Java
bytecode, executable on every machine that supports
Java by having a Java virtual machine installed. This is very useful
because the programmer, whom is developing a compiler for a language,
does not have to construct a compiler for every platform; just one
that translates to Java bytecode, that can then be translated further many
supported platforms. This way the programmer must only construct a
single compiler. If one does not compile
to an intermediate language, then the programmer must construct a
compiler for each specific platform, which will be a lengthy and
cost-heavy process. If you have $m$ compilers and $n$ platforms, then
the programmer must construct $m*n$ compilers to be able to compile
to every platform. The difference between compiling directly to a
platform or to an intermediate language and then further translating is
illustrated in \figref{fig:mtimesn}.


\begin{figure}[ht]
  \begin{center}
    \begin{tikzpicture}[level/.style={sibling distance=30mm/#1}]
      %m+n
      \node [square] (a) {Language $1$};
      \node [square,xshift=8em] (b) {Language $2$};
      \node [square,xshift=16em] (c) {Language $m$};

      \node [ellipse,draw,xshift=8em,yshift=-4em] (il) {Intermediate language};
      
      \node [square,yshift=-8em] (aa) {Platform $1$};
      \node [square,xshift=8em,yshift=-8em] (bb) {Platform $2$};
      \node [square,xshift=16em,yshift=-8em] (cc) {Platform $n$};

      \draw[->, thick,] (a) -- (il);
      \draw[->, thick,] (b) -- (il);
      \draw[->, dashed,] (c) -- (il);
      
      \draw[->, thick,] (il) -- (aa);
      \draw[->, thick,] (il) -- (bb);
      \draw[->, dashed,] (il) -- (cc);
      
      \path (b)--(c) node [midway] {$\cdots$};
      \path (bb)--(cc) node [midway] {$\cdots$};
      
      %m*n
      \node [square,xshift=23em,yshift=-1em] (x) {Language $1$};
      \node [square,xshift=31em,yshift=-1em] (y) {Language $2$};
      \node [square,xshift=39em,yshift=-1em] (z) {Language $m$};
      
      \node [square,xshift=23em,yshift=-7em] (xx) {Platform $1$};
      \node [square,xshift=31em,yshift=-7em] (yy) {Platform $2$};
      \node [square,xshift=39em,yshift=-7em] (zz) {Platform $n$};

      \draw[->, thick,] (x) -- (xx);
      \draw[->, thick,] (x) -- (yy);
      \draw[->, dashed,] (x) -- (zz);

      \draw[->, thick,] (y) -- (xx);
      \draw[->, thick,] (y) -- (yy);
      \draw[->, dashed,] (y) -- (zz);
      
      \draw[->, dashed,] (z) -- (xx);
      \draw[->, dashed,] (z) -- (yy);
      \draw[->, dashed,] (z) -- (zz);
     
      \path (y)--(z) node [midway] {$\cdots$};
      \path (yy)--(zz) node [midway] {$\cdots$};
    \end{tikzpicture}
  \end{center}
  \capt{Difference between compiling to an intermediate language.}
  \label{fig:mtimesn}
\end{figure}



Now it is also possible to optimise the compiled source code before
further translation. This way all
code that has been compiled can be optimised, yielding better
efficiency by noticing common patterns. An intermediate language also
gives the possibility to support multiple platforms and architectures,
if you choose a popular and well supported IL. A well supported IL is a
language that already has a compiler/interpreter built for the target
platforms, saving the programmer from having to write them. Examples
of such ILs can be high-level languages such as $C$, or low-level
languages such as Microsoft's Common Intermediate Language bytecode or
Java bytecode, that abstract away from platform-specific instructions and
registers that other languages, such as assembly language use.

When compiling to an intermediate language before compiling to the target
language (often the object code), it is possible to make optimisations on the
intermediate code. This is one advantage of compiling to an intermediate
language and results in more efficient executable code.

Instead of compiling to native machine code it could be compiled to an
intermediate format such as Java bytecode which is supported on many platforms.
While Java bytecode is interpreted and therefore slower, modern interpreters
use sophisticated methods such as Just-in-Time compilation (JIT), which at
run-time compiles intermediate code into native machine code. This process of
course adds an overhead, and the speed differences are not that great
anymore. \cite{java-speed}

\subsubsection{An intermediate format}
To take it a step further, it could also be possible to use an
intermediate format, which could be stored as an archive file that
contains not only the code, but also sounds, images and other resources
required to play a game. This for instance could make it easier to
distribute games in \productname{}. The source code would not be
available like with a compiled game, however a package format could
allow to optionally include the original source if the developer wanted
to share.

Using an non-existing intermediate format however means that you need to
create a compiler, an interpreter, and the IL, which in turn requires a
significant larger amount of work. Then this IL would need to further be
compiled or interpreted so the machine understands it.

\subsection{Interpretation}
\label{sec:interpretation}
An alternative option is to write an interpreter. Interpreters take the
original source code and execute each instruction at each translation.
This means that a program will be parsed and executed on-the-fly
when using an interpreter. It is required that the end-user has the
interpreter. Different games written in our language would then use
the same copy of the interpreter instead of having a copy of the
engine for each executable. This separation will be further explored
in \secref{subsec:engineseperation}. The execution speed will however
suffer and while techniques such as JIT exists to improve this, it is
beyond the scope of this project.

A pure interpretation of a program lies at the opposite end from compilation in
regards to methods of implementation. With this approach, which is illustrated in
\figref{fig:compileroverviewinterpretation}, no translation is performed at all.

\begin{figure}
	\begin{center}
		\scalebox{0.85}{
			\begin{tikzpicture}
  				[node distance=.8cm, start chain=going below,]
  				\node[punktchain, join,] (sprog) {Source program};
  				\node[punktchain, join,] (interp) {Interpreter};
  				\begin{scope}[start branch=venstre, every join/.style={->, thick, shorten <=1pt}, ]
  					\node[punktchain, on chain=going right, join=by {<-}] (indat) {Input data};
  				\end{scope}
  				\node[punktchain, join,] (res) {Result};
			\end{tikzpicture}}
		\end{center}
	\capt{The different phases of an interpreter. Based on Sebesta \textit{et al.}\cite{sebesta2013} p. 48, Figure 1.4}
	\label{fig:compileroverviewinterpretation}
\end{figure}


An interpreter literally ``interprets'' a program written in the targeted language. It
acts like a virtual machine where instructions are statements of a high-level
language. By purely using interpretation, a source code debugger can easily be
implemented. Various errors that might occur can once they are detected, easily
refer to the location of faulty source code that caused the error. Debugging is
eased because the interpreter works like a software implementation of a virtual
machine, thus the state of the machine and the value of a specific variable can
be outputted at any time when requested. This will of course lead to the
disadvantage that an interpreter uses more space than a compiler. Furthermore,
the execution speed of an interpreter is usually 10 to 100 times slower than
that of a compiler.
\cite[p. 48]{sebesta2013}

\subsection{Hybrid compilation and interpretation}
The compiling or interpreting approach can be combined to form
a hybrid implementation system. This method is illustrated in
\figref{fig:compileroverviewhybrid}, where a program is compiled into
an intermediate code which is then interpreted. By using this approach,
errors in a program can be detected before interpretation, saving time
for the programmer, since the error will most likely ruin later stages
anyway. Great portability can also be achieved when using hybrid system.
The initial implementation of Java was hybrid and allowed Java to be
compiled to an intermediate code that could run on any platform which
had an implementation of Java Virtual Machine. \cite[p. 50]{sebesta2013}

\begin{figure}
	\begin{center}
		\scalebox{0.85}{
			\begin{tikzpicture}
  				[node distance=.8cm, start chain=going below,]
  				\node[punktchain, join,] (sprog) {Source program};
  				
	     		\node[punktchain, join] (leana) {Lexical analyser};
    	 		\node[punktchain, join] (syana) {Syntax analyser};
     			\node[punktchain, join] (seana) {Semantic analyser};
  				\node[punktchain, join,] (interp) {Interpreter};
  	
  				\begin{scope}[start branch=venstre, every join/.style={->, thick, shorten <=1pt}, ]
        			\node[punktchain, on chain=going left, join=by {<-}] (indat) {Input data};
      			\end{scope}
  				
  				\node[punktchain, join,] (res) {Result};
  				
  				\draw[tuborg, decoration={brace}] let 
  				\p1=(leana.south), \p2=(syana.north) in
    			($(2, \y1)$) -- ($(2, \y2)$) node[tubnode] {Token list};
  
  				\draw[tuborg, decoration={brace}] let 
  				\p1=(syana.south), \p2=(seana.north) in
    			($(2, \y1)$) -- ($(2, \y2)$) node[tubnode] {Abstract syntax tree};
  
  				\draw[tuborg, decoration={brace}] let 
  				\p1=(seana.south), \p2=(interp.north) in
    			($(2, \y1)$) -- ($(2, \y2)$) node[tubnode] {Decorated abstract syntax tree};
			\end{tikzpicture}}
		\end{center}
	\capt{The different phases of a hybrid implementation systems. Based on Sebesta \textit{et al.}\cite{sebesta2013} p. 49, Figure 1.5.}
	\label{fig:compileroverviewhybrid}
\end{figure}


\subsection{Separation of game and engine}
\label{subsec:engineseperation}
Keeping the game and the engine separated opens up for the possibility of
changing the game engine while still being able to use the same game. 

One major advantage is that it is possible to update the engine and in result
update all your games. An update which improves the graphics or adds new
features such as network support would work with older games instantly without
having to wait for the developer to update it. If the developer no longer
maintains the game an updated version might never come out.

The disadvantage is however that the responsibility for maintaining
compatibility is moved from the developer of the game to the developers of the
engine. A game developer can simply change his program so it works with a new
engine, however the engine developers would have to support games written for
every version released.

\subsection{Summary of code generation and interpretation}
The advantage of compilation is that the outputted code will run faster
because a complete list of instructions will be ready to be executed.
Although, a disadvantage is the time it takes to compile the code will
take longer because the complete source code must be translated, though
only once before executing the program can be done any number of times.

The advantage of interpretation is that it is possible to begin executing the
program quickly because each instruction is interpreted on-the-fly which makes
it faster than compiling the complete code. 

If code is translated to an IL and then further translated, a lot of
work away when talking about generating compilers to machine code
because these compilers are platform dependant. If we say that we have
$n$ compilers and $m$ platforms, then when compiling to an IL we only
have to develop $n+m$ compilers instead of $n*m$ because when compiling
to an IL every platform can compile to this language and from the IL to
the target platform.

It is possible to combine compilation and interpretation. The program is
compiled to intermediate code which is then interpreted. By using this approach,
errors in a program can be detected before interpretation which can save much
time for a programmer.

We chose to interpret, because speed of executing a game is not a
factor, and we want to be able to support the programmer as much as
possible, providing detailed error messages if any exist.
