\subsection{Chess}
Chess is a board game of two opposing players. Chess is a turn-based game played
on a board of $8 \times 8$ squares. The squares are typically black and white,
but can be any two colors (see figure \ref{fig:chess}). The squares can only
contain one piece at a time, unlike other games e.g. Kalah and Backgammon. 

Each player has a total of $16$ pieces: $8$ pawns, $2$ knights, $2$
bishops, $2$ rooks, $1$ queen and $1$ king. Each piece type has unique
possible moves. For instance a pawn can move only one square vertically
forward or one square diagonal when capturing an enemy piece. A rook
can move unlimited squares either vertically or horizontally. The fact
that each piece type has a unique way to move separates chess from a lot
of other common board games where all pieces have the same abilities,
like Noughts and Crosses, Kalah, Ludo, Backgammon. This means it should
be possible to define a unique movement pattern for numerous different
types of pieces in \productname{}.

Cut to the bone, Chess goes as follows: When a game starts the pieces are in their
starting positions as seen in figure \ref{fig:chess}. The player with the white
pieces always makes the first move and after that the players take turn in
which clever moves are being made and pieces are being captured until one
player has checkmated the other -- and the game is over. The checkmate condition
is obtained when the king piece is in a position to be captured and cannot
escape from capture in the next move\cite{chessrules}. Therefore it should be
possible to look one move ahead to control if the checkmate condition is
obtained.

\subsubsection{Special moves} 
In chess, there are numerous special moves that don't follow the normal pattern
of movements in chess. Earlier we mentioned that a pawn can move only one square
vertically forward or one square diagonally when capturing an enemy piece. But
this is not always the case. If the pawn is in its respective starting position, it
can move either one or two squares vertically forward. After that it can only
move one square forward or one square vertically the rest of the game. 

Another special move is the move called ``Castling''. This move allows a player
to move two pieces in one turn (the king and one of the rooks). But to do the
move several following conditions must be met\cite{chessrules}:
\begin{nlist}
  \item The move has to be the very first move of the king and the rook
  \item There can't be any pieces standing between the king and the rook
  \item There can't be any opposing pieces that could capture the king in his
    original square, the squares he moves through or the square he ends up in
\end{nlist}

There exits two more special moves, called ``En Passant'' and
``Promotion''. These moves are important to the game, but not exactly
relevant. Hence they are not going to be described here, but refer to 
\cite{chessrules} for more information.

So, what's the problem with these special moves? The problem is the fact that
they don't follow the regular pattern of the game and this has to be taken into
consideration when designing \productname{}.

\fig[width=0.25\textwidth]{chess}{The board game chess with the pieces in start
position.}

From brief analysis above, we present this list of interesting game elements we found 
in chess:

\begin{dlist}
\item Pieces have different movement abilities
\item A squared board with a number of squares in it
\item A winning condition - when the king has been checkmated
\item A starting state - how the pieces are placed on the board before the
  game's very first move
\item Special moves like ``Castling'', ``Promotion'', and ``En Passant''
\item Constraints that disallow a piece to move if some condition is true after
  the move has been made (a move that sets your king in check)
\item A piece can be ``captured'' by another piece, which causes the piece to be
  removed from the board
\end{dlist}

From looking at chess, we conclude that these aspects and elements need to be
easily done in \productname{}.
