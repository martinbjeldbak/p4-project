
\section{Summary}
From looking at the 3 board games, Chess, Kalah and Naughts \& Crosses, many different game elements have been recognised. For a programming language that allows all of the 3 games to be described, all of the game elements in each of those must could be described. Some of the elements are very similar, in which case a more generalised description has been provided. For example, the rook piece and the bishop piece in chess do not have equal moves but their moves have been generalised to just \textit{movement by patterns}. The general game elements in the earlier mentioned board games can bee seen here:
\begin{itemize}
\item The game has one single board.
\item The board contains squares in a 2-dimensional grid.
\item The game contains one or more types of pieces.
\item The game has an initial setup.
\item A piece can be either on the board or off the board. If a piece is on the board it is positioned on one specific square.
\item A piece that is off the board may be put on the board.
\item A piece can belong to a player.
\item A pattern may consider the position of occupied squares, empty squares and other pieces on the board in relation to a specific square or a specific piece on the board.
\item A piece has a set of moves which might be an empty set. The possible moves may be based on a pattern.
\item A piece may or shall be removed / exchanged based on a pattern.
\item At any given time, just one player has the turn.
\item If a piece is owned by a player, only that player can use its moves.
\item A player's turn may consist of more than one move.
\item A player can win if the game is in one or more specific states, which can consider a pattern.
\item A game can be a tie if the game is in one or more specific states, which can consider a pattern.
\end{itemize}
