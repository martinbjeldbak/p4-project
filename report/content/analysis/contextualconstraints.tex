\section{Contextual constraints}
\label{sec:contextualconstraints}

Context-free grammars cannot describe syntax which can only be syntactically
correct given a specific context (hence the name of these grammars). This means
that there are constraints which a given context-free grammar cannot describe.

These contextual constraints are i.e.\cite[pg. 39]{plpp}:

\begin{dlist}
\item Declaration before use
\item Scope rules
\item Type correspondence
\item Overriding methods
\end{dlist}

In the following subsections we eplain why these cannot be described by using a
context-free grammar.

%The rule that all variables must be declared before being used is impossible to
%express in a BNF. That would require the BNF to remember things, particularly
%those variables it had seen before, which it cannot. The problem of remembering
%things also shows up when we start to concern about scope rules. Typically, a
%variable declared in one scope cannot be used outside that scope. The BNF cannot
%describe such problems that we describe as static semantics rules. It is named
%static because the analysis required to check the specifications can be done at
%compile-time rather than run-time\cite[p. 153]{sebesta2013}.

%In this semantic analysis phase, the compiler can check for type rules by
%starting to decorate the parse tree from the syntactic analysis with types. If
%the non-terminal \textit{expr} derives the terminal sequence \textit{int}
%\textit{plus} \textit{int} \textit{semicolon}, it can be decorated with the
%\textit{int}-type, and the analysis can proceed further up the tree and check
%that the type of the \textit{expr(int)} is legal. If the \textit{expr(int)} is
%derived from a \textit{expr} $\rightarrow$ \textit{expr(int)} +
%\textit{expr(bool)} production, the static semantic rules must determine if the
%programs semantic is wrong or it the boolean value can be converted to the
%integer values zero or one.

% Scope and their importance
% Semantics at compile-time
\subsection{Declaration before use}

%The rule that all variables must be declared before being used is impossible to
%express in a BNF. That would require the BNF to remember things, particularly
%those variables it had seen before, which it cannot. The problem of remembering
%things also shows up when we start to concern about scope rules. Typically, a
%variable declared in one scope cannot be used outside that scope. The BNF cannot
%describe such problems that we describe as static semantics rules. It is named
%static because the analysis required to check the specifications can be done at
%compile-time rather than run-time\cite[p. 153]{sebesta2013}.

%cite http://www.haskell.org/haskellwiki/The_Monad.Reader/Issue4/Why_Attribute_Grammars_Matter


\section{consideration of scope and type systems}

In the following sections we take a quick look at the consequences of scope and
type systems for \productname{}. We begin by introducing scope rules with a
short discussion of the consequences of static and dynamic scoping. Afterwards,
we introduce and discuss the consequences of static and dynamic type systems.

\subsection{Which scope rules are we considering?}

With static scoping the scopes of variables can be determined prior to
execution. This means that a compiler can easily determine the type of every
variable in the program by just examining the source code. So, when a variable
is referenced in a statically scoped language the value and type of the variable 
is the on it had at the time of declaration.


Static scope rules provide subprograms with access to nonlocal identifiers and
this type of scoping works very well with compilers because the scope can be
determined at compile time.

A disadvantage of static scoping is that it can give too much access and might
need restrictions. But programs are dynamic and are often restructured which
can lead to destruction of initial restrictions.

With dynamic scoping the scopes of variables can only be determine at run time
because it is based on the calling sequence of subprograms. When a variable is
referenced in a language with dynamic scoping then the value and type of the 
variable is what it had on the time of the call to it.
\cite[p. 227]{sebesta2013}
%5.5.6, side 227 (sebesta)

It is not possible to determine scopes statically because the calling sequence
of subprograms is not always known. When a method A calls a method B then B has 
access to variables in that were declared in A. As a result dynamically scoped
languages are much more difficult to read and understand and this results in
them being less reliable.


Comparing two similar programs with different scoping, then the statically
scoped program will be much easier to read, more reliable, and it will execute
much faster than the program written with dynamic scoping.
\cite[p. 229]{sebesta2013}
%5.5.7, side 229 (sebesta)

\subsection{dynamic and static type systems}

The type system is an essential part of the feel and touch of a programming language. There exits two main kinds of type systems: the dynamic and the static type system. Wether to have one over another, is today a hotly discussed topic. In the following section we list some advantages and disadvantages of each. 

When it comes to detecting errors the static type system can make it easier to detect programming errors in an earlier stage than the dynamic type systems. This is due to the fact that it's already possible to check type errors during compile-time rather than at run-time. The readability can also be improved by the static type system because of the precense of type names. This can make it easier for the programmer to get an idea of what a certain sub-program is ment to do. The forced precense of types is also what decreases the writeability of static type systems, since the programmer has to write down the types at all times and, when declaring variable, spend time considering wether a variable should be an integer type, a floating points or another type. This can take valuable programming time. So the dynamic type system is faster to writ and it's more flexible but the static type system is easier to read and more reliable at run-time since type checking is done on compile-time. There is a whole list of other advantages and disadvantages of each type system, which are provocatively explained in \cite{staticvsdynamictypesystem}, where it is argued that perhaps a middle solution between the two should be considered.

But which type system should we then go for? We are aiming to create a programming language in which it's easy and fast to create board games. It should be possible to create a board game within fewest possibles line of code and for that purpose the dynamic type system is the more relevant. 

\input{content/analysis/contextualconstraints/typecorrespondence}
\input{content/analysis/contextualconstraints/overridingmethods}
\subsection{Summary}

% Conclusion: Dynamic typing makes it hard to do typechecking

