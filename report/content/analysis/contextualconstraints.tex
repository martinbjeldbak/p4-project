\section{Contextual constraints for CFGs}
\label{sec:contextualconstraints}

Context-free grammars (CFGs) cannot describe syntax which can only be
syntactically correct given a specific context (hence the name of these
grammars). This means that there are constraints which a given CFG cannot
describe.

These contextual constraints are i.e.\cite[p. 39]{plpp}:

\begin{dlist}
\item Declaration before use
\item Scope rules
\item Type correspondence
\item Overriding methods
\end{dlist}

In the following subsections we eplain why these cannot be described by using a
CFG.

% Scope and their importance
% Semantics at compile-time

\subsection{The issue of remembering things}

%declaration before use
The rule that variables must be declared before they can be used is impossible
to express with CFGs. It would require that the CFG was able remember things,
particularly those variables that have been declared, which it cannot. 

%scope rules
The problem of remembering things also shows up when working with scope rules.
Typically, a variable declared in one scope cannot be used outside that scope.
The CFG cannot describe such scope rules that we describe as static semantics
rules. It is named static because the analysis required to check the
specifications can be done at compile-time rather than run-time.
\cite[p. 153]{sebesta2013}

%type and overriding methods
The same issue is present when working with type systems and the possibility of
overriding methods in a programming language. The CFG must remember names of the
variables and which type they have to be able to compare and devise a conclusion
if a specific type is correct in the given context. Furthermore, the possibility
of overriding methods is not possible because the CFG must remember method
names, formal parameters, and return value to be able to formulate a production
to make it possible to override methods, which is not possible. 

\subsection{Solutions to this problem}
So, the big issue for the CFG is to remember stuff. Are there any grammers that
solve this issue? Yes there is. There exist different kinds of grammars which
are able to describe the semantics of a given language instead of just focusing
on the syntax. Grammars that can describe semantics are specified under a class
of grammars called contextual grammars.
\cite{plpp}
An example of such a grammar is an attribute grammar which actually
``decorates'' a context-free grammar with those attributes we are interested in.
\cite{attrgrammar}

Contextual grammars can take the form of:

\[
  uAv \rightarrow uwv
\]

where $u$ and $v$ is the context which $A$ is in at this transition.  The issue
with contextual grammars is that they are difficult to write, process, and the
biggest issue is that there are no automated generators for efficient generation
of translation.
\cite{attrgrammar}

\subsection{Summary for contextual constraints}
The big issue is that the CFG cannot remember what it has met in the past and
which context a given production must be in to be true. This makes is impossible
to declare rules as declaration before use, scope rules, type rules, etc.

There exist contextual grammars that can describe a language given a specific
context. The biggest issue with these grammars is that it is not possible to
automatically generate efficient translators.

