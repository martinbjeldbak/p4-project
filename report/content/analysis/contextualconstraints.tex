\section{Contextual constraints}
\todo{This is just pasted in, make it fit}
Not all characteristics of programming languages are easy to describe with a BNF and some even cannot be described using a BNF. If a programming language allows a floating-point value to be assigned to an integer variable but not the opposite, this \textit{can} be expressed with a BNF but if all such rules should be specified in the BNF, it would increase the size of it remarkably. With increased size, the formal description gets more clumsy to look at and also increases the risk that an error is contained in the BNF.
The rule that all variables must be declared before being used is impossible to express in a BNF. That would require the BNF to remember things, particularly those variables it had seen before, which it cannot. The problem of remembering things also shows up when we start to concern about scope rules. Typically, a variable declared in one scope cannot be used outside that scope. The BNF cannot describe such problems that we describe as static semantics rules. It is named static because the analysis required to check the specifications can be done at compile-time rather than runtime\cite[p. 153]{sebesta2013}.
In this semantic analysis phase, the compiler can check for type rules by starting to decorate the parse tree from the syntactic analysis with types. If the non-terminal \textit{expr} derives the terminal sequence \textit{int} \textit{plus} \textit{int} \textit{semicolon}, it can be decorated with the \textit{int}-type, and the analysis can proceed further up the tree and check that the type of the \textit{expr(int)} is legal. If the \textit{expr(int)} is derived from a \textit{expr} $\rightarrow$ \textit{expr(int)} +  \textit{expr(bool)} production, the static semantic rules must determine if the programs semantic is wrong or it the boolean value can be converted to the integer values zero or one.

% Scope and their importance
% Semantics at compiletime
\subsection{Scope}
\todo{Write this\ldots}
\subsection{Type system}
\todo{Write this\ldots}

\section{Summary}
From looking at the 3 board games, Chess, Kalah and Naughts \& Crosses, many different game elements have been recognised. For a programming language that allows all of the 3 games to be described, all of the game elements in each of those must could be described. Some of the elements are very similar, in which case a more generalised description has been provided. For example, the rook piece and the bishop piece in chess do not have equal moves but their moves have been generalised to just \textit{movement by patterns}. The general game elements in the earlier mentioned board games can bee seen here:
\begin{itemize}
\item The game has one single board.
\item The board contains squares in a 2-dimensional grid.
\item The game contains one or more types of pieces.
\item The game has an initial setup.
\item A piece can be even on the board or off the board. If a piece is on the board it is also on one specific square.
\item A piece can belong to a player.
\item A pattern may consider the position of occupied squares, empty squares and other pieces on the board in relation to a specific square or a specific piece on the board.
\item A piece has a set of moves which might be an empty set. The possible moves may be based on a pattern.
\item At any given time, just one player has the turn.
\item If a piece is owned by a player, only that player can move it.
\item A turn may consist of more than one move.
\item A player can win if the game is in one or more specific states, which can consider a pattern.
\item A game can be a tie if the game is in one or more specific states, which can consider a pattern.
\end{itemize}

