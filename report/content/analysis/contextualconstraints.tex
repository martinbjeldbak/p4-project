\section{Contextual constraints for context-free grammars}
\label{sec:contextualconstraints}

Context-free grammars (CFGs) cannot describe syntax which can only be
syntactically correct given a specific context (hence the name of these
grammars). This means that there are constraints which a given CFG cannot
describe.
These contextual constraints are i.e.:
\cite[p. 39]{plpp}

\begin{dlist}
\item Declaration before use
\item Scope rules
\item Type correspondence
\item Overriding methods
\end{dlist}

% Scope and their importance
% Semantics at compile-time

\subsection{The issue of remembering}

%declaration before use
The rule that variables must be declared before they can be used is impossible
to express with CFGs. It would require that the CFG was able remember things,
particularly those variables that have been declared, which it cannot. 

%scope rules
The problem of remembering things also shows up when working with scope rules.
Typically a variable declared in one scope cannot be used outside that scope.
The CFG cannot describe such scope rules that we describe as static semantic
rules. It is named static because the analysis required to check the
specifications can be done at compile-time rather than run-time.
\cite[p. 153]{sebesta2013}

%type and overriding methods
The same issue is present when working with type systems and the possibility of
overriding methods in a programming language. The CFG must remember names of the
variables, and which type they have, to be able to compare and devise a conclusion
if a specific type is correct in the given context. Furthermore, the possibility
of overriding methods is not possible because the CFG must remember method
names, formal parameters, and return value to be able to formulate a production
to make it possible to override the methods, which is not possible.

\subsection{Solutions to this problem}
So, the big issue for the CFG is to remember stuff. Are there any grammers that
solve this issue? Yes there is. There exist different kinds of grammars which
are able to describe the semantics of a given language instead of just focusing
on the syntax. Grammars that can describe semantics are specified under a class
of grammars called contextual grammars.
\cite{plpp}
An example of such a grammar is an attribute grammar which actually
``decorates'' a CFG with those attributes we are interested in.
\cite{attrgrammar}

Contextual grammars can take the form of:

\[
  uAv \rightarrow uwv
\]

where $u$ and $v$ is the context which $A$ is in at this transition. The issue
with contextual grammars is that they are difficult to write, process, and there 
are no automated generators for efficient generation of parsers.
\cite{attrgrammar}

\subsection{Dynamic and static scope rules}

In this section we take a quick look at the advantages and disadvantages of
static and dynamic scoping and compare the two types of scoping.

With static scoping the scopes of variables can be determined prior to
execution. This means that a compiler can easily determine the type of every
variable in the program by just examining the source code. So, when a variable
is referenced in a statically scoped language the value and type of the variable 
is the on it had at the time of declaration.
\cite[5.5.1, p. X]{sebesta2013}
%5.5.1, side 219 (sebesta)

Static scope rules provide subprograms with access to nonlocal identifiers and
this type of scoping works very well with compilers because the scope can be
determined at compile time.
\cite[5.5.5, p. X]{sebesta2013}
%5.5.5, side 227 (sebesta)

A disadvantage of static scoping is that it can give too much access and might
need restrictions. But programs are dynamic and are often restructured which
can lead to destruction of initial restrictions.
\cite[5.5.5, p. X]{sebesta2013}
%5.5.5, side 227 (sebesta)

With dynamic scoping the scopes of variables can only be determine at run time
because it is based on the calling sequence of subprograms. When a variable is
referenced in a language with dynamic scoping then the value and type of the 
variable is what it had on the time of the call to it.
\cite[5.5.6, p. X]{sebesta2013}
%5.5.6, side 227 (sebesta)

It is not possible to determine scopes statically because the calling sequence
of subprograms is not always known. When a method A calls a method B then B has 
access to variables in that were declared in A. As a result dynamically scoped
languages are much more difficult to read and understand and this results in
them being less reliable.
\cite[5.5.7, p. X]{sebesta2013}
%5.5.7, side 229 (sebesta)

Comparing two similar programs with different scoping, then the statically
scoped program will be much easier to read, more reliable, and it will execute
much faster than the program written with dynamic scoping.
\cite[5.5.7, p. X]{sebesta2013}
%5.5.7, side 229 (sebesta)

\subsection{Which type systems are we considering?}

The type system is an essential part of the feel and touch of a programming
language. There are two main type systems; the dynamic and the static
type system. Whether or not to choose one over the other is today a very
actively discussed topic. In the following section we list some advantages and
disadvantages for each the two.

When it comes to detecting errors the static type system makes it possible to
detect these in an early stage compared to the dynamic type systems. This is
due to the fact that it is possible to check type errors already during
compile time rather than at run time. The readability is also be improved by
the static type system because of the presence of type names. This for instance
makes it easier for the programmer to get an idea of what a certain
subprogram is ment to do. 

The forced presence of types is also what decreases the writeability of static
type systems, since the programmer has to write down the types at all times, and
when declaring variables, spend time considering whether or not a variable
should be an integer type, a floating point or some other type. This can take
a lot of time from the programmer, instead of just letting the language calculate 
what is best suited when it compiles and runs the program.

So, the dynamic type system is faster to write and it is more flexible, but the
static type system is easier to read and more reliable at run time since type
checking is done at compile time. There is a whole list of other advantages and
disadvantages of each type system, which are provocatively explained in
\cite{staticvsdynamictypesystem}. It is also argued that perhaps a middle
solution between the two type systems could be the optimal solution for a type
system.

Which type system should we then go for in \productname{}? We are aiming to create a
programming language in which it is easy and fast to create board games. It
should be possible to create a board game with as few as possibles lines of code
and for that purpose the dynamic type system is the most suiting.



\subsection{Summary for contextual constraints}
The big issue is that the CFG cannot remember what it has met in the past and
which context a given production must be in to be true. This makes is impossible
to declare rules as declaration before use, scope rules, type rules, etc.

There exist contextual grammars that can describe a language given a specific
context. The biggest issue with these grammars is that it is not possible to
automatically generate efficient translators.

