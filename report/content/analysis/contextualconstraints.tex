\section{Contextual constraints}
\label{sec:contextualconstraints}

Context-free grammars cannot describe syntax which can only be syntactically
correct given a specific context (hence the name of these grammars). This means
that there are constraints which a given context-free grammar cannot describe.

These contextual constraints are i.e.\cite[pg. 39]{plpp}:

\begin{dlist}
\item Declaration before use
\item Scope rules
\item Type correspondence
\item Overriding methods
\end{dlist}

In the following subsections we eplain why these cannot be described by using a
context-free grammar.

%The rule that all variables must be declared before being used is impossible to
%express in a BNF. That would require the BNF to remember things, particularly
%those variables it had seen before, which it cannot. The problem of remembering
%things also shows up when we start to concern about scope rules. Typically, a
%variable declared in one scope cannot be used outside that scope. The BNF cannot
%describe such problems that we describe as static semantics rules. It is named
%static because the analysis required to check the specifications can be done at
%compile-time rather than run-time\cite[p. 153]{sebesta2013}.

%In this semantic analysis phase, the compiler can check for type rules by
%starting to decorate the parse tree from the syntactic analysis with types. If
%the non-terminal \textit{expr} derives the terminal sequence \textit{int}
%\textit{plus} \textit{int} \textit{semicolon}, it can be decorated with the
%\textit{int}-type, and the analysis can proceed further up the tree and check
%that the type of the \textit{expr(int)} is legal. If the \textit{expr(int)} is
%derived from a \textit{expr} $\rightarrow$ \textit{expr(int)} +
%\textit{expr(bool)} production, the static semantic rules must determine if the
%programs semantic is wrong or it the boolean value can be converted to the
%integer values zero or one.

% Scope and their importance
% Semantics at compile-time
\subsection{Declaration before use}



\subsection{Dynamic and static scope rules}

In this section we take a quick look at the advantages and disadvantages of
static and dynamic scoping and compare the two types of scoping.

With static scoping the scopes of variables can be determined prior to
execution. This means that a compiler can easily determine the type of every
variable in the program by just examining the source code. So, when a variable
is referenced in a statically scoped language the value and type of the variable 
is the on it had at the time of declaration.
\cite[5.5.1, p. X]{sebesta2013}
%5.5.1, side 219 (sebesta)

Static scope rules provide subprograms with access to nonlocal identifiers and
this type of scoping works very well with compilers because the scope can be
determined at compile time.
\cite[5.5.5, p. X]{sebesta2013}
%5.5.5, side 227 (sebesta)

A disadvantage of static scoping is that it can give too much access and might
need restrictions. But programs are dynamic and are often restructured which
can lead to destruction of initial restrictions.
\cite[5.5.5, p. X]{sebesta2013}
%5.5.5, side 227 (sebesta)

With dynamic scoping the scopes of variables can only be determine at run time
because it is based on the calling sequence of subprograms. When a variable is
referenced in a language with dynamic scoping then the value and type of the 
variable is what it had on the time of the call to it.
\cite[5.5.6, p. X]{sebesta2013}
%5.5.6, side 227 (sebesta)

It is not possible to determine scopes statically because the calling sequence
of subprograms is not always known. When a method A calls a method B then B has 
access to variables in that were declared in A. As a result dynamically scoped
languages are much more difficult to read and understand and this results in
them being less reliable.
\cite[5.5.7, p. X]{sebesta2013}
%5.5.7, side 229 (sebesta)

Comparing two similar programs with different scoping, then the statically
scoped program will be much easier to read, more reliable, and it will execute
much faster than the program written with dynamic scoping.
\cite[5.5.7, p. X]{sebesta2013}
%5.5.7, side 229 (sebesta)

\input{content/analysis/contextualconstraints/typecorrespondence}
\subsection{Overriding methods}




\section{Summary}
From looking at the 3 board games, Chess, Kalah and Naughts \& Crosses, many different game elements have been recognised. For a programming language that allows all of the 3 games to be described, all of the game elements in each of those must could be described. Some of the elements are very similar, in which case a more generalised description has been provided. For example, the rook piece and the bishop piece in chess do not have equal moves but their moves have been generalised to just \textit{movement by patterns}. The general game elements in the earlier mentioned board games can bee seen here:
\begin{itemize}
\item The game has one single board.
\item The board contains squares in a 2-dimensional grid.
\item The game contains one or more types of pieces.
\item The game has an initial setup.
\item A piece can be even on the board or off the board. If a piece is on the board it is also on one specific square.
\item A piece can belong to a player.
\item A pattern may consider the position of occupied squares, empty squares and other pieces on the board in relation to a specific square or a specific piece on the board.
\item A piece has a set of moves which might be an empty set. The possible moves may be based on a pattern.
\item At any given time, just one player has the turn.
\item If a piece is owned by a player, only that player can move it.
\item A turn may consist of more than one move.
\item A player can win if the game is in one or more specific states, which can consider a pattern.
\item A game can be a tie if the game is in one or more specific states, which can consider a pattern.
\end{itemize}

