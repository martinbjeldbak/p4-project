\subsubsection{Contextual constraints for context-free grammars}
\label{sec:contextualconstraints}

Context-free grammars (CFGs), as explained in \secref{sec:context-freegrammars}, cannot describe syntax which can only be
syntactically correct given a specific context (hence the name of these
grammars). This means that there are constraints which a given CFG cannot
describe.
These contextual constraints are i.e.\:
\cite[p. 39]{plpp}

\begin{dlist}
\item Declaration before use
\item Scope rules
\item Type correspondence
\item Overriding methods
\end{dlist}

% Scope and their importance
% Semantics at compile-time

\subsubsection*{The issue of remembering}

%declaration before use
The rule that variables must be declared before they can be used is impossible
to express with CFGs. It would require that the CFG was able to remember things,
particularly those variables that have been declared, which it cannot. 

%scope rules
The problem of remembering things also shows up when working with
scope rules. Typically, a variable declared in one scope cannot be
used outside that scope, depending on scope rules, of course. The CFG
cannot describe such scope rules that we describe as static semantic
rules. It is named static because the analysis required to check the
specifications can be done at compile-time rather than run-time.
\cite[p. 153]{sebesta2013}

%type and overriding methods
The same issue is present when working with type systems and the
possibility of overriding methods in a programming language. The CFG
must remember names of the variables, which type they have, and to be
able to compare and devise a conclusion if a specific type is correct in
the given context. Furthermore, the possibility of overriding methods
is not possible because the CFG must remember method names, formal
parameters, and return value to be able to formulate a production to
make it possible to override the methods, which is not possible.

\subsubsection*{Solutions to this problem}
The big issue for CFGs is to remember things. Are there any grammars
that solve this issue? Yes there are. There exists different kinds of
grammars which are able to describe the semantics of a given language
instead of just focusing on the syntax. Grammars that can describe
semantics are specified under a class of grammars called contextual
grammars. \cite{plpp} An example of such a grammar is an attribute
grammar which actually ``decorates'' a CFG with those attributes we are
interested in. \cite{attrgrammar} 
Contextual grammars can take the form of:

\[
  uAv \rightarrow uwv
\]

Where $u$ and $v$ are in the context $A$ is in at this transition. The
issue with contextual grammars is that they are difficult to write,
process, and there are no automated generators for efficient generation
of parsers. These kinds of grammars are out of context and will not be
taken much into account.

\cite{attrgrammar}

\subsubsection*{Summary for contextual constraints}
The issue is that CFGs cannot remember what they have met in the past, and
which context a given production must be in to be true. This makes is impossible
to declare rules as declaration before use, scope rules, type rules, etc.

There exists contextual grammars that can describe a language given a specific
context. The largest issue with these grammars is that it is not possible to
automatically generate efficient translators.
