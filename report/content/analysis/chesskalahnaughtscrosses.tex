\section{Chess, Kalah and Naughts \& Crosses}

In the following sections a detailed analysis of the three games: Chess,
Kalah, and Naughts \& Crosses will be performed. The reason why we pick
these three games is because they are among those we have personal
interest in, and feel are essential for a generic board game programming
language's possible descriptions of games. Therefore we want to dig
deeper into the details of the components of these games (e.g. the
pieces, the board, the squares, etc.) to gain a better understanding
of which features are needed in \productname. The respective history
and basic rules of the games and other related information will not
be included in the analysis, since this has no relevance for gaining
understanding of how our programming language should be designed.

\begin{center}
    \begin{tabular}{| l | l | l | l |}                             \hline
      & Production & Input      & Action                        \\ \hline
    1 & S          & int + int  & Predict S $\rightarrow$ E     \\ 
    2 & E          & int + int  & Predict E $\rightarrow$ T + E \\ 
    3 & T + E      & int + int  & Predict T $\rightarrow$ int   \\
    4 & int + E    & int + int  & Match int                     \\ 
    5 & + E        & + int      & Match +                       \\ 
    6 & E          & int        & Predict E $\rightarrow$ T     \\ 
    7 & T          & int        & Predict T $\rightarrow$ int   \\
    8 & int        & int        & Match int                     \\ 
      &            &            & Accept                        \\ \hline
    \end{tabular}
\end{center}

\begin{center}
    \begin{tabular}{| l | l | l | l |}                              \hline
      & Workspace  & Input      & Action                         \\ \hline
    1 &            & int + int  & Shift                          \\ 
    2 & int        & + int      & Reduce T $\rightarrow$ int     \\ 
    3 & T          & + int      & Shift                          \\
    4 & T +        & int        & Shift                          \\ 
    5 & T + int    &            & Reduce T $\rightarrow$ int     \\ 
    6 & T + T      &            & Reduce E $\rightarrow$ T       \\ 
    7 & T + E      &            & Reduce E $\rightarrow$ T + E   \\
    8 & E          &            & Reduce S $\rightarrow$ E       \\ 
      & S          &            & Accept                         \\ \hline
    \end{tabular}
\end{center}

\subsection{Chess}
Chess is a board game of two oppenent players. It's a turn-based game which means one player makes a move, 
then the other player makes a move, then the first player makes a move and so on. Chess is played on a board of $8 \times 8$ squares. The squares are typically black and white, but can be any two colors (see figure \ref{fig:chess}). The squares can only contain one piece at a time, unlike games like Mancala and Backgammon. Each player has a total of 16 pieces: 8 pawns, 2 knights, 2 bishops, 2 rooks, a queen and a king. Each type of piece has unique ways to move. For instance a pawn can move only one square vertically forward or one square diagonal when capturing an enemy piece. A rook can move unlimited squares either forward or backward (vertical movement), or to the right or to the left (horisontal movement). This separates chess from a lot of other common board games where all pieces have the same abilities, like Naughts and Crosses, Mancala, Ludo, Backgammon.  

Cut to the bone Chess goes as follow: When a game starts the pieces are in their starting positions as seen in figure \ref{fig:chess}. The player with the white pieces always makes the first move, and after that the players shifts in turn in which clever moves are beeing taken and pieces are beeing captured until one player has checkmated the other - and the game is over. The checkmate condition is obtained when the king piece is in a position to be captured and cannot escape from capture in the next move. \cite{chessrules}. Therefore it's nessecary to look one move ahead to control if the checkmate condition is optained.

Special moves. In chess there are numerus special moves which doesn't follow the normal pattern of chess. Earlier we mentioned that a pawn can move only one square vertically forward or one square diagonal when capturing an enemy piece. But this is not always true. If the pawn is in it's respective starting position it can move either one or two squares vertically forward. After that it can only move one square forward or one square vertically the rest of the game. Another special move is the move called ``castling''. This move allows a player to move two pieces in one turn (the king and one of the rooks). But to do the move several conditions needs to be met. First: the move has to be the very first move of the king and the rook, second: there can't be any pieces standing between the king and the rook and third: there can't be any opposing pieces that could capture the king in his original square, the squares he moves through or the square he end up in \cite{chessrules}. There exits two more special moves which are called ''En Passant´´ and ''Promotion´´. These are not going to described here, but information about them can be found in \cite{chessrules}. So what is the problem with these special moves? The problem is the fact that they don't follow the regular pattern of the game and this has to be taken into consideration when designing \productname{}.

\fig[scale=0.1]{chess}{The board game chess with the pieces in start position.}

From the above analysis here is a list of interesting game elements we found in chess:
\begin{itemize}[noitemsep]
\item Pieces has different movement abilities.
\item A squared board with a number of squares in it.
\item A winning condition - when the king has been checkmated.
\item A starting state - how the pieces are placed on the board before the game's very first move.
\item Special moves like ``castling'', ``Promotion'' and ``En Passant.
\item Constraints that disallows a piece to move if some condition is true after the move has been made (a move that sets your king in check).
\item A piece can be ``captured'' by another piece, which causes the piece to be removed from the board.
\end{itemize}

\subsection{Kalah}

Kalah is like chess, a turn-based game of two opposing players. The Kalah pieces,
called seeds, are very different from the Chess pieces. They do not have
specific moves but rather functions, as their name also suggest, as seeds. The
board is not like the Chess board either. It consists of 14 squares, sometimes
referred to as houses,\cite{kalahrules} with two of the houses separating
themselves from the rest by being the houses or bases of each of the players.
Furthermore, each player has six houses belonging to them (see
\figref{fig:kalah}). Each house (including the players' houses) can contain an
arbitrary number of seeds, unlike in chess where the squares can only contain
one piece.  

Cut to the bone, Kalah goes as follow; When the game starts, each of the 12
houses contain 4 seeds (in some versions of the game each house contains 5 or 6
seeds) and the player bases are empty. Now the players take turn to pick up
piles of seeds and deal them out to the 12 houses and their own base. The
dealing of seeds works by a player picking up a pile of seeds from one of his
six houses and dropping one seed down in each of the following houses moving
counter-clockwise. If, when dealing out seeds, the player lands in a house
belonging to himself (not including his own base), which is not empty, the
player can pick up the pile of seeds in the house and start dealing these out.
The turn shifts once a player, when dealing out seeds, lands in an empty house
or in one of the opponent's houses. For a more detailed description of the
rules, we refer to \cite{kalahrules}. The game is over once one of the players'
six houses are empty and the winner is the one who has the most seeds in their
base.

\fig[width=0.25\textwidth]{kalah}{The board game Kalah.}

\subsubsection{Special moves}
Like in Chess there are some special moves in Kalah which don't follow the
regular pattern of the Kalah game. We are not going to describe them here, but
they will be present in the list of interesting game elements and can be found
in a detailed description in \cite{kalahrules}.

Here is a list of some interesting game elements we found in Kalah. For
simplicity we are going to refer to houses and bases as squares and refer to
seeds as pieces:

\begin{dlist}
  \item Squares can contain an arbitrary number of pieces
  \item Making a move can be considered as simple as choosing a square
  \item The number of pieces on a square determines how long a move you can make
  \item A turn may contain more than one move
    \begin{dlist}
      \item If the last piece is dropped in a non-empty square, the player can
	make another move
    \end{dlist}
  \item A square can belong to a player
  \item Squares can be related to other squares
    \begin{dlist}
      \item You place pieces on squares counter-clockwise
      \item If you place the last piece on an empty square, the square across
	the board belonging to the opponent is emptied over to your own square
    \end{dlist}
  \item An end game condition - when all of the squares belonging to a player are
    empty
  \item A winning condition - the player with the most pieces in their square
    wins when the end game condition has been met
  \item Only one type of piece
\end{dlist}

\subsection{Naughts \& Crosses}

Naughts \& Crosses is like Kalaha and Chess a game of two opposing players. The game is played on a board with 3 $\times$ 3 squares. Each square has the same properties and allows only one piece to reside on it. There is only one type of piece which can belong to one of the two players. The board starts off empty and each player will in turn drop pieces onto it. In some varieties, every players has only three pieces which may be moved once all on the board. In others, pieces may not be moved but may be dropped on until the board is full and the game is a tie. The game is won when a player has 3 pieces aligned horizontally, vertically or diagonally. 

Interesting game elements in Naughts \& Crosses:
\begin{dlist}
	\item Turn-based
	\item Only one type of pieces
	\item Pieces are dropped onto the board
	\item A game is won when pieces match an alignment (a pattern) 
\end{dlist}      

\section{Summary}
From looking at the 3 board games, Chess, Kalah and Naughts \& Crosses, many different game elements have been recognised. For a programming language that allows all of the 3 games to be described, all of the game elements in each of those must could be described. Some of the elements are very similar, in which case a more generalised description has been provided. For example, the rook piece and the bishop piece in chess do not have equal moves but their moves have been generalised to just \textit{movement by patterns}. The general game elements in the earlier mentioned board games can bee seen here:
\begin{itemize}
\item The game has one single board.
\item The board contains squares in a 2-dimensional grid.
\item The game contains one or more types of pieces.
\item The game has an initial setup.
\item A piece can be even on the board or off the board. If a piece is on the board it is also on one specific square.
\item A piece can belong to a player.
\item A pattern may consider the position of occupied squares, empty squares and other pieces on the board in relation to a specific square or a specific piece on the board.
\item A piece has a set of moves which might be an empty set. The possible moves may be based on a pattern.
\item At any given time, just one player has the turn.
\item If a piece is owned by a player, only that player can move it.
\item A turn may consist of more than one move.
\item A player can win if the game is in one or more specific states, which can consider a pattern.
\item A game can be a tie if the game is in one or more specific states, which can consider a pattern.
\end{itemize}

