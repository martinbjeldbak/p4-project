\subsection{Which type systems are we considering?}
\label{sec:typesystemanalysis}

The type system is an essential part of the touch and feel of a programming
language. There are two main type systems: the dynamic and the static type
system. Whether to choose one over the other is today a very actively discussed
topic. In the following section we list some advantages and disadvantages for
each.

When it comes to detecting errors, static type systems make it possible to
detect these in an early stage compared to the dynamic type systems. This is due
to the fact that it is possible to check type errors already during compile
time, rather than at run time. The readability is also improved by the static
type system because of the presence of type names. This for instance makes it
easier for the programmer to get an idea of what a certain subprogram is meant
to do.

But the forced presence of types is also what decreases the writability of
static type systems, since the programmer has to write down the types at all
times, and when declaring variables, spend time considering whether or not a
variable should be an integer type, a floating point or some other type, for
example. This can take a lot of time for the programmer, instead of just letting
the language calculate what is best suited when it compiles and runs the
program.

So the dynamic type system is faster to write and it is more flexible, but the
static type system can be easier to read and is more reliable at run time, since
type checking is done at compile time. There is a big list of other advantages
and disadvantages of each type system, which are explained in
\cite{staticvsdynamictypesystem}. Here it is argued that perhaps a middle
solution between the two type systems could be the optimal solution for a type
system.

Which type system should we then go for in \productname{}? We are aiming to
create a programming language in which it is easy and fast to create board
games. It should be possible to create a board game with as few as possibles
lines of code, and for that purpose the dynamic type system is suitable.
