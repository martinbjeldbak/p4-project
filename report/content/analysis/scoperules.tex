\section{Consideration of scope and type systems}
\label{sec:anal-scoperules}
In the following sections we take a quick look at the consequences of scope and
type systems for \productname{}. We begin by introducing scope rules with a
short discussion of the consequences of static and dynamic scoping. Afterwards,
we introduce and discuss the consequences of static and dynamic type systems.

\subsection{Which scope rules are we considering?}

With static scoping the scopes of variables can be determined prior to
execution. This means that a compiler can easily determine the type of every
variable in the program by just examining the source code. So, when a variable
is referenced in a statically scoped language, the value and type of the
variable is the one it had at the time of declaration.

Static scope rules provide subprograms with access to non-local identifiers.
This type of scoping works very well with compilers because the scope can be
determined at compile time, which allows a compiler to do certain optimisations.

A disadvantage of static scoping is that it can give too much access and might
need restrictions. But programs are dynamic and are often restructured which can
lead to destruction of initial restrictions.

With dynamic scoping the scopes of variables can only be determined at run time,
because it is based on the calling sequence of subprograms.  When a variable is
referenced in a language with dynamic scoping then the value and type of the
variable is what it had on the time of the call to it\cite[p.
227]{sebesta2013}.

It is not possible to determine scopes statically, because the calling sequence
of subprograms is not always known. When a method \methodref{A} calls a method
\methodref{B}, then \methodref{B} has access to variables in that were declared
in \methodref{A}. As a result dynamically scoped languages are much more
difficult to read and understand, and this results in them being less reliable.

Comparing two similar programs with different scoping, then the statically
scoped program will be much easier to read, more reliable, and it will execute
much faster than the program written with dynamic scoping\cite[p. 229]{sebesta2013}.

\subsection{Which type systems are we considering?}

The type system is an essential part of the feel and touch of a programming
language. There are two main type systems; the dynamic and the static
type system. Whether or not to choose one over the other is today a very
actively discussed topic. In the following section we list some advantages and
disadvantages for each the two.

When it comes to detecting errors the static type system makes it possible to
detect these in an early stage compared to the dynamic type systems. This is
due to the fact that it is possible to check type errors already during
compile time rather than at run time. The readability is also be improved by
the static type system because of the presence of type names. This for instance
makes it easier for the programmer to get an idea of what a certain
subprogram is ment to do. 

The forced presence of types is also what decreases the writeability of static
type systems, since the programmer has to write down the types at all times, and
when declaring variables, spend time considering whether or not a variable
should be an integer type, a floating point or some other type. This can take
a lot of time from the programmer, instead of just letting the language calculate 
what is best suited when it compiles and runs the program.

So, the dynamic type system is faster to write and it is more flexible, but the
static type system is easier to read and more reliable at run time since type
checking is done at compile time. There is a whole list of other advantages and
disadvantages of each type system, which are provocatively explained in
\cite{staticvsdynamictypesystem}. It is also argued that perhaps a middle
solution between the two type systems could be the optimal solution for a type
system.

Which type system should we then go for in \productname{}? We are aiming to create a
programming language in which it is easy and fast to create board games. It
should be possible to create a board game with as few as possibles lines of code
and for that purpose the dynamic type system is the most suiting.

