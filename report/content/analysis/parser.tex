\section{Parser}
\label{sec:parser}

A parser is the component or algorithm that controls whether or not the source code of a given application is set up syntactically correct according to the programming language it is written in. This process is called the syntax analysis. In short, the parser takes as input the stream of tokens produced by the scanner, and checks if the given sequence of tokens corresponds to the program language's grammar. If this is the case the sequence is syntactically correct else a syntax error has occurred, which has to be dealt with before the compiler or interpreter can proceed. In this section we are going to make an analysis of some different types of parsers. There exists two main approaches to parsing, namely top-down parsing and bottom-up parsing, which have very different ways of dealing with the parsing process. A variety of parsers derive from each approach. For instance the LL parser, the LR parser and the LALR parser, which we are going to analyse. Further more we are going to look at different methods for making parsers, more specifically we are going to analyse some of the pros and cons against writing the parser by hand versus using parser generator tools to generate it. This section is an important rationale as to our choice of parser type and how we want to implement it. 

\subsection{LL-parsers and LR-parsers}

The LL-parsers is a family of parsers which derives from the top-down parsing approach. In this approach the parsers starts at the start symbol of a grammar and through a series of leftmost derivations tries to match the input string, if possible. On the opposite the LR-parsers starts with the input string and through a series of reductions tries to get back to the start symbol.  

In both approaches a parse tree is build. The root of the parse tree contains the start symbol, and the branches the input string. The LL-parsers
build the parse tree from the root down to the branches - hench the name top-down parsing. While LR-parser builds the parse tree from the branches up to the root - hench buttom-up. A simple illustration of a parse tree can be seen in figure ref(fig:XXXX). Here the parser gets the input string "int + int". 
And it's based on the simple context free grammar seen in table ref{table:XXXX).

The LL-parsers has two actions: predict and match. The predict action is used when the parser 
is trying to guess the next production to apply in order to get closer to the input string. 
While the match action eats the next unconsumed input symbol if it corresponds to the leftmost 
predicted terminal. These two actions are continuosly called until the entire input string has 
been eaten and thereby has been matched.

The LR-parsers also has two actions: the shift action and the reduce action. The shift action
adds the next input symbol into a buffer for consideration. The reduce action reduces a collection
of nonterminals and terminals into a nonterminal by reversing a production. These two actions are
continously called until the input string is reduced to the start symbol. 
cite{http://stackoverflow.com/questions/5975741/what-is-the-difference-between-ll-and-lr-parsing}.
An example of a LR(2)-parser and LL(1)-parser in action can be seen in table \ref{table:LL(1)} and 
table \ref{table:LR(2)}.  

\begin{center}
    \begin{tabular}{| l | l | l | l |}                             \hline
      & Production & Input      & Action                        \\ \hline
    1 & S          & int + int  & Predict S $\rightarrow$ E     \\ 
    2 & E          & int + int  & Predict E $\rightarrow$ T + E \\ 
    3 & T + E      & int + int  & Predict T $\rightarrow$ int   \\
    4 & int + E    & int + int  & Match int                     \\ 
    5 & + E        & + int      & Match +                       \\ 
    6 & E          & int        & Predict E $\rightarrow$ T     \\ 
    7 & T          & int        & Predict T $\rightarrow$ int   \\
    8 & int        & int        & Match int                     \\ 
      &            &            & Accept                        \\ \hline
    \end{tabular}
    \label{table:LL(1)}
\end{center}

\begin{center}
    \begin{tabular}{| l | l | l | l |}                              \hline
      & Workspace  & Input      & Action                         \\ \hline
    1 &            & int + int  & Shift                          \\ 
    2 & int        & + int      & Reduce T $\rightarrow$ int     \\ 
    3 & T          & + int      & Shift                          \\
    4 & T +        & int        & Shift                          \\ 
    5 & T + int    &            & Reduce T $\rightarrow$ int     \\ 
    6 & T + T      &            & Reduce E $\rightarrow$ T       \\ 
    7 & T + E      &            & Reduce E $\rightarrow$ T + E   \\
    8 & E          &            & Reduce S $\rightarrow$ E       \\ 
      & S          &            & Accept                         \\ \hline
    \end{tabular}
    \label{table:LR(2)}
\end{center}

Compared to the LR-parser, the LL-parsers are much easier to write by hand and understand. They
are not as powerfull as the LR-parsers in the sense that they do not accept as many grammars, and
the LR-parser are generally parsing faster. ****** KILDER og flere sammenligninger følger *******





   