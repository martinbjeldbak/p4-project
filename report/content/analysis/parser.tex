\section{Parser}
\label{sec:parser}

A parser is the component or algorithm that controls whether or not the source code of a given application is set up syntactically correct according to the programming language it is written in. This process is called the syntax analysis. In short, the parser takes as input the stream of tokens produced by the scanner, and checks if the given sequence of tokens corresponds to the program language's grammar. If this is the case the sequence is syntactically correct else a syntax error has occurred, which has to be dealt with before the compiler or interpreter can proceed. In this section we are going to make an analysis of some different types of parsers. There exists two main approaches to parsing, namely top-down parsing and bottom-up parsing, which have very different ways of dealing with the parsing process. A variety of parsers derive from each approach. For instance the LL parser, the LR parser and the LALR parser, which we are going to analyse. Further more we are going to look at different methods for making parsers, more specifically we are going to analyse some of the pros and cons against writing the parser by hand versus using parser generator tools to generate it.

\subsection{Top-down parsing}

The 



   