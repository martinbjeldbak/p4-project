\section{Chess, Draughts and Naughts \& Crosses}

In the following sections a detailed analysis of the three games: Chess, Draughts, and Naughts \& Crosses will be performed. The reason for picking these three games in particular, is first of all the fact that none of the games includes dice, cards or other related objects, accept for it's pieces and a board. The second reason why we pick these three games is because they are among those we have biggest personal interest in. Therefore we want to dig deeper into the details of the components of these games (e.g. the pieces, the board, the fields etc.) to gain a better understanding of which features are needed in \productname. The respective history of the games and other related information will not be included in the analysis, since this has no relevans for gaining understandig of how our programming language could be designed.  

\subsection{Chess}
Chess is a game of two players playing against each other. It is played on a board of 8 $\times$ 8 fields. The fields are typically black and white, but can be any two colors (see picture) %\figref{fig:chess}). 

%\begin{figure}
	%\centering
	%	\includegraphics[scale=0.6]{pictures/rigebilleder/problemomraade.png}
	%	\capt{Rigt billede, der visualiserer problemet ved at skulle genbruge madrester.}
%\label{fig:rigbillede1}
%\end{figure}

\subsection{Naughts}

\subsection{Naughts \& Crosses}       