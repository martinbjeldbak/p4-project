\section{The syntactic translation phase}
\label{sec:syntacticanalysis}

In this section we describe and analyse the phase where a compiler performs
syntatic analysis. The syntatic analysis is performed by the algorithm called a
parser. A parser checks whether or not the source code of a given program is set
up syntactically correct according to the programming language it is written in.
All programming languages have rules for how its tokens can be combined (tokens
are described in \secref{sec:lexicalanalysis}).  A common way to describe these
rules is by using formal language-generation mechanisms called grammars or
context-free grammars (context-free grammers are described in
\secref{sec:context-freegrammars}). 

There exist two main approaches to perform syntatic analysis; one is top-down parsing
from which the family of LL($k$) parsers derive (the $k$ defines the number of
tokens needed as look ahead). The other is bottom-up parsing from which the
family of LR($k$) parser derive. In \secref{subsec:llparsersandlrparsers} we
analyse how the LL- and LR-parsers work and we look at the advantages and
disadvantages of each by comparing them. Furthermore, we look at different
methods for making parsers, more specifically we analyse the pros and cons
against writing the parser by hand vs. using parser-generating tools to generate
a parser, which is presented in \secref{subsec:constructingaparser}.

\subsection{Lexical analysis}
\label{sec:lexicalanalysis}
Before the syntax analysis can be performed a scanner must perform a lexical
analysis. The scanner's job is basically to check a given source code for lexical
errors and translating the input stream of characters from the source code into
a stream of tokens which the parser can work with, this
is done by identifying every lexeme in the source and attaching a potential
token to it.
\cite[p. 57]{fischer2009}

Lexemes are strings of characters described by regular expressions.  Typical
examples of lexemes in a programming language are: variable names, integer
literals, operators and special keywords etc. A variable name lexeme could be
defined by the following regular expression: $[a-z, A-Z, "\_"][a-z, A-Z, 0-9,
"\_"]^*$. Which means a variable name can start with either an uppercase letter,
lowercase letter or an underscore followed by zero or more lowercase letters,
uppercase letters, numbers or underscores. In \tableref{table:lexandtokens} we
give examples of lexemes and the tokens they have been paired with. If both
\textit{a} and \textit{b} are lexemes describing variable names and \textit{102}
and \textit{42} are lexemes describing integer literals, then \textit{a} and
\textit{b} or \textit{102} and \textit{42} can typically be used interchangeably
and still give a syntatic meaningful program.

\tab[10cm]{lexandtokens}{1}{Lexemes and their corresponding token group.}
		             {               }
       {Lexemes             }{\textbf{Tokens} } {
\tabrow{ x                  }{VAR\_NAME       } 
\tabrow{ random\_var\_name  }{VAR\_NAME       }
\tabrow{ RANdom\_var\_name2 }{VAR\_NAME       }
\tabrow{ 1 		    }{INT\_LITERAL    }
\tabrow{ 342 		    }{INT\_LITERAL    }
\tabrow{ 52890 		    }{INT\_LITERAL    }
\tabrow{ +		    }{PLUS\_OPERATOR  }
\tabrow{ - 		    }{MINUS\_OPERATOR }
\tabrow{ * 		    }{MULT\_OPERATOR  }
\tabrow{ if		    }{KEYWORD         }
\tabrow{ while 		    }{KEYWORD  	      }
\tabrow{ switch 	    }{KEYWORD  	      }
}

A scanner is a relatively simple component which can be constructed by writing
it by hand or using a scanner-generating tool such as Lex, which generates an
executable scanner by feeding it with a set of regular expressions. When
implementing the scanner for \productname{}, we would likely benefit more from
crafting the scanner by hand than by using a scanner-generating tool. By
crafting it by hand we will know excatly how it is implemented. There might be
some advantages of using a scanner-generating tool such as the fact that it is
more reliable, it is easier to maintain and it is faster to implement if one
already knows how it works, if not, a handwritten scanner might be just as fast. 

\subsection{Context-free grammars}
\label{sec:context-freegrammars}
In the previous section we described how to transfrom an input stream of
characters into an output stream of tokens. We will now describe
Grammars are defined using Backus-Naur Form (BNF).

BNF contains a set of terminals and a set of non-terminals. The terminals are
the tokens from the lexical analysis. The non-terminals all have a set of
productions, from which a mix of terminals and non-terminals can be derived
from. A start production specifies a single non-terminal, from where all
syntactically valid strings that are in the language can be derived from by
using the production rules until only a sequence of terminals (the tokens) are
left. The syntax analysis takes a sequence of tokens as input and tries to
create a set of derivations from the start symbol that creates the given
sequence of tokens. If success, the input has been parsed and the parse tree is
kept for later analysis. The parse tree is the information concerning how the
start symbol was derived into the sequence of tokens, which yields a tree
structure. This tree is called an abstract syntax tree.

\begin{ebnf}
%Expressions
\grule{program}{\gter{print} \gcat expr}
\grule{expr}{\gter{(} \gcat term \gcat \gter{)} \gcat operator \gcat \gter{(}
\gcat term \gcat \gter{)}}
\grule{operator}{\gter{=}}
\galt{\gter{>}}
\galt{\gter{<}}
\grule{term}{number}
\galt{expr}
\grule{number}{\textbf{any number}}
\end{ebnf}

\section{Parser}
\label{sec:parserimplementation}

In this section we present the handwritten parser of \productname{}'s.
We have written a top-down recursive descent parser, which is within the class
of LL(1) parsers. The grammar for \productname{} is suited for this because
e.g.\ it does not have left-recursive productions. In the end of the section we present our work 
with SableCC and the reason why we chose not to continue working with this tool.

\subsection{Constructing the parser}
%structured as the grammar
The parser was very simple to implement, because it is structured exactly the same way as the
grammar which can be found in \chapref{ap:CFG}.  For instance if the grammar
expresses that the next set of terminals must begin with a left bracket (`['),
  then the parser will expect the next token to be a \tokenref{LBRACKET} which
  is the token name for a left bracket. If the grammar then expects a
  non-terminal, then the parser simply calls the method for that non-terminal,
  allowing it to finish, possibly calling more non-terminals and expecting
  terminals, before continuing parsing the next part of the rule.

%discuss the if expression
In \lstref{lst:ifexpr} we give an example of how this structure looks like in
our handwritten parser. The production rule for an if-expression is presented in
\secref{sec:conditionalexpressions}.

%\begin{ebnf}
%\grule{if\_expr}{\gter{if} \gcat expression \gcat \gter{then} \gcat expression
%\gcat \gter{else} \gcat expression}
%\end{ebnf}

The production for if-expression says that every expression of this type
must start with the combination of the two symbols which spell the word
\gter{if}. When the parser meets this word in an expression, it knows
that it has to parse an if-expression.

\lstinputlisting[caption="How if-expressions are parsed using top-down parsing in Java.",
label=lst:ifexpr, language=Java]{listings/ifexpr.java}

\subsection{Building an abstract syntax tree}
%astNode()
In \lstref{lst:ifexpr}, the parser initialises the node for the
expected if-expression. The parser starts by calling the method
\methodref{astNode} to create a node for the Abstract Syntax Tree (AST).
We call the method with information about what type of expression this
is (\tokenref{IF\_EXPR}). The method calls the \methodref{expect} method
to verify that the next token is what we are expecting. If the two
tokens do not match, the parser throws a syntax error with information
about the error. If everything is syntactically correct the parser
constructs a node for the AST for the given expression. The first child
of the node is the boolean expression, and the next two siblings of that
child are the expression branches of the if-expression.

\subsubsection{Terminal and nonterminals}
Every grammar has a finite set of nonterminals and terminals that
constitute the productions of the grammar. We have defined tokens in the
parser for every nonterminal in our grammar. The if-expression has the
token name of \tokenref{IF\_EXPR}.

In the production for the if-expression, we have three terminals: the \gter{if},
\gter{then}, and the \gter{else}. These are all required in the method for any
if-expression. When the parser finishes reading a terminal, it knows that the
following token will be an expression, and therefore a new child for the node is
made with a call to the \methodref{expression} method wherein we parse
expressions. Finally the method returns the node containing every child for the
whole if expression.

\subsection{Looking ahead in the input}
%lookAhead methods - atomic
We mentioned earlier that the parser is an LL(1) parser, which means that the
parser is able to look ahead in the sequence of tokens. We have shown the
\methodref{lookAhead} method to determine if the next token is part of an
atomic expression. The production for the atomic expression is presented in
\secref{sec:atomicexpressions}.

%\begin{ebnf}
%\grule{atomic}{\gter{(} \gcat expression \gcat \gter{)}}
%\galt{variable}
%\galt{list}
%\galt{\gter{/} \gcat pattern \gcat \gter{/}}
%\galt{\gter{this}}
%\galt{\gter{super}}
%\galt{direction}
%\galt{coordinate}
%\galt{integer}
%\galt{string}
%\galt{type}
%\galt{constant}
%\end{ebnf}

An atomic expression can derive quite a few productions. This is why we have
constructed a specific method to determine whether the next token is part of an
atomic expression. This method is shown in \lstref{lst:lookaheadatomic}.

\lstinputlisting[caption="The lookAhead method to determine if the next
  expression is an atomic type.", label=lst:lookaheadatomic,
language=Java]{listings/method_lookAheadAtomic.java}

The method \methodref{lookAheadAtomic} makes use of two methods to figure out if
the next token is part of an atomic expression. The first method is the
\methodref{lookAhead} method that takes a token as an argument and figures out
if the next token in the sequence of tokens are equal to each other. The second
method is the \methodref{lookAheadLiteral} method which is similar to the method 
in \lstref{lst:lookaheadatomic} but instead of checking for atomic expressions it 
checks for literals. All these methods return true or false.

%example of lookAheadAtomic
%LL(1)
In \lstref{lst:examplelookahead} we show an example of how the
\methodref{lookAhead} method is used in the parser. The example is taken from
the \methodref{expression} method. The productions for expressions are presented
in \secref{sec:expressions}.
The production of an expression is reflected in the code of the parser. An
example of this is given in \lstref{lst:examplelookahead}.

%\begin{ebnf}
%\grule{expression}{assignment}
%\galt{if\_expr}
%\galt{lambda\_expr}
%\galt{\gter{not} \gcat expression}
%\galt{operation}
%\end{ebnf}

\lstinputlisting[caption="Use of the \methodref{lookAhead}-method. This example
is from the \methodref{expression}-method.", label=lst:examplelookahead,
language=Java]{listings/example_lookAheadAtomic.java}

The code presented in \lstref{lst:examplelookahead} is a small section of the
\methodref{expression} method. We have removed code from the section which is
not relevant for the example we are trying to give. The removed code is
presented as \{\ldots\}. In \lstref{lst:examplelookahead} we wish to present how
the \methodref{lookAhead} methods are used.

An assignment begins with the reserved word \gter{let} and the first
\methodref{lookAhead} method peeks for exactly that token to determine
if the next production is an assignment. If the method returns true then
the next token is in fact the \gter{let} word, and the parser enters a
new method, namely the \methodref{assignment} method which checks to
determine if the rest of the production is correctly written. The same
is done for the if expression, lambda expression and operations which
begins with the ``loSequence()'' (logical operators).

The operation production is a bit different, because it needs two lookAhead
methods to determine if the next production is an operation. An operation can
begin with either an atomic value or a minus operator. So the code uses a
\methodref{lookAheadAtomic} and a regular \methodref{lookAhead} with the
specific token as a parameter to check if the next production is an operation.

The methods return nodes which are connected with each other to form
a complete AST\@. When the parser has parsed every token of the input,
it can produce an AST that corresponds to the program written in
\productname{}. This shows that the parser is built systematically
according to the grammar, producing a parse tree consisting of AST
nodes.

\subsection{SableCC}
We have also implemented a scanner and parser using a
compiler/interpreter generator, or a compiler compiler
known as SableCC\cite{sableccdoc}. As described in
\secref{subsec:generatedparsers}, it is an automated scanner and
LALR($1$) parser generator written in Java, with support for making
compilers and interpreters. We have implemented an early version of
\productname{} in SableCC to evaluate the capabilities of such a tool.

\subsubsection{Choice of SableCC}
We chose SableCC instead of various other popular tools such as
ANTLR\cite{antlr} and JavaCC\cite{javacc}. Though ANTLR and JavaCC are far more
well-documented than SableCC, we still chose this tool because it's parser
generates a LALR($1$) parser, whereas ANTLR and JavaCC create simpler
LL($k$) parsers, which our hand-written implementation already takes advantage of.
Therefore, we felt LALR($1$) parsing to be more interesting, since it
also supports more powerful grammars (such as left-recursion).

SableCC outputs an abstract syntax node type fokokr each alternation in
every rule in the grammar specifications file. It's then possible to
iterate over these nodes via extending the visitor pattern SableCC also
supplies, generating code or directly interpreting a syntactically
correct program. This is all done in classes separate from the grammar
specifications, which is also desirable and different from ANTLR
and JavaCC, where action code is injected directly in the grammar
specification. This is all done in Java, which is also desirable, since
it would work well with the rest of the project (also written in Java).

\subsubsection{Experience with SableCC}
Our experience with the tool has been rather cumbersome, in that it took
quite a while to read the documentation before and during writing the
specifications, as it simply isn't just copy/pasting EBNF grammar
into a file. An example of the if-expression rule is seen here

\begin{lstlisting}[caption={Part of the grammar specifications file of SableCC, with focus on if-expressions.}]
Tokens
  else          = 'else';
  then          = 'then';
  if            = 'if';

Productions
  expression   = {elopexp} element operator expression
               | {assign} assignment
               | {if} if_expr
               | {lambda} lambda_expr
               | {el} element list?
               | {not} not expression;
  if_expr      = if [left]:expression then [mid]:expression else [right]:expression;
\end{lstlisting}

% Pros
This example brings out the strengths of SableCC, as it looks very
similar to the EBNF for if-expressions, with a few additions. As long as
you know the special syntax and how helpers, tokens, and productions works, it
is possible to create scanners and parsers very quickly. This was not
our case, as no one had experience with any form of compiler-compilers.
Given that it generates a LALR parser, it grants the ability to have
more powerful grammar than what we had designed. Lastly, the fact that
there's a clear and clean separation between automated, generated code
and user code makes the grammar and compilation/interpretation parts
easier to maintain. When adding new features to the language, you simply
have to update the specifications file and generate a new scanner/parser
combination. On the other hand when adding new features to the language while
using a handwritten scanner and parser many lines of codes needs to be change
in order to implement the new feature. 

% Cons
Even though SableCC looks like a prime candidate to continue
interpretation with, we chose not to use the tool. This is because it
took an unreasonable amount of time to figure out how to precisely
define the grammar to keep it from being ambiguous. And with poor
documentation, it took even longer. Also, it offers less control and
customizability, compared to writing our own from scratch. As an
example, the tool offers an application-specific interface to tree
walking the AST nodes with the visitor pattern, requiring knowledge of
how SableCC implements it. SableCC also generates around 17000 lines of
Java code, even for our simple grammar, which seems superfluous compared
to the handwritten code, consisting of around 1500 lines of code.
%  - Don't learn as much about different parser techniques
%  - Old project, not as active anymore

% WHAT TO CALL THIS SUBSUBSECTION?
\subsubsection{Discussion}
We chose not to continue using SableCC on our updated grammar, due
to the weight of cons against pros, and the fact that the time spent
working on implementing SableCC was also spent making the handwritten
scanner and parser and making them work exactly the way we want them to. It might
not be as easy to modify our language with this solution, but the time
spend on modifying and adding features to our language with the use of a
handwritten scanner and parser is not wasted time, but learning time, which
gives us better understanding of their underlying functionality.  


\subsection{Constructing a parser}
\label{subsec:constructingaparser}
As we described in \secref{sec:lexicalanalysis} a scanner can be generetad by a
scanner generator tool. The same can be done for parsers. Since the principle of 
crafting parsers is very systematic there exists
different automated tools to generate parsers with, for grammars that meets
some specific standards. A grammar must for instance not be ambiguous otherwise the tools
cannot make a distinct parser for the grammar. A grammar is ambiguous if a
string can be expressed by more than one different parse tree. In the following section we
anlyse the subjects of contructing a parser by hand and by use of a parser generator
tool. In the end we sum up the pros and cons of each of the methods. 

\subsubsection{Handwritten parsers}
\label{subsec:handwrittenparsers}
Why would you write a parser on your own when you have automated tools for this
job? If we were to construct our own handwritten parser, and not use the tools
already built for this, it would provide us with a greater understanding of how a parser 
work. It is commonly known that one of the best ways to learn is from making mistakes and 
later correcting them. But there are the pitfalls in taking on this task. First of all we 
could be stumbling upon many errors in the code. These errors must be solved before the
parser can be functional. Therefore the construction of the parser in this way, will be
time-consuming.

Programming languages evolve and their grammars can change. When this is the case a 
parser must be maintained so that it still outputs the correct result.
When we have constructed a parser by hand this task will be time-consuming because
we must search through the code of the parser and correct it. 

Whenever we work on correcting existing code we are likely going to run into new errors 
that need to be resolved. This brings us to the topic of how reliable our handwritten 
parsers are. So far we have discussed that the produced code for a handwritten parser 
will be error-prone so this will naturally bring us to conclude that this code must be 
less reliable than the automated generators produced code. This is a big disadvantage because we have to be able to rely on the
output of the parser. To check the validity of a parser, it can be given a set
of inputs that are almost in its language but contains a small error, which must
cause the parser to reject the inputs. Additionally, a set of inputs can also be
constructed, which is in the language, hence the parser must build a
correct abstract syntax tree.

\subsubsection{Generated parsers}
\label{subsec:generatedparsers}
There exists quite a few parser generators. To construct af parser with a generator 
the developer must input a grammar into the generator, and it will output a parser 
for that specific grammar. This can be a bit different from software to software but the grammar is often expressed
using the Extended Backus Naur Form (EBNF). A context-free grammar is on EBNF if it
satifies a certain set of rules and contains some special abilities. We will not describe
these here, but for further information we refer to \cite[152]{fischer2009}. The outputted parser will be
produced in the language specified by the generator. We take a look at two different parser
generator tools, namely SableCC and JavaCC.

SableCC is a bottom-up parser generator that generates LALR(1) parsers. This
generator runs on the Java-platform and produces object-oriented code with
clearly seperated machine-generated code and handwritten code. This contributes
to the simplification and ease of maintaining the code.\cite[pp. 11]{sableccdoc}

JavaCC is a top-down parser generator that generates LL(k) parsers. As the name
of the parser generator suggests, it produces the output code in Java source
code.\cite{wiki-javacc}

What differences are there between a handwritten and a generated parser? We need
a grammar in both methods so what is the advantages of a generated parser? First
of all, it takes less time to construct the parser because once we have a
grammar we can input it in the generator and it will automatically generate the
parser for us. Before we can use the software we have to figure out how to use
it but this is not a complicated process. For every software there is some kind
of tutorial on how to use the software.

The software that generate the parsers have been under development for quite a
while and therefore the developers are using efficient algorithms to implement
the parsers. This means that the parser we construct by hand will not be as fast
and reliable as the ones generated by the software - unless the programmer is
very experienced with a wide knowledge base about this subject. Maintenance of the 
parser is also much easier because everything is machine-produced and can easily be 
changed to correspond with a grammar if it has been changed.

\subsection{Summary}
\label{subsec:summary-parser}
From the above section about handwritten parser and parser generators we conclude the
following advantages and disadvantages for handwriting a parser and generating a parser by a parser
generator tool:

\begin{dlist}
\item Handwriting a parser means we gain a better understanding of how the parser work
\item Both handwritten parsers and generated parsers are time-consuming to construct
\item Handwritten parsers are more time-consuming to maintain than generated parsers
\item Handwritten parsers are less reliable than generated parsers
\item Handwritten parsers are slower than generated parsers
\end{dlist}

It is clear from the above list that there are many benefits in using a parser generator tool to
construct a parser. It will be much easier to reach our goal by using such as tool and gaining 
experience in how they work can be a valuable skill. But we believe that it is also very important 
to gain experience in writting parsers ourself and knowledge on how they work from the beginning to the end. 
Therefore we think it would be beneficial to both write a parser by hand and generate a parser with a parser 
generator tool as well.

