\section{The syntactic translation phase}
\label{sec:syntacticanalysis}

In this section we describe and analyse the phase where a compiler performs
syntatic analysis. The syntatic analysis is performed by the algorithm called a
parser. A parser checks whether or not the source code of a given program is set
up syntactically correct according to the programming language it is written in.
All programming languages have rules for how its tokens can be combined (tokens
are described in \secref{sec:lexicalanalysis}).  A common way to describe these
rules is by using formal language-generation mechanisms called grammars or
context-free grammars. 

Context-free grammers are described in \secref{sec:context-freegrammars}.  There
exists two main approaches to perform syntatic analysis, one is top-down parsing
from which the family of LL(k) parsers derive (the k defines the number of
tokens needed as look ahead). The other is bottom-up parsing from which the
family of LR(k) parser derive.  In \secref{subsec:llparsersandlrparsers} we
analyse how the LL- and LR-parsers work and we look at the advantages and
disadvantages of each by comparing them. Furthermore, we look at different
methods for making parsers, more specifically we analyse the pros and cons
against writing the parser by hand vs. using parser generating tools to generate
a parser, which is presented in \secref{subsec:constructingaparser}.

\subsection{Lexical analysis}
\label{sec:lexicalanalysis}
The lowest level syntactic units of a language is called lexemes. A language's formal description does not often include these. They are instead described by a lexical specification, regular expressions i.e., separated from the syntactic specification\cite[p. 135]{sebesta2013}. Typical lexemes for a programming language includes integer literals, operators and special keywords like \textit{if} and \textit{while}. If both \textit{\$a} and \textit{\$b} are lexemes describing a variable and \textit{102} and \textit{42} are lexemes describing an integer, then \textit{\$a} and \textit{\$b} or \textit{102} and \textit{42} can typically be used interchangeably and still give a meaningful program. Therefore the lexemes are grouped into tokens. The name of a variable or the value of an integer is preserved when tokenising. The tokens are an abstraction that makes it easier to analyse if correct syntax of the language. An example of the grouping of lexemes into tokens can be seen by \tableref{table:lexandtokens}. After the lexical analysis an input stream of characters has been converted to an output stream of tokens.

\tab[4cm]{lexandtokens}{1}{Lexemes and their corresponding token group.}
		    {               }
{Lexemes   }{\textbf{Tokens}}{
\tabrow{\$a}{var(a) 		}
\tabrow{=  }{assign 		}
\tabrow{3  }{int(3) 		}
\tabrow{\$b}{var(b) 		}
\tabrow{+  }{plus   		}
\tabrow{4  }{int(4) 		}
\tabrow{\$a}{var(a) 		}
}
\subsection{Context-free grammars}
\label{sec:context-freegrammars}

In the previous section, we described how the scanner transforms an input stream
of characters into an output stream of tokens. In the following section we will
describe context-free grammars, a component which contains the programming
languages grammar, that is the set of rules for how its tokens can be combined.

A context-free grammar contains a set of terminals, a set of non-terminals, a
set of productions and a start symbol. The terminals are the tokens from the
lexical analysis. The non-terminals all have a set of productions, from which a
mix of terminals and non-terminals can be derived. A production typically
has the following form:

\begin{ebnf}
\grule{non-terminal} {\{terminal \gor \gcat non-terminal\}}
\end{ebnf}

A start production specifies a single non-terminal, from where all syntactically
valid strings, that are in the language, can be derived by using the
production rules until only a sequence of terminals (the tokens) are left. The
syntax analysis takes a sequence of tokens as input and tries to create a set of
derivations from the start symbol that creates the given sequence of tokens. If
successful, the input has been parsed and the parse tree is kept for later
analysis. The parse tree is the information concerning how the start symbol was
derived into the sequence of tokens, which yields a tree structure. This tree
is called an abstract syntax tree.


\subsection{LL- and LR-parsers}
\label{subsec:llparsersandlrparsers}
As mentioned in the introduction to the section, the LL-parsers derive from the
top-down parsing approach. In terms of grammars, this means the LL-parsers
attempt to parse a string by starting at the start symbol of the grammar and
through a series of left-most derivations match the input string. On the
opposite, the LR-parsers derive from the bottom-up parsing approach. Here the
LR-parsers attempt to parse by starting with the input string and through a
series of reductions get back to the start symbol.

The LL-parsers have two
actions; predict and match. The predict action is used when the parser is trying
to guess the next production to apply in order to get closer to the input
string. While the match action eats the next unconsumed input symbol if it
corresponds to the left-most predicted terminal. These two actions are
continuously called until the entire input string has been eaten and thereby has
been matched. An example of a LL(1)-parser can be seen in
\tableref{table:LL1}. In the example the parser is based on the simple grammar: 

\begin{centering}
\begin{ebnf}
  \grule{S}{E}
  \grule{E}{T \gcat + \gcat E}
  \galt{T}
  \grule{T}{int}
\end{ebnf}
\end{centering}

\tab[11cm]{LL1}{3}{A LL(1) parser seen in action parsing the string ``int + int''.}
	      {The process                                          }
{Step  	 }{Production & Input       & Action                        }{
\tabrow{1}{$S$        & $int + int$ & Predict $S \rightarrow E$     }
\tabrow{2}{$E$	      & $int + int$ & Predict $E \rightarrow T + E$ }
\tabrow{3}{$T+E$      & $int + int$ & Predict $T \rightarrow int$   }
\tabrow{4}{$int+E$    & $int + int$ & Match $int$  		    }
\tabrow{5}{$+E$       & $+\; int$   & Match $+$		    	    }
\tabrow{6}{$E$ 	      & $int$ 	    & Predict $E \rightarrow T$     }
\tabrow{7}{$T$ 	      & $int$ 	    & Predict $T \rightarrow int$   }
\tabrow{8}{$int$      & $int$       & Match $int$   		    }
\tabrow{ }{           &             & Accept			    }
}

$S$, $E$ and $T$ are non-terminals, and $+$ and $int$ are terminals. 

The LR-parsers
also have two actions; shift and reduce. The shift action adds the next input
symbol of the input string into a buffer for consideration. The reduce action
reduces a collection of non-terminals and terminals into a non-terminal by
reversing a production. These two actions are continuously called until the
input string is reduced to the start symbol.
\cite{LL(1)andLR(2)inaction} 
An example of a LR(2)-parser in action is illustrated in \tableref{table:LR2}.

\tab[11cm]{LR2}{3}{A LR(2) parser seen in action parsing the string ``int + int''.}
	  {The process	    					 }
{Step  	 }{Production & Input       & Action                     }{
\tabrow{1}{           & $int + int$ & Shift   			 }
\tabrow{2}{$int$      & $+\; int$   & Reduce $T \rightarrow int$ }
\tabrow{3}{$T$        & $+\; int$   & Shift     		 }
\tabrow{4}{$T+$       & $int$ 	    & Shift			 }
\tabrow{5}{$T+int$    & 	    & Reduce $T \rightarrow int$ }
\tabrow{6}{$T+T$      &             & Reduce $E \rightarrow T$   }
\tabrow{7}{$T+E$      &      	    & Reduce $E \rightarrow T+E$ }
\tabrow{8}{$E$        &             & Reduce $S \rightarrow E$   }
\tabrow{ }{$S$        &             & Accept			 }
}

\subsubsection{Comparison of the parsers}
Compared to the LL-parsers, the LR-parsers are more complex and they are
generally harder to construct,\cite[p. 193]{sebesta2013} thus by the use of
automated generator tools this might not be the case. We take a loot at how to 
construct a parser in \secref{subsec:constructingaparser}. 

The LR-parsers are more powerful than the LL-parsers, because they accept a
bigger variety of grammars. For instance LL-parsers can't handle grammars with
left-recursion, while LR-parsers can. The ``power'' and complexity of a parser
is very dependent on the number of lookahead tokens, $k$, which the parser makes
use of. The bigger $k$ is, the more complex and difficult the parser is to
contruct, but the bigger variety of grammars the parser also accepts. As
illustrated in \figref{fig:LL-parserandLR-parser} the LL-parser is a proper
subset of the LR-parsers.

\fig[width=0.75\textwidth]{LL-parserandLR-parser}{The set of grammars accepted
by different parsers. As illustrated LL(k)-parsers are a subsets of
LR(k)-parsers for different number of lookahead tokens, $k$. The figure is
modified from slides presented in the ``Languages and Compilers''
course from Aalborg University in the spring of 2013.}

\subsection{Constructing a parser}
\label{subsec:constructingaparser}
The art of crafting parsers is very systematic and therefore different automated
tools have been written to generate parsers for grammars that meet some specific
standards. A grammar must for instance not be ambiguous, otherwise tools cannot
make a distinct parser for the grammar. A grammar is ambiguous if more than
one parse tree can be constructed the same string expressed in the language of
that grammar.

In the following section, we analyse the elements of constructing a parser by
hand and by use of a parser-generating tool. In the end we sum up the pros and
cons of each of the methods. 

\subsubsection{Handwritten parsers}
\label{subsec:handwrittenparsers}
Why would you write a parser on your own when you have automated tools for this
job? If we were to construct our own handwritten parser, and not use the tools
already built for this, it would provide us with a greater understanding of how
parsers work. It is commonly known that one of the best ways to learn is by
trial and error. There are some pitfalls in taking on this task. First of all,
bugs are inevitable, when writing something new. These errors must be solved
before the parser can be used. Therefore, the construction of the parser in this
way, will be time-consuming.

Programming languages evolve and their grammars can change. When this is the
case a parser must be maintained so that it still outputs the correct result.
When we have constructed a parser by hand, this task will be time-consuming
because we must search through the code of the parser and update it.

Whenever we work on correcting existing code we are likely going to run into new
errors that need to be resolved. This brings us to the topic of how reliable our
handwritten parsers are. So far we have discussed that the produced code for a
handwritten parser will be error-prone, which will naturally bring us to the
conclusion that this code must be less reliable than the automated generators
produced code. This is a big disadvantage because we have to be able to rely on
the output of the parser, because if it fails, everything else that builds on
top of it will too. To check the validity of a parser, it can be given a set of
inputs that are almost in its language but contain a small error, which must
cause the parser to reject the input. Additionally, a set of inputs can also be
constructed, which is in the language, hence the parser must build a correct
abstract syntax tree.

A big advantage for handwritten parsers is that we are $100 \%$ sure of what is
written in the parser and we can customise the code for our specific needs. If
we step back and look at this from a learning perspective, then this must be the
optimal choice to take. Of course, this does not mean that there is nothing to
be learned from using a tool. By using a tool we will learn to be critical
against automated solutions and make decisions on whether or not a given tool
would be optimal as a solution in the future.

\subsubsection{Generated parsers}
\label{subsec:generatedparsers}
To construct a parser with a generator the developer must input a grammar into
the generator, and it will output a parser for that specific grammar. This can
be a bit different from software to software, but the grammar is often expressed
using the Extended Backus Naur Form (EBNF). A context-free grammar is on EBNF if
it satisfies a certain set of rules and contains some special abilities. We will
not describe these here, but for further information we refer to \cite{fischer2009}.

The generated parser will be produced in the language specified by the
generator. An example of a parser generator is the SableCC parser project, which
is a scanner and bottom-up parser generator, that generates LALR($1$) parsers.
This generator runs on the Java-platform and produces object-oriented code with
clearly separated generated code. This contributes to the simplification and
ease of maintaining the code\cite[p.  11]{sableccdoc}, clearly making it one
of its selling points.

What differences are there between a handwritten and a generated parser?  We
need a grammar in both methods, so what are the advantages of a generated
parser? First of all, it takes less time to construct the parser because once we
have a grammar, we can input it in the generator, and it will automatically
generate the parser (and maybe even a scanner) for us. Before we can use the
software we have to figure out how to use it, but this is not a complicated
process. Most software includes documentation on its features and usage.

The software that generates the parsers have been under development for quite a
while and therefore the developers are using efficient algorithms to implement
the parsers. This means that the parser we construct by hand will most likely
not be as fast and reliable as the ones generated by the software -- unless the
programmer is very experienced with a wide knowledge base about this subject.
Maintenance of the parser is also much easier because everything is
machine-produced and can easily be changed to correspond with a grammar if it
has been changed.

Though a disadvantage with generated parsers is that a lot of code will be
generated without making much sense to the language designer, as we will see in
\secref{sec:parserimplementation}. To sit down and go through it all would be a
daunting task. We could call this produced code for a black box because we
really don't know what is inside it and how it works -- it just does.

\subsection{Summary of constructing scanners and parsers}
\label{subsec:summary-parser}
From the above section about handwritten parser and parser generators we
conclude the following advantages and disadvantages for handwriting a parser and
generating a parser by a parser generator tool:

\begin{dlist}
  \item Handwriting a parser means we gain a better understanding of how the
    parser work
  \item Handwriting a parser gives us the opportunity to customise the code
    specifically for our needs.
  \item Both handwritten parsers and generated parsers are time-consuming to
    construct
  \item Handwritten parsers are more time-consuming to maintain than generated
    parsers
  \item Handwritten parsers are less reliable than generated parsers
  \item Handwritten parsers are slower than generated parsers
  \item A generated parser is like a black box -- we don't really know how it
    works
\end{dlist}

It is clear from the above list that there are many benefits in using a parser
generator tool to construct a parser. It will be much easier to reach our goal
by using such a tool and gaining experience in how they work can be a valuable
skill. But we believe that it is also very important to gain experience in
writing parsers ourselves, causing us to think about how they work from the
beginning to the end. Furthermore, we can customise the code specifically for
our needs. Therefore, we believe it will be beneficial to both write a parser by
hand and generate a parser with a parser generator tool as well to see which one
we want to build on top of.

