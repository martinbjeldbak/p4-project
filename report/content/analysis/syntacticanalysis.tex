\section{Syntactic analysis}
All languages whether natural or artificial is a set of strings of characters over some alphabet. There are rules for how the strings can look that are in a language and how they can be combined. The lexemes described how the strings can look and now the tokens are useful when analysis how the lexemes can be combined. The rules can be specified formally to describe the syntax of a language\cite[p. 135]{sebesta2013}. A common way to describe a language's syntax is by a formal language-generation mechanism (also called grammars or context free grammars). By describing a grammar that can generate all possible strings in a language, the language has also been formally described. Backus-Naur Form is such a mechanism which in the 1950's became the most widely used method for describing programming language syntax\cite[p. 137]{sebesta2013}.

\subsection{Lexical analysis}
The lowest level syntactic units of a language is called lexemes. A language's formal description does not often include these. They are instead described by a lexical specification, regular expressions i.e., separated from the syntactic specification\cite[p. 135]{sebesta2013}. Typical lexemes for a programming language includes integer literals, operators and special keywords like \textit{if} and \textit{while}. If both \textit{\$a} and \textit{\$b} are lexemes describing a variable and \textit{102} and \textit{42} are lexemes describing an integer, then \textit{\$a} and \textit{\$b} or \textit{102} and \textit{42} can typically be used interchangeably and still give a meaningful program. Therefore the lexemes are grouped into tokens. The name of a variable or the value of an integer is preserved when tokenising. The tokens are an abstraction that makes it easier to analyse if correct syntax of the language. An example of the grouping of lexemes into tokens can be seen by \tableref{table:lexandtokens}. After the lexical analysis an input stream of characters has been converted to an output stream of tokens.

\tab[4cm]{lexandtokens}{1}{Lexemes and their corresponding token group.}
		    {               }
{Lexemes   }{\textbf{Tokens}}{
\tabrow{\$a}{var(a) 		}
\tabrow{=  }{assign 		}
\tabrow{3  }{int(3) 		}
\tabrow{\$b}{var(b) 		}
\tabrow{+  }{plus   		}
\tabrow{4  }{int(4) 		}
\tabrow{\$a}{var(a) 		}
}

\subsection{Context-free grammars}
Grammars are defined using Backus-Naur Form (BNF). \todo{Finish\ldots}

BNF contains a set of terminals and a set of non-terminals. The terminals are the tokens from the lexical analysis. The non-terminals all have a set of productions, from which a mix of terminals and non-terminals can be derived from. A start production specifies a single non-terminal, from where all syntactically valid strings that are in the language can be derived from by using the production rules until only a sequence of terminals (the tokens) are left. The syntax analysis takes a sequence of tokens as input and tries to create a set of derivations from the start symbol that creates the given sequence of tokens. If success, the input has been parsed and the parse tree is kept for later analysis. The parse tree is the information concerning how the start symbol was derived into the sequence of tokens, which yields a tree structure. This tree is called an abstract syntax tree.

\begin{ebnf}
%Expressions
\grule{program}{\gter{print} \gcat expr}
\grule{expr}{\gter{(} \gcat term \gcat \gter{)} \gcat operator \gcat \gter{(} \gcat term \gcat \gter{)}}
\grule{operator}{\gter{=}}
\galt{\gter{>}}
\galt{\gter{<}}
\grule{term}{number}
\galt{expr}
\grule{number}{\textbf{any number}}
\end{ebnf}

% This subsection is massive and from the previous structure,
% hence it is inputted
\subsection{LL- and LR-parsers}
\label{subsec:llparsersandlrparsers}
As mentioned in the introduction to the section, the LL-parsers derive from the
top-down parsing approach. In terms of grammars, this means the LL-parsers
attempt to parse a string by starting at the start symbol of the grammar and
through a series of left-most derivations match the input string. On the
opposite, the LR-parsers derive from the bottom-up parsing approach. Here the
LR-parsers attempt to parse by starting with the input string and through a
series of reductions get back to the start symbol.

The LL-parsers have two
actions; predict and match. The predict action is used when the parser is trying
to guess the next production to apply in order to get closer to the input
string. While the match action eats the next unconsumed input symbol if it
corresponds to the left-most predicted terminal. These two actions are
continuously called until the entire input string has been eaten and thereby has
been matched. An example of a LL(1)-parser can be seen in
\tableref{table:LL1}. In the example the parser is based on the simple grammar: 

\begin{centering}
\begin{ebnf}
  \grule{S}{E}
  \grule{E}{T \gcat + \gcat E}
  \galt{T}
  \grule{T}{int}
\end{ebnf}
\end{centering}

\tab[11cm]{LL1}{3}{A LL(1) parser seen in action parsing the string ``int + int''.}
	      {The process                                          }
{Step  	 }{Production & Input       & Action                        }{
\tabrow{1}{$S$        & $int + int$ & Predict $S \rightarrow E$     }
\tabrow{2}{$E$	      & $int + int$ & Predict $E \rightarrow T + E$ }
\tabrow{3}{$T+E$      & $int + int$ & Predict $T \rightarrow int$   }
\tabrow{4}{$int+E$    & $int + int$ & Match $int$  		    }
\tabrow{5}{$+E$       & $+\; int$   & Match $+$		    	    }
\tabrow{6}{$E$ 	      & $int$ 	    & Predict $E \rightarrow T$     }
\tabrow{7}{$T$ 	      & $int$ 	    & Predict $T \rightarrow int$   }
\tabrow{8}{$int$      & $int$       & Match $int$   		    }
\tabrow{ }{           &             & Accept			    }
}

$S$, $E$ and $T$ are non-terminals, and $+$ and $int$ are terminals. 

The LR-parsers
also have two actions; shift and reduce. The shift action adds the next input
symbol of the input string into a buffer for consideration. The reduce action
reduces a collection of non-terminals and terminals into a non-terminal by
reversing a production. These two actions are continuously called until the
input string is reduced to the start symbol.
\cite{LL(1)andLR(2)inaction} 
An example of a LR(2)-parser in action is illustrated in \tableref{table:LR2}.

\tab[11cm]{LR2}{3}{A LR(2) parser seen in action parsing the string ``int + int''.}
	  {The process	    					 }
{Step  	 }{Production & Input       & Action                     }{
\tabrow{1}{           & $int + int$ & Shift   			 }
\tabrow{2}{$int$      & $+\; int$   & Reduce $T \rightarrow int$ }
\tabrow{3}{$T$        & $+\; int$   & Shift     		 }
\tabrow{4}{$T+$       & $int$ 	    & Shift			 }
\tabrow{5}{$T+int$    & 	    & Reduce $T \rightarrow int$ }
\tabrow{6}{$T+T$      &             & Reduce $E \rightarrow T$   }
\tabrow{7}{$T+E$      &      	    & Reduce $E \rightarrow T+E$ }
\tabrow{8}{$E$        &             & Reduce $S \rightarrow E$   }
\tabrow{ }{$S$        &             & Accept			 }
}

\subsubsection{Comparison of the parsers}
Compared to the LL-parsers, the LR-parsers are more complex and they are
generally harder to construct,\cite[p. 193]{sebesta2013} thus by the use of
automated generator tools this might not be the case. We take a loot at how to 
construct a parser in \secref{subsec:constructingaparser}. 

The LR-parsers are more powerful than the LL-parsers, because they accept a
bigger variety of grammars. For instance LL-parsers can't handle grammars with
left-recursion, while LR-parsers can. The ``power'' and complexity of a parser
is very dependent on the number of lookahead tokens, $k$, which the parser makes
use of. The bigger $k$ is, the more complex and difficult the parser is to
contruct, but the bigger variety of grammars the parser also accepts. As
illustrated in \figref{fig:LL-parserandLR-parser} the LL-parser is a proper
subset of the LR-parsers.

\fig[width=0.75\textwidth]{LL-parserandLR-parser}{The set of grammars accepted
by different parsers. As illustrated LL(k)-parsers are a subsets of
LR(k)-parsers for different number of lookahead tokens, $k$. The figure is
modified from slides presented in the ``Languages and Compilers''
course from Aalborg University in the spring of 2013.}


\subsection{Constructing a parser}
\label{sec:ana-parsers}
The principle of generating parsers is very systematic and therefore there are different automated tools to generate parsers for a specific grammar that meets some standards. A grammar must for instance not be ambiguous otherwise the tools cannot make a distinct parser for the grammar. A grammar is ambiguous if a string can be generated with more than one parse tree. We start by taking a look at the constructing of a handwritten parser. Then we take a look a two different parser generators that produce different parsers. Finally we sum up the pros and cons of the analysis on handwritten and generated parsers. 

\subsection{Handwritten parsers}
\label{sec:handparser}
Why would you write a parser on your own when you have automated tools for this job? If we were to construct our own handwritten parser, and not use the tools alreadt built for this, it would be so that we would gain a greater understanding of how these parsers work. How are they constructed? What kind of errors could occur when trying to develop a parser? One of the best ways to learn is to fail - learn from the mistakes and correct them. But this is also a pain in the neck if a lot of errors are popping up and the person trying to get the job done does not have the expertise to resolve the errors and find a solution.

So we would gain experience and probably learn a lot by writing our own parser - but what are the pitfalls of taking on this task? First of all we could be stumbling upon many errors in the code. These errors must be solved before the parser can be finished. Therefore the construction of the parser will be time-consuming. So we will be gaining knowledge about the process but it will take a lot of time compared to an automatic generator.

Programming languages evolve and their grammar can change. When this is the case the parsers must be maintained so that they still output the correct result. When we have generated a parser by hand this task will be time-consuming because we must search through the code of the parser and tweak it so it will be correct again. Whenever we work on tweaking existing code we are most likely going to run into new errors that need to be resolved. 

This brings us to the topic of how reliable our handwritten parsers are compared to the generated parsers. So far we have discussed that the produced code for a handwritten parser will be error-prone so this will naturally bring us to conclude that this code must be less reliable than the automated generators produced code. This is a big con because we have to be able to rely on the output of the parser. To check the validity of a parser, it can be given a set of inputs that are almost in its language but contains a small error, which must cause the parser to reject the inputs. Additionally, a set of inputs can also be constructed, which is \textbf{in} the language, hence the parser must build a correct abstract syntax tree. For each input, the AST outputted by the handwritten parser can be compared to the one outputted by the parse generator tool. If the parse generator has not been constructed yet, one can choose to manually write the expected AST in XML, which most languages / programming frameworks support, using the grammar as reference.

\subsection{Generated parsers}
\label{sec:ana-genparser}
There are quite a few automated parser generaters (compiler compilers). To construct af parser with a generator the developer must input the grammar into the generator, and it will output a parser for that specific grammar. This can be a bit different from software to software but the grammar is often expressed using the Extended Backus Naur Form (EBNF) and the output parser will be produced in the language the generator is meant to output. We take a short look at SableCC and JavaCC in the following two section that both produce Java source code.

What differences are there between a handwritten and a generated parser? We need a grammar in both methods so what is the advantages of a generated parser? First of all, it takes less time to construct the parser because once we have a grammar we can input it in the generator and it will automatically generate the parser for us. Before we can use the software we have to figure out how to use it but this is not a complicated process. For every software there is some kind of tutorial on how to use the software.

The software that generate the parsers have been under development for quite a while and therefore the developers are using efficient algorithms to implement the parsers. This means that the parser we construct by hand will not be as fast and reliable as the ones generated by the software - unless the programmer is very experienced with a wide knowledge base about this subject. So the generated parser is most-likely more efficient.

Maintenance of the parser is also much easier because everything is machine-produced and can easily be changed to correspond witht he new grammar if the grammar has been changed.

There are different methods for constructing parsers. We have top-down parsers, where the parse trees are built from the root (the top) to the bottom, and bottom-up parsers, where the parse trees are built from the bottom to the root. Different grammars have different limitations and  the different types of parsers work on specific grammars. We will shortly discuss this in the following sections.

\subsubsection{SableCC, a bottom-up parser}
\label{sec:ana-sablecc}
SableCC is a bottom-up parser generator that generates LALR(1) parsers. This generator runs on the Java-platform and produces object-oriented code with clearly seperated machine-generated code and handwritten code. This contributes to the simplification and ease of maintaining the code.\cite[pp. 11]{sableccdoc}

The following is a list of advantages that LR parsers have:\cite[pp. 193]{sebesta2013} \todo{kontroller sidetal! afsnit 4.5.3}


The only disadvantage a LR parser has is that it is verey difficult to produce by hand. We have the automated generators to solve this disadvantage.

\subsubsection{JavaCC, a top-down parser}
\label{sec:ana-javacc}
JavaCC is a top-down parser generator that generetes LL(k) parsers. As the name of the parser generator it also produces the output code as Java source code.\cite{wiki-javacc}

\subsection{Summary}
\label{sec:ana-parsersum}
By reading the above section about handwritten parser we can conclude the following advantages by handwriting a parser:

\begin{dlist}
\item Gain experience in constructing parsers
\item Gain a better understanding of how parsers work
\end{dlist}

We will be gaining a lot of experience by writing a parser by hand and solving the problems that arise along the way. But there are quite a few disadvantages accompanied with handwriting a parser, and they are as follows.

\begin{dlist}
\item Can be error-prone
\item Can be time-consuming to construct
\item Time-consuming to maintain
\item Less reliable than generated parsers
\item Slower than generated parsers
\end{dlist}

We've summed up the advantages and disadvantages of handwritten parsers. Know we take a short look at the advantages of generated parsers. By reading the above section about the automatically generated parsers we can conclude the following advantages:

\begin{dlist}
\item Efficient
\item Reliable
\item Fast
\item Easy to maintain
\end{dlist}

These parsers will most-likely be more efficient and faster than handwritten parsers because they include efficient algorithms developed and maintained throughout the lifetime of the software. This also makes them more reliable than handwritten parsers because there will be much less errors in the process of constructing the parser.

This brings us towards a conclusion on handwritten and generated parsers. It is very clear that there are more benefits in using an automated generator to construct a parser. It is quite obvious that it will be much easier to reach our goal of constructing a parser. But we believe that it is very important that we try to gain experience and therefore we will be both handwriting a parser and using a generator as well.
