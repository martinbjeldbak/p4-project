\subsection{LL- and LR-parsers}
\label{subsec:llparsersandlrparsers}
As mentioned in the introduction to the section, the LL-parsers derive from the
top-down parsing approach. In terms of grammars, this means the LL-parsers
attempt to parse a string by starting at the start symbol of the grammar and
through a series of left-most derivations match the input string. On the
opposite, the LR-parsers derive from the bottom-up parsing approach. Here the
LR-parsers attempt to parse by starting with the input string and through a
series of reductions get back to the start symbol.

The LL-parsers have two
actions; predict and match. The predict action is used when the parser is trying
to guess the next production to apply in order to get closer to the input
string. While the match action eats the next unconsumed input symbol if it
corresponds to the left-most predicted terminal. These two actions are
continuously called until the entire input string has been eaten and thereby has
been matched. An example of a LL(1)-parser can be seen in
\tableref{table:LL1}. In the example the parser is based on the simple grammar: 

\begin{centering}
\begin{ebnf}
  \grule{S}{E}
  \grule{E}{T \gcat + \gcat E}
  \galt{T}
  \grule{T}{int}
\end{ebnf}
\end{centering}

\tab[11cm]{LL1}{3}{A LL(1) parser seen in action parsing the string ``int + int''.}
	      {The process                                          }
{Step  	 }{Production & Input       & Action                        }{
\tabrow{1}{$S$        & $int + int$ & Predict $S \rightarrow E$     }
\tabrow{2}{$E$	      & $int + int$ & Predict $E \rightarrow T + E$ }
\tabrow{3}{$T+E$      & $int + int$ & Predict $T \rightarrow int$   }
\tabrow{4}{$int+E$    & $int + int$ & Match $int$  		    }
\tabrow{5}{$+E$       & $+\; int$   & Match $+$		    	    }
\tabrow{6}{$E$ 	      & $int$ 	    & Predict $E \rightarrow T$     }
\tabrow{7}{$T$ 	      & $int$ 	    & Predict $T \rightarrow int$   }
\tabrow{8}{$int$      & $int$       & Match $int$   		    }
\tabrow{ }{           &             & Accept			    }
}

$S$, $E$ and $T$ are non-terminals, and $+$ and $int$ are terminals. 

The LR-parsers
also have two actions; shift and reduce. The shift action adds the next input
symbol of the input string into a buffer for consideration. The reduce action
reduces a collection of non-terminals and terminals into a non-terminal by
reversing a production. These two actions are continuously called until the
input string is reduced to the start symbol.
\cite{LL(1)andLR(2)inaction} 
An example of a LR(2)-parser in action is illustrated in \tableref{table:LR2}.

\tab[11cm]{LR2}{3}{A LR(2) parser seen in action parsing the string ``int + int''.}
	  {The process	    					 }
{Step  	 }{Production & Input       & Action                     }{
\tabrow{1}{           & $int + int$ & Shift   			 }
\tabrow{2}{$int$      & $+\; int$   & Reduce $T \rightarrow int$ }
\tabrow{3}{$T$        & $+\; int$   & Shift     		 }
\tabrow{4}{$T+$       & $int$ 	    & Shift			 }
\tabrow{5}{$T+int$    & 	    & Reduce $T \rightarrow int$ }
\tabrow{6}{$T+T$      &             & Reduce $E \rightarrow T$   }
\tabrow{7}{$T+E$      &      	    & Reduce $E \rightarrow T+E$ }
\tabrow{8}{$E$        &             & Reduce $S \rightarrow E$   }
\tabrow{ }{$S$        &             & Accept			 }
}

\subsubsection{Comparison of the parsers}
Compared to the LL-parsers, the LR-parsers are more complex and they are
generally harder to construct,\cite[pp. 193]{sebesta2013} thus by the use of
automated generator tools this might not be the case. We take a loot at how to 
construct a parser in \secref{subsec:constructingaparser}. 

The LR-parsers are more powerful than the LL-parsers, because they accept a
bigger variety of grammars. For instance LL-parsers can't handle grammars with
left-recursion, while LR-parsers can. The ``power'' and complexity of a parser
is very dependent on the number of lookahead tokens, $k$, which the parser makes
use of. The bigger $k$ is, the more complex and difficult the parser is to
contruct, but the bigger variety of grammars the parser also accepts. As
illustrated in \figref{fig:LL-parserandLR-parser} the LL-parser is a proper
subset of the LR-parsers.

\fig[width=0.75\textwidth]{LL-parserandLR-parser}{The set of grammars accepted
by different parsers. As illustrated LL(k)-parsers are a subsets of
LR(k)-parsers for different number of lookahead tokens, $k$. The figure is
modified from slides presented in the ``Languages and Compilers''
course from Aalborg University in the fall of 2013.}
