\subsection{Context-free grammars}
\label{sec:context-freegrammars}
In the previous section we described how the scanner transforms an input stream of
characters into an output stream of tokens. In the following section we
will describe context-free grammars, a component which contains the programming
languages grammar, that is the set of rules for how its tokens can be
combined.

A context-free grammar contains a set of terminals, a set of non-terminals, a
set of productions and a start production. The terminals are the tokens from the
lexical analysis. The non-terminals all have a set of productions, from which a
mix of terminals and non-terminals can be derived from. A production typically
has the following form:

\begin{ebnf}
\grule{Nonterminal} {\{Terminal \gor \gcat Nonterminal\}}
\end{ebnf}

A start production specifies a single non-terminal, from where all syntactically
valid strings that are in the language can be derived from by using the
production rules until only a sequence of terminals (the tokens) are left. 
The syntax analysis takes a sequence of tokens as input and tries to create a
set of derivations from the start symbol that creates the given sequence of
tokens. If success, the input has been parsed and the parse tree is kept for
later analysis. The parse tree is the information concerning how the start
symbol was derived into the sequence of tokens, which yields a tree structure.
This tree is called an abstract syntax tree.

