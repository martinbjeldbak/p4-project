\subsection{Context-free grammars}
\label{sec:context-freegrammars}
In the previous section we described how to transfrom an input stream of
characters into an output stream of tokens. We will now describe
Grammars are defined using Backus-Naur Form (BNF).

BNF contains a set of terminals and a set of non-terminals. The terminals are
the tokens from the lexical analysis. The non-terminals all have a set of
productions, from which a mix of terminals and non-terminals can be derived
from. A start production specifies a single non-terminal, from where all
syntactically valid strings that are in the language can be derived from by
using the production rules until only a sequence of terminals (the tokens) are
left. The syntax analysis takes a sequence of tokens as input and tries to
create a set of derivations from the start symbol that creates the given
sequence of tokens. If success, the input has been parsed and the parse tree is
kept for later analysis. The parse tree is the information concerning how the
start symbol was derived into the sequence of tokens, which yields a tree
structure. This tree is called an abstract syntax tree.

\begin{ebnf}
%Expressions
\grule{program}{\gter{print} \gcat expr}
\grule{expr}{\gter{(} \gcat term \gcat \gter{)} \gcat operator \gcat \gter{(}
\gcat term \gcat \gter{)}}
\grule{operator}{\gter{=}}
\galt{\gter{>}}
\galt{\gter{<}}
\grule{term}{number}
\galt{expr}
\grule{number}{\textbf{any number}}
\end{ebnf}
