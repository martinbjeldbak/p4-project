\subsection{Constructing a parser}
\label{subsec:constructingaparser}
As we described in \secref{sec:lexicalanalysis} a scanner can be generetad by a
scanner generator tool. The same can be done for parsers. Since the principle of 
crafting parsers is very systematic there exists
different automated tools to generate parsers with, for grammars that meets
some specific standards. A grammar must for instance not be ambiguous otherwise the tools
cannot make a distinct parser for the grammar. A grammar is ambiguous if a
string can be expressed by more than one different parse tree. In the following section we
anlyse the subjects of contructing a parser by hand and by use of a parser generator
tool. In the end we sum up the pros and cons of each of the methods. 

\subsubsection{Handwritten parsers}
\label{subsec:handwrittenparsers}
Why would you write a parser on your own when you have automated tools for this
job? If we were to construct our own handwritten parser, and not use the tools
already built for this, it would provide us with a greater understanding of how a parser 
work. It is commonly known that one of the best ways to learn is from making mistakes and 
later correcting them. But there are the pitfalls in taking on this task. First of all we 
could be stumbling upon many errors in the code. These errors must be solved before the
parser can be functional. Therefore the construction of the parser in this way, will be
time-consuming.

Programming languages evolve and their grammars can change. When this is the case a 
parser must be maintained so that it still outputs the correct result.
When we have constructed a parser by hand this task will be time-consuming because
we must search through the code of the parser and correct it. 

Whenever we work on correcting existing code we are likely going to run into new errors 
that need to be resolved. This brings us to the topic of how reliable our handwritten 
parsers are. So far we have discussed that the produced code for a handwritten parser 
will be error-prone so this will naturally bring us to conclude that this code must be 
less reliable than the automated generators produced code. This is a big disadvantage because we have to be able to rely on the
output of the parser. To check the validity of a parser, it can be given a set
of inputs that are almost in its language but contains a small error, which must
cause the parser to reject the inputs. Additionally, a set of inputs can also be
constructed, which is in the language, hence the parser must build a
correct abstract syntax tree.

\subsubsection{Generated parsers}
\label{subsec:generatedparsers}
There exists quite a few parser generators. To construct af parser with a generator 
the developer must input a grammar into the generator, and it will output a parser 
for that specific grammar. This can be a bit different from software to software but the grammar is often expressed
using the Extended Backus Naur Form (EBNF). A context-free grammar is on EBNF if it
satifies a certain set of rules and contains some special abilities. We will not describe
these here, but for further information we refer to \cite[152]{fischer2009}. The outputted parser will be
produced in the language specified by the generator. We take a look at two different parser
generator tools, namely SableCC and JavaCC.

SableCC is a bottom-up parser generator that generates LALR(1) parsers. This
generator runs on the Java-platform and produces object-oriented code with
clearly seperated machine-generated code and handwritten code. This contributes
to the simplification and ease of maintaining the code.\cite[pp. 11]{sableccdoc}

JavaCC is a top-down parser generator that generates LL(k) parsers. As the name
of the parser generator suggests, it produces the output code in Java source
code.\cite{wiki-javacc}

What differences are there between a handwritten and a generated parser? We need
a grammar in both methods so what is the advantages of a generated parser? First
of all, it takes less time to construct the parser because once we have a
grammar we can input it in the generator and it will automatically generate the
parser for us. Before we can use the software we have to figure out how to use
it but this is not a complicated process. For every software there is some kind
of tutorial on how to use the software.

The software that generate the parsers have been under development for quite a
while and therefore the developers are using efficient algorithms to implement
the parsers. This means that the parser we construct by hand will not be as fast
and reliable as the ones generated by the software - unless the programmer is
very experienced with a wide knowledge base about this subject. Maintenance of the 
parser is also much easier because everything is machine-produced and can easily be 
changed to correspond with a grammar if it has been changed.

\subsection{Summary}
\label{subsec:summary-parser}
From the above section about handwritten parser and parser generators we conclude the
following advantages and disadvantages for handwriting a parser and generating a parser by a parser
generator tool:

\begin{dlist}
\item Handwriting a parser means we gain a better understanding of how the parser work
\item Both handwritten parsers and generated parsers are time-consuming to construct
\item Handwritten parsers are more time-consuming to maintain than generated parsers
\item Handwritten parsers are less reliable than generated parsers
\item Handwritten parsers are slower than generated parsers
\end{dlist}

It is clear from the above list that there are many benefits in using a parser generator tool to
construct a parser. It will be much easier to reach our goal by using such as tool and gaining 
experience in how they work can be a valuable skill. But we believe that it is also very important 
to gain experience in writting parsers ourself and knowledge on how they work from the beginning to the end. 
Therefore we think it would be beneficial to both write a parser by hand and generate a parser with a parser 
generator tool as well.
