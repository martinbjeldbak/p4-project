\subsection{Lexical analysis}
\label{sec:lexicalanalysis}
Before the syntax analysis can be performed a scanner must perform a lexical
analysis. The scanner's job is basically to check a given source code for lexical
errors and translating the input stream of characters from the source code into
a stream of tokens which the parser can work with, this
is done by identifying every lexeme in the source and attaching a potential
token to it.
\cite[p. 57]{fischer2009}

Lexemes are strings of characters described by regular expressions.  Typical
examples of lexemes in a programming language are: variable names, integer
literals, operators and special keywords etc. A variable name lexeme could be
defined by the following regular expression: $[a-z, A-Z, "\_"][a-z, A-Z, 0-9,
"\_"]^*$. Which means a variable name can start with either an uppercase letter,
lowercase letter or an underscore followed by zero or more lowercase letters,
uppercase letters, numbers or underscores. In \tableref{table:lexandtokens} we
give examples of lexemes and the tokens they have been paired with. If both
\textit{a} and \textit{b} are lexemes describing variable names and \textit{102}
and \textit{42} are lexemes describing integer literals, then \textit{a} and
\textit{b} or \textit{102} and \textit{42} can typically be used interchangeably
and still give a syntatic meaningful program.

\tab[10cm]{lexandtokens}{1}{Lexemes and their corresponding token group.}
		             {               }
       {Lexemes             }{\textbf{Tokens} } {
\tabrow{ x                  }{VAR\_NAME       } 
\tabrow{ random\_var\_name  }{VAR\_NAME       }
\tabrow{ RANdom\_var\_name2 }{VAR\_NAME       }
\tabrow{ 1 		    }{INT\_LITERAL    }
\tabrow{ 342 		    }{INT\_LITERAL    }
\tabrow{ 52890 		    }{INT\_LITERAL    }
\tabrow{ +		    }{PLUS\_OPERATOR  }
\tabrow{ - 		    }{MINUS\_OPERATOR }
\tabrow{ * 		    }{MULT\_OPERATOR  }
\tabrow{ if		    }{KEYWORD         }
\tabrow{ while 		    }{KEYWORD  	      }
\tabrow{ switch 	    }{KEYWORD  	      }
}

A scanner is a relatively simple component which can be constructed by writing
it by hand or using a scanner-generating tool such as Lex, which generates an
executable scanner by feeding it with a set of regular expressions. When
implementing the scanner for \productname{}, we would likely benefit more from
crafting the scanner by hand than by using a scanner-generating tool. By
crafting it by hand we will know excatly how it is implemented. There might be
some advantages of using a scanner-generating tool such as the fact that it is
more reliable, it is easier to maintain and it is faster to implement if one
already knows how it works, if not, a handwritten scanner might be just as fast. 
