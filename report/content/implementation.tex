\chapter{Implementation}
\label{chap:implementation}

In the following chapter we present how the different modules of \productname{} are implemented. In \secref{sec:scannerimplementation} the scanner implementation is specified and examples are given of how the tokens are identified. In \secref{sec:parserimplementation} we present the parser of
\productname{}. We explain how it's constructed and how it works. Furthermore we present a parser and scanner generated by SableCC and the reason why we did not proceed with this approach. In \secref{sec:ant} our abstract node types are explained and illustrated. In \secref{sec:staticsemantics} we present the scope checker of \productname{}. Here the visitor pattern and the use of it is also introduced. In \secref{sec:interpreter} the interpreter is presented. Here we explain how patterns are implemented which is a feature used in most programs written in \productname{} and we introduce our implementation of operators and values. In \secref{sec:errorhandling} we present how error handling is implemented. In \secref{sec:gameabstractionlayer} we explain the game abstraction layer which is the layer that binds the interpreter and the simulator together and makes them able to communicate with each other. Finally in \secref{sec:simulator-impl} the game simulator of \productname{} is presented.    


\section{Scanner}
The scanner is the first part of the interpreter that analyses the input. The scanners takes a raw source code for a program written in \productname{} as input. The main purpose of the scanner is to validate the lexically correctness of a \productname{} program. The scanner must analyse the input and find tokens existing in \productname{}. If an input is met which can not be recognized as a valid token, the source code is not a valid program in \productname{}. The strings that are converted to tokens are called lexemes. Many lexemes can be converted to the same type of token. Some programs contains many different identifiers, which in \productname{} all will have a token instantiated with the type \textit{IDENTIFIER}. Consider the example of a single line from a chess game written in \productname{}. The raw input is:
\begin{lstlisting}
Black{ Pawn [A7 B7 C7 D7 E7 F7 G7 H7] }
\end{lstlisting}
The result of analysing the input can be seen in \tableref{table:lexemestotokens}. Tokens are needed for abstraction. When later on the parser will determine if the code respects the grammar of \productname{}, it is useful to have these abstractions. It makes it possible to describe that a list of \textit{COORD\_LIT}'s can be encapsulated between the characters ``[ ]'' without having to list all possible coordinate literals, which in fact are an infinite set, since the grammar of \productname{} which is described later, allows proceeding in both dimensions after \textit{Z9}, namely \textit{Z10} and \textit{AA9}. However, when converting lexemes to tokens, the value of a coordinate, the name of an identifier, e.g. is still kept since that information will be needed later for the subsequent parts of the interpreter.

\begin{figure}
\begin{tabular}{|l|l|}
        \hline
        Lexemes & Tokens             \\ \hline
        Black   & IDENTIFIER (Black) \\ 
        \{       & LBRACE             \\ 
        Pawn    & IDENTIFIER (Pawn)  \\ 
        $[$       & LBRACKET           \\ 
        A7      & COORD\_LIT (A7)     \\ 
        B7      & COORD\_LIT (B7)     \\ 
        C7      & COORD\_LIT (C7)     \\ 
        D7      & COORD\_LIT (D7)     \\ 
        E7      & COORD\_LIT (E7)     \\ 
        F7      & COORD\_LIT (F7)     \\ 
        G7      & COORD\_LIT (G7)     \\ 
        H7      & COORD\_LIT (H7)     \\ 
        $]$       & RBRACKET           \\ 
        \}       & RBRACE             \\
        \hline
\end{tabular}
\capt{Analysing an input stream for lexemes and tokens}\label{table:lexemestotokens}
\end{figure}

A scanner can be hard coded by hand or one can choose to generate one using existing tools, e.g. JLex (for Java). When using tools, you must define tokens to match input on regular expressions but input can also be matched on string equality for instance when describing keywords. If a programming language is complex, writing a scanner by hand can be very time consuming error prone. However, we think the syntax of \productname{} as quite simple and have therefore decided that the time we would spend learning how to use tools like JLex could in the meantime have given us a hand coded scanner instead.

The scanner contains 2 classes, \classref{Scanner} and \classref{Token}. \classref{Token} contains an enum named \classref{Type} that enumerates all the types of tokens in \productname{}. When a lexeme is found in the input stream, the scanner analyses which token type it belongs to. A new token is then instantiated and yielded by the scanner. The constructor for \classref{Token} takes the arguments (\classref{Token.Type}, \textit{line}, \textit{offset}). The \textit{line} and \textit{offset} represent where in the source code the lexeme of any token where found, which can be used to inform a programmer where in his source code an error was found, if any was found.
 

\subsection{LL- and LR-parsers}
\label{subsec:llparsersandlrparsers}
As mentioned in the introduction to the section, the LL-parsers derive from the
top-down parsing approach. In terms of grammars, this means the LL-parsers
attempt to parse a string by starting at the start symbol of the grammar and
through a series of left-most derivations match the input string. On the
opposite, the LR-parsers derive from the bottom-up parsing approach. Here the
LR-parsers attempt to parse by starting with the input string and through a
series of reductions get back to the start symbol.

The LL-parsers have two
actions; predict and match. The predict action is used when the parser is trying
to guess the next production to apply in order to get closer to the input
string. While the match action eats the next unconsumed input symbol if it
corresponds to the left-most predicted terminal. These two actions are
continuously called until the entire input string has been eaten and thereby has
been matched. An example of a LL(1)-parser can be seen in
\tableref{table:LL1}. In the example the parser is based on the simple grammar: 

\begin{centering}
\begin{ebnf}
  \grule{S}{E}
  \grule{E}{T \gcat + \gcat E}
  \galt{T}
  \grule{T}{int}
\end{ebnf}
\end{centering}

\tab[11cm]{LL1}{3}{A LL(1) parser seen in action parsing the string ``int + int''.}
	      {The process                                          }
{Step  	 }{Production & Input       & Action                        }{
\tabrow{1}{$S$        & $int + int$ & Predict $S \rightarrow E$     }
\tabrow{2}{$E$	      & $int + int$ & Predict $E \rightarrow T + E$ }
\tabrow{3}{$T+E$      & $int + int$ & Predict $T \rightarrow int$   }
\tabrow{4}{$int+E$    & $int + int$ & Match $int$  		    }
\tabrow{5}{$+E$       & $+\; int$   & Match $+$		    	    }
\tabrow{6}{$E$ 	      & $int$ 	    & Predict $E \rightarrow T$     }
\tabrow{7}{$T$ 	      & $int$ 	    & Predict $T \rightarrow int$   }
\tabrow{8}{$int$      & $int$       & Match $int$   		    }
\tabrow{ }{           &             & Accept			    }
}

$S$, $E$ and $T$ are non-terminals, and $+$ and $int$ are terminals. 

The LR-parsers
also have two actions; shift and reduce. The shift action adds the next input
symbol of the input string into a buffer for consideration. The reduce action
reduces a collection of non-terminals and terminals into a non-terminal by
reversing a production. These two actions are continuously called until the
input string is reduced to the start symbol.
\cite{LL(1)andLR(2)inaction} 
An example of a LR(2)-parser in action is illustrated in \tableref{table:LR2}.

\tab[11cm]{LR2}{3}{A LR(2) parser seen in action parsing the string ``int + int''.}
	  {The process	    					 }
{Step  	 }{Production & Input       & Action                     }{
\tabrow{1}{           & $int + int$ & Shift   			 }
\tabrow{2}{$int$      & $+\; int$   & Reduce $T \rightarrow int$ }
\tabrow{3}{$T$        & $+\; int$   & Shift     		 }
\tabrow{4}{$T+$       & $int$ 	    & Shift			 }
\tabrow{5}{$T+int$    & 	    & Reduce $T \rightarrow int$ }
\tabrow{6}{$T+T$      &             & Reduce $E \rightarrow T$   }
\tabrow{7}{$T+E$      &      	    & Reduce $E \rightarrow T+E$ }
\tabrow{8}{$E$        &             & Reduce $S \rightarrow E$   }
\tabrow{ }{$S$        &             & Accept			 }
}

\subsubsection{Comparison of the parsers}
Compared to the LL-parsers, the LR-parsers are more complex and they are
generally harder to construct,\cite[p. 193]{sebesta2013} thus by the use of
automated generator tools this might not be the case. We take a loot at how to 
construct a parser in \secref{subsec:constructingaparser}. 

The LR-parsers are more powerful than the LL-parsers, because they accept a
bigger variety of grammars. For instance LL-parsers can't handle grammars with
left-recursion, while LR-parsers can. The ``power'' and complexity of a parser
is very dependent on the number of lookahead tokens, $k$, which the parser makes
use of. The bigger $k$ is, the more complex and difficult the parser is to
contruct, but the bigger variety of grammars the parser also accepts. As
illustrated in \figref{fig:LL-parserandLR-parser} the LL-parser is a proper
subset of the LR-parsers.

\fig[width=0.75\textwidth]{LL-parserandLR-parser}{The set of grammars accepted
by different parsers. As illustrated LL(k)-parsers are a subsets of
LR(k)-parsers for different number of lookahead tokens, $k$. The figure is
modified from slides presented in the ``Languages and Compilers''
course from Aalborg University in the spring of 2013.}

\section{Abstract node types}
\label{sec:ant}

This section illustrates how we have chosen to implement the grammar described in
\secref{sec:grammar} into abstract syntax trees (ASTs). 
We have split every program part into their own subsections with an abstract
node type (ANT) that presents the program parts construction when it is parsed into an
AST. Each node type corresponds to a production in the grammar. These ANTs can
then be combined to form the AST when some piece of code has been parsed.

Firstly, the section begins with a description of how an AST differs from a parse tree 
to make this clear to the reader.

When a piece of code is parsed by a parser that understands the specific programming 
language, the output of the parser will be an AST which consists of the abstract node 
types of each program part. The difference between an AST and a parse tree is that a 
parse tree contains every detail of the input code. A parse tree includes for instance
the parentheses and keywords where an AST does not contain anything other than the 
abstract node types.\cite{parsevsast}

Now, the section is divided into three subsections. First we present the actual
parts of a program structure. Then we present which expressions the language
consists of. Finally, we present the different patterns.

\subsection{Program structure}
The following subsections present the structure of programs. The sections are 
structured as follows:

\begin{dlist}
  \item Program
  \item Definition
  \item Constant definition
  \item Type definition
  \item Type body and member definition
  \item Abstract definition
  \item Variable list
\end{dlist}

\subsubsection{Program}
Every program written in \productname{} begins with an abstract node type which 
we call ``program'' that consists of either zero, one, or more definitions. The 
production for this rule is as follows:

\begin{ebnf}
\grule{program}{\grep{definition}}
\end{ebnf}

It is from this production each and every program is derived from. The ANT for 
this production is illustrated in \figref{ast:program}.


\begin{figure}[ht]
\begin{center}
\begin{tikzpicture}[level/.style={sibling distance=40mm/#1}]
\node [square] {Program}
  child {node [square,xshift=0.5cm] (a) {\textit{Definition}} edge from parent[dashed];}
  child {node [square,xshift=-0.5cm] (b) {\textit{Definition}} edge from parent[dashed];};
\path (a)--(b) node [midway] {$\cdots$};
\end{tikzpicture}
\end{center}
\capt{The abstract syntax tree for the program node.}
\label{ast:program}
\end{figure}


Figure \ref{ast:program} consists of one root which is called ``Program'' and
this root can have zero, one, or more children, called ``Definition''. The
children are optional because the production says that a program can consist of
either zero, one, or more of these definitions. We illustrate this choice by
making the connecting edges dashed from the parent node. Between these two child
nodes there are three dots ($\cdots$) which illustrate that it is possible to
have more of these nodes following each other.

This means that a program is legal if it does not contain anything at all.

\subsection{Expressions}
The following nine subsections present the different expressions in
\productname{}. The sections are structured as follows:

\begin{dlist}
  \item Operations with precedence and negation
  \item Element
  \item Member access
  \item Call sequence
  \item Assignment
  \item If expression
  \item Lambda expression
  \item List
  \item No-operator
\end{dlist}

\subsubsection{Operations with precedence and negation}
In this section we introduce the five different groupings of operations such as
logical operators, equality operators, etc., and finally we present the
production called negation. We illustrate the five groupings of operators but we
have omitted the negation node, because it is merely a root with one child. The
following grammar presents the productions for the different operations:

\begin{ebnf}
\grule{lo\_sequence}{eq\_sequence \gcat \grep{\ggrp{\gter{and} \gor \gter{or}} \gcat eq\_sequence}}
\grule{eq\_sequence}{cm\_sequence \gcat \grep{\ggrp{\gter{==} \gor \gter{!=}} \gcat cm\_sequence}}
\grule{cm\_sequence}{as\_sequence \gcat \grep{\ggrp{\gter{<} \gor \gter{>} \gor \gter{<=} \gor \gter{>=}} \gcat as\_sequence}}
\grule{as\_sequence}{md\_sequence \gcat \grep{\ggrp{\gter{+} \gor \gter{-}} \gcat md\_sequence}}
\grule{md\_sequence}{negation \gcat \grep{\ggrp{\gter{*} \gor \gter{/} \gor{\%}} \gcat negation}}
\end{ebnf}

The meaning of these different groupings of operators have been described in
\secref{sec:grammar}. To refresh the readers memory, the sequences are
intentionally placed in this specific order to ensure the correct precedence for
these operators.

Since the productions look a lot like each other we will only illustrate an
abstract ANT which shows how we have implemented the different operations.
Figure \ref{ast:operation} shows this.


\begin{figure}[ht]
\begin{center}
\begin{tikzpicture}[level/.style={sibling distance=30mm/#1}]
\node [square] {\textit{LHS}}
  child {node [square] {\textit{RHS}}}
  child {node [square] {\textit{RHS}}}
  child {node [square] (a) {\textit{RHS}} edge from parent[dashed];}
  child {node [square] (b) {\textit{RHS}} edge from parent[dashed];};
  
\path (a)--(b) node [midway] {$\cdots$};
\end{tikzpicture}
\end{center}
\end{figure}


Figure \ref{ast:operation} illustrates that the ANT for each production, with
the left-hand side (LHS) of the production as the root, will only be constructed
if there is at least two of the right-hand sides (RHS) with one operator between
them. This means that the sequence should for instance look like the following:

\begin{ebnf}
\grule{lo\_sequence}{eq\_sequence \gcat \gter{and} \gcat eq\_sequence}
\end{ebnf}

So, in the above example the RHS is $eq\_sequence$ and the operator is
$\gter{and}$. In \figref{ast:operation} we do not show the operators, which can
make it a bit cryptic to look at and understand. If there is only one RHS then
the node is not constructed but the next production will be evaluated. Otherwise
the AST would end up with many single-child nodes. This bad implementation is 
illustrated in \figref{ast:badexample}.


\begin{figure}[ht]
\begin{center}
\begin{tikzpicture}[level 3/.style={sibling distance = 20mm}]

\node [square] {LHS}
  child {node [square] {RHS}
    child {node [square] {RHS}
      child {node [square] {RHS}}
      child {node [square] (a) {RHS} edge from parent[dashed]}
      child {node [square] (b) {RHS} edge from parent[dashed]}
    }
  }; 

\path (a)--(b) node [midway] {$\cdots$};
\end{tikzpicture}
\end{center}
\end{figure}


Figure \ref{ast:badexample} illustrates very clearly that the AST quickly would
end up with mant single child nodes. If we do not expect two RHS then We will 
end up with a long list, which is not necessary and it will just make it less 
efficient to read the AST. With our implementation we will have a more efficient 
and more compact AST.

The grammar specified earlier in this section presented that the last operation
($md\_sequence$) consists of negations and the choice to add an operator between
negations. The negation production can be an element or begin with the \gter{-}
symbol followed by another negation.
The following productions present the grammar for the negation expression:

\begin{ebnf}
\grule{negation}{element}
\galt{\gter{-} \gcat negation}
\end{ebnf}

This production is not illustrated with any ANT.

\subsubsection{Assignment}
The following grammar rule specifies the production of an assignment in
\productname{}:

\begin{ebnf}
\grule{assignment}{\gter{let} \gcat variable \gcat \gter{=} \gcat expression \gcat \grep{\gter{,} \gcat variable \gcat \gter{=} \gcat expression} \gcat \gter{in} \gcat expression}
\end{ebnf}

The production specifies that any assignment must begin with the keyword
\gter{let} and end with the keyword \gter{in} followed by an expression. In
between these beginning and ending keywords, the production consists of at least
one sequence of a variable followed by an assignment-operator followed by an
expression. The production specifies that it is possible to have zero, one, or
more of these variable-expression pairs (comma separated) following the first
pair.


\begin{figure}
\begin{center}
\begin{tikzpicture}
[level/.style={sibling distance=40mm},
level 1/.style={sibling distance = 39mm},
level 2/.style={sibling distance = 20mm}]

\node [square] (z) {Assignment}
  child {node [square,left of=b,xshift=-5.1cm] (a) {Variable}}
  child {node [ellipse,draw,left of=c,xshift=-4.5cm] (b) {\textit{Expression}}}
  child {node [square] (c) {Assignment} edge from parent[dashed]
  	child {node [square,xshift=-1cm] (x) {Variable} edge from parent[solid]}
  	child {node [ellipse,draw,solid,xshift=-1cm] (y) {\textit{Expression}} edge from parent[solid]}
  }
  child {node [square,xshift=-1cm] (d) {Assignment} edge from parent[dashed]
  	child {node [square,xshift=1cm] (o) {Variable} edge from parent[solid]}
  	child {node [ellipse,draw,solid,xshift=1cm] (p) {\textit{Expression}} edge from parent[solid]}
  }
  child {node [ellipse,draw,right of=d,xshift=1.5cm](e) {\textit{Expression}}};

\path (c)--(d) node [midway] {$\cdots$};
\end{tikzpicture}
\end{center}
\capt{The abstract syntax tree for the assignment node.}
\label{ast:assignment}
\end{figure}


Figure \ref{ast:assignment} illustrates this with an ANT that omits the
keywords, commas and the assignment-operators, which can make it rather complex
to look at. The figure actually states that an assignment consists of variables
and expression where an expression can be many different things, including
another assignment. So, this means that it is possible to have assignments
nested within each other. 

We have chosen to implement the comma separated nodes as new assignment nodes
which has two nodes that are not optioanl. These nested assignments are
connected to the parent with dashed edges which mean that they are optional and
it is possible to have zero, one, or more of these following each other.

\section{Contextual constraints}
\label{sec:staticsemantics}

The \classref{ScopeChecker} is the class responsible for enforcing some of the
static semantic rules of \productname{} (described in \chapref{chap:design}) at
compile time. This section aims to explain how these static semantic checks
are performed by the \classref{ScopeChecker} in simple sequential steps. Any
error detected by the \classref{ScopeChecker} will cause a \classref{ScopeError}
exception to be thrown. This error contains helpful information about the type
of error and where in the input program the error is located. The checking of
static semantics as well as the interpretation of the code both use the visitor
pattern which is a commonly used approach for both purposes.

After the AST has been created, the visitor pattern allows us to traverse the
AST and execute encapsulated pieces of code for each specific type of
\classref{AstNode}. 

\subsection{TypeVisitor}
The first visitor used is the \classref{TypeVisitor}. This visitor traverses the
AST, finds all type definitions in the input program and for each type
definition an object of class \classref{TypeSymbolInfo} is instantiated. After
running the \classref{TypeVisitor}, each type definition in the input program
has an associated object of class \classref{TypeSymbolInfo} which makes it easy
to get information about any declared type in the input program by accessing the
following members contained in the \classref{TypeSymbolInfo} object:

\begin{dlist}
  \item name (\classref{String}): The type's name
  \item parentName (\classref{String}): The name of its super type (null if
    not a derived type)  
  \item args (\classref{Integer}): The number of arguments in the type's
    constructor
  \item parentArgs (\classref{Integer}): The number of arguments given in the
    call to its parent constructor
  \item data (\classref{List of Data}): Each data defined in the type body has a
    corresponding \classref{Data} object describing its name and position in the
    input program
  \begin{dlist}
    \item The input program position is stored as a line and an offset and makes
      it possible to produce error messages with information about where in the
      input program an error was found
  \end{dlist}
  \item members (\classref{List of Member}): Each constant and function defined
    in the type body has a corresponding \classref{Member} object describing its
    name, argument count, an abstract flag, a \classref{TypeSymbolInfo}
    reference to the type defining the \classref{Member} and a pointer to the line
    and offset in the code
  \item node (\classref{AstNode}): A reference to the
    TYPE\_DEF-\classref{AstNode} which defines the type
  \begin{dlist}
    \item This reference is used to get the input program position where the
      type was defined (for generating useful error messages), but is also used
      for marking abstract type definitions (described in detail in
      \secref{sec:abstractTypeMarker}) 
  \end{dlist}
  \item parent (\classref{TypeSymbolInfo}): This will contain an object
    reference to its parent type if it has one
  \begin{dlist}
    \item This is however a null pointer until running the
      \classref{TypeTableCleaner} described in \secref{sec:typetablecleaner}
  \end{dlist} 
\end{dlist}

All the \classref{TypeSymbolInfo} objects are kept in an object of class
\classref{TypeTable}. The \classref{TypeTable} class is a layer of abstraction
which provides easy and fast access information about the types contained in the
input program. The underlying implementation is a hashmap from the type's name
as a \classref{String} to its object reference, which provides quick lookup on
type names, and a list of \classref{TypeSymbolInfo}, which makes iteration over the
\classref{TypeSymbolInfo}s convenient and makes it possible to sort the types
with a purpose described in \secref{sec:typetablecleaner}. For convenience, we
say that a type is added to a type table which means that a
\classref{TypeSymbolInfo} object representing the type is added to the
\classref{TypeTable} object representing the type table.

When a type is added to the type table, it is checked that no other types with
the same name exist.

\subsection{TypeParentRefMaker}
\label{sec:TypeParentRefMaker}
The \classref{TypeSymbolInfo} objects only contain the name of their super type
as a \classref{String} or a null value if there is no super type. By making a
lookup in the \classref{TypeTable} on the parent name, the real object
references can be found and stored for faster and more convenient parent lookups
which is used greatly by the visitor described in
\secref{sec:usesaredeclaredvisitor}.

\subsection{TypeMemberPropagator}
\label{sec:TypeMemberPropagator}
Some of the later checks that will be performed requires us to determine whether
or not a type member (constant or function) with a specific name is visible in a
given type. This requires searching in the given type and recursively in all
super types for the member. When this kind of lookup is done many times on the
same member, the traversal of the same long chains of parent references become
inefficient which results in clumsy code. To simplify and speed up this process
the \classref{TypeMemberPropagator} ensures that all members of a type A are
also present in a type C, if A is a super type of C. With this approach,
checking if a member is visible in type C, only requires looking in C instead of
following the chain parent types.

This propagation of members is done by first doing a topological sort on the
\classref{TypeSymbolInfo} objects, such that when iterating over the type table,
any type yielded will always appear before all of its subtypes. This makes the
afterwards propagation of members possible in linear time by iterating over the
topologically sorted types. If a type C is met, the members in its parent type B
are put in C as well. If B has a parent A, we know that B has already the
members from A due to the topological sorting.

The topological sorting is done using the algorithm that goes by the same name, from the book Introduction
to Algorithms \cite[p. 612]{ad} working on a graph $G = (V, E)$. Each type
represents a vertex $v$. An edge $e$ exist from $v_1$ to $v_2$ if the type $v_2$
is a parent of the type $v_1$. However this results in a topological order where
a type always appears before its super types. This problem is quickly solved
simply by reversing the list.

Given the graph in \figref{fig:topological}, the following sequences are examples
of correct and wrong topologically sorted orders after the list has been reversed:

\begin{align*}
 Correct &: \texttt{a, d, b, e, c, f} \\
 Correct &: \texttt{a, b, e, c, f, d} \\
 Wrong &: \texttt{a, b, c, \textbf{f}, \textbf{e}, d} \\
 Wrong &: \texttt{\textbf{b}, c, e, \textbf{a}, f, d}
\end{align*}


\begin{figure}[ht]
  \begin{center}
    \begin{tikzpicture}[level/.style={sibling distance=30mm/#1}]      
      \node [square] (a) {A};
      \node [square, yshift=-4em, xshift=-2.5em] (b) {B};
      \node [square, yshift=-4em, xshift=2.5em] (d) {D};
      \node [square, yshift=-8em, xshift=-5em] (c) {C};
      \node [square, yshift=0em, xshift=12em] (e) {E};
      \node [square, yshift=-4em, xshift=12em] (f) {F};

      \draw[<-, thick,] (a) -- (b);
      \draw[<-, thick,] (b) -- (c);
      \draw[<-, thick,] (a) -- (d);
      \draw[<-, thick,] (e) -- (f);
    \end{tikzpicture}
  \end{center}
  \capt{Example of topologically sorted types.}
  \label{fig:topological}
\end{figure}



Another great advantage from topological sorting is the fact that it reveals
cycles in the graph. A cycle in the graph means an extend cycle between types
exists, e.g: A extends B, B extends C and C extends A, which is not accepted.

\subsection{AbstractTypeMarker}
\label{sec:abstractTypeMarker}
The interpreter needs to know whether or not a given type contains any abstract
members. Such a type is an abstract type and should not be allowed to be
instantiated. Marking these abstract types is now smooth. Due to the propagated
members it can just be checked whether or not any abstract members are present in the
given type. This check is done by the \classref{AbstractTypeMarker}. Any type
described by a \classref{TypeSymbolInfo} has a reference to the
\classref{AstNode} it was defined from. If the type is found to be an abstract
type, the type of the \classref{AstNode} is changed from TYPE\_DEF to
ABSTRACT\_TYPE\_DEF. This is the only way to make information visible to the
interpreter since the \classref{Interpreter} does not use the same
\classref{TypeTable} class used by the \classref{ScopeChecker}.

\subsection{TypeSuperCallChecker}
This checker ensures that any type that extends another type provides the right
amount of arguments when calling the parent's constructor. A constructor can have
$x$ arguments and may or may not contain a variable amount of additional
arguments. Consider the type constructor \texttt{Type A[\$var1, \$var2, \dots
\$varargs]}. When calling the constructor from another type, e.g. \texttt{Type
B[] extends A[5, 2, 7, 4]}, it must be checked that the type B provides \textit{at
least} the number of arguments in A's constructor (not counting the variable
amount of extra arguments). If A does not have a variable amount of additional
arguments, the argument count must match exactly. The implemented code for doing
this check can be seen in \lstref{lst:tscc}.

\lstinputlisting[caption={\emph{How the TypeSuperCallChecker is implemented.}},
label=lst:tscc, language=Java]{listings/typeSuperCallChecker.java}

\subsection{UsesAreDeclaredVisitor}
\label{sec:usesaredeclaredvisitor}
This visitor ensures that any use of a variable, constant, function, data
member, or type can be bound to a declaration. The visitor uses a variable
(\classref{TypeSymbolInfo} \varref{currentType}) which updates upon visiting a
TYPE\_DEF or an ABSTRACT\_TYPE\_DEF \classref{AstNode}, to keep track of which
type it is currently visiting inside. If the visitor is not traversing inside a
type (\classref{TypeSymbolInfo} \varref{currentType}) references a special type
called \varref{globalType}, which is used only to contain the standard- and game
environment as well as the global constants and functions declared in a
\productname{} game. 

It is important to realise that \varref{globalType} is not a super type of all
other types, it is a stand alone type that no type can derive from. Its name
contains an invalid character for a type name to ensure that no type can derive
from it. This becomes handy when checking if constants and functions used can be
bound to a declaration.

\subsubsection{Constants and functions}
When a constant or a function is referenced it is necessary to know two things
about the context in which it was referenced:

\begin{nlist}
  \item In what type did the reference occur?
  \item Is the reference a member access?
\end{nlist}

The first thing is easy to check since we have the \varref{currentType}
variable. This variable may however point to the global type. The structure of
the AST makes it easy to determine if it was a member access, since we would
have been visiting a MEMBER\_ACCESS \classref{AstNode} prior to the referenced
constant or function. In the expression: \texttt{A[].B.C[2,3]}, both B and C are
member accesses, but A is not. Given this information, a different check can be
done regarding to the context of the reference:

\begin{nlist}
  \item Type was global and a member access
  \begin{dlist}
    \item Must be visible in at least one type
  \end{dlist} 
  \item Type was global but not a member access
  \begin{dlist}
    \item Must be visible in the global scope
  \end{dlist}
  \item Type was A and a member access
  \begin{dlist}
    \item If prefixed by this, it must be visible in A or any super type of A
    \item If prefixed by super, it must be visible in any super type of A
    \item If prefixed by a variable name, it must be visible in at least one type
  \end{dlist}
  \item Type was A but not a member access: Must be visible in A, a super type
    of A or global scope
\end{nlist}

One may wonder why an accessed member is accepted if the accessed member is
visible in at least one type. Consider the member access \texttt{randomType.B}.
Here it is unknown in what type we shall look for the member B. The constant
\texttt{randomType} could literally return a random type, or the type returned
could be determined by an arbitrary complex algorithm. Therefore, we can only
enforce the rule that the member \texttt{B} must exist in at least one type.

%Skal nedenstående  afsnit med?

%One may think that it is also nice to know if a referenced constant or function
%has a number of parameters along with it and whether the actual number of
%arguments correctly matches the formal number of arguments. This is however
%quite hard to determine. In the example, if \texttt{randomConstant} was declared
%as a constant, the expression \texttt{randomConstant[2]} would still make sense if
%the constant returned a list, in which 2 was an index. This is however
%something the scope checker cannot look into. Given a function declared as
%\texttt{randomFunction[\$a, \$b] = \dots} the expression \texttt{let \$var =
%randomFunction in \dots} would also be correct, in which case \texttt{\$var} is just
%a reference to \texttt{randomFunction} in the \texttt{in}-scope. So it is valid to
%use a constant followed by a parameter list as well as it is valid to not apply
%a parameter list behind a function. 

\subsubsection{Variables}
For any variable, a declaration must always exist before it is used. A variable
can only declared in four ways:

\begin{dlist}
  \item As a type constructor
  \item As a formal parameter in a function declaration
  \item In a lambda expression
  \item In a \texttt{let-in} expression
\end{dlist}

In all cases the \productname{} semantics require that a new scope is opened, in
which the declared variable is known while the body of the expression is
executed. When the scope closes the declared variables are removed. The body of
an expression can also contain new variable declarations, e.g. a \texttt{let-in}
expression in the body of a \texttt{let-in} expression. 

The scope checker uses a \classref{SymbolTable} class which is basically a
symbol table with a list of variable names and a reference to a parent symbol
table.  The reference to the parent symbol table is exactly how the scopes
inside other scopes are implemented. 

\codesample{openscopeexpressions.junta}

Notice how the four code samples in in the above codesample all result in the same
scope checking routine, which can be seen in \figref{fig:scope1}; First, a new
symbol table is instantiated in which the variables \$a and \$b are put in. The
symbol table's parent reference is updated so it points to the current symbol
table, which is referred by \classref{SymbolTable} \varref{currentST}. Next, the
current symbol table is updated to the newly created symbol table, and the body
(the triple dots) are executed. Lastly, the current scope is closed, which
updates the current symbol table reference to point to the parent symbol table
of the current symbol table.  Notice that the symbol tables maintain a
stack-like structure, where opening a scope pushes a symbol table on the stack
and closing a scope pops one. The variable \classref{SymbolTable}
\varref{currentST} points to the element on top of the stack.

When a variable is used, it is checked that the variable exists in any of the
symbol tables by first looking in the current symbol table and recursively
following the parent reference until a null reference is found. If a variable
declaration with the same name as the used variable cannot be found, an error is
generated.

\fig[height=5em]{scope1}{Four different expressions that all result in the scope
action depicted.}

It is important to realise the reason for maintaining the stack-like structure
of symbol tables. It might seem like a single symbol table would be enough and
that all variable declarations could just be put in there. This is indeed wrong,
since the scope checker must also check for double declarations. A double
declaration exists if a symbol table contains the same variable twice. Notice
how \figref{fig:scope2} contains two symbol tables, each containing a
declaration of \$a. 

\fig[height=5em]{scope2}{The variable \$a declared in two different scopes.}

This is completely valid and is caused by the following codesample. If only a
single symbol table was used, an incorrect double declaration would be detected.

\codesample{scope2.junta}

\subsubsection{Data members}
When visiting a type body a new scope is opened and the data members of
\classref{TypeSymbolInfo} \varref{currentType} are immediately inserted into that
scope. The children of the type body is then visited and the scope is closed.
This ensures that the data members of a type can be used anywhere in the type
body but in that type body only. When exiting that type body and closing the
scope the symbol table containing the data members are no longer visible.

\subsection{Evaluation}
Many different static semantic checks are implemented in the scope checker.
Though many other checks could have been included as well, the scope of the
static semantics has been limited due to a few constraints. First of all, there
is a deadline for this project, and with an almost endless set of semantic
checks one can keep developing these checks. Furthermore, with new techniques
being discovered once in a while, a compiler or interpreter can simply not
include them all. A big set of the checks not included in \productname{}
requires type checking, which is cumbersome in a dynamic programming language.
However, it is generally possible to use type inference to find at least some of
the types and errors associated with the use of them. It is important to realise
that everything cannot always be inferred, for instance an algorithm could be so
complex that it would need to be executed to determine all possible outcomes.
Running the algorithm is not possible since you cannot know if the algorithm
will ever halt.
\cite[p. 173]{itttoc}

\section{Interpreter}
The final step is the interpretation of the AST generated by the parser.
Here choices are taken based on the different node types in the tree
for a given program. This interpreter class is, like the scope checker,
also implemented with the visitor pattern, visiting all the nodes and
taking appropriate actions depending on what type of node is visited.
The difference here is that values, types, and other language constructs
are created and evaluated directly, propagating the values up the AST.

In this section, we will look at the inner workings of the interpreter
and see some examples of the different evaluation methods used and how
they're created. We also see how the symbol table keeps track of scopes
and the different values. Thereafter the implementation of patterns is
described, as they play a central role in most programs and functions
written in \productname{}.

\subsection{Symbol table and scopes}
Symbol tables in the interpreter are used to add and get constants, variables, types, and to push and pop the scope stack.

\subsection{Values and their operators}
Each one of our base values is represented internally by a class used in the interpreter. These classes are all sub-classes of a general class \classref{Value}, that offers the sub-classes an interface to implement various different operations, such as comparison, addition, calling, and so on, throwing an error if trying to use the operator between incompatible values or if the operation yields an invalid result.

In the following part of this section, a few important values and their features are highlighted.

\subsubsection{Coordinates}

\subsubsection{Lists}

\subsubsection{Types}
% Abstract type values are just an extension of types

\subsubsection{Functions}

% Our types are probably most interesting to mention here?


\subsection{Pattern evaluation}

\subsection{Evaluation of interpreter}
It is obviously not the fastest implementation of an interpreter, but we don't see that as a hindrance, because speed is not what we're after.

Under development, we have kept flexibility in mind, allowing us to easily add new base values and extend and maintain the feature set and semantics of our language, without having to change any existing code.

\section{Error handling}
% Talk about how, where, and why errors are thrown in the different packages
% ..and how they're implemented

\section{Game abstraction layer}
\label{sec:gameabstractionlayer}
We need an abstraction layer to act as the glue between the interpreter
and simulator, allowing a simulator access to different elements in a
written program. This has caused us to write an abstraction layer on top
of the interpreter, offering interfaces such as our graphical simulator
the player access to elements in the game.

There are three different packages relating to the game abstraction
layer. The first is the central class in the Game Abstraction Layer
package, the second are wrappers in the Interpreter package, providing
the actual implementation, and lastly is the game application
programming interface in our Utilities package. Each is described in
this section.

\subsection{The main layer}
The entire program package offered by \productname{} is encapsulated by
the class \classref{GameAbstractionLayer}. This small class is basically
only defined by its constructor, seen below in \lstref{lst:gal}:

\lstinputlisting[language=Java,label={lst:gal},caption={\emph{The game abstraction layer's constructor, initializing different constructs based off of an input of characters.}}]{listings/gal.java}

Here all the constructs are tied together: The handwritten scanner
is instantiated with the input provided in the constructor, creating
a stream of tokens fed to the parser, that then creates an abstract
syntax tree traversed by the interpreter. The interpreter and its
information is then used when getting the main game wrapper. The class
is also complimented by the method \methodref{getGame}, called after
instantiating the \classref{GameAbstractionLayer}. This method returns a
\classref{GameWrapper} (described in the next sub-section), containing
all the needed information about the written game.

This class is what's called and used by the simulator when starting up,
as the simulator needs access information about the game, which in turn
builds on top of our scanner and parser. It's meant to be used by any
interface that wishes to access and modify a game's state.

\subsection{The application programming interface}
The application programming interface (API) provided is simply a collection of interfaces used by the wrappers. This means that classes implementing these interfaces guarantee different methods are available. As an example, the \classref{Square} interface is seen in \lstref{lst:squareinterface} below:

\lstinputlisting[language=Java,label={lst:squareinterface},caption={\emph{The \classref{Square} interface, one among many such interfaces provided by \productname{}.}}]{listings/squareInterface.java}

Currently $12$ such interfaces exist for different aspects of a game,
such as getting information about the game itself, its players,
board, action sequences, etc.

\subsection{Wrappers}
Wrappers provide a way of accessing the values created by the
interpreter, allowing a simulator to use the properties of these
values to, for example, display the game's title, supply information
about the winning conditions, squares \& pieces, actions \& move
history, and so on. The wrappers implement the interfaces in the
API described above. The main wrapper returned from the method
\methodref{getGame} provides access to all the other wrappers
(implementing their respective interfaces) via methods in its body.

% Write about gamewrapper
When instantiating \classref{GameWrapper} in method \methodref{getGame} of class \classref{GameAbstractionLayer}, an instance of the interpreter is passed. All wrapper classes have the following signature:

\begin{lstlisting}[language=Java,caption={\emph{The signaure of all API wrapper classes.}}]
  public class xxWrapper extends Wrapper implements yy { ... }
\end{lstlisting}

Where $xx$ is one of the $12$ wrappers and $yy$ is the matching interface.

\section{Simulator}
\label{sec:simulator-impl}

In this section, we begin by giving an overview of the relations of classes in
the Simulator in \secref{sec:overview}. Afterwards, we present
\classref{Widget}s and their properties in \secref{sec:widget}. We also explain
the different actions that are available in \secref{sec:actions}. Furthermore,
we explain how the different \classref{Widget}s communicate with each other in
\secref{sec:communication}. Lastly, we present how we have made the game
interactive in \secref{sec:interaction}, and how everything is connected in
\secref{sec:connection}.

%Overview
%Widget
%Propagated actions
%Communication between widgets
%Making games interactive
%Connecting everything

The Simulator has been constructed to provide a visual interface to the API
provided by the Game Abstraction Layer (GAL). The interface should give a visual
representation of the board and pieces, similar to how the board game would look
in real life. Furthermore, it should provide an easy way to interact with the
game. This interaction should be sufficient enough to make it possible to play
the game.

The result is a Java application with a graphical user interface, which takes a
\productname{} code file and makes it playable by the use of GAL.

\subsection{Overview}
\label{sec:overview}

The implementation is based around the class \classref{Widget}, which simplifies the
process of distributing drawings and handling input. Some \classref{Widget}s
manage other \classref{Widget}s, while others provide visual and interactive
content. The game \classref{Widget}s provide visual and interactive intefaces to
GAL.

To work with graphics and input, the game framework slick2d is used\cite{slick2d}.
\classref{SimulatedGame} is used to bind slick2d, \classref{Widget}s, and GAL
together to provide a complete system.

Figure \ref{fig:simulator-overview} shows all these components and their most
important relationships.

\fig[width=0.7\textwidth]{simulator-overview}{Generalised overview of classes in
the Simulator and their relations. Triangles means inheriance, while arrows means that one class makes use of the one it points to.}

\subsection{Widget}
\label{sec:widget}

\classref{Widget} specifies an object which can have a size and a position, can be
drawn, take mouse input, and send messages to each other. The most important
property of \classref{Widget} however is that a \classref{Widget} can contain
several sub-\classref{Widget}s which each can contain futher sub-\classref{Widget}s. We use this tree-structure to control how drawing and mouse input is handled.

\subsubsection{Placement}

A \classref{Widget} has a position specified with a $(x, y)$ coordinate. Its
position is relative to its parent, so if a \classref{Widget} has position
$(7, 10)$ and its parent has $(23, 50)$, its absolute position is $(30, 60)$.

Furthermore, a \classref{Widget} has a size specified with a width and height,
but it also contains an allowed range for each dimension. This allows us to
specify that a \classref{Widget} might be dynamic in size and can be adjusted if
wanted.

\subsubsection{Automatic placement and sizing}

Instead of setting sizes and positions manually, we create container classes
that manage the position and size of their sub-\classref{Widget}s. By using
\classref{Widget}s which are dynamic in size, we can create a layout which works
independent of the window and board size.

\classref{ScaleContainer} is such a container \classref{Widget} and positions
\classref{Widget}s along an axis. For \classref{Widget}s whose size is dynamic,
the remaining available space is distributed evenly amoung them. An example is
shown in \figref{fig:ScaleContainer}. The top-level \classref{Widget} is a
\classref{ScaleContainer} set to position \classref{Widget}s vertically. It does
not affect its own size, only its sub-\classref{Widget}s. The second
\classref{ScaleContainer} (containing two buttons, to be positioned
horizontally) is thus resized by the first \classref{ScaleContainer}. The
ordering of the sub-\classref{Widget}s determines the sequence they are
positioned in the \classref{ScaleContainer}.

\fig[width=0.7\textwidth]{ScaleContainer}{How \classref{ScaleContainers} (marked
with color) affects the positioning and resizing of sub-\classref{Widget}s.}

Creating a layout is now only a matter of building a hierarchy of
\classref{Widget}s, not deciding the exact position and size of each and every
single \classref{Widget}.

\subsection{Propagated actions}
\label{sec:actions}

Drawing a \classref{Widget} should not only draw the \classref{Widget}, but also
all its sub-\classref{Widget}s and their sub-\classref{Widget}s. To do this,
\classref{Widget} has two methods, \texttt{draw()} and \texttt{handleDraw()}.
\texttt{handleDraw()} needs to be overridden in sub-\classref{Widget}s that 
wants to provide a custom drawing method. \texttt{draw()} handles all the logic
for drawing sub-\classref{Widget}s, so the inherited class only needs to worry
about itself. An overview of \texttt{draw()} is given in
\lstref{lst:simulatorDraw}.

\lstinputlisting[caption={\emph{Pseudocode for the \texttt{draw()} method.}},
label={lst:simulatorDraw}, language=Java]{listings/simulatorDraw.java} 

\todo{looks wrong, draw() will never be called on the top-level
  \classref{Widget}.  Secondly, draw() should be called before handleDraw() }

To further ease development, the coordinate system is translated so
\texttt{handleDraw()} will be done using local coordinates instead of absolute
coordinates.  Furthermore, we apply clipping, so that any drawing outside the
\classref{Widget} will be clipped and not displayed. This way we can ensure that
\classref{Widget}s can't mess with other \classref{Widget}s.

We enforce this by restricting the \texttt{draw()} method with Java's final
keyword so it can't be overwridden, and \texttt{handleDraw()} is protected, so
the calling class can't call \texttt{handleDraw()} instead of \texttt{draw()} by
accident.

\subsubsection{Mouse input}

The same pattern is used for mouse input, but here we use it to determine which
\classref{Widget} is responsible for handling it. An overview is given in
\lstref{lst:simulatorMouseClicked}.

\lstinputlisting[caption={\emph{Pseudocode for the \texttt{mouseClicked()}
method.}}, label={lst:simulatorMouseClicked},
language=Java]{listings/simulatorMouseClicked.java}

Like with drawing, we translate coordinates into local coordinates, however
notice that it returns a boolean which is used to determine if the event was
handled, and it will stop as soon as any callee returns true. A second
difference is that in constrast to drawing, input is handled bottom up. The
reasoning is that the lower we get in the hierarchy, the more specific the
behaviour of each \classref{Widget} is. Thus, we try to see if the more specific
\classref{Widget}s will handle the input and if not, less and less specific
\classref{Widget}s are tried.

When mouse buttons are pushed and released, it will only try \classref{Widget}s
which contain the position at which the mouse is currently pointioned at. For
mouse dragging the situation is different, it will try any \classref{Widget}
which has initiated a drag, even if the mouse has moved outside it. If this was
not the case, a scrollbar for example would only move if we kept the mouse
exactly on top of it, which usually is tricky as they are long and slim.

\subsection{Communication between Widgets}
\label{sec:communication}

Consider the case where a \classref{Widget} represent a button. The user might
click on it, but the button by itself is not interested in what this should
signify. Thus, we need some way of notifying \classref{Widget}s that some events
have happened inside other \classref{Widget}s. For this, the Observer design
pattern is used in \classref{Widget}.

\subsection{Making games interactive}
\label{sec:interaction}

Two "game \classref{Widget}s" which interact with GAL are used to present the
game to the user. They are \classref{GameInfoWidget}s which provide information
like move history, and \classref{BoardWidget} which displays an interactive
board with pieces based on GAL.

\subsubsection{BoardWidget}

For interaction, \classref{BoardWidget} supports selecting \classref{Action}s by
the use of either Drag\&Drop or Click\&Select. While Drag\&Drop only allows
you to move a \classref{Piece} from one \classref{Square} to another, Click\&Select 
will work on any two \classref{Square}s whether or not it contains any
\classref{Piece}s. It will go through all available \classref{Action}s and find
the ones which are related to those Squares. To help ease this process, usable
\classref{Square}s are hinted as shown in \figref{fig:multiple-moves}.

\fig[width=0.7\textwidth]{multiple-moves}{The Square at E2 was selected and shows the 4 pieces which can move there.}

In reality we have two \classref{Widget}s: \classref{BoardWidget} and
\classref{GridBoardWidget} which are specialisations of \classref{BoardWidget}.
While it is not necessary at this point, it is an attempt to generalise
\classref{GirdBoard} so that a future addition with new \classref{Board} types
will be easier to implement.

\subsection{Binding everything together}
\label{sec:connection}

The class \classref{SimulatedGame} has the responsibility to connect slick2d
with the \classref{Widget} structure, and GAL with the game \classref{Widget}s.

\classref{SimulatedGame} contains one \classref{ScaleContainer}, which it resizes
to fit the whole window and tells it to adjust the sizes of its
sub-\classref{Widgets}. Secondly, it sends all mouse-events to this
\classref{Widget} and draws it whenever slick2d wants to be redrawn.

On construction of \classref{SimulatedGame}, it reads the \productname{} code file
and attempts to load it through GAL. It then creates the game
\classref{Widget}s, but it does not pass a reference to the game directly. As
the game object changes each time an \classref{Action} is applied, which
requires us to update it in every game object each time. Instead we pass a
reference to this instance of \classref{SimulatedGame} and the game
\classref{Widget}s must
then access the game object through its accessor methods directly, without
caching it.

One final task of \classref{SimulatedGame} is to handle any exceptions in GAL or
the \classref{Simulator} and show them to the user, without the application
crashing.

\section{An overview of the implementation of \productname{}}
\label{anoverviewofjunta}

Now that the implementation of the different constructs that make up our
implementation of \productname{} have been described, we wish to provide
an overview of how they fit together, attempting to provide a concise
overview. Figure \ref{fig:cake} presents how these components actually
are connected.


\begin{figure}[ht]
  \begin{center}
    \begin{tikzpicture}      
      \node [rectangle,draw,rounded corners,thick,text width=7cm,text centered,]
      (a) {\textbf{Simulator}};
      \node [rectangle,draw,rounded corners,thick,text width=7cm,text
      centered,yshift=-1cm,text centered,] (b) {\textbf{Game Abstraction
      Layer}};
      \node [rectangle,draw,rounded corners,thick,yshift=-2cm,xshift=-2.95cm,text
      centered,] (c) {\textbf{Scanner}};
      \node [rectangle,draw,rounded corners,thick,yshift=-2cm,xshift=-1.3cm,text
      centered,] (d) {\textbf{Parser}};
      \node [rectangle,draw,rounded corners,thick,yshift=-2cm,xshift=0.6cm,text
      centered,] (e) {\textbf{Scope checker}};
      \node [rectangle,draw,rounded corners,thick,yshift=-2cm,xshift=2.8cm,text
      centered,] (f) {\textbf{Interpreter}};

      \node[rectangle,xshift=-5cm] (invis1) { };
      \node[rectangle,xshift=5cm] (invis2) { };

      \draw[line] ($(a.south)-(-2.95,0)$) |- ($(b.north)-(-2.95,0)$) 
        node [midway, xshift=0.9em,yshift=0.7em] {11};
      \draw[line] ($(b.north)-(3.1,0)$) |- ($(a.south)-(3.1,0)$)
        node [midway, xshift=-0.8em,yshift=-0.7em] {2};
      \draw[line] ($(b.south)-(2.8,0)$) |- ($(c.north)-(-0.15,0)$) 
        node [midway, xshift=0.7em,yshift=0.7em] {4};
      \draw[line] ($(c.north)-(0.15,0)$) |- ($(b.south)-(3.1,0)$)
        node [midway, xshift=-0.7em,yshift=-0.7em] {3};
      \draw[line] ($(b.south)-(1.15,0)$) |- ($(d.north)-(-0.15,0)$) 
        node [midway, xshift=0.7em,yshift=0.7em] {6};
      \draw[line] ($(d.north)-(0.15,0)$) |- ($(b.south)-(1.45,0)$)
        node [midway, xshift=-0.7em,yshift=-0.7em] {5};
      \draw[line] ($(b.south)-(-0.8,0)$) |- ($(e.north)-(-0.2,0)$) 
        node [midway, xshift=0.7em,yshift=0.7em] {8};
      \draw[line] ($(e.north)-(0.3,0)$) |- ($(b.south)-(-0.3,0)$)
        node [midway, xshift=-0.7em,yshift=-0.7em] {7};
      \draw[line] ($(b.south)-(-2.95,0)$) |- ($(f.north)-(-0.15,0)$) 
        node [midway, xshift=0.7em,yshift=0.7em] {10};
      \draw[line] ($(f.north)-(0.15,0)$) |- ($(b.south)-(-2.65,0)$)
        node [midway, xshift=-0.7em,yshift=-0.7em] {9};
      \draw[line] (a) -- (invis1) node [midway, sloped, above] {1};
      \draw[line] (invis2) -- (a) node [midway, sloped, above] {12};
    \end{tikzpicture}
  \end{center}
  \capt{A diagram showing the components of \productname{} and how they are
connected.}
  \label{fig:cake}
\end{figure}



The numbers seen in the figure is the sequence of stages in which programs 
written in \productname{} will go through before they are playable. The first
arrow ($1$) is when a user inputs a program and the last arrow ($12$) is when
the program is ready to be played.

The following enumeration explains what each number the arrows have:

\begin{nlist}
\item File location
\item Stream characters
\item Stream of characters
\item Stream of tokens
\item Stream of tokens
\item Abstract Syntax Tree
\item Abstract Syntax Tree
\item Abstract Syntax Tree
\item Symbol table and Abstract Syntax Tree
\item Symbol table
\item A game object 
\item Fun with playable board games
\end{nlist}

The simulator begins by accepting a file location as input. Then the
game abstraction layer (GAL) receives a stream of characters from
this file at that location (the simulator opens the file and reads
it, instantiating GAL with the input stream). The GAL forwards the
stream of characters to the scanner, which scans the input and outputs
a stream of tokens to the GAL. The GAL again forwards the stream of
tokens to the parser, which parses the input and outputs an abstract
syntax tree (AST) to the GAL. The GAL once again forwards the AST to
the scope checker, which checks the AST for errors and passes the
checked (and possibly a modified) AST back to the GAL. One last time,
the GAL forwards the AST to the interpreter (instantiated with the game
environment), which visits every node and takes action depending on the
node types, meanwhile building up its symbol table. The interpreter
outputs the symbol table to GAL, and GAL sends an object with information
about the game to the simulator, which now makes it possible to play the
board game.

All this can of course be interrupted an error is thrown in one of the
stages. It is up to the simulator to handle these errors. If no errors
are encountered, then it is possible to see the game in the simulator and
the visualised game can be played.

