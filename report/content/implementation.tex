\chapter{Implementation}
\label{chap:implementation}

In the following chapter we present how the different modules of \productname{} are implemented. In \secref{sec:scannerimplementation} the scanner implementation is specified and examples of how the tokens are identified are given. In \secref{sec:parserimplementation} we present the parser of
\productname{}. We explain how it's constructed and how it works. Furthermore we present the implementation of SableCC and the reason why we did not proceed with this. In \secref{sec:ant} our abstract node types are explained and illustrated. In \secref{sec:staticsemantics} we present the scope checker of \productname{}. Here the visitor pattern and the use of it is also introduced. In \secref{sec:interpreter} the interpreter is presented. Here we for example explain how patterns are implemented which is a feature used in most programs written in \productname{} and we introduce our implementation of operators and values. In \secref{sec:errorhandling} we present how error handling is implemented. In \secref{sec:gameabstractionlayer} we explain the game abstraction layer which is the layer that binds the interpreter and the simulator together and makes them able to communicate with each other. Finally in \secref{sec:simulator} the game simulator of \productname{} is presented.    


\chapter{Scanner}
The scanner is the first part of the interpreter that analyses the input. The input for the scanner is a raw source code for a program written in \productname{}. The main purpose of the scanner is to validate the lexically correctness of a \productname{} program. The scanner must analyse the input and find tokens existing in \productname{}. If an input is met which can not be recognized as a valid token, the source code is not a valid program in \productname{}. The strings that are converted to tokens are called lexemes. Many lexemes can be converted to the same type of token. Some programs contains many different identifiers, which in \productname{} all will have a token instantiated with the type \textit{IDENTIFIER}. Consider the example of a single line from a chess game written in \Productname{}. The raw input is:
\begin{lstlisting}
Black{ Pawn [A7 B7 C7 D7 E7 F7 G7 H7] }
\end{lstlisting}
The result of analysing the input can be seen in \tableref{table:lexemestotokens}. Tokens are needed for abstraction. When later on the parser will determine if the code respects the grammar of \productname{}, it is useful to have these abstractions. It makes it possible to describe that a list of \textit{COORD_LIT}'s can be encapsulated between the characters ``[ ]'' without having to list all possible coordinate literals, which in fact are an infinite set, since the grammar of \productname{} which is described later, allows proceeding in both dimensions after \textit{Z9}, namely \textit{Z10} and \textit{AA9}. However, when converting lexemes to tokens, the value of a coordinate, the name of an identifier, e.g. is still kept since that information will be needed for the subsequent parts of the interpreter.

\begin{figure}
\begin{table}
    \begin{tabular}{|l|l|}
        \hline
        Lexemes & Tokens             \\ \hline
        Black   & IDENTIFIER (Black) \\ 
        {       & LBRACE             \\ 
        Pawn    & IDENTIFIER (Pawn)  \\ 
        [       & LBRACKET           \\ 
        A7      & COORD_LIT (A7)     \\ 
        B7      & COORD_LIT (B7)     \\ 
        C7      & COORD_LIT (C7)     \\ 
        D7      & COORD_LIT (D7)     \\ 
        E7      & COORD_LIT (E7)     \\ 
        F7      & COORD_LIT (F7)     \\ 
        G7      & COORD_LIT (G7)     \\ 
        H7      & COORD_LIT (H7)     \\ 
        ]       & RBRACKET           \\ 
        }       & RBRACE             \\
        \hline
    \end{tabular}
\end{table}
\caption{Analysing an input stream for lexemes and tokens}\label{table:lexemestotokens}
\end{figure}

The scanner contains 2 classes, \classref{Scanner} and \classref{Token}. \classref{Token} contains an enum named \classref{Type} that enumerates all the types of tokens in \productname{}. When a lexeme is found in the input stream, the scanner analyses which token type it belongs to. A new token is then instantiated and yielded by the scanner. The constructor for \classref{Token} takes the arguments (\classref{Token.Type}, \textit{line}, \textit{offset}). The \textit{line} and \textit{offset} represent where in the source code the lexeme of any token where found, which can be used to inform a programmer where in his source code an error was found, if any was found.
 
\section{Parser}
\label{sec:parserimplementation}

In this section we present the handwritten parser of \productname{}'s.
We have written a top-down recursive descent parser, which is within the class
of LL(1) parsers. The grammar for \productname{} is suited for this because
e.g.\ it does not have left-recursive productions. In the end of the section we present our work 
with SableCC and the reason why we chose not to continue working with this tool.

\subsection{Constructing the parser}
%structured as the grammar
The parser was very simple to implement, because it is structured exactly the same way as the
grammar which can be found in \chapref{ap:CFG}.  For instance if the grammar
expresses that the next set of terminals must begin with a left bracket (`['),
  then the parser will expect the next token to be a \tokenref{LBRACKET} which
  is the token name for a left bracket. If the grammar then expects a
  non-terminal, then the parser simply calls the method for that non-terminal,
  allowing it to finish, possibly calling more non-terminals and expecting
  terminals, before continuing parsing the next part of the rule.

%discuss the if expression
In \lstref{lst:ifexpr} we give an example of how this structure looks like in
our handwritten parser. The production rule for an if-expression is presented in
\secref{sec:conditionalexpressions}.

%\begin{ebnf}
%\grule{if\_expr}{\gter{if} \gcat expression \gcat \gter{then} \gcat expression
%\gcat \gter{else} \gcat expression}
%\end{ebnf}

The production for if-expression says that every expression of this type
must start with the combination of the two symbols which spell the word
\gter{if}. When the parser meets this word in an expression, it knows
that it has to parse an if-expression.

\lstinputlisting[caption="How if-expressions are parsed using top-down parsing in Java.",
label=lst:ifexpr, language=Java]{listings/ifexpr.java}

\subsection{Building an abstract syntax tree}
%astNode()
In \lstref{lst:ifexpr}, the parser initialises the node for the
expected if-expression. The parser starts by calling the method
\methodref{astNode} to create a node for the Abstract Syntax Tree (AST).
We call the method with information about what type of expression this
is (\tokenref{IF\_EXPR}). The method calls the \methodref{expect} method
to verify that the next token is what we are expecting. If the two
tokens do not match, the parser throws a syntax error with information
about the error. If everything is syntactically correct the parser
constructs a node for the AST for the given expression. The first child
of the node is the boolean expression, and the next two siblings of that
child are the expression branches of the if-expression.

\subsubsection{Terminal and nonterminals}
Every grammar has a finite set of nonterminals and terminals that
constitute the productions of the grammar. We have defined tokens in the
parser for every nonterminal in our grammar. The if-expression has the
token name of \tokenref{IF\_EXPR}.

In the production for the if-expression, we have three terminals: the \gter{if},
\gter{then}, and the \gter{else}. These are all required in the method for any
if-expression. When the parser finishes reading a terminal, it knows that the
following token will be an expression, and therefore a new child for the node is
made with a call to the \methodref{expression} method wherein we parse
expressions. Finally the method returns the node containing every child for the
whole if expression.

\subsection{Looking ahead in the input}
%lookAhead methods - atomic
We mentioned earlier that the parser is an LL(1) parser, which means that the
parser is able to look ahead in the sequence of tokens. We have shown the
\methodref{lookAhead} method to determine if the next token is part of an
atomic expression. The production for the atomic expression is presented in
\secref{sec:atomicexpressions}.

%\begin{ebnf}
%\grule{atomic}{\gter{(} \gcat expression \gcat \gter{)}}
%\galt{variable}
%\galt{list}
%\galt{\gter{/} \gcat pattern \gcat \gter{/}}
%\galt{\gter{this}}
%\galt{\gter{super}}
%\galt{direction}
%\galt{coordinate}
%\galt{integer}
%\galt{string}
%\galt{type}
%\galt{constant}
%\end{ebnf}

An atomic expression can derive quite a few productions. This is why we have
constructed a specific method to determine whether the next token is part of an
atomic expression. This method is shown in \lstref{lst:lookaheadatomic}.

\lstinputlisting[caption="The lookAhead method to determine if the next
  expression is an atomic type.", label=lst:lookaheadatomic,
language=Java]{listings/method_lookAheadAtomic.java}

The method \methodref{lookAheadAtomic} makes use of two methods to figure out if
the next token is part of an atomic expression. The first method is the
\methodref{lookAhead} method that takes a token as an argument and figures out
if the next token in the sequence of tokens are equal to each other. The second
method is the \methodref{lookAheadLiteral} method which is similar to the method 
in \lstref{lst:lookaheadatomic} but instead of checking for atomic expressions it 
checks for literals. All these methods return true or false.

%example of lookAheadAtomic
%LL(1)
In \lstref{lst:examplelookahead} we show an example of how the
\methodref{lookAhead} method is used in the parser. The example is taken from
the \methodref{expression} method. The productions for expressions are presented
in \secref{sec:expressions}.
The production of an expression is reflected in the code of the parser. An
example of this is given in \lstref{lst:examplelookahead}.

%\begin{ebnf}
%\grule{expression}{assignment}
%\galt{if\_expr}
%\galt{lambda\_expr}
%\galt{\gter{not} \gcat expression}
%\galt{operation}
%\end{ebnf}

\lstinputlisting[caption="Use of the \methodref{lookAhead}-method. This example
is from the \methodref{expression}-method.", label=lst:examplelookahead,
language=Java]{listings/example_lookAheadAtomic.java}

The code presented in \lstref{lst:examplelookahead} is a small section of the
\methodref{expression} method. We have removed code from the section which is
not relevant for the example we are trying to give. The removed code is
presented as \{\ldots\}. In \lstref{lst:examplelookahead} we wish to present how
the \methodref{lookAhead} methods are used.

An assignment begins with the reserved word \gter{let} and the first
\methodref{lookAhead} method peeks for exactly that token to determine
if the next production is an assignment. If the method returns true then
the next token is in fact the \gter{let} word, and the parser enters a
new method, namely the \methodref{assignment} method which checks to
determine if the rest of the production is correctly written. The same
is done for the if expression, lambda expression and operations which
begins with the ``loSequence()'' (logical operators).

The operation production is a bit different, because it needs two lookAhead
methods to determine if the next production is an operation. An operation can
begin with either an atomic value or a minus operator. So the code uses a
\methodref{lookAheadAtomic} and a regular \methodref{lookAhead} with the
specific token as a parameter to check if the next production is an operation.

The methods return nodes which are connected with each other to form
a complete AST\@. When the parser has parsed every token of the input,
it can produce an AST that corresponds to the program written in
\productname{}. This shows that the parser is built systematically
according to the grammar, producing a parse tree consisting of AST
nodes.

\subsection{SableCC}
We have also implemented a scanner and parser using a
compiler/interpreter generator, or a compiler compiler
known as SableCC\cite{sableccdoc}. As described in
\secref{subsec:generatedparsers}, it is an automated scanner and
LALR($1$) parser generator written in Java, with support for making
compilers and interpreters. We have implemented an early version of
\productname{} in SableCC to evaluate the capabilities of such a tool.

\subsubsection{Choice of SableCC}
We chose SableCC instead of various other popular tools such as
ANTLR\cite{antlr} and JavaCC\cite{javacc}. Though ANTLR and JavaCC are far more
well-documented than SableCC, we still chose this tool because it's parser
generates a LALR($1$) parser, whereas ANTLR and JavaCC create simpler
LL($k$) parsers, which our hand-written implementation already takes advantage of.
Therefore, we felt LALR($1$) parsing to be more interesting, since it
also supports more powerful grammars (such as left-recursion).

SableCC outputs an abstract syntax node type fokokr each alternation in
every rule in the grammar specifications file. It's then possible to
iterate over these nodes via extending the visitor pattern SableCC also
supplies, generating code or directly interpreting a syntactically
correct program. This is all done in classes separate from the grammar
specifications, which is also desirable and different from ANTLR
and JavaCC, where action code is injected directly in the grammar
specification. This is all done in Java, which is also desirable, since
it would work well with the rest of the project (also written in Java).

\subsubsection{Experience with SableCC}
Our experience with the tool has been rather cumbersome, in that it took
quite a while to read the documentation before and during writing the
specifications, as it simply isn't just copy/pasting EBNF grammar
into a file. An example of the if-expression rule is seen here

\begin{lstlisting}[caption={Part of the grammar specifications file of SableCC, with focus on if-expressions.}]
Tokens
  else          = 'else';
  then          = 'then';
  if            = 'if';

Productions
  expression   = {elopexp} element operator expression
               | {assign} assignment
               | {if} if_expr
               | {lambda} lambda_expr
               | {el} element list?
               | {not} not expression;
  if_expr      = if [left]:expression then [mid]:expression else [right]:expression;
\end{lstlisting}

% Pros
This example brings out the strengths of SableCC, as it looks very
similar to the EBNF for if-expressions, with a few additions. As long as
you know the special syntax and how helpers, tokens, and productions works, it
is possible to create scanners and parsers very quickly. This was not
our case, as no one had experience with any form of compiler-compilers.
Given that it generates a LALR parser, it grants the ability to have
more powerful grammar than what we had designed. Lastly, the fact that
there's a clear and clean separation between automated, generated code
and user code makes the grammar and compilation/interpretation parts
easier to maintain. When adding new features to the language, you simply
have to update the specifications file and generate a new scanner/parser
combination. On the other hand when adding new features to the language while
using a handwritten scanner and parser many lines of codes needs to be change
in order to implement the new feature. 

% Cons
Even though SableCC looks like a prime candidate to continue
interpretation with, we chose not to use the tool. This is because it
took an unreasonable amount of time to figure out how to precisely
define the grammar to keep it from being ambiguous. And with poor
documentation, it took even longer. Also, it offers less control and
customizability, compared to writing our own from scratch. As an
example, the tool offers an application-specific interface to tree
walking the AST nodes with the visitor pattern, requiring knowledge of
how SableCC implements it. SableCC also generates around 17000 lines of
Java code, even for our simple grammar, which seems superfluous compared
to the handwritten code, consisting of around 1500 lines of code.
%  - Don't learn as much about different parser techniques
%  - Old project, not as active anymore

% WHAT TO CALL THIS SUBSUBSECTION?
\subsubsection{Discussion}
We chose not to continue using SableCC on our updated grammar, due
to the weight of cons against pros, and the fact that the time spent
working on implementing SableCC was also spent making the handwritten
scanner and parser and making them work exactly the way we want them to. It might
not be as easy to modify our language with this solution, but the time
spend on modifying and adding features to our language with the use of a
handwritten scanner and parser is not wasted time, but learning time, which
gives us better understanding of their underlying functionality.  


\section{Abstract node types}
\label{sec:ant}

This section illustrates how we have chosen to implement the grammar described in
\secref{sec:grammar} into abstract syntax trees (ASTs). 
We have split every program part into their own subsections with an abstract
node type (ANT) that presents the program parts construction when it is parsed into an
AST. Each node type corresponds to a production in the grammar. These ANTs can
then be combined to form the AST when some piece of code has been parsed.

Firstly, the section begins with a description of how an AST differs from a parse tree 
to make this clear to the reader.

When a piece of code is parsed by a parser that understands the specific programming 
language, the output of the parser will be an AST which consists of the abstract node 
types of each program part. The difference between an AST and a parse tree is that a 
parse tree contains every detail of the input code. A parse tree includes for instance
the parentheses and keywords where an AST does not contain anything other than the 
abstract node types.\cite{parsevsast}

Now, the section is divided into three subsections. First we present the actual
parts of a program structure. Then we present which expressions the language
consists of. Finally, we present the different patterns.

\subsection{Program structure}
The following subsections present the structure of programs. The sections are 
structured as follows:

\begin{dlist}
  \item Program
  \item Definition
  \item Constant definition
  \item Type definition
  \item Type body and member definition
  \item Abstract definition
  \item Variable list
\end{dlist}

\subsubsection{Program}
Every program written in \productname{} begins with an abstract node type which 
we call ``program'' that consists of either zero, one, or more definitions. The 
production for this rule is as follows:

\begin{ebnf}
\grule{program}{\grep{definition}}
\end{ebnf}

It is from this production each and every program is derived from. The ANT for 
this production is illustrated in \figref{ast:program}.


\begin{figure}[ht]
\begin{center}
\begin{tikzpicture}[level/.style={sibling distance=40mm/#1}]
\node [square] {Program}
  child {node [square,xshift=0.5cm] (a) {\textit{Definition}} edge from parent[dashed];}
  child {node [square,xshift=-0.5cm] (b) {\textit{Definition}} edge from parent[dashed];};
\path (a)--(b) node [midway] {$\cdots$};
\end{tikzpicture}
\end{center}
\capt{The abstract syntax tree for the program node.}
\label{ast:program}
\end{figure}


Figure \ref{ast:program} consists of one root which is called ``Program'' and
this root can have zero, one, or more children, called ``Definition''. The
children are optional because the production says that a program can consist of
either zero, one, or more of these definitions. We illustrate this choice by
making the connecting edges dashed from the parent node. Between these two child
nodes there are three dots ($\cdots$) which illustrate that it is possible to
have more of these nodes following each other.

This means that a program is legal if it does not contain anything at all.

\subsection{Expressions}
The following nine subsections present the different expressions in
\productname{}. The sections are structured as follows:

\begin{dlist}
  \item Operations with precedence and negation
  \item Element
  \item Member access
  \item Call sequence
  \item Assignment
  \item If expression
  \item Lambda expression
  \item List
  \item No-operator
\end{dlist}

\subsubsection{Operations with precedence and negation}
In this section we introduce the five different groupings of operations such as
logical operators, equality operators, etc., and finally we present the
production called negation. We illustrate the five groupings of operators but we
have omitted the negation node, because it is merely a root with one child. The
following grammar presents the productions for the different operations:

\begin{ebnf}
\grule{lo\_sequence}{eq\_sequence \gcat \grep{\ggrp{\gter{and} \gor \gter{or}} \gcat eq\_sequence}}
\grule{eq\_sequence}{cm\_sequence \gcat \grep{\ggrp{\gter{==} \gor \gter{!=}} \gcat cm\_sequence}}
\grule{cm\_sequence}{as\_sequence \gcat \grep{\ggrp{\gter{<} \gor \gter{>} \gor \gter{<=} \gor \gter{>=}} \gcat as\_sequence}}
\grule{as\_sequence}{md\_sequence \gcat \grep{\ggrp{\gter{+} \gor \gter{-}} \gcat md\_sequence}}
\grule{md\_sequence}{negation \gcat \grep{\ggrp{\gter{*} \gor \gter{/} \gor{\%}} \gcat negation}}
\end{ebnf}

The meaning of these different groupings of operators have been described in
\secref{sec:grammar}. To refresh the readers memory, the sequences are
intentionally placed in this specific order to ensure the correct precedence for
these operators.

Since the productions look a lot like each other we will only illustrate an
abstract ANT which shows how we have implemented the different operations.
Figure \ref{ast:operation} shows this.


\begin{figure}[ht]
\begin{center}
\begin{tikzpicture}[level/.style={sibling distance=30mm/#1}]
\node [square] {\textit{LHS}}
  child {node [square] {\textit{RHS}}}
  child {node [square] {\textit{RHS}}}
  child {node [square] (a) {\textit{RHS}} edge from parent[dashed];}
  child {node [square] (b) {\textit{RHS}} edge from parent[dashed];};
  
\path (a)--(b) node [midway] {$\cdots$};
\end{tikzpicture}
\end{center}
\end{figure}


Figure \ref{ast:operation} illustrates that the ANT for each production, with
the left-hand side (LHS) of the production as the root, will only be constructed
if there is at least two of the right-hand sides (RHS) with one operator between
them. This means that the sequence should for instance look like the following:

\begin{ebnf}
\grule{lo\_sequence}{eq\_sequence \gcat \gter{and} \gcat eq\_sequence}
\end{ebnf}

So, in the above example the RHS is $eq\_sequence$ and the operator is
$\gter{and}$. In \figref{ast:operation} we do not show the operators, which can
make it a bit cryptic to look at and understand. If there is only one RHS then
the node is not constructed but the next production will be evaluated. Otherwise
the AST would end up with many single-child nodes. This bad implementation is 
illustrated in \figref{ast:badexample}.


\begin{figure}[ht]
\begin{center}
\begin{tikzpicture}[level 3/.style={sibling distance = 20mm}]

\node [square] {\textbf{LHS}}
  child {node [square,yshift=0.5cm] {\textbf{RHS}}
    child {node [square,yshift=0.5cm] {\textbf{RHS}}
      child {node [square,yshift=0.5cm] {\textbf{RHS}}}
      child {node [square,yshift=0.5cm] (a) {\textbf{RHS}} edge from parent[dashed]}
      child {node [square,yshift=0.5cm] (b) {\textbf{RHS}} edge from parent[dashed]}
    }
  }; 

\path (a)--(b) node [midway] {$\cdots$};
\end{tikzpicture}
\end{center}
\capt{An example of why we need at least two RHS before we construct a node.}
\label{ast:badexample}
\end{figure}


Figure \ref{ast:badexample} illustrates very clearly that the AST quickly would
end up with mant single child nodes. If we do not expect two RHS then We will 
end up with a long list, which is not necessary and it will just make it less 
efficient to read the AST. With our implementation we will have a more efficient 
and more compact AST.

The grammar specified earlier in this section presented that the last operation
($md\_sequence$) consists of negations and the choice to add an operator between
negations. The negation production can be an element or begin with the \gter{-}
symbol followed by another negation.
The following productions present the grammar for the negation expression:

\begin{ebnf}
\grule{negation}{element}
\galt{\gter{-} \gcat negation}
\end{ebnf}

This production is not illustrated with any ANT.

\subsubsection{Assignment}
The following grammar rule specifies the production of an assignment in
\productname{}:

\begin{ebnf}
\grule{assignment}{\gter{let} \gcat variable \gcat \gter{=} \gcat expression \gcat \grep{\gter{,} \gcat variable \gcat \gter{=} \gcat expression} \gcat \gter{in} \gcat expression}
\end{ebnf}

The production specifies that any assignment must begin with the keyword
\gter{let} and end with the keyword \gter{in} followed by an expression. In
between these beginning and ending keywords, the production consists of at least
one sequence of a variable followed by an assignment-operator followed by an
expression. The production specifies that it is possible to have zero, one, or
more of these variable-expression pairs (comma separated) following the first
pair.


\begin{figure}
\begin{center}
\begin{tikzpicture}
[level/.style={sibling distance=40mm},
level 1/.style={sibling distance = 39mm},
level 2/.style={sibling distance = 20mm}]

\node [square] (z) {Assignment}
  child {node [square,left of=b,xshift=-4cm] (a) {Variable}}
  child {node [ellipse,draw,left of=c,xshift=-4.5cm] (b) {\textit{Expression}}}
  child {node [square] (c) {Assignment} edge from parent[dashed]
  	child {node [square,xshift=-1cm] (x) {Variable} edge from parent[solid]}
  	child {node [ellipse,draw,solid,xshift=-1cm] (y) {\textit{Expression}} edge from parent[solid]}
  }
  child {node [square,xshift=-1cm] (d) {Assignment} edge from parent[dashed]
  	child {node [square,xshift=1cm] (o) {Variable} edge from parent[solid]}
  	child {node [ellipse,draw,solid,xshift=1cm] (p) {\textit{Expression}} edge from parent[solid]}
  }
  child {node [ellipse,draw,right of=d,xshift=1.5cm](e) {\textit{Expression}}};

\path (c)--(d) node [midway] {$\cdots$};
\end{tikzpicture}
\end{center}
\capt{The abstract syntax for the assignment node.}
\label{ast:assignment}
\end{figure}


Figure \ref{ast:assignment} illustrates this with an ANT that omits the
keywords, commas and the assignment-operators, which can make it rather complex
to look at. The figure actually states that an assignment consists of variables
and expression where an expression can be many different things, including
another assignment. So, this means that it is possible to have assignments
nested within each other. 

We have chosen to implement the comma separated nodes as new assignment nodes
which has two nodes that are not optioanl. These nested assignments are
connected to the parent with dashed edges which mean that they are optional and
it is possible to have zero, one, or more of these following each other.

\section{Contextual constraints}
\label{sec:staticsemantics}

The \classref{ScopeChecker} is the class responsible for enforcing some of the
static semantic rules of \productname{} (described in \chapref{chap:design}) at
compile time. This section aims to explain how these static semantic checks
are performed by the \classref{ScopeChecker} in simple sequential steps. Any
error detected by the \classref{ScopeChecker} will cause a \classref{ScopeError}
exception to be thrown. This error contains helpful information about the type
of error and where in the input program the error is located. The checking of
static semantics as well as the interpretation of the code both use the visitor
pattern which is a commonly used approach for both purposes.

After the AST has been created, the visitor pattern allows us to traverse the
AST and execute encapsulated pieces of code for each specific type of
\classref{AstNode}. 

\subsection{TypeVisitor}
The first visitor used is the \classref{TypeVisitor}. This visitor traverses the
AST, finds all type definitions in the input program and for each type
definition an object of class \classref{TypeSymbolInfo} is instantiated. After
running the \classref{TypeVisitor}, each type definition in the input program
has an associated object of class \classref{TypeSymbolInfo} which makes it easy
to get information about any declared type in the input program by accessing the
following members contained in the \classref{TypeSymbolInfo} object:

\begin{dlist}
  \item name (\classref{String}): The type's name
  \item parentName (\classref{String}): The name of its super type (null if
    not a derived type)  
  \item args (\classref{Integer}): The number of arguments in the type's
    constructor
  \item parentArgs (\classref{Integer}): The number of arguments given in the
    call to its parent constructor
  \item data (\classref{List of Data}): Each data defined in the type body has a
    corresponding \classref{Data} object describing its name and position in the
    input program
  \begin{dlist}
    \item The input program position is stored as a line and an offset and makes
      it possible to produce error messages with information about where in the
      input program an error was found
  \end{dlist}
  \item members (\classref{List of Member}): Each constant and function defined
    in the type body has a corresponding \classref{Member} object describing its
    name, argument count, an abstract flag, a \classref{TypeSymbolInfo}
    reference to the type defining the \classref{Member} and a pointer to the line
    and offset in the code
  \item node (\classref{AstNode}): A reference to the
    TYPE\_DEF-\classref{AstNode} which defines the type
  \begin{dlist}
    \item This reference is used to get the input program position where the
      type was defined (for generating useful error messages), but is also used
      for marking abstract type definitions (described in detail in
      \secref{sec:abstractTypeMarker}) 
  \end{dlist}
  \item parent (\classref{TypeSymbolInfo}): This will contain an object
    reference to its parent type if it has one
  \begin{dlist}
    \item This is however a null pointer until running the
      \classref{TypeParentRefMaker} described in \secref{sec:TypeParentRefMaker}
  \end{dlist} 
\end{dlist}

All the \classref{TypeSymbolInfo} objects are kept in an object of class
\classref{TypeTable}. The \classref{TypeTable} class is a layer of abstraction
which provides easy and fast access information about the types contained in the
input program. The underlying implementation is a hashmap from the type's name
as a \classref{String} to its object reference, which provides a quick way to map a type's name
to the \classref{TypeSymbolInfo} object that represents the type. One use of this mapping is 
described in \secref{sec:TypeParentRefMaker}. Another feature of \classref{TypeTable} is a 
convenient way to iterate over all the \classref{TypeSymbolInfo}s. This is for instance used
to topological sort the types, which is described in \secref{sec:TypeMemberPropagator}.
For convenience, we say that a type is added to a type table which means that a
\classref{TypeSymbolInfo} object representing the type is added to the
\classref{TypeTable} object representing the type table.

When a type is added to the type table, it is checked that no other types with
the same name exist.

\subsection{TypeParentRefMaker}
\label{sec:TypeParentRefMaker}
The \classref{TypeSymbolInfo} objects only contain the name of their super type
as a \classref{String} or a null value if there is no super type. By making a
lookup in the \classref{TypeTable} on the parent name, the real object
references can be found and stored for faster and more convenient parent lookups
which is used greatly by the visitor described in 
\secref{sec:usesaredeclaredvisitor}.

\subsection{TypeMemberPropagator}
\label{sec:TypeMemberPropagator}
Some of the later checks that will be performed requires us to determine whether
or not a type member (constant or function) with a specific name is visible in a
given type. This requires searching in the given type and recursively in all
super types for the member. When this kind of lookup is done many times on the
same member, the traversal of the same long chains of parent references become
inefficient which results in clumsy code. To simplify and speed up this process
the \classref{TypeMemberPropagator} ensures that all members of a type A are
also present in a type C, if A is a super type of C. With this approach,
checking if a member is visible in type C, only requires looking in C instead of
following the chain parent types.

This propagation of members is done by first doing a topological sort on the
\classref{TypeSymbolInfo} objects, such that when iterating over the type table,
any type yielded will always appear before all of its subtypes. This makes the
afterwards propagation of members possible in linear time by iterating over the
topologically sorted types. If a type C is met, the members in its parent type B
are put in C as well. If B has a parent A, we know that B has already the
members from A due to the topological sorting.

The topological sorting is done using the algorithm that goes by the same name, from the book Introduction
to Algorithms \cite[p. 612]{ad} working on a graph $G = (V, E)$. Each type
represents a vertex $v$. An edge $e$ exist from $v_1$ to $v_2$ if the type $v_2$
is a parent of the type $v_1$. However this results in a topological order where
a type always appears before its super types. This problem is quickly solved
simply by reversing the list.

Given the graph in \figref{fig:topological}, the following sequences are examples
of correct and wrong topologically sorted orders after the list has been reversed:

\begin{align*}
 Correct &: \texttt{a, d, b, e, c, f} \\
 Correct &: \texttt{a, b, e, c, f, d} \\
 Wrong &: \texttt{a, b, c, \textbf{f}, \textbf{e}, d} \\
 Wrong &: \texttt{\textbf{b}, c, e, \textbf{a}, f, d}
\end{align*}


\begin{figure}[ht]
  \begin{center}
    \begin{tikzpicture}[level/.style={sibling distance=30mm/#1}]
      \node [square] (c) {C};
      \node [square, yshift=-4em, xshift=2.5em] (b) {B};
      \node [square, yshift=-4em, xshift=7.5em] (d) {D};
      \node [square, yshift=-8em, xshift=5em] (a) {A};
      \node [square, yshift=-2em, xshift=12em] (f) {F};
      \node [square, yshift=-6em, xshift=12em] (e) {E};

      \draw[->, thick,] (c) -- (b);
      \draw[->, thick,] (b) -- (a);
      \draw[->, thick,] (d) -- (a);
      \draw[->, thick,] (f) -- (e);
    \end{tikzpicture}
  \end{center}
  \capt{Example of topologically sorted types. An edge goes from type X to type Y if X is a subtype of Y.}
  \label{fig:topological}
\end{figure}



Another great advantage from topological sorting is the fact that it reveals
cycles in the graph. A cycle in the graph means an extend cycle between types
exists, e.g: A extends B, B extends C and C extends A, which is not accepted.

\subsection{AbstractTypeMarker}
\label{sec:abstractTypeMarker}
The interpreter needs to know whether or not a given type contains any abstract
members. Such a type is an abstract type and should not be allowed to be
instantiated. Marking these abstract types is now smooth. Due to the propagated
members it can just be checked whether or not any abstract members are present in the
given type. This check is done by the \classref{AbstractTypeMarker}. Any type
described by a \classref{TypeSymbolInfo} has a reference to the
\classref{AstNode} it was defined from. If the type is found to be an abstract
type, the type of the \classref{AstNode} is changed from TYPE\_DEF to
ABSTRACT\_TYPE\_DEF. This is the only way to make information visible to the
interpreter since the \classref{Interpreter} does not use the same
\classref{TypeTable} class used by the \classref{ScopeChecker}.

\subsection{TypeSuperCallChecker}
This checker ensures that any type that extends another type provides the right
amount of arguments when calling the parent's constructor. A constructor can have
$x$ arguments and may or may not contain a variable amount of additional
arguments. Consider the type constructor \texttt{Type A[\$var1, \$var2, \dots
\$varargs]}. When calling the constructor from another type, e.g. \texttt{Type
B[] extends A[5, 2, 7, 4]}, it must be checked that the type B provides \textit{at
least} the number of arguments in A's constructor (not counting the variable
amount of extra arguments). If A does not have a variable amount of additional
arguments, the argument count must match exactly. The implemented code for doing
this check can be seen in \lstref{lst:tscc}.

\lstinputlisting[caption={\emph{How the TypeSuperCallChecker is implemented.}},
label=lst:tscc, language=Java]{listings/typeSuperCallChecker.java}

\subsection{UsesAreDeclaredVisitor}
\label{sec:usesaredeclaredvisitor}
This visitor ensures that any use of a variable, constant, function, data
member, or type can be bound to a declaration. The visitor uses a variable
(\classref{TypeSymbolInfo} \varref{currentType}) which updates upon visiting a
TYPE\_DEF or an ABSTRACT\_TYPE\_DEF \classref{AstNode}, to keep track of which
type it is currently visiting inside. If the visitor is not traversing inside a
type (\classref{TypeSymbolInfo} \varref{currentType}) references a special type
called \varref{globalType}, which is used only to contain the standard- and game
environment as well as the global constants and functions declared in a
\productname{} game. 

It is important to realise that \varref{globalType} is not a super type of all
other types, it is a stand alone type that no type can derive from. Its name
contains an invalid character for a type name to ensure that no type can derive
from it. This becomes handy when checking if constants and functions used can be
bound to a declaration.

\subsubsection{Constants and functions}
When a constant or a function is referenced it is necessary to know two things
about the context in which it was referenced:

\begin{nlist}
  \item In what type did the reference occur?
  \item Is the reference a member access?
\end{nlist}

The first thing is easy to check since we have the \varref{currentType}
variable. This variable may however point to the global type. The structure of
the AST makes it easy to determine if it was a member access, since we would
have been visiting a MEMBER\_ACCESS \classref{AstNode} prior to the referenced
constant or function. In the expression: \texttt{A[].B.C[2,3]}, both B and C are
member accesses, but A is not. Given this information, a different check can be
done regarding to the context of the reference:

\begin{nlist}
  \item Type was global and a member access
  \begin{dlist}
    \item Must be visible in at least one type
  \end{dlist} 
  \item Type was global but not a member access
  \begin{dlist}
    \item Must be visible in the global scope
  \end{dlist}
  \item Type was A and a member access
  \begin{dlist}
    \item If prefixed by this, it must be visible in A or any super type of A
    \item If prefixed by super, it must be visible in any super type of A
    \item If prefixed by a variable name, it must be visible in at least one type
  \end{dlist}
  \item Type was A but not a member access: Must be visible in A, a super type
    of A or global scope
\end{nlist}

One may wonder why an accessed member is accepted if the accessed member is
visible in at least one type. Consider the member access \texttt{randomType.B}.
Here it is unknown in what type we shall look for the member B. The constant
\texttt{randomType} could literally return a random type, or the type returned
could be determined by an arbitrary complex algorithm. Therefore, we can only
enforce the rule that the member \texttt{B} must exist in at least one type.

%Skal nedenstående  afsnit med?

%One may think that it is also nice to know if a referenced constant or function
%has a number of parameters along with it and whether the actual number of
%arguments correctly matches the formal number of arguments. This is however
%quite hard to determine. In the example, if \texttt{randomConstant} was declared
%as a constant, the expression \texttt{randomConstant[2]} would still make sense if
%the constant returned a list, in which 2 was an index. This is however
%something the scope checker cannot look into. Given a function declared as
%\texttt{randomFunction[\$a, \$b] = \dots} the expression \texttt{let \$var =
%randomFunction in \dots} would also be correct, in which case \texttt{\$var} is just
%a reference to \texttt{randomFunction} in the \texttt{in}-scope. So it is valid to
%use a constant followed by a parameter list as well as it is valid to not apply
%a parameter list behind a function. 

\subsubsection{Variables}
For any variable, a declaration must always exist before it is used. A variable
can only declared in four ways:

\begin{dlist}
  \item As a type constructor
  \item As a formal parameter in a function declaration
  \item In a lambda expression
  \item In a \texttt{let-in} expression
\end{dlist}

In all cases the \productname{} semantics require that a new scope is opened, in
which the declared variable is known while the body of the expression is
executed. When the scope closes the declared variables are removed. The body of
an expression can also contain new variable declarations, e.g. a \texttt{let-in}
expression in the body of a \texttt{let-in} expression. 

The scope checker uses a \classref{SymbolTable} class which is basically a
symbol table with a list of variable names and a reference to a parent symbol
table.  The reference to the parent symbol table is exactly how the scopes
inside other scopes are implemented. 

\codesample{openscopeexpressions.junta}

Notice how the four code samples in in the above code sample all result in the same
scope checking routine, which can be seen in \figref{fig:scope1}; First, a new
symbol table is instantiated in which the variables \$a and \$b are put in. The
symbol table's parent reference is updated so it points to the current symbol
table, which is referred by \classref{SymbolTable} \varref{currentST}. Next, the
current symbol table is updated to the newly created symbol table, and the body
(the triple dots) are executed. Lastly, the current scope is closed, which
updates the current symbol table reference to point to the parent symbol table
of the current symbol table.  Notice that the symbol tables maintain a
stack-like structure, where opening a scope pushes a symbol table on the stack
and closing a scope pops one. The variable \classref{SymbolTable}
\varref{currentST} points to the element on top of the stack.

When a variable is used, it is checked that the variable exists in any of the
symbol tables by first looking in the current symbol table and recursively
following the parent reference until a null reference is found. If a variable
declaration with the same name as the used variable cannot be found, an error is
generated.

\fig[height=5em]{scope1}{Four different expressions that all result in the scope
action depicted.}

It is important to realise the reason for maintaining the stack-like structure
of symbol tables. It might seem like a single symbol table would be enough and
that all variable declarations could just be put in there. This is indeed wrong,
since the scope checker must also check for double declarations. A double
declaration exists if a symbol table contains the same variable twice. Notice
how \figref{fig:scope2} contains two symbol tables, each containing a
declaration of \$a. 

\fig[height=5em]{scope2}{The variable \$a declared in two different scopes.}

This is completely valid and is caused by the following code sample. If only a
single symbol table was used, an incorrect double declaration would be detected.

\codesample{scope2.junta}

\subsubsection{Data members}
When visiting a type body a new scope is opened and the data members of
\classref{TypeSymbolInfo} \varref{currentType} are immediately inserted into that
scope. The children of the type body is then visited and the scope is closed.
This ensures that the data members of a type can be used anywhere in the type
body but in that type body only. When exiting that type body and closing the
scope the symbol table containing the data members are no longer visible.

\subsection{Summary of scope checking}
Many different static semantic checks are implemented in the scope checker.
Though many other checks could have been included as well, the scope of the
static semantics has been limited due to a few constraints. First of all, there
is a deadline for this project, and with an almost endless set of semantic
checks one can keep developing these checks. Furthermore, with new techniques
being discovered once in a while, a compiler or interpreter can simply not
include them all. A big set of the checks not included in \productname{}
requires type checking, which is cumbersome in a dynamic programming language.
However, it is generally possible to use type inference to find at least some of
the types and errors associated with the use of them. It is important to realise
that everything cannot always be inferred, for instance an algorithm could be so
complex that it would need to be executed to determine all possible outcomes.
Running the algorithm is not possible since you cannot know if the algorithm
will ever halt.
\cite[p. 173]{itttoc}

\section{Interpreter}
The final step is the interpretation of the AST generated by the parser.
Here choices are taken based on the different node types in the tree
for a given program. This interpreter class is, like the scope checker,
also implemented with the visitor pattern, visiting all the nodes and
taking appropriate actions depending on what type of node is visited.
The difference here is that values, types, and other language constructs
are created and evaluated directly, propagating the results up the AST,
and that every type of node is visited, as each node is significant in
that they carry important information about the written program.

The implementation of the interpreter consists of $40$ visitor methods,
excluding private helper methods, each providing a specific behaviour
depending on the type of node visited, possibly calling other visit
methods on the node to evaluate to get a \classref{Value}.

In this section, we will look at the inner workings of the interpreter
and see some examples of the different evaluation methods used and how
they're created. We also see how the symbol table keeps track of scopes
and the different values. Thereafter the implementation of patterns is
described, as they play a central role in most programs written in our
language.

% To add (maybe a section here about how the interpreter behaves?):
% The game env is loaded on startup
% Methods are created for each type of node, possibly visiting other nodes and passing values around

\subsection{Symbol table and scopes}
Symbol tables in the interpreter are used to add and get constants,
variables, types, and to push and pop scopes. The interpreter keeps a
single, global instance of a root symbol table which it calls methods
upon when needed. This symbol table is unaffiliated with the scope
checker's symbol table, which is built differently up, and hence does
not share any information with it at all.

As an example of how the symbol table is operated, an example of
visiting an assignment (let-in) node, which utilizes some of these
features, is shown below:

\lstinputlisting[caption={\emph{Code taken from the interpreter to
show how the symbol table is used when visiting let--expressions.}},
label={lst:let}, language=Java]{listings/visitAssignment.java}

Listing \ref{lst:let} shows how ``let-in'' expressions are evaluated
when the abstract node type shown in \secref{sec:letexpressions} is
visited. A new scope is pushed onto the current stack of scopes,
whereafter the variables are hereafter pushed into the scope that was
just opened. Because let-in expressions, like all \productname{}'s
expressions, return a \classref{Value}, this value is retrieved after
the body of the assignment is computed but before the scope is closed,
as we will most likely need the values of the different variables.

\subsection{Values and their operators}
Each one of the base values offered by \productname{} is represented
internally by a class written in the interpreter. These classes are
all sub-classes of a general class \classref{Value}, that offers the
sub-classes an interface to implement various different operations, such
as comparison, addition, calling, and so on, throwing an error if trying
to use the operator between incompatible values or if the operation
yields an invalid result.

In the following part of this section, a few important values and their
features are highlighted. The most basic values such as integers,
strings, directions, coordinates, and lists principally support the
same operations, and are therefore relatively uninteresting to discuss
compared to the more complex values, such as types and functions. For
completeness, the implementation of coordinates, types, and
functions are discussed below.

Most properties (variables) of values are declared as Java
\classref{final}, due to the fact that \productname{} is functional and
that the result of an expression always results in a new instance of an
object. This means that a simple plus operation yields a new value, so
the interpreter constantly instantiates these new values without ever
needing to update a specific property, possibly leaving many young, dead
objects on the heap for Java's garbage collection to deallocate.

\subsubsection{Coordinates}
Coordinates are represented internally by a tuple $(x, y)$ specifying
a coordinate on the game board. The maximum board size is limited by
an $x$ and $y$, which are both 32-bit signed two's compliment Java
integers\footnote{Which gives a maximum coordinate of $(2,147,483,647;
2,147,483,647)$, much more than any realistically imaginable board
game!}. If we wished to represent larger boards, we could simply use
a built-in larger natural number representation, or create a custom
representation.

As specified in \secref{sec:standardenvironment}, coordinates consist of
a horizontal axis represented by letter, or multiple letters, $x$, and a
numerical $y$, representing the vertical squares on a grid-shaped board.
When displaying a coordinate to the user, a simple method is called to
convert the $x$ coordinate to its alphabetical form.

Coordinate values support equals comparison, done by value. It
also allows addition with strings, directions, and lists. Subtracting
other coordinates yields a direction vector and subtracting a direction
yields a new coordinate with an offset defined by subtracting the
coordinates $x$ and $y$ properties with the direction's $x$ and $y$
properties, as seen in the following listing.

\lstinputlisting[caption={\emph{How subtraction of other values on \classref{Coordinate} is handled.}}, label={lst:coordSubtract}, language=Java]{listings/coordSubtract.java}

Listing \ref{lst:coordSubtract} shows the implementation of the
subtraction operation on coordinates. This method is principally the
same implementation for every mathematical operation across all the
different above-mentioned sub-classes of \classref{Value}. Type checking
is always done to see if the right-hand side of the operation is
compatible before taking the correct actions. The method \methodref{is}
on all \classref{Value} types checks to see if the type is the specified
type or any sub-type extending the type in \productname{}. If the RHS
is allowed, it's up-casted to the appropriate type if it isn't used
directly, whereafter the appropriate operation is performed between the
two types.

%\subsubsection{Lists}
%Lists are implemented very much like Coordinates, supporting comparison,
%addition with other lists and any singular \classref{Value}-type,
%subtraction also with other lists and singular \classref{Value}-types
%existing in the current \classref{List} being subtracted from.
%
%The most interesting feature of \classref{List} types is that they take
%an optional list of one to two parameters, allowing access to an element
%at a specific index in the list, or defining a range of indexes desired
%in the list, resulting in a new list containing the elements between the
%two indexes.


\subsubsection{Types}
\classref{Type} values are one of the more complex sub-classes of
\classref{Value}, as they, when defined, can have formal parameters, a
parent \classref{Type}, and a complex body. These are all represented
and stored internally in a class \classref{Type}, instantiated when
visiting a type definition node (defined in \apref{ap:typedef}).

% Declaring
There are a handful of different ways to instantiate a \classref{Type},
all depending on the contents of the type definition node visited.
If a type is declared to have a parent, a \classref{Value} of type
\classref{Type} is simply instantiated by the interpreter with the
appropriate constructor, saving the nodes for the parent and the parents
formal parameters. If the type declaration also has a body, then each
declaration is added to the newly instantiated type by the interpreter.
Finally, the type is added to the symbol table, so it can be called and
instantiated at a later point in the program.

% Instantiating
Types are implemented to allow instantiating the type, returning an
\classref{ObjectValue} sub-class with the actual parameters bound to the
formal parameters, scope bindings, and a reference to the interpreter
so it can evaluate the methods in its body (if one exists) when called.
This instantiating is done when the interpreter visits an actual type,
where it looks up the type's name in the symbol table, returning it if
found, or an error if not.

% Abstract types
Abstract type values are an extension of \classref{Type}, but only
consisting of abstract members, and do not allow instantiating, throwing
a \classref{TypeError} if attempted.

% Quick mini-summary
Unfortunately the methods that make up the interface to the
\classref{Type} class are too large to be shown in this section,
therefore we refer to the \classref{TypeValue} class in the source code
for a quick overview. The body of types can consist of constants and
functions, where the implementation of functions is described blow.

\subsubsection{Functions}
Functions are yet another sub-class of the \classref{Value} type,
though only supporting the addition operation to store the function in
a \classref{ListValue} (letting functions be first class citizens).
Functions are represented internally as constants, and can therefore be
defined in the global scope or within any sub-scope that may exist, each
updating the symbol table's current scope's constant declarations.

% Creating function values 
Functions are created with the AST node expression body, which is
stored and not evaluated until the function is called. They do not keep
track of their names, as that's the symbol table's job (allowing lambda
expressions to also have the \classref{FunValue} type), so functions
merely incorporate formal parameters and an expression. Functions also
allow the ability to have a variable list of formal parameters. If the
function is a lambda expression, then the current scope is passed to the
\classref{FunValue} when instantiating. On the other hand, if a function
is defined with a name identifying it, the interpreter adds it to the
symbol table's constant table, instantiating a \classref{FunValue} with
the AST node expression and AST node formal parameters.

% Calling a function
When a function is called, it is called with a reference to the
interpreter and the actual parameters passed to the function. The
interpreter is needed to evaluate the body, and is used for access
to the symbol table for scopes, adding variables, and so forth.
For optimization, an error is thrown if the length of the actual
parameters does not match the length of the formal parameters, already
short-circuiting the call if attempted. Else the function body is
evaluated with the actual parameters in a new, temporary scope that only
exists while the call is running. This scope has the previous scope as a
parent, giving it complete access to all the existing, declared members
outside the function's body inside the function's body.

% Tail recursion detection
Special action is taken during the call when tail recursion is
detected. An example of a tail recursive function, a small method
\methodref{recsum} that makes use of it, is seen below:

\todo{Pygmentize below (commented out) code sample}
%\codesample{tailrecursion.junta}

This simple function adds the first $N$ integers, where $N$ is defined as the first parameter to the function. The tail recursion optimizer converts the recursive function into a loop if it detects a trail recursion construction, guaranteeing stack overflow prevention when using tail recursion. How tail recursion is detected by the function when it is called is shown by the pseudocode below:

\begin{lstlisting}[language=Pascal,label=alg:tailrecursion, caption={Detecting tail recursion during call of a function.}]
  if 'second time we see this function' then
    return new CallValue(interpreter, this, actualParam)

  body := 'the AST node body of this function'
  openScope()

  recursion := false
  repeat
    For i := 0 to 'formal parameter size' n do
       currentScope.addVariable(formalParam.i, actualParam.i)
    End

    evaluatedBody := visit(body)

    if evaluatedBody 'is an instance of CallValue' and evaluatedBody.getFunction() = this then
      recursion := true
      actualParameters := evaluatedBody.getParameters()
    else
      recursion := false
  until recursion = false 
closeScope()
return evaluatedBody
\end{lstlisting}

%\begin{algorithm}[H]
%\DontPrintSemicolon
%recursion $\leftarrow$ false\;
%body $\leftarrow$ this function's AST node body\;
%openScope()\;
%
%\Repeat{recursion $=$ false}{
%  \For{$i = 0$ \KwTo formal parameter size $n$}{
%    curentScope.addVariable(formalParam $i \mapsto$ actualParam $i$)\;
%  }
%  
%  evaluatedBody $\leftarrow$ visit(body)\tcc*[r]{returns a value}
%  
%  \eIf{evaluatedBody is a callValue and body.getFunction() $=$ this function}{
%    recursion $\leftarrow$ true\;
%    actualParameters $\leftarrow$ evaluatedBody.getParameters()\;
%  }{
%    recursion $\leftarrow$ false\;
%  }
%}
%closeScope()\;
%\Return evaluatedBody\;
%\capt{Detecting tail recursion during call of a function.}
%\label{alg:tailrecursion}
%\end{algorithm}


Here it is important to note that the type \classref{CallValue} is an internal \classref{Value}
\todo{Finish this\ldots}

\subsection{Pattern evaluation}
Patterns in \productname{} are a very important feature, being used
in pretty much every one of our test programs to evaluate winning
conditions, find squares or actions that have a specific property, and
various other constructs.

\subsubsection{Failed implementation with finite automaton}
We looked at other languages in an attempt to find an existing solution
to a similar problem and quickly found that patterns should be very
similar to regular expressions. Unfortunately, regular expressions
actually do not share the same properties as our patterns, ending up
giving us quite a headache when trying to use automaton to describe
patterns. One of the primary troubles we had using automatons was the fact 
that automatons takes an input stream and decide whether the input is to be accepted 
or rejected. For a pattern check, an input stream can be seen as a path 
between two squares on the board. For a person who plays a \productname{}-game, it is however 
not convenient that he has to specify the exact path he wants to move. Consider trying to move the 
knight from B1 to C3 in a game of chess. Such a move could be chosen by 
sequentially clicking the 4 squares  B1, B2, B3, C3. This inconvenient approach 
would also lack the support of clicking a piece and have the squares it can 
move to highlighted. An automaton would be useful if we had decided to use backtracking 
to feed it with the set of all possible input streams. However, this approach is 
not serious, because of the infinite possibilities of input streams. 

\subsubsection{Actual implementation}
The idea of a pattern-match is explained in \secref{sec:patterns}. In the implementation, a pattern matching method takes a pattern, a game object, and a position as input and then return either true or false, depending on how the pattern complies with the game objects (board, squares and pieces) relative to the input position.

To aid the understanding of how pattern-matching is implemented, this small example is introduced before the full implementation: 
When evaluating a pattern, a single branch is created containing the input coordinate. A branch can be seen as a thing that ``digests'' the pattern sequentially and moves around on the board accordingly. The branch is the thing verifying that the pattern complies with board set up. For instance, given the pattern \texttt{/n | w empty/} and the start position B2, one branch starts on B2, meets the \texttt{|}-operator, and then splits into two branches. One branch moves one square north, the other one square west. The two branches are then united to a single branch containing the set of coordinates \{B3, A2\} The next digested part of the pattern is the \texttt{empty}-keyword. The \texttt{empty}-keyword causes all coordinates in the current branch to be removed if the square on the board on that coordinate is not empty. The pattern is now fully digested, and if the current branch still contains any coordinates, it means that it was somehow possible to make a route from the starting position that fully complied with the patter, thus the pattern match returns true. If the branch did not contain any coordinates after digesting the entire pattern, this means that it is in no way possible to choose a branch which complied with the pattern, this the pattern match returns false.

In our full implementation of how a pattern is matched, these notations are used:

The notations used:
\begin{dlist}
\item $B$: a branch containing a set of coordinates.
\item $p$: a pattern.
\item $c$: a coordinate.
\item $d$: a direction, e.g. n (\textit{north}) or sw (\textit{south west}).
\item clone($p$): returns an exact clone of p.
\item evaluate($p$, $B$): modifies the branch $B$ in a way depending on the value of $p$.
\item union($B_1$, $B_2$): adds all coordinates in $B_2$ to $B_1$ if they were not already contained in $B_1$. 
\item concat($p$, c): returns $p_1 p_2 ... p_c$, where $p_i = p$.
\end{dlist}

The full implementation of a pattern match is explained now. When checking if a pattern $p$ matches on a coordinate $c$ given the game object $\text{game}$ this is what happens:
A new branch $B$ is created, containing the input coordinate $c$. 
evaluate($p$, $B$) is called, and when the function returns, it is checked that if $B$ contains any coordinates. If it does the pattern check returns true, and false otherwise.

The evaluate($p$, $B$) function depends on the composition of $p$, the pseudo-code for its implementation can be seen by \lstref{lst:patternalg}.

\lstinputlisting[caption={\emph{Pseudo-code of how a pattern is evaluated on a branch}},
label={lst:patternalg}, language=Java]{listings/patternalg.pseudo}

\subsubsection{Visualisation of pattern-algorithm}
\label{sec:vispatternalg}

\fig[scale=2]{patternboard}{A possible board set up.}
\fig[scale=0.35]{patternalg1}{The algorithm trying to match the pattern \texttt{/s? w* empty n !empty/} on the square D3, using the board set up in \figref{fig:patternboard}.}
\fig[scale=0.35]{patternalg2}{The algorithm trying to match the pattern \texttt{/n|w n|w !empty/} on the square C2, using the board set up in \figref{fig:patternboard}. The two branches created when meeting a \texttt{|}-operator are merged again. Notice how this increases efficiency, as the two red coordinates (B3) are merged into a single branch before trying to match the successive pattern values.}

\subsection{Evaluation of interpreter}
We have implemented a complete interpreter with semantics for every
single node type, allowing us to explore and demonstrate every feature
of \productname{} without many limits.

Under development, we have kept flexibility in mind, allowing us to
easily add new base values and extend and maintain the feature set and
semantics of our language, without having to change any existing code.
If we wanted to add an addition operation between coordinates, it's only
a matter of adding a few lines of code to the coordinate value class.

It is obviously not the fastest implementation of an interpreter, but we
don't see that as a hindrance, because speed is not what we're after.

There are a few optimizations that can be taken into account when
writing an interpreter to make it run faster, such as detecting tail
recursion. Since \productname{} has no loop-constructs, recursive
functions are the only way to repeatedly run through some code.

\section{Error handling}
\label{sec:errorhandling}

In the different phases of the interpreter many different errors can occur.
In the scanning phase for instance, a scanning error can occur and be
thrown if the scanner detects a string, which doesn't correspond to one
of \productname{}'s specified lexemes. In the parsing phase a syntax
error can occur if two or more tokens are not set up syntactically
correct according to our grammar. In order to catch these errors we
have implemented error handlers, which extends the built-in Java class
\classref{Exception} (\classref{Java.lang.Exception}).

An error can cause serious flaws and is annoying for a programmer if no
information of what caused it is provided. The function of the error
handlers are to give the programmer of \productname{} games a better chance of
figuring out what has been done wrong and make it easier to correct the errors
by, for instance, referring to where in the source code an error has occurred and
what caused it. We have created an error hierarchy in order to give an overview
of our different error handlers. The hierarchy is illustrated in
\figref{ast:errorhierarchy}. 

\begin{figure}[ht]
  \begin{center}
    \begin{tikzpicture}
      \node [square,yshift=-2em] (a) {Exception};
      \node [square,yshift=-5em] (b) {Error};
      \node [square,yshift=-8em,xshift=-7em] (c) {SyntaxError};
      \node [square,yshift=-8em] (d) {StandardError};
      \node [square,yshift=-8em,xshift=7.5em] (e) {SimulatorError};
      \node [square,yshift=-12em,xshift=-21em] (f) {ScannerError};
      \node [square,yshift=-12em,xshift=-14em] (g) {ScopeError};
      \node [square,yshift=-12em,xshift=-7em] (h) {InternalError};
      \node [square,yshift=-12em] (i) {TypeError};
      \node [square,yshift=-12em,xshift=7em] (j) {ArgumentError};
      \node [square,yshift=-12em,xshift=14em] (k) {NameError};
      \node [square,yshift=-15em,xshift=7em] (l) {DivideByZeroError};

       \draw[<-, thin,] (a) -- (b);
       \draw[<-, thin,] (b) -- (d);
       \draw[-|,-,thin,] (c.north) |-+(0,0.75em)-| (b.south);
       \draw[-|,-,thin,] (e.north) |-+(0,0.86em)-| (b.south);
       \draw[-|,->,thin,] (f.north) |-+(0,1.7em)-| (c.south);
       \draw[<-,thin,] (d) -- (i);
       \draw[-|,-,thin,] (g.north) |-+(0,1.0em)-| (d.south);
       \draw[-|,-,thin,] (h.north) |-+(0,1.11em)-| (d.south);
       \draw[-|,-,thin,] (k.north) |-+(0,1.11em)-| (d.south);
       \draw[-|,-,thin,] (j.north) |-+(0,0.98em)-| (d.south);
       \draw[<-, thin,] (j) -- (l);
    \end{tikzpicture}
  \end{center}
  \capt{An illustration of the error hierarchy as it is implemented in the
  interpreter.}
  \label{ast:errorhierarchy}
\end{figure}


\begin{description}
\item[\textbf{Exception}] is Java's standard error handling concept and is located 
  at the top of the error hierarchy. Exception is a built-in Java class that
  extends the \classref{Throwable} class which is the superclass of all errors
  and exceptions in Java. Only \classref{Throwable} or subclasses of
  \classref{Throwable} can be the argument type in a catch clause and only
  objects that are instances of this class or it's subclasses can be thrown by
  the Java throw statement\cite{throwable}.

\item[\textbf{Error}] extends the \classref{Exception} class.
  \classref{Error} is an abstract class which contains two abstract
  methods: \methodref{getColumn()} and \methodref{getLine()}, allowing
  subclasses to provide more detailed error messages.

\item[\textbf{SyntaxError}] extends the \classref{Error} class and is used in
  the parser to handle syntactic errors. A syntactic error occurs if a token
  doesn't correspond to the currently expected token according to the grammar.
  \classref{SyntaxError} outputs a message and information about the line and
  offset of the token that caused the error. 

\item[\textbf{ScannerError}] extends the \classref{SyntaxError} class and is
  used in the scanner to handle lexical errors. A lexical error occurs if the
  programmer makes a typo, e.g. writes ``defin'' instead of ``define'' or if it 
  in other ways doesn't follow the regulations of \productname{}'s lexemes e.g.
  writes the first letter in a type name in lower case letters. Like
  \classref{SyntaxError}, \classref{ScannerError} outputs a message and
  information about the line and offset of the token that cause the error.  

\item[\textbf{StandardError}] extends the \classref{Error} class. The
  \classref{StandardError} class is used in the interpretation phase to handle
  errors with nodes in the abstract syntax tree.  

\item[\textbf{ScopeError}] extends the \classref{StandardError} class. It is
  thrown by the scope checker to notify the programmer of scope errors.

\item[\textbf{InternalError}] extends the \classref{StandardError} class. It is
  an error that encapsulates Java's own exceptions, if they are thrown. It
  essentially represents an error in the implementation or environment in which
  it is running, and is not caused directly by the programmers code.

\item[\textbf{TypeError}] extends the \classref{StandardError} class. It is
  thrown at runtime when a value is of the wrong type, for instance when using
  an operator such as the \texttt{+} operator, the operands must be applicable
  for that operator.

\item[\textbf{ArgumentError}] extends the \classref{StarndardError} class. It is
  thrown at runtime if a function is supplied the wrong number of arguments.

\item[\textbf{NameError}] extends the \classref{StandardError} class. It is
  thrown at runtime if a used constant, type or variable is not defined in the current scope.

\item[\textbf{DivideByZeroError}] extends the \classref{ArgumentError} class. It
  is thrown when dividing by zero.

\item[\textbf{SimulatorError}] extends the \classref{Error} class. It is thrown
  if the simulator encounters an error.
\end{description}

% Talk about how, where, and why errors are thrown in the different packages
% ..and how they're implemented

\section{Game abstraction layer}
\label{sec:gameabstractionlayer}
We need an abstraction layer to act as the glue between the interpreter
and simulator, allowing a simulator access to different elements in a
written program. This has caused us to write an abstraction layer on top
of the interpreter, offering interfaces such as our graphical simulator
the player access to elements in the game.

There are three different packages relating to the game abstraction
layer. The first is the central class in the Game Abstraction Layer
package, the second are wrappers in the Interpreter package, providing
the actual implementation, and lastly is the game application
programming interface in our Utilities package. Each is described in
this section.

\subsection{The main layer}
The entire program package offered by \productname{} is encapsulated by
the class \classref{GameAbstractionLayer}. This small class is basically
only defined by its constructor, seen below in \lstref{lst:gal}:

\lstinputlisting[language=Java,label={lst:gal},caption={The game abstraction layer's constructor, initializing different constructs based off of an input of characters.}]{listings/gal.java}

Here all the constructs are tied together: The handwritten scanner
is instantiated with the input provided in the constructor, creating
a stream of tokens fed to the parser, that then creates an abstract
syntax tree traversed by the interpreter. The interpreter and its
information is then used when getting the main game wrapper. The class
is also complimented by the method \methodref{getGame}, called after
instantiating the \classref{GameAbstractionLayer}. This method returns a
\classref{GameWrapper} (described in the next subsection), containing
all the needed information about the written game.

This class is what's called and used by the simulator when starting up,
as the simulator needs access information about the game, which in turn
builds on top of our scanner and parser. It's meant to be used by any
interface that wishes to access and modify a game's state.

\subsection{The application programming interface}
The application programming interface (API) provided is simply a collection of interfaces used by the wrappers. This means that classes implementing these interfaces guarantees different methods are available. As an example, the \classref{Square} interface is seen in \lstref{lst:squareinterface} below:

\lstinputlisting[language=Java,label={lst:squareinterface},caption={The \classref{Square} interface, one among many such interfaces provided by \productname{}.}]{listings/squareInterface.java}

Currently $12$ such interfaces exist for different aspects of a game,
such as getting information about the game itself, its players, board,
action sequences, etc. These interfaces all build on top of the default
types in the game environment described in \secref{sec:predefined}. As a
rule, an interface exists for every default type defined in the standard
environment. If we want to expand the standard environment, an interface
and its wrapper would need to be added, considering the fact that it's
one of the only ways 3\textsuperscript{rd} party simulators can interact
with \productname{}.

\subsection{Wrappers}
Wrappers provide a way of accessing the values created by the
interpreter, allowing a simulator to use the properties of these
values to, for example, display the game's title, supply information
about the winning conditions, squares \& pieces, actions \& move
history, and so on. The wrappers implement the interfaces in the
API described above. The main wrapper returned from the method
\methodref{getGame} provides access to all the other wrappers
(implementing their respective interfaces) via methods in its body.

When instantiating \classref{GameWrapper} in method \methodref{getGame}
of class \classref{GameAbstractionLayer}, an instance of the interpreter
is passed to the method, which means full access to the symbol table and
standard environment. All wrapper classes have the following signature:

\begin{lstlisting}[language=Java,caption={The signaure of all API wrapper classes.}]
  public class xxWrapper extends Wrapper implements yy { ... }
\end{lstlisting}

Where $xx$ is one of the $12$ wrappers and $yy$ is the matching
interface. The root class \classref{Wrapper} houses a series of methods
used by all the wrapper subclasses that retrieve different values in
the type the specific wrapper abstracts over (as in the implemented
interface).

This construction is the only way applications such as our simulator
can interact with the programming language. A simple instantiation of
\classref{GameAbstractionLayer} provides everything needed for this to
happen (granted it's a Java application).

\section{A game simulator}
\label{sec:simulator}
Considering the fact that board games consist of physical entities in the real
world and rely purely on user-to-game and user-to-user interaction, we find it
necessary to analyse how we can emulate this behaviour in the most ``realistic''
way. To do this, we look at what a simulator is and could be, what we can use it
for, and set up some features an optimal simulator for our programming language
would include, ending with a final definition of the simulator for our language.

So what is a simulator and what does it consist of? A simulator can be seen as a
front end to an interpreter (or a compiler, though not as practical). It is the
glue between the user and code execution. A user interacts with the simulator,
which in turn interprets the user's input and does something with it, such as
updating a graphical user interface or supplying some other kind of feedback.

Examples of simulators are seen in various different contexts, such as the
Ruby\cite{rubyLang} programming language's interactive shell $irb$, which is run
from the command line and allows programmers to interact, experiment, and write
code with immediate response, calling Ruby's interpreter upon every command
entered. The $irb$ keeps track of all current code entered, allowing programmers
to write an entire program in $irb$. Another example could be various different
kinds of environmental simulators, such as physics simulators created by the
University of Colorado at Boulder\cite{colSim}. These simulators offer a
computerized environment that allows changing of different factors within a
simulated world, such as changing the pressure and gravity of an environment,
providing instant feedback.

\subsection{Usage}
We see the need for a simple simulator because board games consist of so much
interactivity between the players and the board, that we need to mimic it.
Nobody wants to sit and play noughts and crosses or chess in front of a
terminal; that'd be both awkward and impractical. Therefore, we see the
simulator playing a crucial role as the engine that drives the graphics and
gameplay of a written game - in part being a front end to everything in the
interpretation/compilation phases.

A board game designer could program his game in \productname{} and see it
displayed with the current implementation fully working and playable on the
screen in a matter of a few clicks. Another advantage with having such a
simulator is that it can be used to prototype games before they physically need
to be produced. Such a construction will allow quickly changing the game rules
and board layout, etc.\ and support experimentation with different set-ups. This
type of simulator could allow dynamically changing board game parameters, such
as the board size, the amount of players, how the pieces behave, etc.

Another, more simple version of the simulator directed at the end users can be
used to merely play the games. All they would have to do is open a game file in
the simulator or set the simulator as the default program for game files written
in our language. This is useful for games that don't necessarily need a physical
version or when the game designer wants to test it with a broad group of people
before putting it into production.

\subsection{Possible features}
We decide that creating a set of potential features for a simulator will also be
useful when it comes to designing the programming language itself, as these
features can influence the syntax and semantics of the \productname{} language.
Described below are some descriptions of possible features we have discussed and
deem important for the simulator to offer.

\begin{description}

  \item[Interactive design] As a board game designer, it could be possible to
    quickly change pieces around and edit some things directly from the
    interface. This could influence the written code, creating a new game based
    off of the old one, much like the physics simulators mentioned previously.
    An alternative option to this would be to dynamically reload the file used
    as input if it is changed from an external source, allowing quick feedback
    if you're just editing a few lines in the game's source code.

  \item[Loading pictures] Pieces and illustrations of various entities in the
    game can be automatically found and determined from their names definitions
    in a \productname{} file. This lets the designer think about writing a game
    and not how to load specific files from a directory and so on, easily
    influencing cluttered code.

  \item[AI] As long as the code and game rules are well defined, an automatic AI
    could be implemented as a module in the simulator to simulate other players
    following the exact same set of rules, allowing the designer to test his
    entire game or parts of it without constantly needing other people. This
    could be very interesting, but unfortunately is out of the scope of this
    project.

  \item[Multi-player] Multi-player support using the same computer or over a
    network. Each real player could take turns sitting at a physical computer,
    replacing non-existent players or computer-controlled players. As long as
    the simulator is implemented optimally, supporting multi-player games should
    be considerably simple, as the simulator needs to handle commands from a
    single player anyway. Scaling this up and handling multiple turns from
    multiple players shouldn't be too much of a challenge. A better, yet not
    always more practical solution is to allow players to play against each
    other across a network. Sending turn commands back and forth could be
    established via a simple protocol.

  \item[Tracking moves] The simulator could offer a simple turn list displaying
    all the previous moves in the board game. Then it'd be possible to go back
    to a specific turn to ``rewind'' the game to a previous state.
\end{description}

These features could easily influence the syntax of our programming language.
There could be specific reserved constructs to determine how the board and
players are defined, making the simulator's job at displaying things easier.

\subsection{Definition of a simulator}
We define a simulator as a package consisting of the language's
interpreter/compiler and a GUI that is in direct contact with the users of our
programming language. Whether these users are designers or players is
irrelevant, as different versions of the simulator could easily be written and
implemented. It can support many different features and could allow changes to
be made as the user notices something that needs to be changed. The simulator
sends commands to the interpreter/compiler and responds to the commands returned
from it, such as updating a score, changing the position of a piece, or
displaying an error message upon an attempting an illegal move.

An example of this could be that the user clicks and drags on a knight in an
implementation of chess, moving it to another position on the game board. The
simulator would send this behaviour to the interpreter or compiler (which
recompiles), which checks it against the game's source code to see if the move
itself is legal, and also any side-effects this move could have, such as
eliminating an opposing player's piece.

We see spending time on writing a simulator useful because it links all the
different stages together and will act as the final product containing all the
other parts of the project. That said, it'd be ideal to separate the
interpreter/compiler and simulator, allowing greater modularity if the
interpreter/compiler is to be used in another implementation of a simulator or
something entirely different.

Considering the fact that most board games are very visual and consist of
different kinds of pieces placed at various different locations on a board, we
conclude that we need a simulator. This simulator needs to be graphical and
support all the elements a normal gaming session would, such as a board, pieces,
rules for moving pieces, multiple players, and so on. Adding the ability to
dynamically change programmatic features from the user interface is not rated as
important, because this can simply already be done from the source code. It
would help make testing and playing games as authentic as possible.

