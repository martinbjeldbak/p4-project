\section{Writing games in \productname{}}
\label{sec:writinggames}
We have written a handful of games in the final version \productname{}
to gain a better practical understanding of writing in the language
along with finding possible holes and features that may exist. The games
vary from classic board games to small, unknown puzzle games invented by
individual group members. All game screenshots in this section are taken
from the simulator while running an implementation of a particular game.

This section introduces four of those games, namely Noughts and Crosses,
Connect Four, Ice, and Kent's game. Some of these have their source code
listed to show the simplicity of \productname{}. Chess could also be
listed here, but it is not as simple and doesn't provide the overview
of our language we wish to create with this section. It is possible
to see the source code of Chess in the games folder in this project's
repository.

\subsection{Noughts and Crosses}
The first game implemented in \productname{} is Noughts and
Crosses. A picture of a possible board layout is shown below in
\figref{fig:nacsimulated}. The source code for the entire game is presented 
in \csref{naccode.junta}:

\fig[scale=0.3]{nacsimulated}{An implementation of Noughts and Crosses
  in \productname{}, played in the simulator.}

\codesample{naccode.junta}

\subsection{Connect Four}
The famous Connect Four game was written in approximately five minutes
using the Noughts And Crosses implementation as a template, modifying
board set up, win condition, etc. The game loaded in the simulator can
be seen in \figref{fig:connect4simulated}. The source code is shown in
\csref{connect4code.junta} underneath.

\fig[scale=0.3]{connect4simulated}{An implementation of Connect Four
  \productname{}, played in the simulator.}

\codesample{connect4code.junta}

\subsection{Ice}
The Ice game is a single-player puzzle game partly invented by one
of the group members. Partly here means that it is not based on any
particular game, but the game mechanics is used in many other puzzle
games. The goal is to move the player (the ring-shaped piece) onto the
green square. The player can move continuously in one of the directions
north, east, west, or south until he meets a black wall. He cannot stop
halfway on the path. A screenshot of the game loaded in the simulator
can be seen in \figref{fig:icesimulated}. The red numbers show the moves
needed to solve the puzzle, they are not visible in the game.

\fig[scale=0.3]{icesimulated}{A \productname{} implementation of a
custom puzzle game, Ice, played in the simulator.}

\subsection{Kent's Game}
Kent's Game is another game invented by the group.
\figref{fig:kentgamesimulated} shows the initial board set up. The
goal is to swap the position of all red and blue pieces. A blue piece
can move one square north or one square east if it lands on an empty
square. It can however move two squares north or two squares east by
jumping over a red piece, landing on an empty field. The moves of a
red piece are identical to those of a blue piece, where moving is
mirrored in the opposite direction. The game is hard to solve because
pieces can never move back, and the narrow passage in the middle allows
only one piece to pass at once. One of the group members liked the
game idea and implemented it in JavaScript so he could post it on his
website. He included a backtracking solver which verified that the
game is solvable with multiple solutions after dozens of moves. The
source code is not listed here, but is only about $35$ lines. Below, in
\figref{fig:kentgamesimulated}, the initial board setup in the simulator
is shown.

\fig[scale=0.3]{kentgamesimulated}{A \productname{} implementation of a
custom puzzle game, Kent's Game, played in the simulator.}
