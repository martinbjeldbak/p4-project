\section{Unit testing}
A built-in type in \productname{} makes it easy to unit test a game
written in \productname{}. If a source code contains a type which
extends the \typeref{TestCase} type, it means that any constant defined
in the type must evaluate to true.

An example of a handful unit tests of Noughts and Crosses can be seen
below:

\codesample{nactest.junta}

The data member \varref{state1} contain a Noughts and Crosses game in
its starting state. The first constant \methodref{testTitle} verifies
that the game title really is set to ``Noughts and Crosses''. The data
member \varref{state2} contains the game state after the first player
has performed the first action in the list of possible actions. This
first action will cause a piece to be placed on one of the squares,
which means that $3*3 - 1 = 8$ squares must still be empty. This is
verified by the test constant \varref{testEmptySquares}.

This construct allows the programmer to define one or more test types
for the game in question (in the same file, as \productname{} does not
support multi-file imports). If one of the definitions within a type
extending \type{TestCase} returns false, then the interpreter throws an
error to notify the programmer of a failing test.
