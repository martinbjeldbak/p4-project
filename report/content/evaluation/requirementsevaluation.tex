\section{Requirements evaluation}
\label{sec:requirementsevaluation}

In \chapref{chap:requirements}, we formulated a series of requirements.
The chapter featured requirements for the functionality and performance
of \productname{} and also requirements for the final solution. In the
following section we will evaluate the requirements. Those we find most
interesting and essential for the project are evaluated more detailed
than those which are less relevant.

\begin{description}
  \item[\textbf{Requirement $1$a:}]
    One of the major requirements of the project is requirement
    $1$a: ``It must be possible to implement Chess, including the
    special rules of Chess''. This however has not been possible,
    due to time pressure. The special rules of Chess, and especially
    \emph{En Passant} and \emph{Castling} moves proves to be difficult
    challenges, requiring a good amount of time to get implemented. We
    do though believe that \productname{} provides the features write a
    complex game such as chess.

  \item[\textbf{Requirements $2$-$5$:}]
    Are all accomplished.

  \item[\textbf{Requirement $6$:}]
    Are all accomplished. Requirement $6$j: ``Check if the current move
    about to be made for a piece is the first move made by that piece'' is
    not directly implemented, but can be accomplished by for example using a
    counter, \variable{Moves}, which counts one up for every move taken by
    the piece and checking if this is larger than one.

  \item[\textbf{Requirements $7$-$16$:}]
    Are all accomplished.

  \item[\textbf{Requirement $17$:}]
    The requirement states: ``The created board games must be playable
    in a graphical simulator''. We have provided an API that our
    simulator builds on top of, proving that this is possible. If one
    wished to create a graphical interface for playing board games in
    Android, they would simply need to manipulate the object returned by
    our API, just like our simulator does.

  \item[\textbf{Requirement $18$:}]
    The requirement and its sub points are not accomplished. Requirement
    $18$a and $18$ state that ``It must be possible to save the move
    history'' and ``It must be possible to start a game from a saved
    move history''. In the right hand side of our simulator, the game
    info widget is located, displaying the current moves taken other
    information. So the requirements could be accomplished indirectly by
    saving this history in a screen shot for instance, restart the game
    and then perform all the moves that one wants to keep again. This
    is a clumsy and very manual solution. A better solution would be to
    save the game in its current state as a file for every action taken.
    Nothing hinders this solution from being accomplished, the solution
    has been down-prioritised. Requirement $18$c that stated: ``It must
    be possible to play over a network'' is a ``could-have'' feature,
    which has also been down-prioritized.

  \item[\textbf{Requirement $19$:}]
    The requirement states that ``The programmer must be able to
    implement board games with relatively few lines of code''. It is difficult to
    make a quality comparison with other programming languages, since
    two of the board games are invented by group members and therefore
    haven't been programmed in other languages. Noughts and Crosses
    and Connect Four are however well-known board games and these were 
    created in around $20$ - $25$ lines of code, which we consider
    as relatively few lines of code.  

  \item[\textbf{Requirements $20$-$24$:}]
    Are all accomplished.

  \item[\textbf{Requirements $25$-$27$:}]
    These requirements can not be evaluated as they are merely a listing
    of project limitations.

  \item[\textbf{Requirement $28$:}]
    The requirement states that ``The programming language must make
    it easy and quick for programmers to develop board games''. In
    \secref{sec:connectfour}, we describe how a game of Connect Four
    was created in approximately $5$ minutes. The speed of which a
    game can be created in a programming language depends on a number
    of parameters. First of all, the programmer's familiarity with
    the programming language and the programmer's experience with a
    programming mindset. Also, the programming language's support for
    relevant functionality and its overall writability is a parameter.
    The fact that we were able to create a fully functional Connect Four
    game in $5$ minutes can be considered as quick and supports that it
    must be easy to develop board games in \productname{}.

  \item[\textbf{Requirement $29$:}]
    The requirement is accomplished. Since all of the modules (the
    scanner, parser, scope checker, interpreter, etc.) of \productname{}
    are compiled to Java byte code, and hence every system supporting
    the Java virtual machine is able to run \productname{} games. The
    simulator on the other hand is implemented using a Java library,
    that only works on personal computers due to the graphics library
    used (openGL).
\end{description} 
