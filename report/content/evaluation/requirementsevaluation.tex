\section{Requirements evaluation}
\label{sec:requirementsevaluation}

In \chapref{chap:requirements} we formulated a series of requirements. The chapter featured requirements for the functionality and performance of \productname{} and also requirements for the final solution. In the following section we will evaluate the requirements we find the most interesting, since going through every single requirement would be uninteresting. 

\begin{description}
\item[\textbf{Requirement 1.a:}]
One of the major requirements of the project is requirement 1.a: ``It must be possible to implement Chess, including the special rules of Chess''. This however has not been possible to do due to time pressure. The special rules of Chess and especially the \emph{En Passant} and \emph{Castling} moves proves to be difficult challenges which requires a good amount of time to get implemented though we believe that \productname{} provides the features to get it done.

\item[\textbf{Requirement 2-5:}]
Are all accomplished

\item[\textbf{Requirement 6:}]
Are all accomplished. Requirement 6.j: ``Check if the current move about to be made for a piece is the first move made by that piece'' is not directly implemented but can be accomplished by for example using a counter, \variable{Moves}, which counts one up for every move taken by the piece and checking if this is bigger than one.

\item[\textbf{Requirement 7-10:}]
Are all accomplished



\item[\textbf{Requirement 20:}]
The requirement states that ``The programmer must be able to implement board games with relatively few lines of code''. Based on the games currently programmed in \productname{} it is difficult to make a quality comparison with other programming languages, since two of the board games are invented by group members and therefore hasn't been programmed in other languages. Noughts and crosses is however a well-known board game and this was possible to be created in around 20 lines of code which can be considered as relatively few lines of code.





\end{description} 