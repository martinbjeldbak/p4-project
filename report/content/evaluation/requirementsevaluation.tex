\section{Requirements evaluation}
\label{sec:requirementsevaluation}

In \chapref{chap:requirements} we formulated a series of requirements. The chapter featured requirements for the functionality and performance of \productname{} and also requirements for the final solution. In the following section we will evaluate the requirements. Those we find most interesting and essential for the project are evaluated more detailed than those which are less relevant. 

\begin{description}
\item[\textbf{Requirement 1.a:}]
One of the major requirements of the project is requirement 1.a: ``It must be possible to implement Chess, including the special rules of Chess''. This however has not been possible to do due to time pressure. The special rules of Chess and especially the \emph{En Passant} and \emph{Castling} moves proves to be difficult challenges which requires a good amount of time to get implemented though we believe that \productname{} provides the features to get it done.

\item[\textbf{Requirement 2-5:}]
Are all accomplished

\item[\textbf{Requirement 6:}]
Are all accomplished. Requirement 6.j: ``Check if the current move about to be made for a piece is the first move made by that piece'' is not directly implemented but can be accomplished by for example using a counter, \variable{Moves}, which counts one up for every move taken by the piece and checking if this is bigger than one.

\item[\textbf{Requirement 7-15:}]
Are all accomplished

\item[\textbf{Requirement 16:}]
\todo{something clever about this requirement}

\item[\textbf{Requirement 17:}]
The requirement states that  ``The created board games must be playable in a graphical simulator''. This is accomplished and description of how this
simulator functions is found in \secref{sec:simulator-impl}.

\item[\textbf{Requirement 18:}]
The requirement is not accomplished. Requirement 18.b and 18.b.I states that ``It must be possible to save the move history'' and ``It must be possible to start a game from a saved move history''. In the right hand side of the simulator the game info widget is located which displays the current moves taken amongst other information. So the requirements could be accomplished indirectly by saving this history in a screen shot for instance, restart the game and then perform all the moves that one wants to keep again. This is a clumsy solution though. A better solution would be if it was possible to save the game in it's current state as a file. Nothing hinders this solution from being accomplished but as for now it has been down prioritised. Requirement 18.c that stated: ``It must be possible to play over a network'' is a ``could-have'' feature which has also been down prioritized. 

\item[\textbf{Requirement 19:}]
The requirement states that ``The programmer must be able to implement board games with relatively few lines of code''. Based on the games currently programmed in \productname{} it is difficult to make a quality comparison with other programming languages, since two of the board games are invented by group members and therefore hasn't been programmed in other languages. Noughts and crosses and connect four are however well-known board games and these were possible to create in around 20 lines of code and 22 lines of code respectively which can be considered as relatively few lines of code.

\item[\textbf{Requirement 20-24:}]
Are all accomplished

\item[\textbf{Requirement 25-27:}]
These requirements can not be evaluated as they are merely a listing of project limitations. 

\item[\textbf{Requirement 28:}]
The requirement states that ``The programming language must make it easy and quick for programmers to develop board games''. In \secref{sec:connectfour} we described how a game of Connect Four was created in approximately 5 minutes using the already programmed Noughts and Crosses game as a template. The speed of which a game can be created in a programming language depends on a number of parameters. First of all the programmers familiarity with the programming language and the programmers experience with programming overall. Also the programming languages support for relevant functionality and it's overall writability is a parameter. The fact that we were able to create a fully functional Connect Four game in 5 minutes can be considered as quick and witnesses that it must be easy to develop board games in \productname{}.

\item[\textbf{Requirement 29:}]
The requirement is accomplished. Since all of the modules (the scanner, the parser, the scope checker, the interpreter etc.) of \productname{} are compiled to Java byte code, every system supporting the Java virtual machine is able to run \productname{} games.   
\end{description} 