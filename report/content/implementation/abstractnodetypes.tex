\section{Abstract node types}
\label{sec:ant}

Abstractions are the interpreter's or compiler's internal representation of a program. It is represented as an abstract syntax tree which is a series of nodes and leaves connected, forming a so-called tree.%

This section covers all aspects of our abstract syntax tree (AST), and it begins with a description of how an AST differs from a parse tree. 
The grammar specifications are specified in \secref{sec:grammar}. In this section each program part is presented individually with an AST within its respective subsection.%

When a piece of code is parsed by a parser that understands the specific programming language, the output of the parser will be a parse tree which corresponds to the abstract syntax of each program part. The difference between an AST and a parse tree is that a parse tree contains every detail of the source code. A parse tree includes for instance the parentheses and keywords where an AST does not contain anything other than the abstract node types (function definition, game declaration, expression etc.).%

\subsection{Program}%
Every program written in \productname{} begins with an abstract node type which we call ``program'' that consists of either zero or more function definitions followed by a single game declaration. The production for this rule is a follows:%

\begin{ebnf}%
\grule{program}{\grep{function\_def} \gcat game\_decl}%
\end{ebnf}%

It is from this production each and every program is derived from. The AST for this abstract node type is illustrated in \figref{ast:program}.%


\begin{figure}[ht]
\begin{center}
\begin{tikzpicture}[level/.style={sibling distance=40mm/#1}]
\node [square] {Program}
  child {node [square,xshift=0.5cm] (a) {\textit{Definition}} edge from parent[dashed];}
  child {node [square,xshift=-0.5cm] (b) {\textit{Definition}} edge from parent[dashed];};
\path (a)--(b) node [midway] {$\cdots$};
\end{tikzpicture}
\end{center}
\capt{The abstract syntax tree for the program node.}
\label{ast:program}
\end{figure}
%

Figure \ref{ast:program} consists of one root which is called ``Program'' and this root has three children, two called ``Function definition'' and one called ``Game declaration''. The nodes named ``Function definition'' are optional because the production says that a program can begin with either zero or more of these abstract node types. The AST illustrates this by making the connecting lines dashed. Within these two nodes there are three dots ($\cdots$) which illustrate that it is possible to have more of these abstract node types following each other.%

This means that a program can begin with either a function or a game declaration. The following section will illustrate the abstract node type of a function definition which is part of the program abstract node type.%

\subsection{Function definition}%
Programmars have the opportunity of defining functions in their programs. The parser knows that the following piece of code is a function definition when it meets the keyword ``define'' and then the following three abstract node types are part of this definition. This is also evident in the grammar for a \textit{function\_def}:%

\begin{ebnf}%
\grule{function\_def}{\gter{define} \gcat function \gcat varlist \gcat expression}%
\end{ebnf}%

\begin{figure}[ht]
\begin{center}
\begin{tikzpicture}[level/.style={sibling distance=30mm/#1}]
\node [square] {Function definition}
  child {node [square,xshift=0.9cm] {Function}}
  child {node [square] {Variable list}}
  child {node [ellipse,draw,xshift=-0.5cm] {\textit{Expression}}};
\end{tikzpicture}
\end{center}
\capt{The abstract syntax tree for the function definition node.}
\label{ast:funcdef}
\end{figure}%

Figure \ref{ast:funcdef} illustrates the AST for this production. The AST does not show the ``define'' keyword because it is not part of the abstraction. The keyword is there to make sure that the parser knows that the following production is a function definition and nothing else.%

The production defines that after the keyword ``define'' a function, a variable list, and an expression are followed in that specific order. If anything else is read then a syntax error will be thrown by the parser because then the input is not syntactically correct and does not correspond with the specification.%

It is clear that \figref{ast:funcdef} has three children from the root and these children are all drawn with solid lines. This means that they are not optional and must occur in that specific order. The \textit{expression} child is different from the others in its form and that it is written with an italic font. This is done to illustrate that an expression can be many different things.%

So, the AST in \figref{ast:funcdef} is different from the AST for the program abstract node type in that is has no optional children. The next section will cover the final part of the program which is the game declaration from which the rest of the program is derived from.%

\subsection{Function call}
\todo{write description}


\begin{figure}[ht]
\begin{center}
\begin{tikzpicture}[level/.style={sibling distance=30mm/#1}]
\node [square] {Function call}
  child {node [ellipse, draw, xshift=0.5cm] {\textit{Element}}}
  child {node [square, xshift=-0.5cm] {List}};
\end{tikzpicture}
\end{center}
\capt{The abstract syntax tree for the variable list node.}
\label{ast:funccall}
\end{figure}


\subsection{Game declaration}%
The game declaration is the final part of the program abstract node type and it is from this abstract node type that the game is actually declared. The production for the game declaration is as follows:%

\begin{ebnf}%
\grule{game\_decl}{\gter{game} \gcat declaration\_struct}%
\end{ebnf}%

A function definition had a ``define'' keyword in front of every function defintion and the game declaration has another keyword - namely the ``game'' keyword. This makes sure that the parser is aware of the fact that the following declaration is the game declaration. The next production must be a declaration\_struct which is described in the next section.%


\begin{figure}
\begin{center}
\begin{tikzpicture}[level/.style={sibling distance=30mm/#1}]
\node [square] {Game declaration}
  child {node [square] {Declaration struct}};
\end{tikzpicture}
\end{center}
\capt{The abstract syntax for the game declaration node.}
\label{ast:gamedecl}
\end{figure}
%

Figure \ref{ast:gamedecl} illustrates the AST for the game declaration. This is a rather simply AST with just one root and one child. The keyword in front of the declaration is not shown because it is not part of the tree. If anything else than a declaration struct is read by the parser a syntax error will be thrown indicating that the source code is not valid.%

\subsection{Declaration struct}%

The declaration struct consists of at least one declaration. The parser knows that the following is a declaration struct by ready the first symbol as being a left brace ( ``\{'' ) followed by one or more declarations and finalised with a right brace ( ``\}'' ). The following production specifies this:%

\begin{ebnf}%
\grule{declaration\_struct}{\gter{\{} \gcat declaration \gcat \grep{declaration} \gter{\}}}%
\end{ebnf}%


\begin{figure}[ht]
\begin{center}
\begin{tikzpicture}[level/.style={sibling distance=30mm/#1}]
\node [square] {Declaration struct}
  child {node [square, xshift=0.5cm]  {Declaration}}
  child {node [square] (b) {Declaration} edge from parent[dashed];}
  child {node [square] (c) {Declaration} edge from parent[dashed];};
  
\path (b)--(c) node [midway] {$\cdots$};
\end{tikzpicture}
\end{center}
\capt{The abstract syntax tree for the declaration structure node.}
\label{ast:declstruct}
\end{figure}
%

Figure \ref{ast:declstruct} illustrates this by have one root with three children where the first child is connected with a solid line to the root followed by two similar children but with dashed lines specifying that the first declaration can be followed by zero or more declarations. The braces are not visible in the AST.%

\subsection{Declaration}%
The declaration is the last abstract node type which is described in this section that constitutes the game declaration. A declaration is identified by either a keyword or an identifier which will be followed by a structure. This is specified in the following production:%

\begin{ebnf}%
\grule{declaration}{\ggrp{keyword \gor identifier} \gcat structure}%
\end{ebnf}%

So, the parser reads either a keyword or a identifier and the following must be some kind of structure or else the parser will throw a syntax error.%

A structure can be many different things and \figref{ast:decl} illustrates this by encapsulating the child with the abstract node type of a structure in an ellipse formed box with italic font. This is similar to the expressions illustrated earlier.%


\begin{figure}
\begin{center}
\begin{tikzpicture}[level/.style={sibling distance=30mm/#1}]
\node [square] {Declaration}
  child {node [ellipse split,draw] {Keyword \nodepart{lower} Identifier}}
  child {node [ellipse,draw] {\textit{Structure}}};
\end{tikzpicture}
\end{center}
\capt{The abstract syntax for the declaration node.}
\label{ast:decl}
\end{figure}
%

So, this means that a program written in \productname{} will always have at least one structure because the declaration struct in \figref{ast:declstruct} specifies that at least one occurence of a declaration must be included.%

This finalises the abstract node types for the game declaration leaving the abstract node types for the function definitions which will be specified in the following sections.%

\subsection{Variable list}%

The function definition is constituted of a function followed by a variable list and an expression. The only part of a function definition that has an abstract node type is the variable list because for instance a function is not something that is specified in the grammar other than it can consist of any type of letter. The other part of a function definition is the expression which again can consist of many different things.%

A variable list can consist of variables be an empty list containing nothing. This is illustrated in the following grammar:%

\begin{ebnf}%
\grule{varlist}{\gter{[} \gcat \gopt{variable \gcat \grep{\gter{,} \gcat variable} \gcat \gopt{\gter{,} \gcat vars} \gor vars} \gcat \gter{]}}%
\end{ebnf}%

The production specifies that a variable list begins with an occurence of a left bracket ( ``['' ) and ends with an occurence of a right bracket ( ``]'' ). The brackets in between the outer bracket mean that the variables are optional. This means that a variable list can be empty.%

The last part of the production is a bit hard to read, but it says that if the list begins with one or more variable types then these variables can be followed by a comma and the variable argument which is a list of parameters. If there are no variable types then the last variable argument can be included. So, the production consists of two parts - one part containing two zero, one or more variables followed by an occurence of variable argument or just a single variable argument. This is due to the comma that must follow the last variable before a variable argument.%


\begin{figure}
\begin{center}
\begin{tikzpicture}[level/.style={sibling distance=30mm/#1}]
\node [square] {Variable list}
  child {node [square] (a) {Variable} edge from parent[dashed];}
  child {node [square] (b) {Variable} edge from parent[dashed];};
\path (a)--(b) node [midway] {$\cdots$};
\end{tikzpicture}
\end{center}
\capt{The abstract syntax for the variable list node.}
\label{ast:variablelist}
\end{figure}
%

Figure \ref{ast:variablelist} illustrates this by having a root and three children with dashed lines from the root. But this is actually a bit misleading because of the optional parameters in the specification of the production. The production specifies that if the programmer does not want the list to be empty then at least one occurence of a variable must be present or a single variable argument. After the first variable it is possible to have zero or more variables following the first one and another optional variable argument. The easiest way to illustrate this is as it has been done in \figref{ast:variablelist}.%

\subsection{Assignment}%

An assignment is actually one of the many different abstract node types an expression can be. The following grammar rule specifies the production of an assignment in \productname{}:%

\begin{ebnf}%
\grule{assignment}{\gter{let} \gcat variable \gcat \gter{=} \gcat expression \gcat \grep{\gter{,} \gcat variable \gcat \gter{=} \gcat expression} \gcat \gter{in} \gcat expression}%
\end{ebnf}%

The production specifies that any assignment must begin with the keyword ``let'' and end with the keyword ``in'' followed by an expression. In between these beginning and ending keywords the production consists of at least one pair of a variable followed by an assignment-operator followed by an expression. The production specifies that it is possible to have zero or more of these variable-expression pairs (comma-separated) following the first pair.%


\begin{figure}
\begin{center}
\begin{tikzpicture}
[level/.style={sibling distance=40mm},
level 1/.style={sibling distance = 39mm},
level 2/.style={sibling distance = 20mm}]

\node [square] (z) {Assignment}
  child {node [square,left of=b,xshift=-5.1cm] (a) {Variable}}
  child {node [ellipse,draw,left of=c,xshift=-4.5cm] (b) {\textit{Expression}}}
  child {node [square] (c) {Assignment} edge from parent[dashed]
  	child {node [square,xshift=-1cm] (x) {Variable} edge from parent[solid]}
  	child {node [ellipse,draw,solid,xshift=-1cm] (y) {\textit{Expression}} edge from parent[solid]}
  }
  child {node [square,xshift=-1cm] (d) {Assignment} edge from parent[dashed]
  	child {node [square,xshift=1cm] (o) {Variable} edge from parent[solid]}
  	child {node [ellipse,draw,solid,xshift=1cm] (p) {\textit{Expression}} edge from parent[solid]}
  }
  child {node [ellipse,draw,right of=d,xshift=1.5cm](e) {\textit{Expression}}};

\path (c)--(d) node [midway] {$\cdots$};
\end{tikzpicture}
\end{center}
\capt{The abstract syntax tree for the assignment node.}
\label{ast:assignment}
\end{figure}
%

Figure \ref{ast:assignment} illustrates this with an AST that omits the keywords, commas and the assignment-operators which can make it rather complex to look at. The figure actually states that an assignment consists of variables and expression where an expression can be many different things, including another assignment. So, this means that it is possible to have assignments nested within each other which is also illustrated in \figref{ast:assignment}. These nested assignments are connected to the root with dashed lines which means that they are optional and it is possible to have zero or more of these following each other.%

\subsection{If expression}%

The if expression is yet another expression as the name indicates. These expressions have one form and it is specified in the following production:%

\begin{ebnf}%
\grule{if\_expr}{\gter{if} \gcat expression \gcat \gter{then} \gcat expression \gcat \gter{else} \gcat expression}%
\end{ebnf}%

An if expression begins with the keyword ``if'' and ends with the keyword ``else'' followed by an expression. In between the beginning and the end the keyword ``then'' appears as a must in the if expression. It is for instance not possible to omit the ``else'' statement. The AST for this production is illustrated in \figref{ast:ifexpr} and consists of a root with three children which are expressions. The children are connected to the root with solid lines which means they cannot be omitted.%


\begin{figure}
\begin{center}
\begin{tikzpicture}[level/.style={sibling distance=30mm/#1}]
\node [square] {If expression}
  child {node [label={[xshift=1.2cm, yshift=0.2cm]\textbf{if}}] [ellipse,draw] {\textit{Expression}}}
  child {node [label={[xshift=-0.38cm, yshift=0.2cm]\textbf{then}}] [ellipse,draw] {\textit{Expression}}}
  child {node [label={[xshift=-2.2cm, yshift=0.2cm]\textbf{else}}] [ellipse,draw] {\textit{Expression}}};
\end{tikzpicture}
\end{center}
\capt{The abstract syntax tree for the if expression node.}
\label{ast:ifexpr}
\end{figure}
%

\subsection{Lambda expression}%

The next expression is a lambda expression and is specified in the following production:%

\begin{ebnf}%
\grule{lambda\_expr}{\gter{\#} \gcat varlist \gcat \gter{=>} \gcat expression}%
\end{ebnf}%

Any lambda expression begins with the \# symbol which makes it clear for the parser that this is a lambda expression. A lambda expression consists of a variable list and an expression seperated by the $=>$ symbol.%


\begin{figure}
\begin{center}
\begin{tikzpicture}[level/.style={sibling distance=30mm/#1}]
\node [square] {Lambda expression}
  child {node [square] {Variable list}}
  child {node [ellipse,draw] {\textit{Expression}}};
\end{tikzpicture}
\end{center}
\capt{The abstract syntax for the lambda expression node.}
\label{ast:lambdaexpr}
\end{figure}
%

Figure \ref{ast:lambdaexpr} illustrates the AST for this expression. Because a lambda expression is an expression it is possible to have lambda expression inside other lambda expressions, so it is possible to nest these expressions.%

\subsection{List}%

We also have an abstract node type of the list element. Any list can be empty just like a variable list but instead of containing variables a list contains expressions. It can consist of either zero, one, or more expressions. The specification for the abstract node type is specified in the following production:%

\begin{ebnf}%
\grule{list}{\gter{[} \gcat \gopt{expression \gcat \grep{\gter{,} \gcat expression}} \gcat \gter{]}}%
\end{ebnf}%

This element begins and ends like a variable list with brackets with optional elements as the elements of the list. The AST for this production is illustrated in \figref{ast:list}.%


\begin{figure}
\begin{center}
\begin{tikzpicture}[level/.style={sibling distance=30mm/#1}]
\node [square] {List}
  child {node [ellipse,draw] (a) {\textit{Element}} edge from parent[dashed]}
  child {node [ellipse,draw] (b) {\textit{Element}} edge from parent[dashed]};

\path (a)--(b) node [midway] {$\cdots$};
\end{tikzpicture}
\end{center}
\capt{The abstract syntax for the list node.}
\label{ast:list}
\end{figure}
%

The AST in \figref{ast:list} lacks the same descpritiveness as the AST for the variable list does because of the optional parameters in the production. The AST illustrates that it is possible to have zero or more expressions in the list, but it is in fact not entirely correct. The correct statement is that it is possible to have either none or at least one expression in the list.%

\subsection{Pattern}%

The next element is a pattern which consists of a single pattern expression followed by zero or more pattern expressions. This is specified in the following production:%

\begin{ebnf}%
\grule{pattern}{pattern\_expr \gcat \grep{pattern\_expr}}%
\end{ebnf}%

This production is very straightforward, a pattern must contain at least one pattern expression. The AST for this production is illustrated in \figref{ast:pattern}. This AST has one solid line to the first pattern expression followed by two dashed lines to two other pattern expressions illustrating that these can occur zero or more times.%


\begin{figure}
\begin{center}
\begin{tikzpicture}[level/.style={sibling distance=40mm/#1}]
\node [square] {Pattern}
  child {node [ellipse,draw] {\textit{Pattern expression}}}
  child {node [ellipse,draw] (b) {\textit{Pattern expression}} edge from parent[dashed]}
  child {node [ellipse,draw] (c) {\textit{Pattern expression}} edge from parent[dashed]};

\path (b)--(c) node [midway] {$\cdots$};
\end{tikzpicture}
\end{center}
\capt{The abstract syntax tree for the pattern node.}
\label{ast:pattern}
\end{figure}
%

\subsection{Pattern operators}%

\productname{} consists of three different pattern operators. We have an or-operator (\figref{ast:pattern-or}), a multiplier-operator (\figref{ast:pattern-mult}), and a not-operator (\figref{ast:pattern-not}).%

The or-operator derives a pattern value followed by a pattern expression that both can be many different things. This is why they are written in an italic font. This is illustrated in \figref{ast:pattern-or}.%


\begin{figure}
\begin{center}
\begin{tikzpicture}[level/.style={sibling distance=40mm/#1}]
\node [square] {Pattern, or-operator}
  child {node [square] {Pattern value}}
  child {node [square] {Pattern expression}};
\end{tikzpicture}
\end{center}
\capt{The abstract syntax for the pattern or-operator node.}
\label{ast:pattern-or}
\end{figure}
%

The multiplier-operator derives a single child namely a pattern value. This is illustrated in \figref{ast:pattern-mult}.%


\begin{figure}
\begin{center}
\begin{tikzpicture}[level/.style={sibling distance=40mm/#1}]
\node [square] {Pattern, multiplier-operator}
  child {node [square] {Pattern value}};
\end{tikzpicture}
\end{center}
\capt{The abstract syntax for the pattern multiplier-operator node.}
\label{ast:patter-mult}
\end{figure}
%

The not-operator also derives a single child namely a pattern check which is also written in an italic font within an ellipse formed box that indicates that it can be many different things. This is illustrated in \figref{ast:pattern-not}.%


\begin{figure}[ht]
\begin{center}
\begin{tikzpicture}[level/.style={sibling distance=40mm/#1}]
\node [square] {Pattern, not-operator}
  child {node [ellipse,draw] {\textit{Pattern check}}};
\end{tikzpicture}
\end{center}
\capt{The abstract syntax tree for the pattern not-operator node.}
\label{ast:pattern-not}
\end{figure}
%

\subsection{Not-operator}%

We also have a not-operator that has to do with expressions and not patterns. This operators AST is illustrated in \figref{ast:not} and derives an expression.%


\begin{figure}[ht]
\begin{center}
\begin{tikzpicture}[level/.style={sibling distance=40mm/#1}]
\node [square] {Not-operator}
  child {node [ellipse, draw] {\textit{Expression}}};
\end{tikzpicture}
\end{center}
\capt{The abstract syntax tree for the not-operator node.}
\label{ast:not}
\end{figure}%

\subsection{Operator}%

Furthermore \productname{} is able to handle arithmetic operators. The AST for operators is illustrated in \figref{ast:operator} and consists of an element followed by an expression.%


\begin{figure}[ht]
\begin{center}
\begin{tikzpicture}[level/.style={sibling distance=25mm/#1}]
\node [square] {Operator}
  child {node [square] {Element}}
  child {node [square] {\textit{Expression}}};
\end{tikzpicture}
\end{center}
\capt{The abstract syntax tree for the operator node.}
\label{ast:operator}
\end{figure}
%


\subsection{Expression and value node types}
\todo{write description and add trees etc.}

These are used throughout the production of source code written in \productname{}.

%
\begin{figure}[ht]
\begin{center}
\begin{tikzpicture}[level/.style={sibling distance=30mm/#1}]
\node [ellipse, draw] {Identifiers}
  child {node [square,xshift=1.0cm] {Function} edge from parent [dashed];}
  child {node [square] {Identifier} edge from parent [dashed];}
  child {node [square,xshift=-1.0cm] {Variable} edge from parent [dashed];};
\end{tikzpicture}
\end{center}
\capt{The abstract node type for the identifiers.}
\label{ast:identifer}
\end{figure}

%\input{content/implementation/abstractnodetypes/keywords}
%
\begin{figure}[ht]
\begin{center}
\begin{tikzpicture}
  [node distance=0.4cm, start chain=going below]
  %    \node (wid) [combtree]  {Width};     
  %    \begin{scope}[start branch=venstre,]
  %      \node (hei) [combtree] {Height};
  %    	 \node (pie) [combtree] {Piece};
  %      \node (wic) [combtree] {Win condition};
  %      \node (tit) [combtree] {Title};
  %    \end{scope}

  %    \node (play) [combtree]  {Players};     
  %    \begin{scope}[start branch=venstre,]
  %    	 \node (tic) [combtree] {Tie condition};
  %    	 \node (boa) [combtree] {Board};
  %      \node (tur) [combtree] {Turn order};
  %      \node (pom) [combtree]  {Possible moves};   
  %	   \end{scope}	  

  %    \node (set) [combtree]  {Setup};     
  %    \begin{scope}[start branch=venstre,]
  %      \node (nam) [combtree] {Name};
  %      \node (posd) [combtree] {Possible drops};
  %      \node (gri) [combtree] {Grid};
  %      \node (wal) [combtree] {Wall};
  %    \end{scope}

  % BRACE 
  %\draw[tuborg] let
  %  \p1=(wid.west), \p2=(tit.east) in
  %  ($(\x1,\y1+1.5em)$) -- ($(\x2,\y2+1.5em)$) node[above, midway]  {\textbf{Keywords}};
\end{tikzpicture}
\end{center}
\capt{The abstract node type for the keywords.}
\label{ast:keywords}
\end{figure}

%\section{Expressions}
\label{sec:expressions}

Throughout this section we will present all the expressions that are included in
\productname{}. We have provided big-step semantics for six central expressions
in this section. These are the ones that we have deemed most necessary and they
are not as any other construct. Which sections that include semantics will be
explained in the following.

The smallest parts of expressions are presented in
\secref{sec:atomicexpressions}. In \secref{sec:lists} the notion of
lists is presented and explained. Here we present big-step semantics. Then we
present the let expressions in \secref{sec:letexpressions}. Also here we present
big-step semantics. Afterwards, we present the conditional expressions in 
\secref{sec:conditionalexpressions} followed by a lambda expressions in
\secref{sec:lambdaexpressions}. We provide big-step semantics for lambda
expressions. Furthermore, we present the concept of a set expression in
\secref{sec:setexpressions}. We also supply big-step semantics for set
expressions. Lastly, we present the different operators and calls in
\secref{sec:operatorsandcalls}. Here we provide big-step semantics for function
calls and for member access.

We begin here by listing the main expressions below this
paragraph:

\begin{ebnf}
%Expressions
\grule{expression}{let\_expr}
\galt{if\_expr}
\galt{set\_expr}
\galt{lambda\_expr}
\galt{\gter{not} \gcat expression}
\galt{lo\_sequence}
\end{ebnf}

Two statements hold about expressions in \productname{}:

\begin{nlist}
\item An expression \textbf{always} has a value.
\item An expression \textbf{cannot} have side effects.
\end{nlist}

\subsection{Atomic expressions}
\label{sec:atomicexpressions}

These are the smallest possible parts of expressions in \productname{}, defined by
the rule:

\begin{ebnf}
\grule{atomic}{\gter{(} \gcat expression \gcat \gter{)}}
\galt{constant}
\galt{type}
\galt{variable}
\galt{\gter{this}}
\galt{\gter{super}}
\galt{integer}
\galt{string}
\galt{direction}
\galt{coordinate}
\galt{\gter{/} \gcat pattern \gcat \gter{/}}
\galt{list}
\end{ebnf}

The first atomic expression is $\gter{(} \gcat expression \gcat \gter{)}$, which means
that it is possible to embed expressions within other expressions, and manually control
the precedence of operations. Consider the following two expressions:

\codesample{parentheses1.junta}
\codesample{parentheses2.junta}

In \csref{parentheses2.junta} the parentheses are in fact unnecessary because the
\texttt{*}-operator has precedence over the \texttt{+}-operator (see
\secref{sec:operatorsandcalls}).

Names (constants, types, and variables) are also atomic expressions, and are evaluated
to whatever value they are associated with, based on the current scope. The keywords
($\gter{this}$ and $\gter{super}$) are atomic, but only applicable within type
definitions, where $\gter{this}$ refers to the current object and $\gter{super}$
refers to the current object casted to its parent type (if it has one).

The literals ($integer$, $string$, $direction$, and $coordinate$) are evaluated to their
respective values, while patterns are evaluated to \type{Pattern}-values according to
the grammar in \secref{sec:patterns} and lists are evaluated to \type{List}-values
according to the grammar in \secref{sec:lists}.

\subsection{Lists}
\label{sec:lists}

The \type{List}-type is one of the basic types in \productname{}. A \type{List}-value is
created using the following syntax:

\begin{ebnf}
\grule{list}{\gter{[} \gcat \gopt{expression \gcat \grep{\gter{,} \gcat
expression}} \gcat \gter{]}}
\end{ebnf}

Essentially this means that a list is created from zero or more expressions (separated
by commas). The following statements can be made about lists:

\begin{nlist}
\item A list can be empty: \texttt{[]}
\item Lists are ordered ($\texttt{[1, 2]} \ne \texttt{[2, 1]}$).
\item Lists are immutable.
\end{nlist}

When a list is evaluated, each expression is evaluated to a value (the order of
evaluation does not matter, since no side-effects are possible, lazy-evaluation
of expressions could even be a possibility), and all values (in the same order
as the expressions) are added to the resulting \type{List}-value. When a
\type{List}-value is created, it can't be altered further (because of the
no-side-effects condition). All operations on that \type{List}-value will create
new \type{List}-values, and leave the original value intact.

Values within lists can be accessed using the \texttt{[]}-operation (same as
with function calls and type instantiation). Lists begin at offset $0$, so in
order to access the second element of a list, one can use the following:

\codesample{listaccess1.junta}

Ranges of elements can also be returned. For instance in the following
expression a new list is returned containing elements from offset 1 up
to and including offset 2 (the list $\texttt{["is", "a"]}$):

\codesample{listaccess2.junta}

An offset can be negative, which means that the offset is dependent on
the size of the list. This way offset $-1$ will always refer to the last element
of the list. For example in order to return the last two elements of a list one
could use the offsets, $-2$ and $-1$:

\codesample{listaccess3.junta}

Some other examples are:

\codesample{listaccess4.junta}

\subsubsection{Big-step semantics}

The semantics presented in \tableref{semantic:lists} are the transition rules
for lists.

\begin{table}[ht]
  \begin{center}
    \begin{tabular*}{\textwidth}{l p{\textwidth}}
      \hline \\
      \hspace{0.5cm} $[\mbox{LIST}_{\mbox{ACCESS-1}}]$ & \infrule{env_{T},
      env_{C}, env_{V} \vdash \lag e\; \rag \ra v_2  \qquad env_{T}, env_{C},
      env_{V} \vdash \lag i \rag \ra v_3}
      {env_{T}, env_{C}, env_{V} \vdash \lag e\; i \rag \ra v_1} \\
       & where $v_2 = \left(l_1, elem_1\right)$ \\
       & and $v_3 = (l_2,elem_2)$ \\
       & and $l_2 = 1$ \\
       & and $d = elem_2\; 0$ \vspace{0.1cm} \\
       & and $v_1 = \left\{
	 \begin{array}{l l}
           elem_1\; d         & \quad \text{if $d \geq 0$}\\
           elem_1\; (l_1 + d) & \quad \text{if $d < 0$}
	 \end{array} \right.$ \\
	 
      \hspace{0.5cm} $[\mbox{LIST}_{\mbox{ACCESS-2}}]$ & \infrule{env_{T},
      env_{C}, env_{V} \vdash \lag e \rag \ra v_2  \qquad env_{T}, env_{C},
      env_{V} \vdash \lag i \rag \ra v_3}
      {env_{T}, env_{C}, env_{V} \vdash \lag e\; i \rag \ra v_1} \\
       & where $v_2 = \left(l_1, elem_1\right)$ \\
       & and $v_3 = (l_2,elem_2)$ \\
       & and $l_2 = 2$ \vspace{0.1cm} \\
       & and $d = \left\{
	 \begin{array}{l l}
           elem_2\; 0         & \quad \text{if $elem_2\; 0 \geq 0$}\\
           l_1 + elem_2\; 0   & \quad \text{if $elem_2\; 0 < 0$}
	 \end{array} \right.$ \vspace{0.1cm} \\
       & and $j = \left\{
	 \begin{array}{l l}
           elem_2\; 1         & \quad \text{if $elem_2\; 1 \geq 0$}\\
           l_1 + elem_2\; 1   & \quad \text{if $elem_2\; 1 < 0$}
	 \end{array} \right.$ \vspace{0.1cm} \\
       & and $elem_3\; z = \left\{
	 \begin{array}{l l}
           elem_1\; d+1       & \quad \text{if $z = 0$}\\
	   \hspace{1cm} \vdots &   \\
           elem_1\; d+n-1     & \quad \text{if $z = n-1$}
	 \end{array} \right.$ \vspace{0.1cm} \\
       & and $n=j-d+1,\; elem_3$ \\	 
       & and $v_1 = n$ \\
       & \\
       \hline
    \end{tabular*}
    \capt{Transition rules for accessing lists.}
    \label{semantic:lists}
  \end{center}
\end{table}



List access requires two rules, since two cases are possible. The first case, is
when just one offset is requested, then that element should be returned. In the
second case, two offsets are requested, and a range of elements should be
returned, also in the form of a list.

\subsection{Let expressions}
\label{sec:letexpressions}

Variables in \productname{} are assigned using let expressions:

\begin{ebnf}
\grule{expression}{\grange \gor let\_expr}
\grule{let\_expr}{\gter{let} \gcat variable \gcat \gter{=} \gcat expression
\gcat \grep{\gter{,} \gcat variable \gcat \gter{=} \gcat expression} \gnl
\gcat \gter{in} \gcat expression}
\end{ebnf}

A let expression consists of one or more assignments and an expression. Each
assignment assigns the value of an expression to a variable name. These
variables can then be used in the expression after the \texttt{in}-keyword.
After the evaluation of the let-expression, the variables cease to exist.

\subsubsection{Informal scope rules for let expressions}

Destructive assignments are not possible \productname{}, meaning that it isn't
possible to reassign a variable. It is however possible to hide a variable.
Consider the following expression:

\codesample{lethiding.garry}

The value of this expression is $13$. This is because within the
\texttt{\variable{x} + 2}-expression the \variable{x}-variable evaluates to $6$.
But in the outer expression \variable{x} evaluates to $5$.

Nested \emph{let}-scopes are possible. Consider for instance:

\codesample{nestedlet.garry}

In the inner scope, both \variable{x} and \variable{y} are available. This is of
course equivalent to:

\codesample{nestedlet2.garry}

\subsubsection{Big-step semantics}

The semantics presented in \tableref{semantic:let} are the transition rules for
the let expression.  This is transition rule is defined recursively to best
illustrate the functionality of the expression.


\begin{table}[ht]
  \begin{tabular*}{\textwidth}{l l}
    \hline \\
    \hspace{1.2cm} $[\mbox{LET-1}]$ & $env_{T}, env_{C}, env_{V}
    \vdash E_{1} \ra v_{1} \vspace{-0.3cm}$ \\
    & \infrule{env_{T}, env_{C}, env_{V}[x_{1} \mapsto v_{1}] \vdash \lag
    \texttt{let}\; x_{2} = E_{2}, \cdots,\; x_{k} = E_{k}\; \texttt{in}\;
    E_{k+1} \rag \ra v_{k+1}}
    {env_{T}, env_{C}, env_{V} \vdash \lag \texttt{let}\; x_{1} = E_{1},\; x_{2}
    = E_{2}, \cdots, x_{k} = E_{k}\; \texttt{in}\; E_{k+1} \rag \ra v_{k+1}} \\
    & where $k \geq 2$\\
    & \\
    
    \hspace{1.2cm} $[\mbox{LET-2}]$ & \hspace{0.4cm} $env_{T}, env_{C}, env_{V}
    \vdash E_{1} \ra v_{1}$ \vspace{-0.3cm} \\
    & \infrule{env_{T}, env_{C}, env_{V}[x_{1} \mapsto v_{1}] \vdash \lag
    E_{2}\rag \ra v_{2}} {env_{T}, env_{C}, env_{V} \vdash \lag \texttt{let}\;
    x_{1} = E_{1}\; \texttt{in}\; E_{2} \rag \ra v_{2}} \\
    \hline \\
  \end{tabular*}
  \capt{Transition rules for let expressions.}
  \label{semantic:let}
\end{table}


The transition rules for $[\mbox{LET-1}]$ is recursive because we must
evaluate each expression $(x_{1}=E_{1})$ individually before we move on to the next one. This
is a must because of the fact that the next expressions can in fact make use of
the previous expressions value. As an example take a look at the following code
sample:

\codesample{letbigstep.junta}

So, each call where there is more than one expression to be evaluated, 
the transition rule $[\mbox{LET-1}]$ where $k \geq 2$ is used. Here the expression first
in line to be evaluated will be evaluated before a new call to one of the two
transition rules is made. When we reach a let expression with only one
expression, we then call the transition rule $[\mbox{LET-2}]$ where $k < 2$.

\subsection{Conditional expressions}
\label{sec:conditionalexpressions}

It is often desirable to base the result of an expression on some sort of condition.
In \productname{} this is achievable using \emph{if}-expressions, as defined by the
following syntax:

\begin{ebnf}
\grule{expression}{\grange \gor if\_expr}
\grule{if\_expr}{\gter{if} \gcat expression \gcat \gter{then} \gcat expression
\gcat \gter{else} \gcat expression}
\end{ebnf}

Unlike in most imperative languages, the conditional construct in \productname{}
is not a statement (\productname{} doesn't have statements) but an expression.
Since all expressions must have a value, the \emph{else}-part of en
if-expression is compulsory.

The if-expression first evaluates the condition (the first expression). The
resulting value must be of type \type{Boolean}. If the value is equal to the
boolean true-value, the \emph{then}-expression is evaluated, and the result is
returned. If the value is false, then the \emph{else}-expression is evaluate,
and the result returned.

\subsection{Lambda expressions}
\label{sec:lambdaexpressions}

Lambda expressions are expressions that evaluate to anonymous functions. In
\productname{} they are defined as:

\begin{ebnf}
\grule{expression}{\grange \gor lambda\_expr}
\grule{lambda\_expr}{\gter{\#} \gcat varlist \gcat \gter{=>} \gcat expression}
\end{ebnf}

The non-terminal $varlist$ represents a list formal parameters, and is further
explained in \secref{sec:constantdefinitions}.

When a lambda expression is created, a reference to the scope it was created in
is saved with it. This is known as a closure, and means that a lambda
expression may access variables outside of its own scope. The accessible
variables are the variables that were available at the time of the creation of
the lambda expression.

Consider the following example:

\codesample{closuredef.garry}

The function \function{getAdder} takes one argument (\variable{a}) and
returns a lambda expression, is defined. Notice how \variable{a} is used within
the lambda expression. This means that when the lambda expression is created, it
must remember the value of the variables that exist in the scope, in which it is
created. The use of the \function{getAdder}-function could look like this:

\codesample{closureuse.garry}

In the first line \function{getAdder} is called with the argument, $25$. A new
scope, $A$, is created, in which the variable \variable{a} is assigned the value
$25$. Then the function expression is evaluated, which results in a new lambda
expression (with a reference to scope $A$).  This is returned and assigned to
\variable{adder} in line 1 of the above example.

In the second line, \variable{adder} is called as a function, meaning
that a new scope, $B$, is created, where the variable \variable{b} is
assigned the value $5$. The important part is that $B$'s parent scope is set to
$A$ (which is saved with the lambda expression). The expression (the right side
of the lambda expression) is then evaluated. First the \variable{a}-variable is
encountered. The interpreter first searches the $B$-scope for \variable{a}, and
when unsuccessful, searches the parent-scope, $A$, for \variable{a}. In $A$ the
variable \variable{a} holds the value $25$, and this is returned. Then the
$B$-scope is searched for the \variable{b}-variable, and the value $5$ is
returned. The two integers are added, and the final return-value of the
lambda-expression ends up being $30$.

\subsubsection{Big-step semantics}

The semantics presented in \tableref{semantic:lambda} is the transition rule for
lambda expressions.


\begin{figure}
\begin{center}
\begin{tikzpicture}[level/.style={sibling distance=30mm/#1}]
\node [square] {Lambda expression}
  child {node [square] {Variable list}}
  child {node [ellipse,draw] {\textit{Expression}}};
\end{tikzpicture}
\end{center}
\capt{The abstract syntax for the lambda expression node.}
\label{ast:lambdaexpr}
\end{figure}


The three environments ($env_{T}, env_{C}, env_{V}$) must be known before it is
possible to execute a lambda expression. We need to know which types, constants
and different variables are given in the specific scope.

The lambda expressions evaluates to a value $v$. The side condition
of the transition rule explains that $v$ is assigned the $4$-tuple, a
function value.

\subsection{Set expressions}
\label{sec:setexpressions}

Set expressions look a bit like let expressions, but are only applicable within
type definitions:

\begin{ebnf}
\grule{expression}{\grange \gor set\_expr}
\grule{set\_expr}{\gter{set} \gcat variable \gcat \gter{=} \gcat expression
\gcat \grep{\gter{,} \gcat variable \gcat \gter{=} \gcat expression}}
\end{ebnf}

Set expressions are used to ``modify'' the value of data-members in objects (see
\secref{sec:typedefinitions} for an explanation of data). Each variable in the
set expression must exist in the current type as data-members. Since modifying a
data-member would be a side-effect, which is not allowed, the set expression
instead returns a clone of the object, with the specified data-members set to
their respective values. This is useful for making setters (or something that
looks like setters). Consider for example the following type:

\codesample{setget.junta}

In the example above, the method \constant{setMyData} returns a new instance of
\type{MyType}, with the data-member \variable{myData} set to something else. The
following example shows the use of a getter and a setter:

\codesample{setget2.junta}

The call to \constant{setMyData} does not change the state of the original
instance of \type{MyType}, instead it returns a new instance.

\subsubsection{Big-step semantics}

The semantics presented in \tableref{semantic:set} is the transition rule for
set expressions.

\begin{table}[ht]
  \begin{tabular*}{\textwidth}{l l}
    \hline \\
    \hspace{1.5cm} $[\mbox{SET}]$ & \infrule{env_C, env_V, env_T \vdash \lag e_1
      \rag\ra u_1 \quad
    \ldots \quad env_C, env_V, env_T \vdash \lag e_k \rag \ra u_k}
    {env_C, env_V, env_T \vdash \lag \texttt{set}\; x_1 = e_1, \ldots, x_k =
    e_k \rag \ra v_1} \\
    & where $env_C\; \texttt{this} = \left(t, env'_C, env'_V, v_2 \right)$ \\
  & and $env''_V = env'_V \left[ x_1 \mapsto u_1, \ldots, x_k \mapsto u_k \right]$ \\
    & and $v_1 = \left( t, env'_C, env''_V, v_2\right)$ \\ 
    & \\
    \hline
  \end{tabular*}
  \capt{Transition rules for set expressions.}
  \label{semantic:set}
\end{table}


The transition rule assumes that the current constant environment $env_C$
contains a pointer to the current object, $\texttt{this}$. It then returns a
copy of that object with a new variable environment, containing the new data.

\subsection{Operators and calls}
\label{sec:operatorsandcalls}

Operators are useful for doing calculations, and \productname{} supports
basic mathematical operators and precedence. In order to prevent
left-recursion, which makes it possible for us to construct an LL-parser
(this was discussed in \secref{subsec:llparsersandlrparsers}), but
preserve left-associativity we have created a hierarchy of operators,
taking advantage of LL-parsing by putting the operators with highest
precedence the lowest in any parse tree that includes them. The grammar
for the operators of \productname{} are described using operator
sequences. A sequence is essentially just a list of operations on that
particular precedence level. In this way all the precedence levels of
\productname{} are described formally:

\begin{ebnf}
\grule{expression}{\grange \gor \gter{not} \gcat expression \gor lo\_sequence}
\grule{lo\_sequence}{eq\_sequence \gcat \grep{\ggrp{\gter{and} \gor \gter{or}}
\gcat eq\_sequence}}
\grule{eq\_sequence}{cm\_sequence \gcat \grep{\ggrp{\gter{==} \gor \gter{!=}
\gor \gter{is}} \gcat cm\_sequence}}
\grule{cm\_sequence}{as\_sequence \gcat \grep{\ggrp{\gter{<} \gor \gter{>} \gor
\gter{<=} \gor \gter{>=}} \gcat as\_sequence}}
\grule{as\_sequence}{md\_sequence \gcat \grep{\ggrp{\gter{+} \gor \gter{-}}
\gcat md\_sequence}}
\grule{md\_sequence}{negation \gcat \grep{\ggrp{\gter{*} \gor \gter{/} \gor
\gter{\%}} \gcat negation}}
\grule{negation}{element}
\galt{\gter{-} \gcat negation}
\grule{element}{call\_sequence \gcat \grep{member\_access}}
\grule{member\_access}{\gter{.} \gcat constant \gcat \grep{list}}
\grule{call\_sequence}{atomic \gcat \grep{list}}
\end{ebnf}

In order to completely understand this grammar, we must first take a look at the list of
operators ordered by precedence. The precedence of operators is presented in
\tableref{table:operatorPrecedence}.

\tab[\textwidth]{operatorPrecedence}{2}{The precedence of operators in \productname{}.}
                  {Operator precedence}
           {Level}{Operator & Description}{
    \tabrow{1}{\texttt{f[]} & Function/constructor invocation and list access}
    \tabrow{2}{\texttt{r.m r.m[]} & Record member access and member invocation}
    \tabrow{3}{\texttt{-} & Unary negation operation}
    \tabrow{4}{\texttt{* / \%} & Multiplication, division, and modulo}
    \tabrow{5}{\texttt{+ -} & Addition and subtraction}
    \tabrow{6}{\texttt{< > <= >=} & Comparison operators}
    \tabrow{7}{\texttt{== != is} & Equality operators and type checking}
    \tabrow{8}{\texttt{and or} & Logical $and$ and $or$}
    \tabrow{9}{\texttt{not} & Logical $not$}
    \tabrow{10}{\texttt{if let set \#} & if-, let-, set-, and lambda-expressions}
}

Each precedence level will correspond to a certain rule in the grammar. For
instance, the fifth precedence level for addition and subtraction is expressed
using the $as\_sequence$-rule. Combined with some multiplication an expression
making use of the $as\_sequence$- and $md\_sequence$-rules could look like
\csref{asandmd.junta}:

\codesample{asandmd.junta}

The resulting parse tree, using the grammar of \productname{}, could look
somewhat like the tree in \figref{fig:parsetreesequences}.

\begin{figure}[ht]
  \begin{center}
      \begin{tikzpicture}[]
	%the nodes
     	\node[square,xshift=4em]      			    (as)     {as\_sequence};
     	\node[circle,draw,yshift=-3em] 		    (plus1)  {$+$};
     	\node[circle,draw,yshift=-3em,xshift=8em]   (minus1) {$-$};
     	\node[circle,draw,yshift=-3em,xshift=16em]  (plus2)  {$+$};
     	\node[square,yshift=-3em,xshift=-8em]       (int1)   {integer $\left(2\right)$};
     	\node[square,yshift=-6em]      		    (md1)    {md\_sequence};
	\node[square,yshift=-6em,xshift=8em]        (int2)   {integer $\left(3\right)$};
     	\node[square,yshift=-6em,xshift=16em]       (md2)    {md\_sequence};
     	\node[square,yshift=-9em,xshift=-4em]      (int3)   {integer $\left(3\right)$};
     	\node[circle,draw,yshift=-9em,xshift=4em]  (mult1)  {$*$};
     	\node[square,yshift=-9em,xshift=12em]      (int4)   {integer $\left(2\right)$};
     	\node[circle,draw,yshift=-9em,xshift=20em] (div1)   {$/$};
     	\node[square,yshift=-12em,xshift=4em]       (int5)   {integer $\left(5\right)$};
     	\node[square,yshift=-12em,xshift=20em]      (int6)   {integer $\left(3\right)$};

	%the solution
	\node[rectangle,yshift=-14.5em,xshift=-8em] (a) {$2$};
     	\node[rectangle,yshift=-14.5em,xshift=-6em]     {$+$};
     	\node[rectangle,yshift=-14.5em,xshift=-4em] (b) {$3$};
     	\node[rectangle,yshift=-14.5em]                 {$*$};
     	\node[rectangle,yshift=-14.5em,xshift=4em]  (c) {$5$};
     	\node[rectangle,yshift=-14.5em,xshift=6em]      {$-$};
	\node[rectangle,yshift=-14.5em,xshift=8em]  (d) {$3$};
     	\node[rectangle,yshift=-14.5em,xshift=10em]     {$+$};
     	\node[rectangle,yshift=-14.5em,xshift=12em] (e) {$2$};
     	\node[rectangle,yshift=-14.5em,xshift=16em]     {$/$};
	\node[rectangle,yshift=-14.5em,xshift=20em] (f) {$3$};

	%first level
	\draw[-,-|,-,thin,] (as.south) |-+(0,-0.75em)-| (int1.north);
	\draw[-,-|,-,thin,] (as.south) |-+(0,-0.75em)-| (plus1.north);
	\draw[-,-|,-,thin,] (as.south) |-+(0,-0.75em)-| (minus1.north);
	\draw[-,-|,-,thin,] (as.south) |-+(0,-0.75em)-| (plus2.north);

	%second level
	\draw[-,thin,] (plus1.south) -- (md1.north);
	\draw[-,thin,] (minus1.south) -- (int2.north);
	\draw[-,thin,] (plus2.south) -- (md2.north);

	%third level
	\draw[-,-|,-,thin,] (md1.south) |-+(0,-0.75em)-| (int3.north);
	\draw[-,thin,] (md1.south) |-+(0,-0.75em)-| (mult1.north);
	\draw[-,-|,-,thin,] (md2.south) |-+(0,-0.75em)-| (int4.north);
	\draw[-,thin,] (md2.south) |-+(0,-0.75em)-| (div1.north);

	%fourth level
	\draw[-,thin,] (mult1.south) -- (int5.north);
	\draw[-,thin,] (div1.south) -- (int6.north);

	%solution level
	\draw[-,dashed,] (int1) -- (a);
	\draw[-,dashed,] (int2) -- (d);
	\draw[-,dashed,] (int3) -- (b);
	\draw[-,dashed,] (int4) -- (e);
	\draw[-,dashed,] (int5) -- (c);
	\draw[-,dashed,] (int6) -- (f);

    \end{tikzpicture}
  \end{center}
  \capt{A parse tree for the expression $2 + 3 * 5 - 3 + 2 / 3$.}
  \label{fig:parsetreesequences}
\end{figure}


The figure clearly shows the precedence, because the lower nodes will be
evaluated before the nodes that are higher in the parse tree. E.g.\ all of the
multiplication and division nodes will be calculated before the additions and
subtraction. This hierarchy is achieved because of the formally described
precedence rules in the previous grammar about sequences.

\subsubsection{Big-step semantics}

The semantics presented in \tableref{semantic:callfun} is transition
rule for function calls. Big-step semantics for the others are left out,
as they are mostly trivial.

\begin{table}[ht]
  \begin{tabular*}{\textwidth}{l l}
    \hline \\
    \hspace{3cm} $[\mbox{CALL}_{\mbox{FUN}}]$ & \hspace{0.1cm} $env_C, env_V,
    env_T \vdash \lag e \rag \ra v_2$ \\
    & \hspace{0.1cm} $env_C, env_V, env_T \vdash \lag i \rag \ra v_3$
    \vspace{-0.3cm} \\
    & \infrule{env'_C, env''_V, env_T \vdash \lag e' \rag \ra v_1}{env_C, env_V,
    env_T \vdash \lag e\; i\; \rag \ra v_1} \\
    & where $v_2 = \left(g, e', env'_V, env'_C\right)$ \\
    & and $v_3 = \left(l, elem\right)$ \\
    & and $env''_V = \left[x_1 \mapsto elem\; 1, \ldots, x_n \mapsto elem\; n \right]$ \\
    & \\
    \hline
  \end{tabular*}
  \capt{Transition rules for function calls.}
  \label{semantic:callfun}
\end{table}



In this rule the expression $E$ is evaluated to a function value, $v_2$, which
has its own variable and constant environments, as per the static scope rules.
The variable environment is then updated with the actual parameters assigned to
the formal parameters, after which the expression, contained within the function
value, is evaluated.

The semantics presented in \tableref{semantic:memaccess} is the transition rule
for member access, also known as dot-notation.

\begin{table}[ht]
  \begin{tabular*}{\textwidth}{l l}
    \hline \\
    \hspace{3cm} $[\mbox{MEMBER}_{\mbox{ACCESS}}]$ & \infrule{env_C, env_V, env_T
    \vdash \lag e \rag \ra v_1}{env_C, env_V, env_T \vdash \lag e\texttt{.}C
  \rag \ra v_3} \\
     & where $v_2 = \left(t, env'_C, env'_V, v_2 \right)$ \\
     & and $env'_C\; C = v_3$ \\
     & \\
     \hline
  \end{tabular*}
  \capt{Transition rules for member access.}
  \label{semantic:memaccess}
\end{table}



A member access is as simple as evaluating the left-side of the object
first (the constant), and then accessing the evaluated constant in that
objects constant environment.

\subsubsection{Valid operands}
\label{sec:validoperands}
In the following paragraphs we present the valid operands of
\productname{}. These will be grouped in the following categories; boolean,
comparison, integer, string, list, and direection and coordinate operators.

\paragraph{Boolean operators}

These operators only accept boolean operands and only return boolean values:

\begin{dlist}
  \item \operator[Boolean]{and}{Boolean}{Boolean}\\
    Returns true when both operands are true and false otherwise. 
  \item \operator[Boolean]{or}{Boolean}{Boolean}\\
    Returns true when at least one of the operands are true and false otherwise.
  \item \operator{not}{Boolean}{Boolean}\\
    Returns true if the single operand is false and false otherwise.
\end{dlist}

\paragraph{Comparison operators}

These operators are used to compare two values, and always returns a boolean value:

\begin{dlist}
  \item \operator[Integer]{<}{Integer}{Boolean}\\
    Returns true if the left operand is less than the right one.
  \item \operator[Integer]{>}{Integer}{Boolean}\\
    Returns true if the left operand is greater than the right one.
  \item \operator[Integer]{<=}{Integer}{Boolean}\\
    Returns true if the left operand is less than or equal to the right one.
  \item \operator[Integer]{>=}{Integer}{Boolean}\\
    Returns true if the left operand is greater than or equal to the right one.
  \item \operator[\opstar]{==}{\opstar}{Boolean}\\
    Returns true if the left operand is equal to the right one.
  \item \operator[\opstar]{!=}{\opstar}{Boolean}\\
    Returns true if the left operand is not equal to the right one.
  \item \operator[\opstar]{is}{Type}{Boolean}\\
    Returns true if the type of the first operand is equal to or inherits from
    the type operand.
\end{dlist}

\paragraph{Integer operators}

The following operations are possible on integers:

\begin{dlist}
  \item \operator{-}{Integer}{Integer} \\
    Integer negation.
  \item \operator[Integer]{+}{Integer}{Integer} \\
    Integer addition.
  \item \operator[Integer]{-}{Integer}{Integer} \\
    Integer subtraction.
  \item \operator[Integer]{*}{Integer}{Integer} \\
    Integer multiplication.
  \item \operator[Integer]{/}{Integer}{Integer} \\
    Integer division.
  \item \operator[Integer]{\%}{Integer}{Integer} \\
    Integer modulo operation.
\end{dlist}

\paragraph{String operators}

It is possible to concatenate strings:

\begin{dlist}
  \item \operator[String]{+}{String}{String} \\
    Returns the concatenation of two strings.
  \item \operator[String]{+}{\opstar}{String} \\
   \operator[\opstar]{+}{String}{String} \\
    Returns the concatenation of a string and the string-representation of another type
\end{dlist}

\paragraph{List operators}

Some operators are available for list values as well:

\begin{dlist}
\item \operator[List]{+}{List}{List} \\
  Returns a list containing all elements from the first list followed
  by all elements from the second list.
\item \operator[List]{-}{List}{List} \\
  Returns a list containing all the elements from the first list that
  do not exist in the second list.
\item \operator[List]{+}{\opstar}{List} \\
  Appends any element on to the end of a list, and returns the resulting list.
\item \operator[List]{-}{\opstar} \\
  Returns a list containing the elements that do not match the right operand.
\item \operator[\opstar]{+}{List}{List} \\
  Prepends any element on to the start of a list, and returns the resulting list.
\end{dlist}

\paragraph{Direction and coordinate operators}

The following operators can manipulate directions and coordinates:

\begin{dlist}
  \item \operator[Direction]{+}{Direction}{Direction} \\
    Add a direction (vector) to another direction.
  \item \operator[Direction]{-}{Direction}{Direction} \\
    Subtract a direction from another direction.
  \item \operator[Direction]{+}{Coordinate}{Coordinate} \\
    Add a coordinate to a direction.
  \item \operator{-}{Direction}{Direction} \\
    Negate a direction.
  \item \operator[Coordinate]{-}{Coordinate}{Direction} \\
    Returns the distance between two coordinates as a direction.
  \item \operator[Coordinate]{+}{Direction}{Coordinate} \\
    Add a direction to a coordinate. 
  \item \operator[Coordinate]{-}{Direction}{Coordinate} \\
    Subtract a direction from a coordinate.
\end{dlist}

For instance adding the directions \texttt{n} and \texttt{e} produces a
direction equivalent with the direction \texttt{ne}. Adding a coordinate and
direction (and vice versa) gives a coordinate. As an example, $\texttt{A2}
\verb!+! \texttt{e}$ gives \texttt{B2}. More information about the coordinate
and direction types is available in \secref{sec:standardenvironment}.


%
\begin{figure}[ht]
\begin{center}
\begin{tikzpicture}[level/.style={sibling distance=30mm/#1}]
\node [ellipse, draw] {Literals}
  child {node [square,xshift=1.5cm] {Integer} edge from parent [dashed];}
  child {node [square,xshift=0.4cm] {Direction} edge from parent [dashed];}
  child {node [square,xshift=-0.4cm] {Coordinate} edge from parent [dashed];}
  child {node [square,xshift=-1.5cm] {String} edge from parent [dashed];};
\end{tikzpicture}
\end{center}
\capt{The abstract node type for the literals.}
\label{ast:literal}
\end{figure}

%
\begin{figure}[ht]
\begin{center}
\begin{tikzpicture}[level/.style={sibling distance=30mm/#1}]
\node [square] {Pattern keyword}
	child {node [square, xshift=1.8cm] {friend} edge from parent [dashed];}
	child {node [square] {foe} edge from parent [dashed];}
	child {node [square, xshift=-1.8cm] {empty} edge from parent [dashed];};
\end{tikzpicture}
\end{center}
\capt{The abstract syntax tree for the variable list node.}
\label{ast:patternkeyword}
\end{figure}

%
\begin{figure}[ht]
\begin{center}
\begin{tikzpicture}
\node [square] {Variable};
\end{tikzpicture}
\end{center}
\capt{The abstract syntax tree for the single node variable.}
\label{ast:variable}
\end{figure}


%literals
%variables
%identifiers
%keywords
%pattern keywords
%expressions
%elements (this is a special keyword)
