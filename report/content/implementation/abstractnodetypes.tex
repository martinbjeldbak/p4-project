\section{Abstract node types}
\label{sec:ant}

Abstractions are the interpreter's or compiler's internal representation of a program. It is represented as an abstract syntax tree which is a series of nodes and leaves connected, forming a so-called tree.%

This section covers how we have implemented the grammar into abstract syntax trees (ASTs), and it begins with a description of how an AST differs from a parse tree. 
The grammar specifications are specified in \secref{sec:grammar}. In this section each program part is presented individually with an AST within its respective subsection. 

The section begins with the structure of programs. Following these subsections is a section presenting the expressions of the grammar. Then we shortly present the patterns of the grammar.

When a piece of code is parsed by a parser that understands the specific programming language, the output of the parser will be an AST which consists of the abstract node types of each program part. The difference between an AST and a parse tree is that a parse tree contains every detail of the input code. A parse tree includes for instance the parentheses and keywords where an AST does not contain anything other than the abstract node types.\cite{parsevsast}%

\subsection{Program structure}
The following 7 subsections present the structure of programs. The sections are structured as follows:

\begin{dlist}
\item Program
\item Definition
\item Constant definition
\item Type definition
\item Type body and member definition
\item Abstract definition
\item Variable list
\end{dlist}

\subsubsection{Program}%
Every program written in \productname{} begins with an abstract node type which we call ``program'' that consists of either zero, one, or more definitions. The production for this rule is a follows:%

\begin{ebnf}%
\grule{program}{\grep{definition}}
\end{ebnf}%

It is from this production each and every program is derived from. The AST for this abstract node type is illustrated in \figref{ast:program}.%


\begin{figure}[ht]
\begin{center}
\begin{tikzpicture}[level/.style={sibling distance=40mm/#1}]
\node [square] {Program}
  child {node [square,xshift=0.5cm] (a) {\textit{Definition}} edge from parent[dashed];}
  child {node [square,xshift=-0.5cm] (b) {\textit{Definition}} edge from parent[dashed];};
\path (a)--(b) node [midway] {$\cdots$};
\end{tikzpicture}
\end{center}
\capt{The abstract syntax tree for the program node.}
\label{ast:program}
\end{figure}
%

Figure \ref{ast:program} consists of one root which is called ``Program'' and this root can have zero, one, or more children, called ``Definition''. The children are optional because the production says that a program can consist of either zero, one, or more of these definitions. The AST illustrates this by making the connecting lines dashed. Within these two nodes there are three dots ($\cdots$) which illustrate that it is possible to have more of these abstract node types following each other.%

This means that a program is legal if it does not contain anything at alle.

\subsubsection{Definition}%
Programmars have the opportunity of defining constants and types in their programs. The parser knows that the following piece of code is a constant when it meets the keyword \gter{define}, and that the following piece of code is a type when it meets the keyword \gter{type}, which will be illustrated in the productions of the the two productions. The production for a defintion is as follows:

\begin{ebnf}%
\grule{definition}{constant\_def}
\galt{type\_def}
\end{ebnf}%

%\input{content/implementation/abstractnodetypes/definition}%funcdef

We do not present a figure to illustrate the AST for this production.
The production says that a defintion can either be a defintion of a constant or a type. But the ``program'' node says that it is possible to have multiple definitions, so in fact the definitions of constants and types can be interleaved.

\subsubsection{Constant definition}%
The production for a constant defintion is as follows:

\begin{ebnf}%
\grule{constant\_def}{\gter{define} \gcat constant \gcat \gopt{varlist} \gcat \gter{=} \gcat \gcat expression}
\end{ebnf}%

Every definition of this type begins with the \gter{define} keyword. The next token must be a constant and this constant can be followed by a list of variables. Then an expression is assigned to the constant.

\begin{figure}[ht]
\begin{center}
\begin{tikzpicture}[level/.style={sibling distance=30mm/#1}]
\node [square] {Constant definition}
  child {node [square,xshift=0.7cm] {Constant}}
  child {node [square] {Variable list} edge from parent [dashed]}
  child {node [square,xshift=-0.7cm] {\textit{Expression}}};
\end{tikzpicture}
\end{center}
\capt{The abstract syntax tree for the constant definition node.}
\label{ast:constdef}
\end{figure}
%gamedecl

Figure \ref{ast:constdef} illustrates the AST for the constant definition. The keyword in front of the definition and the assignment opreator, are not shown because they are not part of the AST.

\subsubsection{Type definition}%

The type definition consists of at least one. The parser knows that the following is a type definition by reading the \gter{type} keyword. The following production specifies this:%

\begin{ebnf}%
\grule{type\_def}{\gter{type} \gcat type \gcat varlist \gcat \gopt{\gter{extends} \gcat type \gcat list} \gopt{type\_body}}
\end{ebnf}%

The production says that a type definition can inheret from another type by the use of the \gter{extends} keyword. Furthermore, it is possible to have a type body which is explained further in the following section.

\input{billeder/tikz/typedef}


%declstruct

Figure \ref{ast:typedef} illustrates the abstract node type of the type definition. The abstract node type consists of a type followed by a variable list. After the variable list there are two optional choices. The first choice is to inheret from a type, and this is done by declaring the type followed by a declaration of a list. The second choice is to include a type body.

\subsubsection{Type body and member definition}

In this subsection we have coupled two productions because they are intertwined and we have choosen not to visualise the abstract node type of the member defintion because it is simply a root with one child.

The production of a type definition and the definition of a member is as follows:

\begin{ebnf}
\grule{type\_body}{\gter{\{} \gcat \grep{member\_def} \gcat \gter{\}}}
\grule{member\_def}{abstract\_def}
\galt{constant\_def}
\end{ebnf}

The type body begins with a left brace and ends with a right brace which encapsulates the entire code of the type body. Within these braces it is possible to have zero, one, or more member definitions. The code which consists of a member definition can be either an abstract defintion or a constant definition. As it was in the ``program'' production it is also to interleave the two definitions in this production.

\begin{figure}[ht]
\begin{center}
\begin{tikzpicture}[level/.style={sibling distance=30mm/#1}]
\node [square] {Type body}
  child {node [square,xshift=-0.2cm] (a) {\textit{Member definition}}}
  child {node [square,xshift=0.2cm] (b) {\textit{Member definition}}};
  
\path (a)--(b) node [midway] {$\cdots$};
\end{tikzpicture}
\end{center}
\capt{The abstract syntax tree for the type body node.}
\label{ast:typebody}
\end{figure}

Figure \ref{ast:typebody} illustrates the abstract node type of the type body. It consists of a root and zero, one, or more member definitions.

A description of the constant defintion was presented earlier in this section and the description of the abstract will follow this subsection.

\subsubsection{Abstract definition}%
The defintion of an abstract constant is similar to the definition of a ``regular'' constant. There are two differences. The first difference is that following the \gter{define} keyword another keyword must appear, namely the \gter{abstract} keyword. The second difference is that there is no expression assigned at the end of the production. Thus being abstract. The following production presents the production from the grammar:

\begin{ebnf}%
\grule{abstract\_def}{\gter{define} \gcat \gter{abstract} \gcat constant \gcat \gopt{varlist}}
\end{ebnf}%

Figure \ref{ast:abstrdef} illustrates the abstract node type of this production.


\begin{figure}[ht]
\begin{center}
\begin{tikzpicture}[level/.style={sibling distance=30mm/#1}]
\node [square] {Abstract definition}
  child {node [square,xshift=0.3cm] {Constant}}
  child {node [square,xshift=-0.3cm] {Variable list} edge from parent[dashed]};
\end{tikzpicture}
\end{center}
\capt{The abstract syntax tree for the abstract definition node.}
\label{ast:abstrdef}
\end{figure}
%decl

In \figref{ast:abstrdef} the only noticeable difference between the constant defintion and this definition is that there is no expression as the last child. This is due to that the abstract node types do not have keywords and such as child nodes.

\subsubsection{Variable list}%
The production for the variable list is presented in the following production:%

\begin{ebnf}%
\grule{varlist}{\gter{[} \gcat \gopt{variable \gcat \grep{\gter{,} \gcat variable} \gcat \gopt{\gter{,} \gcat vars} \gor vars} \gcat \gter{]}}
\grule{vars}{\gter{...} \gcat variable}
\end{ebnf}%

The production specifies that a variable list begins with an occurence of a left bracket and ends with an occurence of a right bracket. The brackets in between the outer bracket mean that the variables are optional. This means that a variable list can be empty.%

The last part of the production is a bit hard to read, but it says that if the list begins with one or more variables separated by commas then these variables can be followed by a final comma and the variable argument (vars) which is a list of parameters. If there are no variables then the last variable argument can be included. So, the production consists of two parts - one part containing zero, one or more variables possibly followed by an occurence of variable argument or just a single variable argument.


\begin{figure}
\begin{center}
\begin{tikzpicture}[level/.style={sibling distance=30mm/#1}]
\node [square] {Variable list}
  child {node [square, xshift=0.5cm] (a) {Variable} edge from parent[dashed];}
  child {node [square] (b) {Variable} edge from parent[dashed];}
  child {node [square, xshift=-0.5cm] {Variable arguments} edge from parent[dashed];};
\path (a)--(b) node [midway] {$\cdots$};
\end{tikzpicture}
\end{center}
\capt{The abstract syntax tree for the variable list node.}
\label{ast:variablelist}
\end{figure}
%

Figure \ref{ast:variablelist} illustrates this by having a root and three children with dashed lines from the root. But this is actually a bit misleading because of the optional parameters in the specification of the production. The production specifies that if the programmer does not want the list to be empty then at least one occurence of a variable must be present or a single variable argument. After the first variable it is possible to have zero, one, or more variables following the first one and another optional variable argument. The easiest way to illustrate this is as it has been done in \figref{ast:variablelist}.%

\subsection{Expressions}
The following eight subsections present the different expressions in \productname{}. The sections are structured as follows:

\begin{dlist}
\item Operations with precedence and negation
\item Element
\item Member access
\item Call sequence
\item Assignment
\item If expression
\item Lambda expression
\item List
\end{dlist}

\subsubsection{Operations with precedence and negation}
In this section we introduce the five different groupings of operations such as logical operators, equality operators, etc., and finally we present the production called negation. We illustrate the five groupings of operators but we have omitted the negation node, because it is merely a root with one child. The following grammar presents the productions for the different operations:

\begin{ebnf}
\grule{lo\_sequence}{eq\_sequence \gcat \grep{\ggrp{\gter{and} \gor \gter{or}} \gcat eq\_sequence}}
\grule{eq\_sequence}{cm\_sequence \gcat \grep{\ggrp{\gter{==} \gor \gter{!=}} \gcat cm\_sequence}}
\grule{cm\_sequence}{as\_sequence \gcat \grep{\ggrp{\gter{<} \gor \gter{>} \gor \gter{<=} \gor \gter{>=}} \gcat as\_sequence}}
\grule{as\_sequence}{md\_sequence \gcat \grep{\ggrp{\gter{+} \gor \gter{-}} \gcat md\_sequence}}
\grule{md\_sequence}{negation \gcat \grep{\ggrp{\gter{*} \gor \gter{/} \gor{\%}} \gcat negation}}
\end{ebnf}

The meaning of these different groupings of operators have been described in \secref{sec:grammar}. To refresh the readers memory, the sequences are intentionally placed in this specific order to ensure the correct precedence for these operators.

Since the productions look a lot like each other we will only illustrate an abstract AST which shows how we have implemented the different operations. Figure \ref{ast:operation} shows this.


\begin{figure}[ht]
\begin{center}
\begin{tikzpicture}[level/.style={sibling distance=30mm/#1}]
\node [square] {\textit{LHS}}
  child {node [square] {\textit{RHS}}}
  child {node [square] {\textit{RHS}}}
  child {node [square] (a) {\textit{RHS}} edge from parent[dashed];}
  child {node [square] (b) {\textit{RHS}} edge from parent[dashed];};
  
\path (a)--(b) node [midway] {$\cdots$};
\end{tikzpicture}
\end{center}
\end{figure}


Figure \ref{ast:operation} illustrates that the node for each production, with the left-hand side (LHS) of the production as the root, will only be constructed if there is at least two of the right-hand sides (RHS) with one operator between them. This means that the sequence should look like the following:

\begin{ebnf}
\grule{lo\_sequence}{eq\_sequence \gcat \gter{and} \gcat eq\_sequence}
\end{ebnf}

So, in the above example the RHS is $eq\_sequence$ and the operator is $\gter{and}$. In \figref{ast:operation} we do not show the operators, which can make it a bit cryptic to look at and understand. If there is only one RHS then the node is not constructed but the next production will be evaluated. Otherwise the AST would end up with many single-child nodes. This is illustrated in \figref{ast:badexample}.


\begin{figure}[ht]
\begin{center}
\begin{tikzpicture}[level 3/.style={sibling distance = 20mm}]

\node [square] {\textbf{LHS}}
  child {node [square,yshift=0.5cm] {\textbf{RHS}}
    child {node [square,yshift=0.5cm] {\textbf{RHS}}
      child {node [square,yshift=0.5cm] {\textbf{RHS}}}
      child {node [square,yshift=0.5cm] (a) {\textbf{RHS}} edge from parent[dashed]}
      child {node [square,yshift=0.5cm] (b) {\textbf{RHS}} edge from parent[dashed]}
    }
  }; 

\path (a)--(b) node [midway] {$\cdots$};
\end{tikzpicture}
\end{center}
\capt{An example of why we need at least two RHS before we construct a node.}
\label{ast:badexample}
\end{figure}


Figure \ref{ast:badexample} illustrates this very clearly. We will end up with a long list, which is not necessary and it will just make it less efficient to read the AST. With our implementation we will have a more efficient and more compact AST.

The grammar specified earlier in this section that the last operation ($md\_sequence$) consists of negations and the choice to add an operator between negations. The negation production can be an element or begin with the \gter{-} symbol followed by another negation.
The following productions present the grammar for the negation expression:

\begin{ebnf}
\grule{negation}{element}
\galt{\gter{-} \gcat negation}
\end{ebnf}

\subsubsection{Element}

The following production presents the grammar for an element:

\begin{ebnf}%
\grule{element}{call\_sequence \gcat \grep{member\_access}}
\end{ebnf}%

An element begins with a call sequence followed by zero, one, or more member access. This is illustrated in \figref{ast:element}.


\begin{figure}[ht]
  \begin{center}\scalebox{0.7}{
\begin{tikzpicture}[level/.style={sibling distance=30mm/#1}]
\node [square] {Element}
  child {node [square, xshift=-1cm] {Call sequence}}
  child {node [square, xshift=-0.3cm] {Member access}}
  child {node [square,xshift=0.3cm] (b) {Member access} edge from parent[dashed]}
  child {node [square,xshift=1cm] (c) {Member access} edge from parent[dashed]};
  
\path (b)--(c) node [midway] {$\cdots$};
\end{tikzpicture}}
\end{center}
\end{figure}


Figure \ref{ast:element} illustrates the element expression which reflects the grammar.

\subsubsection{Member access}

The member access begins with the \gter{.} symbol followed by a constant and zero, one, or more lists. This is presented in the following production:

\begin{ebnf}%
\grule{member\_access}{\gter{.} \gcat constant \gcat \grep{list}}
\end{ebnf}%

Figure \ref{ast:memberaccess} illustrates the abstract node type for this production. It corresponds exactly to the production.

\begin{table}[ht]
  \begin{tabular*}{\textwidth}{l l}
    \hline \\
    \hspace{3cm} $[\mbox{MEMBER}_{\mbox{ACCESS}}]$ & \infrule{env_C, env_V, env_T
    \vdash \lag e \rag \ra v_1}{env_C, env_V, env_T \vdash \lag e\texttt{.}C
  \rag \ra v_3} \\
     & where $v_2 = \left(t, env'_C, env'_V, v_2 \right)$ \\
     & and $env'_C\; C = v_3$ \\
     & \\
     \hline
  \end{tabular*}
  \capt{Transition rules for member access.}
  \label{semantic:memaccess}
\end{table}



\subsubsection{Call sequence}

A call sequence consists of an atomic value, which can be many expressions, followed by zero, one, or more lists. The following product presents this:

\begin{ebnf}%
\grule{call\_sequence}{atomic \gcat \grep{list}}
\end{ebnf}%

We have illustrated the abstract node type of the call sequence and it is presented in \figref{ast:callsequence}. It corresponds exactly to the  production.


\begin{figure}[ht]
\begin{center}
\begin{tikzpicture}[level/.style={sibling distance=30mm/#1}]
\node [square] {Call sequence}
  child {node [square,xshift=2.3cm] {\textit{Atomic}}}
  child {node [square,xshift=0.8cm] {List}}
  child {node [square,xshift=-0.8cm] (a) {List} edge from parent[dashed];}
  child {node [square,xshift=-2.3cm] (b) {List} edge from parent[dashed];};
  
\path (a)--(b) node [midway] {$\cdots$};
\end{tikzpicture}
\end{center}
\capt{The abstract syntax tree for the call sequence node.}
\label{ast:callsequence}
\end{figure}


\subsubsection{Assignment}%

An assignment is actually one of the many different abstract node types an expression can be. The following grammar rule specifies the production of an assignment in \productname{}:%

\begin{ebnf}%
\grule{assignment}{\gter{let} \gcat variable \gcat \gter{=} \gcat expression \gcat \grep{\gter{,} \gcat variable \gcat \gter{=} \gcat expression} \gcat \gter{in} \gcat expression}
\end{ebnf}%

The production specifies that any assignment must begin with the keyword \gter{let} and end with the keyword \gter{in} followed by an expression. In between these beginning and ending keywords the production consists of at least one pair of a variable followed by an assignment-operator followed by an expression. The production specifies that it is possible to have zero, one, or more of these variable-expression pairs (comma-separated) following the first pair.%


\begin{figure}
\begin{center}
\begin{tikzpicture}
[level/.style={sibling distance=40mm},
level 1/.style={sibling distance = 39mm},
level 2/.style={sibling distance = 20mm}]

\node [square] (z) {Assignment}
  child {node [square,left of=b,xshift=-4cm] (a) {Variable}}
  child {node [ellipse,draw,left of=c,xshift=-4.5cm] (b) {\textit{Expression}}}
  child {node [square] (c) {Assignment} edge from parent[dashed]
  	child {node [square,xshift=-1cm] (x) {Variable} edge from parent[solid]}
  	child {node [ellipse,draw,solid,xshift=-1cm] (y) {\textit{Expression}} edge from parent[solid]}
  }
  child {node [square,xshift=-1cm] (d) {Assignment} edge from parent[dashed]
  	child {node [square,xshift=1cm] (o) {Variable} edge from parent[solid]}
  	child {node [ellipse,draw,solid,xshift=1cm] (p) {\textit{Expression}} edge from parent[solid]}
  }
  child {node [ellipse,draw,right of=d,xshift=1.5cm](e) {\textit{Expression}}};

\path (c)--(d) node [midway] {$\cdots$};
\end{tikzpicture}
\end{center}
\capt{The abstract syntax for the assignment node.}
\label{ast:assignment}
\end{figure}
%

Figure \ref{ast:assignment} illustrates this with an AST that omits the keywords, commas and the assignment-operators which can make it rather complex to look at. The figure actually states that an assignment consists of variables and expression where an expression can be many different things, including another assignment. So, this means that it is possible to have assignments nested within each other. These nested assignments (with \textbf{Comma}) are connected to the root with dashed lines which means that they are optional and it is possible to have zero, one, or more of these following each other.%

\subsubsection{If expression}%

The if expression is yet another expression as the name indicates. These expressions have one form and it is specified in the following production:%

\begin{ebnf}%
\grule{if\_expr}{\gter{if} \gcat expression \gcat \gter{then} \gcat expression \gcat \gter{else} \gcat expression}%
\end{ebnf}%

An if expression begins with the keyword ``if'' and ends with the keyword ``else'' followed by an expression. In between the beginning and the end the keyword ``then'' appears as a must in the if expression. It is for instance not possible to omit the ``else'' statement. The AST for this production is illustrated in \figref{ast:ifexpr} and consists of a root with three children which are expressions. The children are connected to the root with solid lines which means they cannot be omitted.%


\begin{figure}
\begin{center}
\begin{tikzpicture}[level/.style={sibling distance=30mm/#1}]
\node [square] {If expression}
  child {node [label={[xshift=1.2cm, yshift=0.2cm]\textbf{if}}] [ellipse,draw] {\textit{Expression}}}
  child {node [label={[xshift=-0.38cm, yshift=0.2cm]\textbf{then}}] [ellipse,draw] {\textit{Expression}}}
  child {node [label={[xshift=-2.2cm, yshift=0.2cm]\textbf{else}}] [ellipse,draw] {\textit{Expression}}};
\end{tikzpicture}
\end{center}
\capt{The abstract syntax tree for the if expression node.}
\label{ast:ifexpr}
\end{figure}
%

\subsubsection{Lambda expression}%

The next expression is a lambda expression and is specified in the following production:%

\begin{ebnf}%
\grule{lambda\_expr}{\gter{\#} \gcat varlist \gcat \gter{=>} \gcat expression}%
\end{ebnf}%

Any lambda expression begins with the \# symbol which makes it clear for the parser that this is a lambda expression. A lambda expression consists of a variable list and an expression seperated by the $=>$ symbol.%

\begin{table}[ht]
  \begin{tabular*}{\textwidth}{l l c}
    \hline \\
    \hspace{0.5cm} $[\mbox{LAMBDA}]$ & $env_{T}, env_{C}, env_{V} \vdash \lag
    \texttt{\#} \; g\;
    \texttt{=>}\; e \rag \ra v$ & \hspace{1cm} where $v = \left(g, e, env_{V}, env_{C}\right)$ \\
    & & \\
    \hline
  \end{tabular*}
  \capt{Transition rules for lambda expressions.}
  \label{semantic:lambda}
\end{table}

%

Figure \ref{ast:lambdaexpr} illustrates the AST for this expression. Because a lambda expression is an expression it is possible to have lambda expression inside other lambda expressions, so it is possible to nest these expressions.%

\subsubsection{List}%

We also have an abstract node type of the list element. Any list can be empty just like a variable list but instead of containing variables a list contains expressions. It can consist of either zero, one, or more expressions. The specification for the abstract node type is specified in the following production:%

\begin{ebnf}%
\grule{list}{\gter{[} \gcat \gopt{expression \gcat \grep{\gter{,} \gcat expression}} \gcat \gter{]}}%
\end{ebnf}%

This element begins and ends like a variable list with brackets with optional elements as the elements of the list. The AST for this production is illustrated in \figref{ast:list}.%


\begin{figure}
\begin{center}
\begin{tikzpicture}[level/.style={sibling distance=30mm/#1}]
\node [square] {List}
  child {node [ellipse,draw] (a) {\textit{Element}} edge from parent[dashed]}
  child {node [ellipse,draw] (b) {\textit{Element}} edge from parent[dashed]};

\path (a)--(b) node [midway] {$\cdots$};
\end{tikzpicture}
\end{center}
\capt{The abstract syntax for the list node.}
\label{ast:list}
\end{figure}
%

The AST in \figref{ast:list} lacks the same descpritiveness as the AST for the variable list does because of the optional parameters in the production. The AST illustrates that it is possible to have zero, one, or more expressions in the list, but it is in fact not entirely correct. The correct statement is that it is possible to have either none or at least one expression in the list.%

\subsection{Pattern}%

The next element is a pattern which consists of a single pattern expression followed by zero, one, or more pattern expressions. This is specified in the following production:%

\begin{ebnf}%
\grule{pattern}{pattern\_expr \gcat \grep{pattern\_expr}}%
\end{ebnf}%

This production is very straightforward, a pattern must contain at least one pattern expression. The AST for this production is illustrated in \figref{ast:pattern}. This AST has one solid line to the first pattern expression followed by two dashed lines to two other pattern expressions illustrating that these can occur zero, one, or more times.%


\begin{figure}[ht]
\begin{center}
\begin{tikzpicture}[level/.style={sibling distance=40mm/#1}]
\node [square] {Pattern}
  child {node [square,xshift=0.5cm] {\textit{Pattern expression}}}
  child {node [square] (b) {\textit{Pattern expression}} edge from parent[dashed]}
  child {node [square,xshift=-0.5cm] (c) {\textit{Pattern expression}} edge from parent[dashed]};

\path (b)--(c) node [midway] {$\cdots$};
\end{tikzpicture}
\end{center}
\capt{The abstract syntax tree for the pattern node.}
\label{ast:pattern}
\end{figure}
%

\subsubsection{Pattern operators}%

\productname{} consists of three different pattern operators. We have an or-operator, a multiplier-operator, and a not-operator .

The or-operator derives a pattern value followed by a pattern expression that both can be many different things. This is why they are written in an italic font. This is illustrated in \figref{ast:pattern-or}.%


\begin{figure}[ht]
\begin{center}
\begin{tikzpicture}[level/.style={sibling distance=40mm/#1}]
\node [square] {Pattern, or-operator}
  child {node [ellipse,draw,xshift=0.2cm] {\textit{Pattern value}}}
  child {node [ellipse,draw,xshift=-0.2cm] {\textit{Pattern expression}}};
\end{tikzpicture}
\end{center}
\capt{The abstract syntax tree for the pattern or-operator node.}
\label{ast:pattern-or}
\end{figure}
%

The multiplier-operator derives a single child namely a pattern value. This is not illustrated because it is just a root with one child.%

%
\begin{figure}
\begin{center}
\begin{tikzpicture}[level/.style={sibling distance=40mm/#1}]
\node [square] {Pattern, multiplier-operator}
  child {node [square] {Pattern value}};
\end{tikzpicture}
\end{center}
\capt{The abstract syntax for the pattern multiplier-operator node.}
\label{ast:patter-mult}
\end{figure}
%

The not-operator also derives a single child namely a pattern check. This is not illustrated because it is just a root with a single child.%

A ``pattern value'' and a ``pattern check'' can derive different things and would be illustrated as an expression - with italic font and a ellipse formed box.

%
\begin{figure}[ht]
\begin{center}
\begin{tikzpicture}[level/.style={sibling distance=40mm/#1}]
\node [square] {Pattern, not-operator}
  child {node [ellipse,draw] {\textit{Pattern check}}};
\end{tikzpicture}
\end{center}
\capt{The abstract syntax tree for the pattern not-operator node.}
\label{ast:pattern-not}
\end{figure}
%

\subsection{Operator}%

Furthermore, \productname{} is able to handle arithmetic operators. The AST for operators is illustrated in \figref{ast:operator} and consists of an element followed by an expression.%


\begin{figure}[ht]
\begin{center}
\begin{tikzpicture}[level/.style={sibling distance=25mm/#1}]
\node [square] {Operator}
  child {node [ellipse,draw] {\textit{Element}}}
  child {node [ellipse,draw] {\textit{Expression}}};
\end{tikzpicture}
\end{center}
\capt{The abstract syntax tree for the operator node.}
\label{ast:operator}
\end{figure}
%

\subsubsection{Not-operator}%

We also have a not-operator that has to do with expressions and not patterns. This operator is not illustrated because it just derives an expression. So, it has one root and one child.

%
\begin{figure}
\begin{center}
\begin{tikzpicture}[level/.style={sibling distance=40mm/#1}]
\node [square] {Not-operator}
  child {node [ellipse, draw] {\textit{Expression}}};
\end{tikzpicture}
\end{center}
\capt{The abstract syntax for the not-operator node.}
\label{ast:not}
\end{figure}%

\subsection{Expression and value node types}
\todo{do we still need this?}

%These are used throughout the production of source code written in \productname{}.

%
\begin{figure}[ht]
\begin{center}
\begin{tikzpicture}[level/.style={sibling distance=30mm/#1}]
\node [ellipse, draw] {Identifiers}
  child {node [square,xshift=1.0cm] {Function} edge from parent [dashed];}
  child {node [square] {Identifier} edge from parent [dashed];}
  child {node [square,xshift=-1.0cm] {Variable} edge from parent [dashed];};
\end{tikzpicture}
\end{center}
\capt{The abstract node type for the identifiers.}
\label{ast:identifer}
\end{figure}

%
\begin{figure}[ht]
\begin{center}
\begin{tikzpicture}
  [node distance=0.4cm, start chain=going below]
      \node (wid) [combtree]  {Width};     
      \begin{scope}[start branch=venstre,]
        \node (hei) [combtree] {Height};
      	\node (pie) [combtree] {Piece};
        \node (wic) [combtree] {Win condition};
        \node (tit) [combtree] {Title};
      \end{scope}

      \node (play) [combtree]  {Players};     
      \begin{scope}[start branch=venstre,]
      	\node (tic) [combtree] {Tie condition};
      	\node (boa) [combtree] {Board};
        \node (tur) [combtree] {Turn order};
        \node (pom) [combtree]  {Possible moves};   
	  \end{scope}	  

      \node (set) [combtree]  {Setup};     
      \begin{scope}[start branch=venstre,]
        \node (nam) [combtree] {Name};
      	\node (posd) [combtree] {Possible drops};
        \node (gri) [combtree] {Grid};
        \node (wal) [combtree] {Wall};
      \end{scope}

  % BRACE 
  \draw[tuborg] let
    \p1=(wid.west), \p2=(tit.east) in
    ($(\x1,\y1+1.5em)$) -- ($(\x2,\y2+1.5em)$) node[above, midway]  {\textbf{Keywords}};
\end{tikzpicture}
\end{center}
\capt{The abstract node type for the keywords.}
\label{ast:keywords}
\end{figure}

%
\begin{figure}[ht]
\begin{center}
\begin{tikzpicture}
  [node distance=0.4cm, start chain=going below]
  %    \node (wid) [combtree]  {Width};     
  %    \begin{scope}[start branch=venstre,]
  %      \node (hei) [combtree] {Height};
  %    	 \node (pie) [combtree] {Piece};
  %      \node (wic) [combtree] {Win condition};
  %      \node (tit) [combtree] {Title};
  %    \end{scope}

  %    \node (play) [combtree]  {Players};     
  %    \begin{scope}[start branch=venstre,]
  %    	 \node (tic) [combtree] {Tie condition};
  %    	 \node (boa) [combtree] {Board};
  %      \node (tur) [combtree] {Turn order};
  %      \node (pom) [combtree]  {Possible moves};   
  %	   \end{scope}	  

  %    \node (set) [combtree]  {Setup};     
  %    \begin{scope}[start branch=venstre,]
  %      \node (nam) [combtree] {Name};
  %      \node (posd) [combtree] {Possible drops};
  %      \node (gri) [combtree] {Grid};
  %      \node (wal) [combtree] {Wall};
  %    \end{scope}

  % BRACE 
  %\draw[tuborg] let
  %  \p1=(wid.west), \p2=(tit.east) in
  %  ($(\x1,\y1+1.5em)$) -- ($(\x2,\y2+1.5em)$) node[above, midway]  {\textbf{Keywords}};
\end{tikzpicture}
\end{center}
\capt{The abstract node type for the keywords.}
\label{ast:keywords}
\end{figure}

%\section{Expressions}
\label{sec:expressions}

\begin{ebnf}
%Expressions
\grule{expression}{assignment}
\galt{if\_expr}
\galt{set\_expr}
\galt{lambda\_expr}
\galt{\gter{not} \gcat expression}
\galt{lo\_sequence}
\end{ebnf}

\subsection{Atomic expressions}

These are the smalles possible parts of expressions in \productname{}.

\begin{ebnf}
\grule{atomic}{\gter{(} \gcat expression \gcat \gter{)}}
\galt{variable}
\galt{list}
\galt{\gter{/} \gcat pattern \gcat \gter{/}}
\galt{\gter{this}}
\galt{\gter{super}}
\galt{direction}
\galt{coordinate}
\galt{integer}
\galt{string}
\galt{type}
\galt{constant}
\end{ebnf}

This group also contains $expression$ contained within parentheses.


\subsection{Lists}

\begin{ebnf}
\grule{list}{\gter{[} \gcat \gopt{expression \gcat \grep{\gter{,} \gcat expression}} \gcat \gter{]}}
\end{ebnf}

\subsection{Let expressions}

\begin{ebnf}
\grule{expression}{\grange \gor let\_expr}
\grule{let\_expr}{\gter{let} \gcat variable \gcat \gter{=} \gcat expression \gcat \grep{\gter{,} \gcat variable \gcat \gter{=} \gcat expression} \gnl
\gcat \gter{in} \gcat expression}
\end{ebnf}

\subsection{Conditional expressions}

\begin{ebnf}
\grule{expression}{\grange \gor if\_expr}
\grule{if\_expr}{\gter{if} \gcat expression \gcat \gter{then} \gcat expression \gcat \gter{else} \gcat expression}
\end{ebnf}

\subsection{Lambda expressions}

\begin{ebnf}
\grule{expression}{\grange \gor lambda\_expr}
\grule{lambda\_expr}{\gter{\#} \gcat varlist \gcat \gter{=>} \gcat expression}
\end{ebnf}

\subsection{Set expressions}

\begin{ebnf}
\grule{expression}{\grange \gor set\_expr}
\grule{set\_expr}{\gter{set} \gcat variable \gcat \gter{=} \gcat expression \gcat \grep{\gter{,} \gcat variable \gcat \gter{=} \gcat expression}}
\end{ebnf}

\subsection{Operators and calls}

\begin{ebnf}
\grule{expression}{\grange \gor \gter{not} \gcat expression \gor lo\_sequence}
\grule{lo\_sequence}{eq\_sequence \gcat \grep{\ggrp{\gter{and} \gor \gter{or}} \gcat eq\_sequence}}
\grule{eq\_sequence}{cm\_sequence \gcat \grep{\ggrp{\gter{==} \gor \gter{!=} \gor \gter{is}} \gcat cm\_sequence}}
\grule{cm\_sequence}{as\_sequence \gcat \grep{\ggrp{\gter{<} \gor \gter{>} \gor \gter{<=} \gor \gter{>=}} \gcat as\_sequence}}
\grule{as\_sequence}{md\_sequence \gcat \grep{\ggrp{\gter{+} \gor \gter{-}} \gcat md\_sequence}}
\grule{md\_sequence}{negation \gcat \grep{\ggrp{\gter{*} \gor \gter{/} \gor{\%}} \gcat negation}}
\grule{negation}{element}
\galt{\gter{-} \gcat negation}
\grule{element}{call\_sequence \gcat \grep{member\_access}}
\grule{member\_access}{\gter{.} \gcat constant \gcat \grep{list}}
\grule{call\_sequence}{atomic \gcat \grep{list}}
\end{ebnf}

\subsubsection{Operator precedence}

\tab[\textwidth]{operatorPrecedence}{2}{The precedence of operators in \productname{}.}
         {Operator precedence}
  {Level}{Operator & Description}{
    \tabrow{1}{\texttt{f[]} & Function/constructor invocation and list access}
    \tabrow{2}{\texttt{r.m r.m[]} & Record member access and member invocation}
    \tabrow{3}{\texttt{-} & Unary negation operation}
    \tabrow{4}{\texttt{* / \%} & Multiplication, division, and modulo}
    \tabrow{5}{\texttt{+ -} & Addition and subtraction}
    \tabrow{6}{\texttt{< > <= >=} & Comparison operators}
    \tabrow{7}{\texttt{== != is} & Equality operators and type checking}
    \tabrow{8}{\texttt{and or} & Logical $and$ and $or$}
    \tabrow{9}{\texttt{not} & Logical $not$}
    \tabrow{10}{\texttt{if let \#} & if-, let-, and lambda-expressions}
}

%
\begin{figure}[ht]
\begin{center}
\begin{tikzpicture}[level/.style={sibling distance=30mm/#1}]
\node [ellipse, draw] {Literals}
  child {node [square,xshift=1.5cm] {Integer} edge from parent [dashed];}
  child {node [square,xshift=0.4cm] {Direction} edge from parent [dashed];}
  child {node [square,xshift=-0.4cm] {Coordinate} edge from parent [dashed];}
  child {node [square,xshift=-1.5cm] {String} edge from parent [dashed];};
\end{tikzpicture}
\end{center}
\capt{The abstract node type for the literals.}
\label{ast:literal}
\end{figure}

%
\begin{figure}[ht]
\begin{center}
\begin{tikzpicture}[level/.style={sibling distance=30mm/#1}]
\node [square] {Pattern keyword}
	child {node [square, xshift=1.8cm] {friend} edge from parent [dashed];}
	child {node [square] {foe} edge from parent [dashed];}
	child {node [square, xshift=-1.8cm] {empty} edge from parent [dashed];};
\end{tikzpicture}
\end{center}
\capt{The abstract syntax tree for the variable list node.}
\label{ast:patternkeyword}
\end{figure}

%
\begin{figure}[ht]
\begin{center}
\begin{tikzpicture}
\node [square] {Variable};
\end{tikzpicture}
\end{center}
\capt{The abstract syntax tree for the single node variable.}
\label{ast:variable}
\end{figure}


%literals
%variables
%identifiers
%keywords
%pattern keywords
%expressions
%elements (this is a special keyword)

\subsection{Function call}
\todo{do we still need function call?}

%
\begin{figure}[ht]
\begin{center}
\begin{tikzpicture}[level/.style={sibling distance=30mm/#1}]
\node [square] {Function call}
  child {node [ellipse, draw, xshift=0.5cm] {\textit{Element}}}
  child {node [square, xshift=-0.5cm] {List}};
\end{tikzpicture}
\end{center}
\capt{The abstract syntax tree for the variable list node.}
\label{ast:funccall}
\end{figure}


This concludes the description of our implementation of the different abstract node types which are reflected in the grammar.