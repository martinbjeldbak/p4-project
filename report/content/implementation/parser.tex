\section{Parser}

In this section we present our handwritten parser for our programming language.
We have written a recursive descent parser, which is within the class of LL(1)
parsers. The grammar for \productname{} is suited for this because e.g. is does
not have left-recursive productions. 

\subsection{Constructing the parser}
%structured as the grammar
The parser was very simple to implement, because it is structured in the same
manner as the grammar is structured. For instance if the grammar expresses that
the next set of terminals must begin with a left bracket (`[') then the parser
will expect the next token to be a \tokenref{LBRACKET} which is the token name
for a left bracket. 

%discuss the if expression
In \lstref{lst:ifexpr} we give an example of how this structure looks like in
the parser. The production rule for an if expression is presented in
\secref{sec:grammar}.

%\begin{ebnf}
%\grule{if\_expr}{\gter{if} \gcat expression \gcat \gter{then} \gcat expression
%\gcat \gter{else} \gcat expression}
%\end{ebnf}

The production for if expression says that every expression of this type must
start with the combination of the two symbols which spell the word \gter{if}.
When the parser meets this word it knows that it has to parse an if expression
and the code for this is reflected in \lstref{lst:ifexpr}.

\lstinputlisting[caption="This shows how if expressions are parsed.",
label=lst:ifexpr, language=Java]{listings/ifexpr.java}

\subsection{Building an AST}
%astNode()
In \lstref{lst:ifexpr} the parser initialises the node for the expected if
expression. The parser starts by calling the method \methodref{astNode} to
create a node for the Abstract Syntax Tree (AST). We call the method with
information about what type of expression this is (\tokenref{IF\_EXPR}). The
method calls the \methodref{expect} method to verify that the next token is what
we are expecting. If the two tokens do not match, the parser throws a syntax
error with information about the error. If everything is syntactically correct
the parser constructs a node for the AST for the given expression.

\subsubsection{Terminal and nonterminals}
%Token
Every grammar has a finite set of nonterminals and terminals that constitute the
productions of the grammar. We have defined tokens for every nonterminal in the
grammar. The if expression have the token name of \tokenref{IF\_EXPR}.

%node.addChild(expression())
In the production for the if expression we have three terminals; the \gter{if},
\gter{then}, and the \gter{else}. These are all expected in the method for any
if expression. When the parser finishes reading a terminal it knows that the
following token will be an expression, and therefore a new child for the node is
made with a call to the \methodref{expression} method wherein we parse
expressions. Finally the method returns the node containing every child for the
whole if expression.

\subsection{Looking ahead in the input}
%lookAhead methods - atomic
We mentioned earlier that the parser is an LL(1) parser which means that the
parser is able to look ahead in the sequence of tokens. We have shown the
\methodref{lookAhead} method to determine if the next token is part of an
atomic expression. The production for the atomic expression was presented in
\secref{sec:grammar}.

%\begin{ebnf}
%\grule{atomic}{\gter{(} \gcat expression \gcat \gter{)}}
%\galt{variable}
%\galt{list}
%\galt{\gter{/} \gcat pattern \gcat \gter{/}}
%\galt{\gter{this}}
%\galt{\gter{super}}
%\galt{direction}
%\galt{coordinate}
%\galt{integer}
%\galt{string}
%\galt{type}
%\galt{constant}
%\end{ebnf}

An atomic expression can derive quite a few productions. This is why we have
constructed a specific method to determine whether the next token is part of an
atomic expression. This method is shown in \lstref{lst:lookaheadatomic}.

\lstinputlisting[caption="The lookAhead method to determine if the next
  expression is an atomic type.", label=lst:lookaheadatomic,
language=Java]{listings/method_lookAheadAtomic.java}

The method \methodref{lookAheadAtomic} makes use of two methods to figure out if
the next token is part of an atomic expression. The first method is the
\methodref{lookAhead} method that takes a token as an argument and figures out
if the next token in the sequence of tokens are equal to each other. The second
method is the \methodref{lookAheadLiteral} method which is similar to the method 
in \lstref{lst:lookaheadatomic} but instead of checking for atomic expressions it 
checks for literals. All these methods return true or false.

%example of lookAheadAtomic
%LL(1)
In \lstref{lst:examplelookahead} we show an example of how the
\methodref{lookAhead} method is used in the parser. The example is taken from
the \methodref{expression} method. The productions for expressions are presented
in \secref{sec:grammar}.
The production of an expression is reflected in the code of the parser. An
example of this is given in \lstref{lst:examplelookahead}.

%\begin{ebnf}
%\grule{expression}{assignment}
%\galt{if\_expr}
%\galt{lambda\_expr}
%\galt{\gter{not} \gcat expression}
%\galt{operation}
%\end{ebnf}


\lstinputlisting[caption="Use of the \methodref{lookAhead}-method. This example
is from the \methodref{expression}-method.", label=lst:examplelookahead,
language=Java]{listings/example_lookAheadAtomic.java}

The code presented in \lstref{lst:examplelookahead} is a small section of the
\methodref{expression} method. We have removed code from the section which is
not relevant for the example we are trying to give. The removed code is
presented as \{...\}. In \lstref{lst:examplelookahead} we wish to present how
the \methodref{lookAhead} methods are used.

An assignment begins with the reserved word \gter{let} and the first
\methodref{lookAhead} method peeks for exactly that token to determine if the
next production is an assignment. If the method returns true then the next token is
in fact the \gter{let} word, and the parser enters a new method namely the
\methodref{assignment} method which checks to determine if the rest of the
production is correctly written. The same is done for the if expression, lambda
expression and operations which begins with the ``loSequence()'' (logical operators).

The operation production is a bit different, because it needs two lookAhead
methods to determine if the next production is an operation. An operation can
begin with either an atomic value or a minus operator. So the code uses a
\methodref{lookAheadAtomic} and a regular \methodref{lookAhead} with the
specific token as a parameter to check if the next production is an operation.

The methods return nodes which are connected with each other to form a complete AST.
When the parser has parsed every token of the input it can produce an AST that
corresponds to the program written in \productname{}. This shows that the parser
is built very systematically according to the grammar.
