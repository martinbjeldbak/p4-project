
\section{Contextual constraints}
\label{sec:staticsemantics}

The \classref{ScopeChecker} is the class responsible for enforcing some of the
static semantic rules of \productname{} (described in \chapref{chap:design}) at
compile time. This section aims to explain how these static semantic checks
are performed by the \classref{ScopeChecker} in simple sequential steps. Any
error detected by the \classref{ScopeChecker} will cause a \classref{ScopeError}
exception to be thrown. This error contains helpful information about the type
of error and where in the input program the error is located. The checking of
static semantics as well as the interpretation of the code both use the visitor
pattern which is a commonly used approach for both purposes.

After the AST has been created, the visitor pattern allows us to traverse the
AST and execute encapsulated pieces of code for each specific type of
\classref{AstNode}. 

\subsection{TypeVisitor}
The first visitor used is the \classref{TypeVisitor}. This visitor traverses the
AST, finds all type definitions in the input program and for each type
definition an object of class \classref{TypeSymbolInfo} is instantiated. After
running the \classref{TypeVisitor}, each type definition in the input program
has an associated object of class \classref{TypeSymbolInfo} which makes it easy
to get information about any declared type in the input program by accessing the
following members contained in the \classref{TypeSymbolInfo} object:

\begin{dlist}
  \item name (\classref{String}): The type's name
  \item parentName (\classref{String}): The name of its super type (null if
    not a derived type)  
  \item args (\classref{Integer}): The number of arguments in the type's
    constructor
  \item parentArgs (\classref{Integer}): The number of arguments given in the
    call to its parent constructor
  \item data (\classref{List of Data}): Each data defined in the type body has a
    corresponding \classref{Data} object describing its name and position in the
    input program
  \begin{dlist}
    \item The input program position is stored as a line and an offset and makes
      it possible to produce error messages with information about where in the
      input program an error was found
  \end{dlist}
  \item members (\classref{List of Member}): Each constant and function defined
    in the type body has a corresponding \classref{Member} object describing its
    name, argument count, an abstract flag, a \classref{TypeSymbolInfo}
    reference to the type defining the \classref{Member} and a pointer to the line
    and offset in the code
  \item node (\classref{AstNode}): A reference to the
    TYPE\_DEF-\classref{AstNode} which defines the type
  \begin{dlist}
    \item This reference is used to get the input program position where the
      type was defined (for generating useful error messages), but is also used
      for marking abstract type definitions (described in detail in
      \secref{sec:abstractTypeMarker}) 
  \end{dlist}
  \item parent (\classref{TypeSymbolInfo}): This will contain an object
    reference to its parent type if it has one
  \begin{dlist}
    \item This is however a null pointer until running the
      \classref{TypeParentRefMaker} described in \secref{sec:TypeParentRefMaker}
  \end{dlist} 
\end{dlist}

All the \classref{TypeSymbolInfo} objects are kept in an object of class
\classref{TypeTable}. The \classref{TypeTable} class is a layer of abstraction
which provides easy and fast access information about the types contained in the
input program. The underlying implementation is a hashmap from the type's name
as a \classref{String} to its object reference, which provides a quick way to map a type's name
to the \classref{TypeSymbolInfo} object that represents the type. One use of this mapping is 
described in \secref{sec:TypeParentRefMaker}. Another feature of \classref{TypeTable} is a 
convenient way to iterate over all the \classref{TypeSymbolInfo}s. This is for instance used
to topologically sort the types, which are described in \secref{sec:TypeMemberPropagator}.
For convenience, we say that a type is added to a type table which means that a
\classref{TypeSymbolInfo} object representing the type is added to the
\classref{TypeTable} object representing the type table.

When a type is added to the type table, it is checked that no other types with
the same name exist.

\subsection{TypeParentRefMaker}
\label{sec:TypeParentRefMaker}
The \classref{TypeSymbolInfo} objects only contain the name of their super type
as a \classref{String} or a null value if there is no super type. By making a
lookup in the \classref{TypeTable} on the parent name, the real object
references can be found and stored for faster and more convenient parent lookups
which is used greatly by the visitor described in
\secref{sec:usesaredeclaredvisitor}.

\subsection{TypeMemberPropagator}
\label{sec:TypeMemberPropagator}
Some of the later checks that will be performed requires us to determine whether
or not a type member (constant or function) with a specific name is visible in a
given type. This requires searching in the given type and recursively in all
super types for the member. When this kind of lookup is done many times on the
same member, the traversal of the same long chains of parent references become
inefficient which results in clumsy code. To simplify and speed up this process,
the \classref{TypeMemberPropagator} ensures that all members of a type A are
also present in a type \type{C}, if \type{A} is a super type of \type{C}. With
this approach, checking if a member is visible in type \type{C}, only requires
looking in \type{C} instead of following the chain parent types.

This propagation of members is done by first topologically sort the
\classref{TypeSymbolInfo} objects, such that when iterating over the type table,
any type yielded will always appear before all of its subtypes. This makes the
afterwards propagation of members possible in linear time by iterating over the
topologically sorted types. If a type \type{C} is met, the members in its parent
type \type{B} are put in \type{C} as well. If \type{B} has a parent \type{A}, we
know that \type{B} has already the members from \type{A} due to the topological
sorting.

The topological sorting is done using the algorithm that goes by the same name,
from the book Introduction to Algorithms\cite[p. 612]{ad} working on a graph $G
= (V, E)$. Each type represents a vertex $v$. An edge $e$ exists from $v_1$ to
$v_2$ if the type $v_2$ is a parent of the type $v_1$. However this results in a
topological order where a type always appears before its super types. This
problem is quickly solved simply by reversing the list.

Given the graph in \figref{fig:topological}, the following sequences are examples
of correct and wrong topologically sorted orders after the list has been reversed:

\begin{align*}
 Correct &: \texttt{a, d, b, e, c, f} \\
 Correct &: \texttt{a, b, e, c, f, d} \\
 Wrong &: \texttt{a, b, c, \textbf{f}, \textbf{e}, d} \\
 Wrong &: \texttt{\textbf{b}, c, e, \textbf{a}, f, d}
\end{align*}


\begin{figure}[ht]
  \begin{center}
    \begin{tikzpicture}[level/.style={sibling distance=30mm/#1}]
      \node [square] (c) {C};
      \node [square, yshift=-4em, xshift=2.5em] (b) {B};
      \node [square, yshift=-4em, xshift=7.5em] (d) {D};
      \node [square, yshift=-8em, xshift=5em] (a) {A};
      \node [square, yshift=-2em, xshift=12em] (f) {F};
      \node [square, yshift=-6em, xshift=12em] (e) {E};

      \draw[->, thick,] (c) -- (b);
      \draw[->, thick,] (b) -- (a);
      \draw[->, thick,] (d) -- (a);
      \draw[->, thick,] (f) -- (e);
    \end{tikzpicture}
  \end{center}
  \capt{Example of topologically sorted types. An edge goes from type X to type Y if X is a subtype of Y.}
  \label{fig:topological}
\end{figure}



Another great advantage from topological sorting is the fact that it reveals
cycles in the graph. A cycle in the graph means an extend cycle between types
exists, e.g: A extends B, B extends C and C extends A, which is not accepted.

\subsection{AbstractTypeMarker}
\label{sec:abstractTypeMarker}
The interpreter needs to know whether or not a given type contains any abstract
members. Such a type is an abstract type and should not be allowed to be
instantiated. Marking these abstract types is now smooth. Due to the propagated
members it can just be checked whether or not any abstract members are present in the
given type. This check is done by the \classref{AbstractTypeMarker}. Any type
described by a \classref{TypeSymbolInfo} has a reference to the
\classref{AstNode} it was defined from. If the type is found to be an abstract
type, the type of the \classref{AstNode} is changed from TYPE\_DEF to
ABSTRACT\_TYPE\_DEF. This is the only way to make information visible to the
interpreter since the \classref{Interpreter} does not use the same
\classref{TypeTable} class used by the \classref{ScopeChecker}.

\subsection{TypeSuperCallChecker}
This checker ensures that any type that extends another type provides the right
amount of arguments when calling the parent's constructor. A constructor can have
$x$ arguments and may or may not contain a variable amount of additional
arguments. Consider the type constructor \texttt{Type A[\$var1, \$var2, \dots
\$varargs]}. When calling the constructor from another type, e.g. \texttt{Type
B[] extends A[5, 2, 7, 4]}, it must be checked that the type B provides \emph{at
least} the number of arguments in A's constructor (not counting the variable
amount of extra arguments). If A does not have a variable amount of additional
arguments, the argument count must match exactly. The implemented code for doing
this check can be seen in \lstref{lst:tscc}.

\lstinputlisting[caption={How the TypeSuperCallChecker is implemented.},
label=lst:tscc, language=Java]{listings/typeSuperCallChecker.java}

\subsection{UsesAreDeclaredVisitor}
\label{sec:usesaredeclaredvisitor}
This visitor ensures that any use of a variable, constant, function, data
member, or type can be bound to a declaration. The visitor uses a variable
(\classref{TypeSymbolInfo} \varref{currentType}) which updates upon visiting a
TYPE\_DEF or an ABSTRACT\_TYPE\_DEF \classref{AstNode}, to keep track of which
type it is currently visiting inside. If the visitor is not traversing inside a
type (\classref{TypeSymbolInfo} \varref{currentType}) references a special type
called \varref{globalType}, which is used only to contain the standard- and game
environment as well as the global constants and functions declared in a
\productname{} game. 

It is important to realise that \varref{globalType} is not a super type of all
other types, it is a stand alone type that no type can derive from. Its name
contains an invalid character for a type name to ensure that no type can derive
from it. This becomes handy when checking if constants and functions used can be
bound to a declaration.

\subsubsection{Constants and functions}
When a constant or a function is referenced it is necessary to know two things
about the context in which it was referenced:

\begin{nlist}
  \item In what type did the reference occur?
  \item Is the reference a member access?
\end{nlist}

The first thing is easy to check since we have the \varref{currentType}
variable. This variable may however point to the global type. The structure of
the AST makes it easy to determine if it was a member access, since we would
have been visiting a MEMBER\_ACCESS \classref{AstNode} prior to the referenced
constant or function. In the expression: \texttt{A[].B.C[2,3]}, both B and C are
member accesses, but A is not. Given this information, a different check can be
done regarding to the context of the reference:

\begin{nlist}
  \item Type was global and a member access
  \begin{dlist}
    \item Must be visible in at least one type
  \end{dlist} 
  \item Type was global but not a member access
  \begin{dlist}
    \item Must be visible in the global scope
  \end{dlist}
  \item Type was A and a member access
  \begin{dlist}
    \item If prefixed by this, it must be visible in A or any super type of A
    \item If prefixed by super, it must be visible in any super type of A
    \item If prefixed by a variable name, it must be visible in at least one type
  \end{dlist}
  \item Type was A but not a member access: Must be visible in A, a super type
    of A or global scope
\end{nlist}

One may wonder why an accessed member is accepted if the accessed member is
visible in at least one type. Consider the member access \texttt{randomType.B}.
Here it is unknown in what type we shall look for the member B. The constant
\texttt{randomType} could literally return a random type, or the type returned
could be determined by an arbitrary complex algorithm. Therefore, we can only
enforce the rule that the member \texttt{B} must exist in at least one type.

%Skal nedenstående  afsnit med?

%One may think that it is also nice to know if a referenced constant or function
%has a number of parameters along with it and whether the actual number of
%arguments correctly matches the formal number of arguments. This is however
%quite hard to determine. In the example, if \texttt{randomConstant} was declared
%as a constant, the expression \texttt{randomConstant[2]} would still make sense if
%the constant returned a list, in which 2 was an index. This is however
%something the scope checker cannot look into. Given a function declared as
%\texttt{randomFunction[\$a, \$b] = \dots} the expression \texttt{let \$var =
%randomFunction in \dots} would also be correct, in which case \texttt{\$var} is just
%a reference to \texttt{randomFunction} in the \texttt{in}-scope. So it is valid to
%use a constant followed by a parameter list as well as it is valid to not apply
%a parameter list behind a function. 

\subsubsection{Variables}
For any variable, a declaration must always exist before it is used. A variable
can only be declared in four ways:

\begin{nlist}
  \item As a type constructor
  \item As a formal parameter in a function declaration
  \item In a lambda expression
  \item In a let expression
\end{nlist}

In all cases the \productname{} semantics require that a new scope is opened, in
which the declared variable is known while the body of the expression is
executed. When the scope closes the declared variables are removed. The body of
an expression can also contain new variable declarations, e.g. a let
expression in the body of a let expression. 

The scope checker uses a \classref{SymbolTable} class which is basically a
symbol table with a list of variable names and a reference to a parent symbol
table.  The reference to the parent symbol table is exactly how the scopes
inside other scopes are implemented. 

\codesample{openscopeexpressions.junta}

Notice how the four code samples above all result in the same
scope checking routine, which can be seen in \figref{fig:scope1}. First, a new
symbol table is instantiated in which the variables \variable{a} and \variable{b} are put in. The
symbol table's parent reference is updated so it points to the current symbol
table, which is referred by \classref{SymbolTable} \varref{currentST}. Next, the
current symbol table is updated to the newly created symbol table, and the body
(the triple dots) are executed. Lastly, the current scope is closed, which
updates the current symbol table reference to point to the parent symbol table
of the current symbol table.  Notice that the symbol tables maintain a
stack-like structure, where opening a scope pushes a symbol table on the stack
and closing a scope pops one. The variable \classref{SymbolTable}
\varref{currentST} points to the element on top of the stack.

When a variable is used, it is checked that the variable exists in any of the
symbol tables by first looking in the current symbol table and recursively
following the parent reference until a null reference is found. If a variable
declaration with the same name as the used variable cannot be found, an error is
generated.

\fig[height=5em]{scope1}{Four different expressions that all result in the scope
action depicted.}

It is important to realise the reason for maintaining the stack-like structure
of symbol tables. It might seem like a single symbol table would be enough and
that all variable declarations could just be put in there. This is indeed wrong,
since the scope checker must also check for double declarations. A double
declaration exists if a symbol table contains the same variable twice. Notice
how \figref{fig:scope2} contains two symbol tables, each containing a
declaration of \$a. 

\fig[height=5em]{scope2}{The variable \variable{a} declared in two different scopes.}

This is completely valid and is caused by the following code sample. If
only a single symbol table was used, an incorrect double declaration
would be detected. See \secref{sec:letexpressions} for more details
about hiding.

\codesample{scope2.junta}

\subsubsection{Data members}
When visiting a type body, a new scope is opened and the data members of
\classref{TypeSymbolInfo} \varref{currentType} are immediately inserted into that
scope. The children of the type body are then visited and the scope is closed.
This ensures that the data members of a type can be used anywhere in the type
body but in that type body only. When exiting that type body and closing the
scope, the symbol table containing the data members are no longer visible.

\subsection{Summary of scope checking}
Many different static semantic checks are implemented in the scope checker.
Though many other checks could have been included as well, the scope of the
static semantics has been limited due to a few constraints. First of all, there
is a deadline for this project, and with an almost endless set of semantic
checks one can keep developing these checks. Furthermore, with new techniques
being discovered once in a while, a compiler or interpreter can simply not
include them all. A big set of the checks not included in \productname{}
requires type checking, which is cumbersome in a dynamic programming language.
However, it is generally possible to use type inference to find at least some of
the types and errors associated with the use of them. It is important to realise
that everything cannot always be inferred, for instance an algorithm could be so
complex that it would need to be executed to determine all possible outcomes.
Running the algorithm is not possible since you cannot know if the algorithm
will ever halt\cite[p. 173]{itttoc}.
