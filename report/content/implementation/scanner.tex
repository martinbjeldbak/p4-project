\chapter{Scanner}
The scanner is the first part of the interpreter that analyses the input. The input for the scanner is a raw source code for a program written in \productname{}. The main purpose of the scanner is to validate the lexically correctness of a \productname{} program. The scanner must analyse the input and find tokens existing in \productname{}. If an input is met which can not be recognized as a valid token, the source code is not a valid program in \productname{}. The strings that are converted to tokens are called lexemes. Many lexemes can be converted to the same type of token. Some programs contains many different identifiers, which in \productname{} all will have a token instantiated with the type \textit{IDENTIFIER}. Consider the example of a single line from a chess game written in \Productname{}. The raw input is:
\begin{lstlisting}
Black{ Pawn [A7 B7 C7 D7 E7 F7 G7 H7] }
\end{lstlisting}
The result of analysing the input can be seen in \tableref{table:lexemestotokens}. Tokens are needed for abstraction. When later on the parser will determine if the code respects the grammar of \productname{}, it is useful to have these abstractions. It makes it possible to describe that a list of \textit{COORD_LIT}'s can be encapsulated between the characters ``[ ]'' without having to list all possible coordinate literals, which in fact are an infinite set, since the grammar of \productname{} which is described later, allows proceeding in both dimensions after \textit{Z9}, namely \textit{Z10} and \textit{AA9}. However, when converting lexemes to tokens, the value of a coordinate, the name of an identifier, e.g. is still kept since that information will be needed for the subsequent parts of the interpreter.

\begin{figure}
\begin{table}
    \begin{tabular}{|l|l|}
        \hline
        Lexemes & Tokens             \\ \hline
        Black   & IDENTIFIER (Black) \\ 
        {       & LBRACE             \\ 
        Pawn    & IDENTIFIER (Pawn)  \\ 
        [       & LBRACKET           \\ 
        A7      & COORD_LIT (A7)     \\ 
        B7      & COORD_LIT (B7)     \\ 
        C7      & COORD_LIT (C7)     \\ 
        D7      & COORD_LIT (D7)     \\ 
        E7      & COORD_LIT (E7)     \\ 
        F7      & COORD_LIT (F7)     \\ 
        G7      & COORD_LIT (G7)     \\ 
        H7      & COORD_LIT (H7)     \\ 
        ]       & RBRACKET           \\ 
        }       & RBRACE             \\
        \hline
    \end{tabular}
\end{table}
\caption{Analysing an input stream for lexemes and tokens}\label{table:lexemestotokens}
\end{figure}

The scanner contains 2 classes, \classref{Scanner} and \classref{Token}. \classref{Token} contains an enum named \classref{Type} that enumerates all the types of tokens in \productname{}. When a lexeme is found in the input stream, the scanner analyses which token type it belongs to. A new token is then instantiated and yielded by the scanner. The constructor for \classref{Token} takes the arguments (\classref{Token.Type}, \textit{line}, \textit{offset}). The \textit{line} and \textit{offset} represent where in the source code the lexeme of any token where found, which can be used to inform a programmer where in his source code an error was found, if any was found.
 