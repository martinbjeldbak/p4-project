\section{Interpreter}
The final step is the interpretation of the AST generated by the parser.
Here choices are taken based on the different node types in the tree
for a given program. This interpreter class is, like the scope checker,
also implemented with the visitor pattern, visiting all the nodes and
taking appropriate actions depending on what type of node is visited.
The difference here is that values, types, and other language constructs
are created and evaluated directly, propagating the results up the AST,
and that every type of node is visited, as each node is significant in
that they carry important information about the written program.

In this section, we will look at the inner workings of the interpreter
and see some examples of the different evaluation methods used and how
they're created. We also see how the symbol table keeps track of scopes
and the different values. Thereafter the implementation of patterns is
described, as they play a central role in most programs written in our
language.

% To add (maybe a section here about how the interpreter behaves?):
% The game env is loaded on startup
% Methods are created for each type of node, possibly visiting other nodes and passing values around

\subsection{Symbol table and scopes}
Symbol tables in the interpreter are used to add and get constants,
variables, types, and to push and pop scopes. The interpreter keeps a
single, global instance of a root symbol table which it calls methods
upon when needed. This symbol table is unaffiliated with the scope
checker's symbol table, which is built differently up, and hence does
not share any information with it at all.

As an example of how the symbol table is operated, an example of
visiting an assignment (let-in) node, which utilizes some of these
features, is shown below:

\lstinputlisting[caption={\emph{Code taken from the interpreter to
show how the symbol table is used when visiting let--expressions.}},
label={lst:let}, language=Java]{listings/visitAssignment.java}

Listing \ref{lst:let} shows how ``let-in'' expressions are evaluated
when the abstract node type shown in \secref{sec:letexpressions} is
visited. A new scope is pushed onto the current stack of scopes,
whereafter the variables are hereafter pushed into the scope that was
just opened. Because let-in expressions, like all \productname{}'s
expressions, return a \classref{Value}, this value is retrieved after
the body of the assignment is computed but before the scope is closed,
as we will most likely need the values of the different variables.

\subsection{Values and their operators}
Each one of the base values offered by \productname{} is represented
internally by a class written in the interpreter. These classes are
all sub-classes of a general class \classref{Value}, that offers the
sub-classes an interface to implement various different operations, such
as comparison, addition, calling, and so on, throwing an error if trying
to use the operator between incompatible values or if the operation
yields an invalid result.

In the following part of this section, a few important values and their
features are highlighted. The most basic values such as integers,
strings, directions, coordinates, and lists principally support the
same operations, and are therefore relatively uninteresting to discuss
compared to the more complex values, such as types and functions. For
completeness, the implementation of coordinates, types, and
functions are discussed below.

Most properties (variables) of values are declared as Java
\classref{final}, due to the fact that \productname{} is functional and
that the result of an expression always results in a new instance of an
object. This means that a simple plus operation yields a new value, so
the interpreter constantly instantiates these new values without ever
needing to update a specific property, possibly leaving many young, dead
objects on the heap for Java's garbage collection to deallocate.

\subsubsection{Coordinates}
Coordinates are represented internally by a tuple $(x, y)$ specifying
a coordinate on the game board. The maximum board size is limited by
an $x$ and $y$, which are both 32-bit signed two's compliment Java
integers\footnote{Which gives a maximum coordinate of $(2,147,483,647;
2,147,483,647)$, much more than any realistically imaginable board
game!}. If we wished to represent larger boards, we could simply use
a built-in larger natural number representation, or create a custom
representation.

As specified in \secref{sec:standardenvironment}, coordinates consist of
a horizontal axis represented by letter, or multiple letters, $x$, and a
numerical $y$, representing the vertical squares on a grid-shaped board.
When displaying a coordinate to the user, a simple method is called to
convert the $x$ coordinate to its alphabetical form.

Coordinate values support equals comparison, done by value. It
also allows addition with strings, directions, and lists. Subtracting
other coordinates yields a direction vector and subtracting a direction
yields a new coordinate with an offset defined by subtracting the
coordinates $x$ and $y$ properties with the direction's $x$ and $y$
properties, as seen in the following listing.

\lstinputlisting[caption={\emph{How subtraction of other values on \classref{Coordinate} is handled.}}, label={lst:coordSubtract}, language=Java]{listings/coordSubtract.java}

Listing \ref{lst:coordSubtract} shows the implementation of the
subtraction operation on coordinates. This method is principally the
same implementation for every mathematical operation across all the
different above-mentioned sub-classes of \classref{Value}. Type checking
is always done to see if the right-hand side of the operation is
compatible before taking the correct actions. The method \methodref{is}
on all \classref{Value} types checks to see if the type is the specified
type or any sub-type extending the type in \productname{}. If the RHS
is allowed, it's up-casted to the appropriate type if it isn't used
directly, whereafter the appropriate operation is performed between the
two types.

%\subsubsection{Lists}
%Lists are implemented very much like Coordinates, supporting comparison,
%addition with other lists and any singular \classref{Value}-type,
%subtraction also with other lists and singular \classref{Value}-types
%existing in the current \classref{List} being subtracted from.
%
%The most interesting feature of \classref{List} types is that they take
%an optional list of one to two parameters, allowing access to an element
%at a specific index in the list, or defining a range of indexes desired
%in the list, resulting in a new list containing the elements between the
%two indexes.


\subsubsection{Types}
\classref{Type} values are one of the more complex sub-classes of
\classref{Value}, as they, when defined, can have formal parameters, a
parent \classref{Type}, and a complex body. These are all represented
and stored internally in a class \classref{Type}, instantiated when
visiting a type definition node (defined in \apref{ap:typedef}).

% Declaring
There are a handful of different ways to instantiate a \classref{Type},
all depending on the contents of the type definition node visited.
If a type is declared to have a parent, a \classref{Value} of type
\classref{Type} is simply instantiated by the interpreter with the
appropriate constructor, saving the nodes for the parent and the parents
formal parameters. If the type declaration also has a body, then each
declaration is added to the newly instantiated type by the interpreter.
Finally, the type is added to the symbol table, so it can be called and
instantiated at a later point in the program.

% Instantiating
Types are implemented to allow instantiating the type, returning an
\classref{ObjectValue} sub-class with the actual parameters bound to the
formal parameters, scope bindings, and a reference to the interpreter
so it can evaluate the methods in its body (if one exists) when called.
This instantiating is done when the interpreter visits an actual type,
where it looks up the type's name in the symbol table, returning it if
found, or an error if not.

% Abstract types
Abstract type values are merely an extension of \classref{Type}, but only
consisting of abstract members, and do not allow instantiating, throwing
a \classref{TypeError} if attempted.

% Quick mini-summary
Unfortunately the methods that make up the interface to the
\classref{Type} class are too large to be shown in this section,
therefore we refer to the \classref{TypeValue} class in the source code
for a quick overview.

The body of types can consist of constants and functions, where the
implementation of functions is described blow.

\subsubsection{Functions}


\subsection{Pattern evaluation}
Patterns in \productname{} are a very important feature, being used
in pretty much every one of our test programs to evaluate winning
conditions, find squares or actions that have a specific property, and
various other constructs.

We looked at other languages in an attempt to find an existing solution
to a similar problem and quickly found that patterns should be very
similar to regular expressions. Unfortunately, regular expressions are
actually do not share the same properties as our patterns, ending up
giving us quite a headache when trying to use automaton to describe
patterns.

\todo{Keeent}

\subsubsection{Implementation with finite automaton}
Our first intuition was to draw inspiration from implementations
of regular expressions. These implementations use nondeterministic
finite automaton (NFAs) to represent the traversal of the expression,
having different constructs of automata nodes for the different logic
operations, such as Kleene star, union, concatenation, among others. An
NFA is constructed for each part of the expression, as they concatenated
together represent the entire regular expression, only accepting strings
in the alphabet that fulfill the desired properties. These NFAs are then
converted to deterministic finite automaton, which can be traversed much
faster than NFAs.

But this implementation caused some problems, because regular
expressions are described as a 4-tuple with an input string, which our
patterns don't require. Patterns are used in function parameters and to
evaluate different conditions. These conditions don't necessarily have
an input, but are based on the current game's state, such as which piece
and player is currently being taken into account.

Therefore we needed another solution, because as practical as
representing patterns as automaton is, they're not a valid solution.
So we went back to the drawing board and thought of a different,
nondeterministic implementation.

\subsection{Evaluation of interpreter}
We have implemented a complete interpreter with semantics for every
single node type, allowing us to explore and demonstrate every feature
of \productname{} without many limits.

Under development, we have kept flexibility in mind, allowing us to
easily add new base values and extend and maintain the feature set and
semantics of our language, without having to change any existing code.
If we wanted to add an addition operation between coordinates, it's only
a matter of adding a few lines of code to the coordinate value class.

It is obviously not the fastest implementation of an interpreter, but we
don't see that as a hindrance, because speed is not what we're after.

There are a few optimizations that can be taken into account when
writing an interpreter to make it run faster, such as detecting tail
recursion. Since \productname{} has no loop-constructs, recursive
functions are the only way to repeatedly run through some code.
