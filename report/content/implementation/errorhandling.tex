\section{Error handling}
\label{sec:errorhandling}

In the different phases of the interpreter a lot of errors can occur. In the
scanning phase for instance a scanning error can occur if the scanner detects a
string, which doesn't correspond to one of \productname{}'s specified lexemes. In
the parsing phase a syntax error can occur if two or more tokens are not set up
syntactically correct according to our grammar. In order to catch these errors
we have implemented error handlers, which extends the built-in Java class
\classref{Exception} (\classref{Java.lang.Exception}). 

An error can cause serious flaws and is annoying for a programmer if no
information of what caused it is provided. So, the function of the error
handlers are to give the programmer of \productname{} games a better chance of
figuring out what has been done wrong and make it easier to correct the errors
by for instance referring to where in the source code an error has occurred and
what caused it. We have created an error hierarchy in order to give an overview
of our different error handlers. The hierarchy is illustrated in
\figref{ast:errorhierarchy}. 

\begin{figure}[ht]
  \begin{center}
    \begin{tikzpicture}
      \node [square,yshift=-2em] (a) {Exception};
      \node [square,yshift=-5em] (b) {Error};
      \node [square,yshift=-8em,xshift=-7em] (c) {SyntaxError};
      \node [square,yshift=-8em] (d) {StandardError};
      \node [square,yshift=-8em,xshift=7.5em] (e) {SimulatorError};
      \node [square,yshift=-12em,xshift=-21em] (f) {ScannerError};
      \node [square,yshift=-12em,xshift=-14em] (g) {ScopeError};
      \node [square,yshift=-12em,xshift=-7em] (h) {InternalError};
      \node [square,yshift=-12em] (i) {TypeError};
      \node [square,yshift=-12em,xshift=7em] (j) {ArgumentError};
      \node [square,yshift=-12em,xshift=14em] (k) {NameError};
      \node [square,yshift=-15em,xshift=7em] (l) {DivideByZeroError};

       \draw[<-, thin,] (a) -- (b);
       \draw[<-, thin,] (b) -- (d);
       \draw[-|,-,thin,] (c.north) |-+(0,0.75em)-| (b.south);
       \draw[-|,-,thin,] (e.north) |-+(0,0.86em)-| (b.south);
       \draw[-|,->,thin,] (f.north) |-+(0,1.7em)-| (c.south);
       \draw[<-,thin,] (d) -- (i);
       \draw[-|,-,thin,] (g.north) |-+(0,1.0em)-| (d.south);
       \draw[-|,-,thin,] (h.north) |-+(0,1.11em)-| (d.south);
       \draw[-|,-,thin,] (k.north) |-+(0,1.11em)-| (d.south);
       \draw[-|,-,thin,] (j.north) |-+(0,0.98em)-| (d.south);
       \draw[<-, thin,] (j) -- (l);
    \end{tikzpicture}
  \end{center}
  \capt{An illustration of the error hierarchy as it is implemented in the
  interpreter.}
  \label{ast:errorhierarchy}
\end{figure}


\begin{description}
\item[\textbf{Exception}] is at the top of the error hierarchy Java's standard
  error handling concept is located, which is called \classref{Exception}
  (\classref{java.lang.Exception}). Exception is a built-in Java class that
  extends the \classref{Throwable} class which is the superclass of all errors
  and exceptions in Java. Only \classref{Throwable} or subclasses of
  \classref{Throwable} can be the argument type in a catch clause and only
  objects that are instances of this class or it's subclasses can be thrown by
  the Java throw statement.
  \cite{throwable}

\item[\textbf{Error}] extends the \classref{Exception} class. \classref{Error}
  is an abstract class which contains two abstract methods:
  \methodref{getColumn()} and \methodref{getLine()}.

\item[\textbf{SyntaxError}] extends the \classref{Error} class and is used in
  the parser to handle syntactic errors. A syntactic error occurs if a token
  doesn't correspond to the currently expected token according to the grammar.
  \classref{SyntaxError} outputs a message and information about the line and
  offset of the token that caused the error. 

\item[\textbf{ScannerError}] extends the \classref{SyntaxError} class and is
  used in the scanner to handle lexical errors. A lexical error occurs if the
  programmer makes a typo, e.g. writes ``defin'' instead of ``define'' or if it 
  in other ways doesn't follow the regulations of \productname{}'s lexemes e.g.
  writes the first letter in a type name in lower case letters. Like
  \classref{SyntaxError}, \classref{ScannerError} outputs a message and
  information about the line and offset of the token that cause the error.  

\item[\textbf{StandardError}] extends the \classref{Error} class. The
  \classref{StandardError} class is used in the interpretation phase to handle
  errors with nodes in the abstract syntax tree.  

\item[\textbf{ScopeError}] extends the \classref{StandardError} class. It is
  thrown by the scope checker to notify the programmer of scope errors.

\item[\textbf{InternalError}] extends the \classref{StandardError} class. It is
  an error that encapsulates Java's own exceptions, if they are thrown. It
  essentially represents an error in the implementation or environment in which
  it is running, and is not caused directly by the programmers code.

\item[\textbf{TypeError}] extends the \classref{StandardError} class. It is
  thrown at runtime when a value is of the wrong type, for instance when using
  an operator such as the \texttt{+}-operator, the operands must be applicable
  for that operator.

\item[\textbf{ArgumentError}] extends the \classref{StarndardError} class. It is
  thrown at runtime if a function is supplied the wrong number of arguments.

\item[\textbf{NameError}] extends the \classref{StandardError} class. It is
  thrown at runtime if a used constant, type or variable is not defined in the current scope.

\item[\textbf{DivideByZeroError}] extends the \classref{ArgumentError} class. It
  is thrown when dividing by zero.

\item[\textbf{SimulatorError}] extends the \classref{Error} class. It is thrown
  if the simulator encounters an error.
\end{description}

% Talk about how, where, and why errors are thrown in the different packages
% ..and how they're implemented
