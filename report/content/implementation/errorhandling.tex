\section{Error handling}

In the interpretation phases lots of different errors can occur. In the scanning phase
a scanning error can occur if the scanner detects a string, which doesn't correspond to one of our language's specified tokens.
This is a lexical error, but we refer to it as a scanning error. In the parsing phase a syntax error can occur if two or more
tokens are not set up syntactically correct according to our grammar. We have created an error hierarchy in order to give an
overview of our different errors, which is illustrated in \figref{ast:errorhierarchy}. The function of an error is to give the
programmer of junta games a better chance of figuring out what he has done wrong and make it easier for him to correct the errors 
by for instance referring to where in the source code and error has occurred.   

\begin{figure}[ht]
	\begin{center}
		\begin{tikzpicture}
			\node [square] (a) {Exception}
  				child {node [square,xshift=0cm] (b) {Error}
  					child {node [square,xshift=-4.5cm] (c) {syntaxError}
  						child {node [square,xshift=0.0cm] (d) {ScannerError}}
  					}
  					child {node [square,xshift=0cm] (e) {StandardError}
  						child {node [square,xshift=-0.5cm] (f) {ScopeError}}
  						child {node [square,xshift=0.5cm] (g) {InternalError}}
  						child {node [square,xshift=1.5cm] (h) {TypeError}}
  						child {node [square,xshift=2.5cm] (i) {ArgumentError}
  							child {node [square,xshift=0.0cm] (j) {DivideByZeroError}}
  						}
  						child {node [square,xshift=3.5cm] (k) {NameError}}
  					}
  					child {node [square,xshift=4.5cm] (l) {SimulatorError}}
  				};
  			\draw[<-, thick,] (a) -- (b);
    		\draw[<-, thick,] (b) -- (c);
    		\draw[<-, thick,] (c) -- (d);
    		\draw[<-, thick,] (b) -- (e);
    		\draw[<-, thick,] (e) -- (f);
    		\draw[<-, thick,] (e) -- (g);
    		\draw[<-, thick,] (e) -- (h);
    		\draw[<-, thick,] (e) -- (i);
    		\draw[<-, thick,] (e) -- (k);
    		\draw[<-, thick,] (i) -- (j);
    		\draw[<-, thick,] (b) -- (l);
		\end{tikzpicture}
	\end{center}
	\capt{An illustration of the error hierarchy as it is implemented in the interpreter}
	\label{ast:errorhierarchy}
\end{figure}

\begin{dlist}
	\item[\textbf{Exception}] At the top of the error hierarchy Java's integrated error handling concept is located, which is called Exception (java.lang.Exception). This
	\item[\textbf{Error}]
	\item[\textbf{SyntaxError}] Inherits from Error and is used in the parser to catch syntactic errors. A syntactic error occurs if a token doesn't correspond to the currently expected token according to the grammar. SyntaxError outputs a message like ``Unexpected token, expected X'' where X is the expected token according to the grammar also it outputs at which line and offset the token which caused the error is located. 
	\item[\textbf{ScannerError}] Inherits from SyntaxError and is used in the scanner to catch lexical errors. A lexical error occurs if the programmer 
	\item[\textbf{StandardError}]
	\item[\textbf{ScopeError}]
	\item[\textbf{InternalError}]
	\item[\textbf{TypeError}]
	\item[\textbf{ArgumentError}]
	\item[\textbf{NameError}]
	\item[\textbf{DivideByZeroError}]
	\item[\textbf{SimulatorError}]
\end{dlist}


    
% Talk about how, where, and why errors are thrown in the different packages
% ..and how they're implemented
