\section{An overview of the implementation of \productname{}}
\label{anoverviewofjunta}

Now that the implementation of the different constructs that make up our
implementation of \productname{} have been described, we wish to provide
an overview of how they fit together, attempting to provide a concise
overview. Figure \ref{fig:cake} presents how these components actually
are connected.


\begin{figure}[ht]
  \begin{center}
    \begin{tikzpicture}      
      \node [rectangle,draw,rounded corners,thick,text width=7cm,text centered,]
      (a) {\textbf{Simulator}};
      \node [rectangle,draw,rounded corners,thick,text width=7cm,text
      centered,yshift=-1cm,text centered,] (b) {\textbf{Game Abstraction
      Layer}};
      \node [rectangle,draw,rounded corners,thick,yshift=-2cm,xshift=-2.95cm,text
      centered,] (c) {\textbf{Scanner}};
      \node [rectangle,draw,rounded corners,thick,yshift=-2cm,xshift=-1.3cm,text
      centered,] (d) {\textbf{Parser}};
      \node [rectangle,draw,rounded corners,thick,yshift=-2cm,xshift=0.6cm,text
      centered,] (e) {\textbf{Scope checker}};
      \node [rectangle,draw,rounded corners,thick,yshift=-2cm,xshift=2.8cm,text
      centered,] (f) {\textbf{Interpreter}};

      \node[rectangle,xshift=-5cm] (invis1) { };
      \node[rectangle,xshift=5cm] (invis2) { };

      \draw[line] ($(a.south)-(-2.95,0)$) |- ($(b.north)-(-2.95,0)$) 
        node [midway, xshift=0.9em,yshift=0.7em] {11};
      \draw[line] ($(b.north)-(3.1,0)$) |- ($(a.south)-(3.1,0)$)
        node [midway, xshift=-0.8em,yshift=-0.7em] {2};
      \draw[line] ($(b.south)-(2.8,0)$) |- ($(c.north)-(-0.15,0)$) 
        node [midway, xshift=0.7em,yshift=0.7em] {4};
      \draw[line] ($(c.north)-(0.15,0)$) |- ($(b.south)-(3.1,0)$)
        node [midway, xshift=-0.7em,yshift=-0.7em] {3};
      \draw[line] ($(b.south)-(1.15,0)$) |- ($(d.north)-(-0.15,0)$) 
        node [midway, xshift=0.7em,yshift=0.7em] {6};
      \draw[line] ($(d.north)-(0.15,0)$) |- ($(b.south)-(1.45,0)$)
        node [midway, xshift=-0.7em,yshift=-0.7em] {5};
      \draw[line] ($(b.south)-(-0.8,0)$) |- ($(e.north)-(-0.2,0)$) 
        node [midway, xshift=0.7em,yshift=0.7em] {8};
      \draw[line] ($(e.north)-(0.3,0)$) |- ($(b.south)-(-0.3,0)$)
        node [midway, xshift=-0.7em,yshift=-0.7em] {7};
      \draw[line] ($(b.south)-(-2.95,0)$) |- ($(f.north)-(-0.15,0)$) 
        node [midway, xshift=0.7em,yshift=0.7em] {10};
      \draw[line] ($(f.north)-(0.15,0)$) |- ($(b.south)-(-2.65,0)$)
        node [midway, xshift=-0.7em,yshift=-0.7em] {9};
      \draw[line] (a) -- (invis1) node [midway, sloped, above] {1};
      \draw[line] (invis2) -- (a) node [midway, sloped, above] {12};
    \end{tikzpicture}
  \end{center}
  \capt{A diagram showing the components of \productname{} and how they are
connected.}
  \label{fig:cake}
\end{figure}



The numbers seen in the figure is the sequence of stages in which programs 
written in \productname{} will go through before they are playable. The first
arrow ($1$) is when a user inputs a program and the last arrow ($12$) is when
the program is ready to be played.

The following enumeration explains what each number the arrows have:

\begin{nlist}
\item File location
\item Stream characters
\item Stream of characters
\item Stream of tokens
\item Stream of tokens
\item Abstract Syntax Tree
\item Abstract Syntax Tree
\item Abstract Syntax Tree
\item Symbol table and Abstract Syntax Tree
\item Symbol table
\item A game object 
\item Fun with playable board games
\end{nlist}

The simulator begins by accepting a file location as input. Then the
game abstraction layer (GAL) receives a stream of characters from
this file at that location (the simulator opens the file and reads
it, instantiating GAL with the input stream). The GAL forwards the
stream of characters to the scanner, which scans the input and outputs
a stream of tokens to the GAL. The GAL again forwards the stream of
tokens to the parser, which parses the input and outputs an abstract
syntax tree (AST) to the GAL. The GAL once again forwards the AST to
the scope checker, which checks the AST for errors and passes the
checked (and possibly a modified) AST back to the GAL. One last time,
the GAL forwards the AST to the interpreter (instantiated with the game
environment), which visits every node and takes action depending on the
node types, meanwhile building up its symbol table. The interpreter
outputs the symbol table to GAL, and GAL sends an object with information
about the game to the simulator, which now makes it possible to play the
board game.

All this can of course be interrupted an error is thrown in one of the
stages. It is up to the simulator to handle these errors. If no errors
are encountered, then it is possible to see the game in the simulator and
the visualised game can be played.
