\section{Simulator}
\label{sec:simulator}

In this section, we begin by giving an overview of the relations of classes in
the Simulator in \secref{sec:overview}. Afterwards, we present
\classref{Widget}s and their properties in \secref{sec:widget}. We also explain
the different actions that are available in \secref{sec:actions}. Furthermore,
we explain how the different \classref{Widget}s communicate with each other in
\secref{sec:communication}. Lastly, we present how we have made the game
interactive in \secref{sec:interaction}, and how everything is connected in
\secref{sec:connection}.

%Overview
%Widget
%Propagated actions
%Communication between widgets
%Making games interactive
%Connecting everything

The Simulator has been constructed to provide a visual interface to the API
provided by the Game Abstraction Layer (GAL). The interface should give a visual
representation of the board and pieces, similar to how the board game would look
in real life. Furthermore, it should provide an easy way to interact with the
game. This interaction should be sufficient enough to make it possible to play
the game.

The result is a Java application with a graphical user interface, which takes a
\productname{} code file and makes it playable by the use of GAL.

\subsection{Overview}
\label{sec:overview}
\todo{Hvad er Widget?}

The implementation is based around \classref{Widget}, which simplifies the
process of distributing drawings and handling input. Some \classref{Widget}s
manage other \classref{Widget}s, while others provide visual and interactive
content. The game \classref{Widget}s provide visual and interactive intefaces to
GAL.

To work with graphics and input, the game framework slick2d is used\cite{slick2d}.
\classref{SimulatedGame} is used to bind slick2d, \classref{Widget}s, and GAL
together to provide a complete system.

Figure \ref{fig:simulator-overview} shows all these components and their most
important relationships.

\fig[width=0.7\textwidth]{simulator-overview}{Generalised overview of classes in
the Simulator and their relations}

\todo{Hvad betyder de forskellige pile?}

\subsection{Widget}
\label{sec:widget}

\classref{Widget} specifies an object which can have a size and a position, can be
drawn, take mouse input, and send messages to each other. The most important
property of \classref{Widget} however is that a \classref{Widget} can contain
several sub-\classref{Widget}s, and that drawing and mouse input is handled
according to this hierarchy.

\todo{Hvilken hierarki?}

\subsubsection{Placement}

A \classref{Widget} has a position specified with a $(x, y)$ coordinate. Its
position is relative to its parent, so if a \classref{Widget} has position
$(7, 10)$ and its parent has $(23, 50)$, its absolute position is $(30, 60)$.

Furthermore, a \classref{Widget} has a size specified with a width and height,
but it also contains an allowed range for each dimension. This allows us to
specify that a \classref{Widget} might be dynamic in size and can be adjusted if
wanted.

\subsubsection{Automatic placement and sizing}

Instead of setting sizes and positions manually, we create container classes
that manage the position and size of their sub-\classref{Widget}s. By using
\classref{Widget}s which are dynamic in size, we can create a layout which works
independent of the window and board size.

\classref{ScaleContainer} is such a container \classref{Widget} and positions
\classref{Widget}s along an axis. For \classref{Widget}s whose size is dynamic,
the remaining available space is distributed evenly amoung them. An example is
shown in \figref{fig:ScaleContainer}. The top-level \classref{Widget} is a
\classref{ScaleContainer} set to position \classref{Widget}s vertically. It does
not affect its own size, only its sub-\classref{Widget}s. The second
\classref{ScaleContainer} (containing two buttons, to be positioned
horizontally) is thus resized by the first \classref{ScaleContainer}. The
ordering of the sub-\classref{Widget}s determines the sequence they are
positioned in the \classref{ScaleContainer}.

\fig[width=0.7\textwidth]{ScaleContainer}{How \classref{ScaleContainers} (marked
with color) affects the positioning and resizing of sub-\classref{Widget}s.}

Creating a layout is now only a matter of building a hierarchy of
\classref{Widget}s, not deciding the exact position and size of each and every
single \classref{Widget}.

\subsection{Propagated actions}
\label{sec:actions}

Drawing a \classref{Widget} should not only draw the \classref{Widget}, but also
all its sub-\classref{Widget}s and their sub-\classref{Widget}s. To do this,
\classref{Widget} has two methods, \texttt{draw()} and \texttt{handleDraw()}.
\texttt{handleDraw()} needs to be overridden in sub-\classref{Widget}s that 
wants to provide a custom drawing method. \texttt{draw()} handles all the logic
for drawing sub-\classref{Widget}s, so the inherited class only needs to worry
about itself. An overview of \texttt{draw()} is given in
\lstref{lst:simulatorDraw}.

\lstinputlisting[caption={\emph{Pseudocode for the \texttt{draw()} method.}},
label={lst:simulatorDraw}, language=Java]{listings/simulatorDraw.java} 

\todo{looks wrong, draw() will never be called on the top-level
  \classref{Widget}.  Secondly, draw() should be called before handleDraw() }

To further ease development, the coordinate system is translated so
\texttt{handleDraw()} will be done using local coordinates instead of absolute
coordinates.  Furthermore, we apply clipping, so that any drawing outside the
\classref{Widget} will be clipped and not displayed. This way we can ensure that
\classref{Widget}s can't mess with other \classref{Widget}s.

We enforce this by restricting the \texttt{draw()} method with Java's final
keyword so it can't be overwridden, and \texttt{handleDraw()} is protected, so
the calling class can't call \texttt{handleDraw()} instead of \texttt{draw()} by
accident.

\subsubsection{Mouse input}

The same pattern is used for mouse input, but here we use it to determine which
\classref{Widget} is responsible for handling it. An overview is given in
\lstref{lst:simulatorMouseClicked}.

\lstinputlisting[caption={\emph{Pseudocode for the \texttt{mouseClicked()}
method.}}, label={lst:simulatorMouseClicked},
language=Java]{listings/simulatorMouseClicked.java}

Like with drawing, we translate coordinates into local coordinates, however
notice that it returns a boolean which is used to determine if the event was
handled, and it will stop as soon as any callee returns true. A second
difference is that in constrast to drawing, input is handled bottom up. The
reasoning is that the lower we get in the hierarchy, the more specific the
behaviour of each \classref{Widget} is. Thus, we try to see if the more specific
\classref{Widget}s will handle the input and if not, less and less specific
\classref{Widget}s are tried.

When mouse buttons are pushed and released, it will only try \classref{Widget}s
which contain the position at which the mouse is currently pointioned at. For
mouse dragging the situation is different, it will try any \classref{Widget}
which has initiated a drag, even if the mouse has moved outside it. If this was
not the case, a scrollbar for example would only move if we kept the mouse
exactly on top of it, which usually is tricky as they are long and slim.

\subsection{Communication between Widgets}
\label{sec:communication}

Consider the case where a \classref{Widget} represent a button. The user might
click on it, but the button by itself is not interested in what this should
signify. Thus, we need some way of notifying \classref{Widget}s that some events
have happened inside other \classref{Widget}s. For this, the Observer design
pattern is used in \classref{Widget}.

\subsection{Making games interactive}
\label{sec:interaction}

Two "game \classref{Widget}s" which interact with GAL are used to present the
game to the user. They are \classref{GameInfoWidget}s which provide information
like move history, and \classref{BoardWidget} which displays an interactive
board with pieces based on GAL.

\subsubsection{BoardWidget}

For interaction, \classref{BoardWidget} supports selecting \classref{Action}s by
the use of either Drag\&Drop or Click\&Select. While Drag\&Drop only allows
you to move a \classref{Piece} from one \classref{Square} to another, Click\&Select 
will work on any two \classref{Square}s whether or not it contains any
\classref{Piece}s. It will go through all available \classref{Action}s and find
the ones which are related to those Squares. To help ease this process, usable
\classref{Square}s are hinted as shown in \figref{fig:multiple-moves}.

\fig[width=0.7\textwidth]{multiple-moves}{The Square at E2 was selected and shows the 4 pieces which can move there.}

In reality we have two \classref{Widget}s: \classref{BoardWidget} and
\classref{GridBoardWidget} which are specialisations of \classref{BoardWidget}.
While it is not necessary at this point, it is an attempt to generalise
\classref{GirdBoard} so that a future addition with new \classref{Board} types
will be easier to implement.

\subsection{Binding everything together}
\label{sec:connection}

The class \classref{SimulatedGame} has the responsibility to connect slick2d
with the \classref{Widget} structure, and GAL with the game \classref{Widget}s.

\classref{SimulatedGame} contains one \classref{ScaleContainer}, which it resizes
to fit the whole window and tells it to adjust the sizes of its
sub-\classref{Widgets}. Secondly, it sends all mouse-events to this
\classref{Widget} and draws it whenever slick2d wants to be redrawn.

On construction of \classref{SimulatedGame}, it reads the \productname{} code file
and attempts to load it through GAL. It then creates the game
\classref{Widget}s, but it does not pass a reference to the game directly. As
the game object changes each time an \classref{Action} is applied, which
requires us to update it in every game object each time. Instead we pass a
reference to this instance of \classref{SimulatedGame} and the game
\classref{Widget}s must
then access the game object through its accessor methods directly, without
caching it.

One final task of \classref{SimulatedGame} is to handle any exceptions in GAL or
the \classref{Simulator} and show them to the user, without the application
crashing.
