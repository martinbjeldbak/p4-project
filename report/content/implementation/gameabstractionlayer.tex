\section{Game abstraction layer}
\label{sec:gameabstractionlayer}
We need an abstraction layer to act as the glue between the interpreter
and simulator, allowing a simulator access to different elements in a
written program. This has caused us to write an abstraction layer on top
of the interpreter, offering interfaces such as our graphical simulator
the player access to elements in the game.

There are three different packages relating to the game abstraction
layer. The first is the central class in the Game Abstraction Layer
package, the second are wrappers in the Interpreter package, providing
the actual implementation, and lastly is the game application
programming interface in our Utilities package. Each is described in
this section.

\subsection{The main layer}
The entire program package offered by \productname{} is encapsulated by
the class \classref{GameAbstractionLayer}. This small class is basically
only defined by its constructor, seen below in \lstref{lst:gal}:

\lstinputlisting[language=Java,label={lst:gal},caption={\emph{The game abstraction layer's constructor, initializing different constructs based off of an input of characters.}}]{listings/gal.java}

Here all the constructs are tied together: The handwritten scanner
is instantiated with the input provided in the constructor, creating
a stream of tokens fed to the parser, that then creates an abstract
syntax tree traversed by the interpreter. The interpreter and its
information is then used when getting the main game wrapper. The class
is also complimented by the method \methodref{getGame}, called after
instantiating the \classref{GameAbstractionLayer}. This method returns a
\classref{GameWrapper} (described in the next subsection), containing
all the needed information about the written game.

This class is what's called and used by the simulator when starting up,
as the simulator needs access information about the game, which in turn
builds on top of our scanner and parser. It's meant to be used by any
interface that wishes to access and modify a game's state.

\subsection{The application programming interface}
The application programming interface (API) provided is simply a collection of interfaces used by the wrappers. This means that classes implementing these interfaces guarantees different methods are available. As an example, the \classref{Square} interface is seen in \lstref{lst:squareinterface} below:

\lstinputlisting[language=Java,label={lst:squareinterface},caption={\emph{The \classref{Square} interface, one among many such interfaces provided by \productname{}.}}]{listings/squareInterface.java}

Currently $12$ such interfaces exist for different aspects of a game,
such as getting information about the game itself, its players, board,
action sequences, etc. These interfaces all build on top of the default
types in the game environment described in \secref{sec:predefined}. As a
rule, an interface exists for every default type defined in the standard
environment. If we want to expand the standard environment, an interface
and its wrapper would need to be added, considering the fact that it's
one of the only ways 3\textsuperscript{rd} party simulators can interact
with \productname{}.

\subsection{Wrappers}
Wrappers provide a way of accessing the values created by the
interpreter, allowing a simulator to use the properties of these
values to, for example, display the game's title, supply information
about the winning conditions, squares \& pieces, actions \& move
history, and so on. The wrappers implement the interfaces in the
API described above. The main wrapper returned from the method
\methodref{getGame} provides access to all the other wrappers
(implementing their respective interfaces) via methods in its body.

When instantiating \classref{GameWrapper} in method \methodref{getGame}
of class \classref{GameAbstractionLayer}, an instance of the interpreter
is passed to the method, which means full access to the symbol table and
standard environment. All wrapper classes have the following signature:

\begin{lstlisting}[language=Java,caption={\emph{The signaure of all API wrapper classes.}}]
  public class xxWrapper extends Wrapper implements yy { ... }
\end{lstlisting}

Where $xx$ is one of the $12$ wrappers and $yy$ is the matching
interface. The root class \classref{Wrapper} houses a series of methods
used by all the wrapper subclasses that retrieve different values in
the type the specific wrapper abstracts over (as in the implemented
interface).

This construction is the only way applications such as our simulator
can interact with the programming language. A simple instantiation of
\classref{GameAbstractionLayer} provides everything needed for this to
happen (granted it's a Java application).
