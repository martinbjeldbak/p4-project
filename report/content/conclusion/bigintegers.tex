\subsection{Alternative integer and decimal data structures}
\label{sec:bigintegers}

Our only primitive data type to represent numerals with, \type{Integer},
corresponds to Java's primitive data type, going by the same name. We
do not however have an alternative for representing decimal numbers
or for representing arbitrary-precision integers, if for instance a
number exceeds what's possible to represent with a $32$-bit signed
two's complement integer (Java's specification). It is imaginable
that a programmer of a board game would want to use decimals or very
big numbers, and therefore implementing this in a later version of
\productname{} could be an idea. A possible solution could be a sort of
implementation like \classref{BigInteger} and \classref{BigDecimal} data
types of Java, maybe even using these as underlying data structures.
The \classref{BigInteger} data type provides analogues to all of Java's
primitive integer operators, exactly the same as the primitive data
type \classref{Integer}\cite{javabigint}, but at the same time it
can represent arbitrary-precision integers. That is integers of any
size limited only by the memory of the computer. The same holds for
\classref{BigDecimal}s.

Decimals do not exist in \productname{}, due to the fact that from all
of the board games we've analysed and discussed, not one uses floating
point numbers to represent any aspect of the game. We have also chosen
not to use either of these, because we do not see a specific need to use
such massive numbers. And given that our \productname{} implementation
is already slow, using these arbitrary-length data strutures will
only bog it further down. It would be rather simple if we ever wished
to implement floating point numerals, we would need to update the
\methodref{isDigit} method in the Scanner to continue picking up numbers
when seeing a period. The parser would then need to create a new type
of node for it, feeding it to the newly updated \classref{Interpreter} that builds
a new type when seeing a floating point node (granted, the operations
and types we wish to support floating points on would need to be updated
too).
