\subsection{Big integers}
\label{sec:bigintegers}

Our primitive data type to represent numerals with, called Integer, corresponds to Java's primitive data type that goes by the same name. We do not however have an alternative for representing decimal numbers or for representing arbitrary-precision integers if for instance a number is exceeding what's possible to represent with a 32-bit signed two's complement integer. It is imaginable that a programmer of a board game would want to use decimals or very big numbers and therefore implementing this to a later version of \productname{} could be an idea. A possible solution could be a sort of implementation like the bigint and bigdecimal data types of Java. The bigint data type provides analogues to all of Java's primitive integer operators as the primitive data type integer \cite{javabigint} but at the same time it can represent arbitrary-precision integers. That is integers of any size limited only by the memory of the computer. 