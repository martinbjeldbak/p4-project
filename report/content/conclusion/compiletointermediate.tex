\subsection{Compile to intermediate language, then translate further}
\label{sec:compiletointermediate}

In this project we decided to implement an interpreter for our programming
language. This decision was made on the basis of our analysis chapter
(\chapref{chap:analysis}) where we could conclude that this would be the optimal
approach for \productname{}, but it is possible to develop a compiler instead. In
this section we will explain what we should have done differently in terms of
the translation method. In \secref{sec:codegenerationandinterpretation} in the
analysis chapter we have analysed and discussed the differences between these
translation methods. Furthermore, in \secref{sec:intermediatelanguage} we
describe what an intermediate language is.

One can compile to an intermediate language such as Java bytecode and the
compiled code can be run on every machine that supports Java, which is very
useful because the programmer whom is developing a compiler for the specific
language does not have to construct a compiler for every platform; just one for
Jave bytecode and then translate further to the platform. This way the
programmer must only construct $m+n$ compilers. If one does not compile to an
intermediate language, then the programmer must construct a compiler for each
specific platform, which will be a lengthy process. If you have $m$ compilers
and $n$ platforms, then the programmer must construct $m*n$ compilers to be able
to compile to every platform.  Compiling to an intermediate language and then
further translating is illustrated in \figref{fig:mtimesn}.


\begin{figure}[ht]
  \begin{center}
    \begin{tikzpicture}[level/.style={sibling distance=30mm/#1}]
      %m+n
      \node [square] (a) {Language $1$};
      \node [square,xshift=8em] (b) {Language $2$};
      \node [square,xshift=16em] (c) {Language $m$};

      \node [ellipse,draw,xshift=8em,yshift=-4em] (il) {Intermediate language};
      
      \node [square,yshift=-8em] (aa) {Platform $1$};
      \node [square,xshift=8em,yshift=-8em] (bb) {Platform $2$};
      \node [square,xshift=16em,yshift=-8em] (cc) {Platform $n$};

      \draw[->, thick,] (a) -- (il);
      \draw[->, thick,] (b) -- (il);
      \draw[->, dashed,] (c) -- (il);
      
      \draw[->, thick,] (il) -- (aa);
      \draw[->, thick,] (il) -- (bb);
      \draw[->, dashed,] (il) -- (cc);
      
      \path (b)--(c) node [midway] {$\cdots$};
      \path (bb)--(cc) node [midway] {$\cdots$};
      
      %m*n
      \node [square,xshift=23em,yshift=-1em] (x) {Language $1$};
      \node [square,xshift=31em,yshift=-1em] (y) {Language $2$};
      \node [square,xshift=39em,yshift=-1em] (z) {Language $m$};
      
      \node [square,xshift=23em,yshift=-7em] (xx) {Platform $1$};
      \node [square,xshift=31em,yshift=-7em] (yy) {Platform $2$};
      \node [square,xshift=39em,yshift=-7em] (zz) {Platform $n$};

      \draw[->, thick,] (x) -- (xx);
      \draw[->, thick,] (x) -- (yy);
      \draw[->, dashed,] (x) -- (zz);

      \draw[->, thick,] (y) -- (xx);
      \draw[->, thick,] (y) -- (yy);
      \draw[->, dashed,] (y) -- (zz);
      
      \draw[->, dashed,] (z) -- (xx);
      \draw[->, dashed,] (z) -- (yy);
      \draw[->, dashed,] (z) -- (zz);
     
      \path (y)--(z) node [midway] {$\cdots$};
      \path (yy)--(zz) node [midway] {$\cdots$};
    \end{tikzpicture}
  \end{center}
  \capt{Difference between compiling to an intermediate language.}
  \label{fig:mtimesn}
\end{figure}



\subsubsection{What should have been done differently?}

In \secref{sec:compilation} we introduced and explained the phases in the
process of compilation, and with \figref{fig:compileroverview} we visualised
these phases. A compiler must also begin with scanner and a parser to get a
stream of tokens and an abstract syntax tree. These steps are the same as we
have described in \chapref{chap:implementation}. Furthermore, the step where we
check if the different variables are visible in their respective scopes is also
the same as described in \chapref{chap:implementation}. When the above mentioned
phases have been run then the compiler should compile the code to machine
language or to an intermediate language. 

According to \figref{fig:compileroverview} the result of the semantic analyser
would be intermediate language and then the code generator would generate
machine language which could be executed. It is possible to compile directly to
machine language.
