\subsection{Compile to intermediate language, then translate further}
\label{sec:compiletointermediate}

In this project we decided to implement an interpreter for our programming
language. This decision was made on the basis of our analysis chapter
(\chapref{chap:analysis}) where we could conclude that this would be the optimal
approach for \productname{}, but it is possible to develop a compiler instead. In
this section we will explain what we should have done differently in terms of
the translation method. In \secref{sec:codegenerationandinterpretation} in the
analysis chapter we have analysed and discussed the differences between these
translation methods. Furthermore, in \secref{sec:intermediatelanguage} we
describe what an intermediate language is.

\subsubsection{Where are the differences between an interpreter and a compiler?}

In \secref{sec:compilation} we introduced and explained the phases in the
process of compilation, and with \figref{fig:compileroverview} we visualised
these phases.

The initial phases of a compiler are quite similar to the initial phases of an
interpreter. For instance a compiler must also begin by scanning the source code
followed by parsing the source code to get a stream of tokens and an abstract
syntax tree, respectively. These steps are the same as we have described in
\chapref{chap:implementation}. Furthermore, the step where we check if the
different variables are visible in their respective scopes (scope checking) is
also the same as described in \chapref{chap:implementation}. When the above
mentioned phases have been run, then the compiler should compile the code to
machine language or to an intermediate language.

According to \figref{fig:compileroverview}, the result of the semantic
analyser would be an intermediate language and then the code generator
would generate machine code, which can then be executed. It is of course
also possible to compile directly to machine language. Although, it is
not the responsible choice to make because it kills opportunities to
e.g.\ optimise the code and support multiple platforms, which would be
possible with an intermediate language.

\subsubsection{What is the advantage of compiling to an intermediate language?}

When compiling to an intermediate language before compiling to the target
language (often the object code), it is possible to make optimisations on the
intermediate code. This is one advantage of compiling to an intermediate
language and results in more efficient executable code.

Furthermore, one can compile to an intermediate language such as Java
bytecode and the compiled code can be run on every machine that supports
Java by having a Java virtual machine installed, which is very useful
because the programmer, whom is developing a compiler for a language,
does not have to construct a compiler for every platform; just one
that translates to Java bytecode, which can be translated further many
supported platforms. This way the programmer must only construct a
single compiler, not one for every platform. If one does not compile
to an intermediate language, then the programmer must construct a
compiler for each specific platform, which will be a lengthy and
cost-heavy process. If you have $m$ compilers and $n$ platforms, then
the programmer must construct $m*n$ compilers to be able to compile
to every platform. The difference between compiling directly to a
platform or to an intermediate language and then further translating is
illustrated in \figref{fig:mtimesn}.


\begin{figure}[ht]
  \begin{center}
    \begin{tikzpicture}[level/.style={sibling distance=30mm/#1}]
      %m+n
      \node [square] (a) {Language $1$};
      \node [square,xshift=8em] (b) {Language $2$};
      \node [square,xshift=16em] (c) {Language $m$};

      \node [ellipse,draw,xshift=8em,yshift=-4em] (il) {Intermediate language};
      
      \node [square,yshift=-8em] (aa) {Platform $1$};
      \node [square,xshift=8em,yshift=-8em] (bb) {Platform $2$};
      \node [square,xshift=16em,yshift=-8em] (cc) {Platform $n$};

      \draw[->, thick,] (a) -- (il);
      \draw[->, thick,] (b) -- (il);
      \draw[->, dashed,] (c) -- (il);
      
      \draw[->, thick,] (il) -- (aa);
      \draw[->, thick,] (il) -- (bb);
      \draw[->, dashed,] (il) -- (cc);
      
      \path (b)--(c) node [midway] {$\cdots$};
      \path (bb)--(cc) node [midway] {$\cdots$};
      
      %m*n
      \node [square,xshift=23em,yshift=-1em] (x) {Language $1$};
      \node [square,xshift=31em,yshift=-1em] (y) {Language $2$};
      \node [square,xshift=39em,yshift=-1em] (z) {Language $m$};
      
      \node [square,xshift=23em,yshift=-7em] (xx) {Platform $1$};
      \node [square,xshift=31em,yshift=-7em] (yy) {Platform $2$};
      \node [square,xshift=39em,yshift=-7em] (zz) {Platform $n$};

      \draw[->, thick,] (x) -- (xx);
      \draw[->, thick,] (x) -- (yy);
      \draw[->, dashed,] (x) -- (zz);

      \draw[->, thick,] (y) -- (xx);
      \draw[->, thick,] (y) -- (yy);
      \draw[->, dashed,] (y) -- (zz);
      
      \draw[->, dashed,] (z) -- (xx);
      \draw[->, dashed,] (z) -- (yy);
      \draw[->, dashed,] (z) -- (zz);
     
      \path (y)--(z) node [midway] {$\cdots$};
      \path (yy)--(zz) node [midway] {$\cdots$};
    \end{tikzpicture}
  \end{center}
  \capt{Difference between compiling to an intermediate language.}
  \label{fig:mtimesn}
\end{figure}



Now it is also possible to optimise the compiled source code before
further translation. So, we can combine these two advantages by
developing an optimiser for the intermediate language. This way all
code that has been compiled can be optimised, which gives better
efficiency by noticing common patterns. An intermediate language also
gives the possibility to support multiple platforms and architectures
if you choose a popular and well supported IL. A well supported IL is a
language that already has a compiler/interpreter built for the target
platforms, saving the programmer from having to write them. Examples
of such ILs can be high-level languages such as $C$, or low-level
languages such as Microsoft's Common Intermediate Language bytecode or
Java bytecode that abstract away from platform-specific instructions and
registers, that other languages such as assembly language use.
