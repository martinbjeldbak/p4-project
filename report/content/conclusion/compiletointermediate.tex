\subsection{Compile to intermediate language, then translate further}
\label{sec:compiletointermediate}

In this project we decided to implement an interpreter for our programming
language. This decision was made on the basis of our analysis chapter
(\chapref{chap:analysis}) where we could conclude that this would be the optimal
approach for \productname{}, but it is possible to develop a compiler instead. In
this section we will explain what we should have done differently in terms of
the translation method. In \secref{sec:codegenerationandinterpretation} in the
analysis chapter we have analysed and discussed the differences between these
translation methods. Furthermore, in \secref{sec:intermediatelanguage} we
describe what an intermediate language is.

One can compile to an intermediate language such as Java bytecode and the
compiled code can be run on every machine that supports Java, which is very
useful because the programmer whom is developing a compiler for the specific
language does not have to construct a compiler for every platform; just one for
Jave bytecode and then translate further to the platform. This way the
programmer must only construct $m+n$ compilers. If one does not compile to an
intermediate language, then the programmer must construct a compiler for each
specific platform, which will be a lengthy process. If you have $m$ compilers
and $n$ platforms, then the programmer must construct $m*n$ compilers to be able
to compile to every platform.  Compiling to an intermediate language and then
further translating is illustrated in \figref{fig:mtimesn}.

\subsection{Compile to intermediate language, then translate further}
\label{sec:compiletointermediate}

In this project we decided to implement an interpreter for our programming
language. This decision was made on the basis of our analysis chapter
(\chapref{chap:analysis}) where we could conclude that this would be the optimal
approach for \productname{}, but it is possible to develop a compiler instead. In
this section we will explain what we should have done differently in terms of
the translation method. In \secref{sec:codegenerationandinterpretation} in the
analysis chapter we have analysed and discussed the differences between these
translation methods. Furthermore, in \secref{sec:intermediatelanguage} we
describe what an intermediate language is.

\subsubsection{Where are the differences between an interpreter and a compiler?}

In \secref{sec:compilation} we introduced and explained the phases in the
process of compilation, and with \figref{fig:compileroverview} we visualised
these phases.

The initial phases of a compiler are quite similar to the initial phases of an
interpreter. For instance a compiler must also begin by scanning the source code
followed by parsing the source code to get a stream of tokens and an abstract
syntax tree, respectively. These steps are the same as we have described in
\chapref{chap:implementation}. Furthermore, the step where we check if the
different variables are visible in their respective scopes (scope checking) is
also the same as described in \chapref{chap:implementation}. When the above
mentioned phases have been run then the compiler should compile the code to
machine language or to an intermediate language.

According to \figref{fig:compileroverview} the result of the semantic analyser
would be an intermediate language and then the code generator would generate
machine code which can then be executed. It is of course also possible to compile
directly to machine language. Although, it is not the responsible choice to make 
because it kills opportunities to e.g. optimise the code which would be possible 
with an intermediate language.

\subsubsection{What is the advantage of compiling to an intermediate language?}

When compiling to an intermediate language before compiling to the target
language (often the object code), it is possible to make optimisations on the
intermediate code. This is one advantage of compiling to an intermediate
language and results in more efficient executable code.

Furthermore, one can compile to an intermediate language such as Java bytecode
and the compiled code can be run on every machine that supports Java, which is
very useful because the programmer whom is developing a compiler for the
specific language does not have to construct a compiler for every platform; just
one for Jave bytecode and then translate further to the platform. This way the
programmer must only construct $m+n$ compilers. If one does not compile to an
intermediate language, then the programmer must construct a compiler for each
specific platform, which will be a lengthy process. If you have $m$ compilers
and $n$ platforms, then the programmer must construct $m*n$ compilers to be able
to compile to every platform.  Compiling to an intermediate language and then
further translating is illustrated in \figref{fig:mtimesn}.

\subsection{Compile to intermediate language, then translate further}
\label{sec:compiletointermediate}

In this project we decided to implement an interpreter for our programming
language. This decision was made on the basis of our analysis chapter
(\chapref{chap:analysis}) where we could conclude that this would be the optimal
approach for \productname{}, but it is possible to develop a compiler instead. In
this section we will explain what we should have done differently in terms of
the translation method. In \secref{sec:codegenerationandinterpretation} in the
analysis chapter we have analysed and discussed the differences between these
translation methods. Furthermore, in \secref{sec:intermediatelanguage} we
describe what an intermediate language is.

\subsubsection{Where are the differences between an interpreter and a compiler?}

In \secref{sec:compilation} we introduced and explained the phases in the
process of compilation, and with \figref{fig:compileroverview} we visualised
these phases.

The initial phases of a compiler are quite similar to the initial phases of an
interpreter. For instance a compiler must also begin by scanning the source code
followed by parsing the source code to get a stream of tokens and an abstract
syntax tree, respectively. These steps are the same as we have described in
\chapref{chap:implementation}. Furthermore, the step where we check if the
different variables are visible in their respective scopes (scope checking) is
also the same as described in \chapref{chap:implementation}. When the above
mentioned phases have been run then the compiler should compile the code to
machine language or to an intermediate language.

According to \figref{fig:compileroverview} the result of the semantic analyser
would be an intermediate language and then the code generator would generate
machine code which can then be executed. It is of course also possible to compile
directly to machine language. Although, it is not the responsible choice to make 
because it kills opportunities to e.g. optimise the code which would be possible 
with an intermediate language.

\subsubsection{What is the advantage of compiling to an intermediate language?}

When compiling to an intermediate language before compiling to the target
language (often the object code), it is possible to make optimisations on the
intermediate code. This is one advantage of compiling to an intermediate
language and results in more efficient executable code.

Furthermore, one can compile to an intermediate language such as Java bytecode
and the compiled code can be run on every machine that supports Java, which is
very useful because the programmer whom is developing a compiler for the
specific language does not have to construct a compiler for every platform; just
one for Jave bytecode and then translate further to the platform. This way the
programmer must only construct $m+n$ compilers. If one does not compile to an
intermediate language, then the programmer must construct a compiler for each
specific platform, which will be a lengthy process. If you have $m$ compilers
and $n$ platforms, then the programmer must construct $m*n$ compilers to be able
to compile to every platform.  Compiling to an intermediate language and then
further translating is illustrated in \figref{fig:mtimesn}.

\subsection{Compile to intermediate language, then translate further}
\label{sec:compiletointermediate}

In this project we decided to implement an interpreter for our programming
language. This decision was made on the basis of our analysis chapter
(\chapref{chap:analysis}) where we could conclude that this would be the optimal
approach for \productname{}, but it is possible to develop a compiler instead. In
this section we will explain what we should have done differently in terms of
the translation method. In \secref{sec:codegenerationandinterpretation} in the
analysis chapter we have analysed and discussed the differences between these
translation methods. Furthermore, in \secref{sec:intermediatelanguage} we
describe what an intermediate language is.

\subsubsection{Where are the differences between an interpreter and a compiler?}

In \secref{sec:compilation} we introduced and explained the phases in the
process of compilation, and with \figref{fig:compileroverview} we visualised
these phases.

The initial phases of a compiler are quite similar to the initial phases of an
interpreter. For instance a compiler must also begin by scanning the source code
followed by parsing the source code to get a stream of tokens and an abstract
syntax tree, respectively. These steps are the same as we have described in
\chapref{chap:implementation}. Furthermore, the step where we check if the
different variables are visible in their respective scopes (scope checking) is
also the same as described in \chapref{chap:implementation}. When the above
mentioned phases have been run then the compiler should compile the code to
machine language or to an intermediate language.

According to \figref{fig:compileroverview} the result of the semantic analyser
would be an intermediate language and then the code generator would generate
machine code which can then be executed. It is of course also possible to compile
directly to machine language. Although, it is not the responsible choice to make 
because it kills opportunities to e.g. optimise the code which would be possible 
with an intermediate language.

\subsubsection{What is the advantage of compiling to an intermediate language?}

When compiling to an intermediate language before compiling to the target
language (often the object code), it is possible to make optimisations on the
intermediate code. This is one advantage of compiling to an intermediate
language and results in more efficient executable code.

Furthermore, one can compile to an intermediate language such as Java bytecode
and the compiled code can be run on every machine that supports Java, which is
very useful because the programmer whom is developing a compiler for the
specific language does not have to construct a compiler for every platform; just
one for Jave bytecode and then translate further to the platform. This way the
programmer must only construct $m+n$ compilers. If one does not compile to an
intermediate language, then the programmer must construct a compiler for each
specific platform, which will be a lengthy process. If you have $m$ compilers
and $n$ platforms, then the programmer must construct $m*n$ compilers to be able
to compile to every platform.  Compiling to an intermediate language and then
further translating is illustrated in \figref{fig:mtimesn}.

\input{figures/compiletointermediate}

Now it is alsp possible to optimise the compiled source code before further
translation. So, we can combine these two advantages by developing an optimiser
for the intermediate language. This way all code that has been compiled can be
optimised, which gives better efficiency.


Now it is alsp possible to optimise the compiled source code before further
translation. So, we can combine these two advantages by developing an optimiser
for the intermediate language. This way all code that has been compiled can be
optimised, which gives better efficiency.


Now it is alsp possible to optimise the compiled source code before further
translation. So, we can combine these two advantages by developing an optimiser
for the intermediate language. This way all code that has been compiled can be
optimised, which gives better efficiency.


\subsubsection{What should have been done differently?}

In \secref{sec:compilation} we introduced and explained the phases in the
process of compilation, and with \figref{fig:compileroverview} we visualised
these phases. A compiler must also begin with scanner and a parser to get a
stream of tokens and an abstract syntax tree. These steps are the same as we
have described in \chapref{chap:implementation}. Furthermore, the step where we
check if the different variables are visible in their respective scopes is also
the same as described in \chapref{chap:implementation}. When the above mentioned
phases have been run then the compiler should compile the code to machine
language or to an intermediate language. 

According to \figref{fig:compileroverview} the result of the semantic analyser
would be intermediate language and then the code generator would generate
machine language which could be executed. It is possible to compile directly to
machine language.
