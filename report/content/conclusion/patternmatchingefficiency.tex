\subsection{Efficiency of pattern matching}
\label{sec:patternmatchingefficiency}
The design of pattern matching is undoubtedly a strong mechanism for describing moves, win condition, or any other check that depends on a particular pattern. The implementation of the pattern matching is highly inefficient. Consider the piece placed at the square D5 in \figref{fig:inefficientpatterns}. The blue piece can make the moves of a knight from a chess game. This can be specified in \productname{} with the pattern \texttt{/(n n e|w) | (s s e|w) | (w w n|s) | (e e n|s) this/}.

\fig[width=0.3\textwidth]{inefficientpatterns}{Pattern matching done on the 8 green squares will return true, given the moves of a knight and the blue square (D5) as input.}

To find the moves of the piece on D5, the pattern must be matched on all 64 squares given D5 as input. Those squares for which the pattern matching returns true are those squares the piece can move to. In \figref{fig:inefficientpatterns}, only 8 of those 64 checks performed are depicted. For a game of chess starting with 32 pieces, the pattern matching is actually done on all 64 squares \textbf{for all 32 pieces}. If we for simplicity assumed that each piece had 8 possible moves, the amount of work related to pattern checks for the first move in chess can be calculated to be $32 * 64 * 8 = 16384$.
Or more generally, $O(p * n * m * c)$, hvor $p$ = the number of pieces, $(n, m)$ = the size of the gridboard, and $c$ is the complexity of each move. It is easy to see how inefficient this approach is so lets now consider a better and even a more intuitive approach.

Consider the green piece placed at the square D5 in \figref{fig:efficientpatterns}. The 8 arrows shows the moves a knight can make. An easy way to find these squares is simply to start at the knight's square (D5), move two squares in an arbitrary direction and then one square in an orthogonal direction. This also seems like a quite efficient approach. This can implemented by modifying the pattern matching to take a square as input and return a list of squares that satisfy a given pattern. A pattern for the knights move could look like \texttt{\%this (n n e|w) | (s s e|w) | (w w n|s) | (e e n|s)\%}. Notice that the \% $\ldots$ \%-encapsulation is used to distinguish between this modified pattern matching mechanism from the actual pattern mechanism used in \productname{}, which encapsulates a pattern using \\ $\ldots$ \\. This modified pattern matching done on the square D5 would return a list containing the green squares in \figref{fig:efficientpatterns}, namely the legal moves of a knight on D3.
Compared to the pattern matching in \productname{}, this approach will have the complexity $O(p * c)$. The amount of work related to the first move of a game of chess would be $32 * 8 = 256$, if we again suppose all pieces are knights. $256$ is much better than $163840$ from the previous example. The complexity here seems to not depend on the actual size of the gridboard, but this is only true in some cases, e.g. when considering the moves of a knight. If the the moves of a rook are considered, its constant $c$ will depend on the size of the gridboard for both pattern matching techniques. This is because a bigger gridboard will increase the amount of squares the rook can slide to. Generally, patterns containing the pattern-value \texttt{*} or \texttt{+} can have a complexity that depends on the size of the gridboard.

\fig[width=0.25\textwidth]{efficientpatterns}{An efficient and intuitive way to check the moves of a knight-piece in a chess game.}

\subsection{Additional functionality with pattern matching}
Many different functionalities could be included in the pattern matching. You may have noticed the big black row of black squares in the bottom of the connect Connect-4 game in \figref{fig:connect4simulated}. These black squares are put there so the pattern check can allow a piece to be dropped one square north of these squares. If a pattern keyword texttt{outofboard} existed that matched a square outside the board, these squares would not be necessary.
A short cut could also be considered for specifying patterns of the form \texttt{/(e3) | (w3) | (s3) | (n3)/}. The \texttt{3} is common, but cannot be put outside parentheses like \texttt{/(e|w|s|n)3/}, as this means \texttt{/(e|w|s|n) (e|w|s|n) (e|w|s|n)/}.