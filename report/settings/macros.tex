\newcommand{\productname}{{\fontspec{OptimusPrinceps}Junta}}

\newcommand{\secref}[1]{section \ref{#1}}
\newcommand{\chapref}[1]{chapter \ref{#1}}
\newcommand{\figref}[1]{figure \ref{#1}}
\newcommand{\lstref}[1]{listing \ref{#1}}
\newcommand{\apref}[1]{appendix \ref{#1}}
\newcommand{\tableref}[1]{table \ref{#1}}
\newcommand{\itemref}[1]{element \ref{#1}}
\newcommand{\pseudoref}[1]{algorithm \ref{#1}}
\newcommand{\capt}[1]{\caption{\emph{#1}}}
\newcommand{\csref}[1]{codesample \ref{cs:#1}}

%code-like refs
\newcommand{\tokenref}[1]{{\textbf{#1}}}
\newcommand{\classref}[1]{\textbf{{#1}}}
\newcommand{\methodref}[1]{\textbf{{#1}}}
\newcommand{\varref}[1]{\textbf{{#1}}}
\newcommand{\typeref}[1]{\textbf{{#1}}}

\newcommand{\todo}[1]{\colorbox{yellow}{\color{red}\textbf{TODO:} \hspace{1ex} #1}}

\newcommand{\codesample}[1]{
  \vspace{-0.6cm}
  \begin{center}
  \begin{tabularx}{0.9\textwidth}{m{10cm} X m{1cm} }
  \parbox[t]{8cm}{
    \vspace{-0.1cm}
    \input{codesamples/#1}}
    & &
    \parbox{1cm}{\hfill
      \begin{equation}\label{cs:#1}
      \end{equation}
    } 
  \end{tabularx}
  \end{center}
  \vspace{-0.4cm}
}

%big-step-semantic
\newcommand{\infrule}[2]
           {\parbox{4.5cm}{$$ \frac{#1}{#2}\hspace{.5cm}$$}}
	
\newcommand{\ra}{\rightarrow}
\newcommand{\lag}{\langle}
\newcommand{\rag}{\rangle}


%itemize
\newenvironment{dlist}{
\begin{itemize}[noitemsep]
}{
\end{itemize}
}

%enumerate
\newenvironment{nlist}{
\begin{enumerate}[noitemsep]
}{
\end{enumerate}
}

\newcommand{\literal}[1]{\texttt{\color[rgb]{0.400 0.000 0.933}#1}}
\newcommand{\type}[1]{\texttt{\color[rgb]{0.000 0.200 0.400}#1}}
\newcommand{\identifier}[1]{\texttt{\color[rgb]{0.000 0.200 0.400}#1}}
\newcommand{\variable}[1]{\texttt{\color[rgb]{0.694 0.537 0.384}\$#1}}
\newcommand{\constant}[1]{\texttt{\color[rgb]{0.012 0.408 0.733}#1}}
\newcommand{\function}[1]{\texttt{\color[rgb]{0.012 0.408 0.733}#1}}
\newcommand{\keyword}[1]{\texttt{\color[rgb]{0.000 0.529 0.000}#1}}

\newcommand{\operator}[4][]{\texttt{\type{#1} \textbf{\color{nicered}#2} \type{#3} $\rightarrow$ \type{#4}}}

\newcommand{\constdef}[3]{\texttt{\constant{#1}#2 : \type{#3}}}
\newcommand{\farg}[2]{\variable{#1} : \type{#2}}

\newcommand{\opstar}{$*$}

\newcommand{\fig}[3][scale=1.0]{\begin{figure}[ht]
  \center
  \includegraphics[#1]{pictures/#2}
  \capt{#3}
  \label{fig:#2}
\end{figure}}

\newenvironment{ebnf} {
  \begin{center}
    \begin{tabular}{>{\hfill}p{2.5cm} c p{9cm}}
}{
  \end{tabular}
\end{center}
}

%rules for ebnf
\newcommand{\grule}[2]{$#1$ & $\rightarrow$ & \parbox[t]{9cm}{$#2$} \\}
\newcommand{\gnl}{$\\$}
\newcommand{\galt}[1]{& | & $#1$ \\}
\newcommand{\grange}{\cdots}
\newcommand{\gcomment}[1]{\textrm{#1}}
\newcommand{\gtsq}{\texttt{"'"}}
\newcommand{\gtbs}{\texttt{"\textbackslash"}}
\newcommand{\gtdq}{\texttt{'"'}}
\newcommand{\gor}{ \: | \: }
\newcommand{\grep}[1]{\{#1\} \: }
\newcommand{\gopt}[1]{[\: #1 \:] \: }
\newcommand{\ggrp}[1]{(\: #1 \:) \: }
\newcommand{\gter}[1]{\texttt{"#1"}}
\newcommand{\gex}{\: - \:}
\newcommand{\gcat}{ \; }

%rules for formation rules
\newcommand{\for}{ \: | \: }

% Table
\newcommand{\tab}[8][\textwidth]{
\begin{table}[ht]
  \centering
  \rowcolors{3}{white}{lightgray}
    \begin{tabularx}{#1}{ r | >{\centering\arraybackslash}X>{\centering\arraybackslash}X>{\centering\arraybackslash}X>{\centering\arraybackslash}X>{\centering\arraybackslash}X>{\centering\arraybackslash}X>{\centering\arraybackslash}X>{\centering\arraybackslash}X>{\centering\arraybackslash}X>{\centering\arraybackslash}X>{\centering\arraybackslash}X>{\centering\arraybackslash}X>{\centering\arraybackslash}X>{\centering\arraybackslash}X>{\centering\arraybackslash}X>{\centering\arraybackslash}X>{\centering\arraybackslash}X>{\centering\arraybackslash}X>{\centering\arraybackslash}X>{\centering\arraybackslash}X>{\centering\arraybackslash}X>{\centering\arraybackslash}X>{\centering\arraybackslash}X>{\centering\arraybackslash}X>{\centering\arraybackslash}X>{\centering\arraybackslash}X>{\centering\arraybackslash}X>{\centering\arraybackslash}X>{\centering\arraybackslash}X>{\centering\arraybackslash}X>{\centering\arraybackslash}X } %Errr, doesn't seem to give any errors, but fix it if possible
  \hiderowcolors
  \hline
  & \multicolumn{#3}{>{\columncolor[rgb]{1,1,1}}c}{\textbf{#5}} \\
   \textbf{#6}    & #7 \\ \hline
   \showrowcolors
	#8
	\hline
    \end{tabularx}
    \capt{#4}
    \label{table:#2}
\end{table}
}

\newcommand{\tabrow}[2]{ #1 & #2 \\ }
