\newcommand{\productname}{{\fontspec{OptimusPrinceps}Garry}}

\newcommand{\secref}[1]{section \ref{#1}}
\newcommand{\chapref}[1]{chapter \ref{#1}}
\newcommand{\figref}[1]{figure \ref{#1}}
\newcommand{\lstref}[1]{example \ref{#1}}
\newcommand{\apref}[1]{appendix \ref{#1}}
\newcommand{\tableref}[1]{table \ref{#1}}
\newcommand{\itemref}[1]{element \ref{#1}}
\newcommand{\pseudoref}[1]{algorithm \ref{#1}}
\newcommand{\capt}[1]{\caption{\emph{#1}}}

%code-like refs
\newcommand{\tokenref}[1]{{\textbf{#1}}-token}
\newcommand{\classref}[1]{\textbf{{#1}}}
\newcommand{\methodref}[1]{\textbf{{#1}}}
\newcommand{\varref}[1]{\textbf{{#1}}}
\newcommand{\typeref}[1]{\textbf{{#1}}}

\newcommand{\todo}[1]{\colorbox{yellow}{\color{red}\textbf{TODO:} \hspace{1ex} #1}}

\newcommand{\codesample}[1]{\input{codesamples/#1}}

%itemize
\newenvironment{dlist}{
\begin{itemize}[noitemsep]
}{
\end{itemize}
}

%enumerate
\newenvironment{nlist}{
\begin{enumerate}[noitemsep]
}{
\end{enumerate}
}

\newcommand{\operator}[4][]{\texttt{#1 \textbf{\color{nicered}#2} #3 $\rightarrow$ #4}}

\newenvironment{ebnf} {
  \begin{center}
    \begin{tabular}{p{3cm} c p{7cm}}
}{
  \end{tabular}
\end{center}
}

\newcommand{\fig}[3][scale=1.0]{\begin{figure}[ht]
  \center
  \includegraphics[#1]{pictures/#2}
  \capt{#3}
  \label{fig:#2}
\end{figure}}

\newcommand{\grule}[2]{$#1$ & $\rightarrow$ & $#2$ \\}
\newcommand{\galt}[1]{& | & $#1$ \\}
\newcommand{\grange}{\cdots}
\newcommand{\gcomment}[1]{\textrm{#1}}
\newcommand{\gtsq}{\texttt{"'"}}
\newcommand{\gtbs}{\texttt{"\textbackslash"}}
\newcommand{\gtdq}{\texttt{'"'}}
\newcommand{\gor}{ \: | \: }
\newcommand{\grep}[1]{\{#1\} \: }
\newcommand{\gopt}[1]{[\: #1 \:] \: }
\newcommand{\ggrp}[1]{(\: #1 \:) \: }
\newcommand{\gter}[1]{\texttt{"#1"}}
\newcommand{\gex}{\: - \:}
\newcommand{\gcat}{ \: }

% Table
\newcommand{\tab}[8][\textwidth]{
\begin{table}[ht]
  \centering
    \begin{tabularx}{#1}{ r | >{\centering\arraybackslash}X>{\centering\arraybackslash}X>{\centering\arraybackslash}X>{\centering\arraybackslash}X>{\centering\arraybackslash}X>{\centering\arraybackslash}X>{\centering\arraybackslash}X>{\centering\arraybackslash}X>{\centering\arraybackslash}X>{\centering\arraybackslash}X>{\centering\arraybackslash}X>{\centering\arraybackslash}X>{\centering\arraybackslash}X>{\centering\arraybackslash}X>{\centering\arraybackslash}X>{\centering\arraybackslash}X>{\centering\arraybackslash}X>{\centering\arraybackslash}X>{\centering\arraybackslash}X>{\centering\arraybackslash}X>{\centering\arraybackslash}X>{\centering\arraybackslash}X>{\centering\arraybackslash}X>{\centering\arraybackslash}X>{\centering\arraybackslash}X>{\centering\arraybackslash}X>{\centering\arraybackslash}X>{\centering\arraybackslash}X>{\centering\arraybackslash}X>{\centering\arraybackslash}X>{\centering\arraybackslash}X } %Errr, doesn't seem to give any errors, but fix it if possible
  \hline
  & \multicolumn{#3}{c}{\textbf{#5}} \\
   \textbf{#6}    & #7 \\ \hline
	#8
	\hline
    \end{tabularx}
    \capt{#4}
    \label{table:#2}
\end{table}
}

\newcommand{\tabrow}[2]{ #1 & #2 \\ }
