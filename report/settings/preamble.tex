\documentclass[openright,twoside,10pt]{memoir}
\usepackage{fontspec}

% DOCUMENT LAYOUT
\setromanfont[Ligatures=TeX,Numbers={Lining}]{Garamond}
\setmonofont[Scale=0.8]{Consolas}
\setsansfont[Scale=0.9]{Optima}
%\usepackage[protrusion=true,factor=2000]{microtype} need beta 2.5 for XeTeX support

% LANGUAGE
\usepackage{polyglossia} %XeTeX babel
\setdefaultlanguage[variant=uk]{english}

% FONTS
\usepackage{xunicode}

%TIKZ
\usepackage{tikz}
\usetikzlibrary{calc,trees,positioning,arrows,chains,shapes.geometric,%
    decorations.pathreplacing,decorations.pathmorphing,shapes,%
    matrix,shapes.symbols}
        
\tikzset{
>=stealth',
  punktchain/.style={
    rectangle, rounded corners, % fill=black!10,
    draw=black, very thick, text width=10em, minimum height=3em, 
    text centered, on chain},
  line/.style={draw, thick, <-},
  element/.style={tape, top color=white, 
  	bottom color=blue!50!black!60!, minimum width=8em,
    draw=blue!40!black!90, very thick, text width=10em, 
    minimum height=3.5em, text centered, on chain},
  every join/.style={->, thick,shorten >=1pt},
  decoration={brace},
  tuborg/.style={decorate},
  tubnode/.style={midway, right=2pt},
}

% OTHER
\usepackage{epstopdf} % to include eps files
\usepackage{enumitem} % for [noitemsep] in enumerate/itemize lists
\graphicspath{ {images/} }
\usepackage{color}
\usepackage{multicol}
\usepackage{array}
\usepackage{hyperref}
\usepackage{listings}	%for code examples
 \lstset{ % http://stackoverflow.com/questions/741985/latex-source-code-listing-like-in-professional-books
%         basicstyle=\footnotesize\ttfamily,% the size of the fonts
                                % that are used for the code
        basicstyle=\footnotesize\ttfamily,% the size of the fonts that are used for the code
         numbers=left,                     % where to put the line-numbers
         numberstyle=\tiny,                % the size of the fonts that are used for the line-numbers
         stepnumber=1,                     % line numbering steps
         numbersep=5pt,                    % how far the line-numbers are from the code
         tabsize=2,                        % sets default tabsize to 2 spaces
         breaklines=true,                  % sets automatic line breaking
         keywordstyle=\color{shkeyword},        % language function text color
                frame=b,
 %        keywordstyle=[1]\textbf,    % Stil der Keywords
 %        keywordstyle=[2]\textbf,    %
 %        keywordstyle=[3]\textbf,    %
 %        keywordstyle=[4]\textbf,   \sqrt{\sqrt{}} %
         stringstyle=\color{shstring}, % string text color
%         stringstyle=\color{red},
        commentstyle=\color{shcomment},
         showspaces=false,           % show spaces adding particular underscores
         showtabs=false,             % show tabs within strings adding particular underscores   
         showstringspaces=false,     % underline spaces within strings
         xleftmargin=17pt,
         framexleftmargin=17pt,
         framexrightmargin=5pt,
         framexbottommargin=4pt,
         language=Java
 }

% grammar stuff

% PARAGRAPH SPACING
\parskip=5pt plus 2pt minus 1pt
\parindent=0pt

% WIDER MARGINS
\let\oldmarginpar\marginpar
\renewcommand\marginpar[1]{\-\oldmarginpar[\raggedleft\footnotesize #1]%
{\raggedright\footnotesize\color{red} #1}}
% Margins
\setstocksize{11in}{8.5in}
\settrimmedsize{11in}{8.5in}{*}
\settrims{0in}{0in}
\settypeblocksize{9.0in}{6in}{*}
\setlrmargins{1.25in}{*}{*}
\setulmargins{1.0in}{*}{*}
\setheadfoot{14pt}{26pt}
\setheaderspaces{*}{13pt}{*}

% Fra settings.tex
\setlrmarginsandblock{3.5cm}{2.5cm}{*}
\setulmarginsandblock{2.5cm}{3.0cm}{*}

\checkandfixthelayout

% CHANGE BULLETS IN ITEMIZE
\renewcommand{\labelitemi}{$\bullet$}
\renewcommand{\labelitemii}{$\circ$}
\renewcommand{\labelitemiii}{$\cdot$}

%colors
\definecolor{shcomment}{rgb}{0.12, 0.38, 0.18 }
\definecolor{shkeyword}{rgb}{0.37, 0.08, 0.25}  % #5F1441
\definecolor{shstring}{rgb}{0.06, 0.10, 0.98} % #101AF9

% macros
\newcommand{\productname}{{\fontspec{OptimusPrinceps}Junta}}
\newcommand{\CS}{C$^\sharp$}

\newcommand{\secref}[1]{section \ref{#1}}
\newcommand{\chapref}[1]{chapter \ref{#1}}
\newcommand{\figref}[1]{figure \ref{#1}}
\newcommand{\lstref}[1]{listing \ref{#1}}
\newcommand{\apref}[1]{appendix \ref{#1}}
\newcommand{\tableref}[1]{table \ref{#1}}
\newcommand{\itemref}[1]{element \ref{#1}}
\newcommand{\pseudoref}[1]{algorithm \ref{#1}}
\newcommand{\capt}[1]{\caption{\emph{#1}}}
\newcommand{\csref}[1]{code sample (\ref{cs:#1})}

%code-like refs
\newcommand{\tokenref}[1]{{\textbf{#1}}}
\newcommand{\classref}[1]{\textbf{{#1}}}
\newcommand{\methodref}[1]{\textbf{{#1}}}
\newcommand{\varref}[1]{\textbf{{#1}}}
\newcommand{\typeref}[1]{\textbf{{#1}}}

\newcommand{\todo}[1]{\colorbox{yellow}{\color{red}\textbf{TODO:} \hspace{1ex} #1}}

\newcommand{\codesample}[1]{
  \vspace{-0.6cm}
  \begin{center}
  \begin{tabularx}{0.9\textwidth}{m{9cm} X m{2cm} }
  \parbox[t]{8cm}{
    \vspace{-0.1cm}
    \input{codesamples/#1}}
    & &
    \parbox{2cm}{\hfill
      \begin{equation}\label{cs:#1}
      \end{equation}
    } 
  \end{tabularx}
  \end{center}
  \vspace{-0.4cm}
}

%big-step-semantic
\newcommand{\infrule}[2]
           {\parbox{4.5cm}{$$ \frac{#1}{#2}\hspace{.5cm}$$}}
	
\newcommand{\ra}{\rightarrow}
\newcommand{\lag}{\langle}
\newcommand{\rag}{\rangle}


%itemize
\newenvironment{dlist}{
\begin{itemize}[noitemsep]
}{
\end{itemize}
}

%enumerate
\newenvironment{nlist}{
\begin{enumerate}[noitemsep]
}{
\end{enumerate}
}

\newcommand{\literal}[1]{\texttt{\color[rgb]{0.400 0.000 0.933}#1}}
\newcommand{\type}[1]{\texttt{\color[rgb]{0.000 0.200 0.400}#1}}
\newcommand{\identifier}[1]{\texttt{\color[rgb]{0.000 0.200 0.400}#1}}
\newcommand{\variable}[1]{\texttt{\color[rgb]{0.694 0.537 0.384}\$#1}}
\newcommand{\constant}[1]{\texttt{\color[rgb]{0.012 0.408 0.733}#1}}
\newcommand{\function}[1]{\texttt{\color[rgb]{0.012 0.408 0.733}#1}}
\newcommand{\keyword}[1]{\texttt{\color[rgb]{0.000 0.529 0.000}#1}}

\newcommand{\operator}[4][]{\texttt{\type{#1} \textbf{\color{nicered}#2} \type{#3} $\rightarrow$ \type{#4}}}

\newcommand{\constdef}[3]{\texttt{\constant{#1}#2 : \type{#3}}}
\newcommand{\farg}[2]{\variable{#1} : \type{#2}}
\newcommand{\typedef}[3]{\texttt{\type{#1} \variable{#2} : \type{#3}}}

\newcommand{\opstar}{$*$}

\newcommand{\fig}[3][scale=1.0]{\begin{figure}[ht]
  \center
  \includegraphics[#1]{pictures/#2}
  \capt{#3}
  \label{fig:#2}
\end{figure}}

\newenvironment{ebnf} {
  \begin{center}
    \begin{tabular}{>{\hfill}p{2.5cm} c p{9cm}}
}{
  \end{tabular}
\end{center}
}

%rules for ebnf
\newcommand{\grule}[2]{$#1$ & $\rightarrow$ & \parbox[t]{9cm}{$#2$} \\}
\newcommand{\gnl}{$\\$}
\newcommand{\galt}[1]{& | & $#1$ \\}
\newcommand{\grange}{\cdots}
\newcommand{\gcomment}[1]{\textrm{#1}}
\newcommand{\gtsq}{\texttt{"'"}}
\newcommand{\gtbs}{\texttt{"\textbackslash"}}
\newcommand{\gtdq}{\texttt{'"'}}
\newcommand{\gor}{ \: | \: }
\newcommand{\grep}[1]{\{#1\} \: }
\newcommand{\gopt}[1]{[\: #1 \:] \: }
\newcommand{\ggrp}[1]{(\: #1 \:) \: }
\newcommand{\gter}[1]{\texttt{"#1"}}
\newcommand{\gex}{\: - \:}
\newcommand{\gcat}{ \; }

%rules for formation rules
\newcommand{\for}{ \: | \: }

% Table
\newcommand{\tab}[8][\textwidth]{
\begin{table}[ht]
  \centering
  \rowcolors{3}{white}{lightgray}
    \begin{tabularx}{#1}{ r | >{\centering\arraybackslash}X>{\centering\arraybackslash}X>{\centering\arraybackslash}X>{\centering\arraybackslash}X>{\centering\arraybackslash}X>{\centering\arraybackslash}X>{\centering\arraybackslash}X>{\centering\arraybackslash}X>{\centering\arraybackslash}X>{\centering\arraybackslash}X>{\centering\arraybackslash}X>{\centering\arraybackslash}X>{\centering\arraybackslash}X>{\centering\arraybackslash}X>{\centering\arraybackslash}X>{\centering\arraybackslash}X>{\centering\arraybackslash}X>{\centering\arraybackslash}X>{\centering\arraybackslash}X>{\centering\arraybackslash}X>{\centering\arraybackslash}X>{\centering\arraybackslash}X>{\centering\arraybackslash}X>{\centering\arraybackslash}X>{\centering\arraybackslash}X>{\centering\arraybackslash}X>{\centering\arraybackslash}X>{\centering\arraybackslash}X>{\centering\arraybackslash}X>{\centering\arraybackslash}X>{\centering\arraybackslash}X } %Errr, doesn't seem to give any errors, but fix it if possible
  \hiderowcolors
  \hline
  & \multicolumn{#3}{>{\columncolor[rgb]{1,1,1}}c}{\textbf{#5}} \\
   \textbf{#6}    & #7 \\ \hline
   \showrowcolors
	#8
	\hline
    \end{tabularx}
    \capt{#4}
    \label{table:#2}
\end{table}
}

\newcommand{\tabrow}[2]{ #1 & #2 \\ }

\usepackage{color,calc,graphicx,soul}

\definecolor{nicered}{rgb}{.647,.129,.149}
%\definecolor{nicered}{rgb}{.047,.136,.290} % aau color
\makeatletter
\newlength\dlf@normtxtw
\setlength\dlf@normtxtw{\textwidth}
\def\myhelvetfont{\def\sfdefault{mdput}}
\newsavebox{\feline@chapter}
\newcommand\feline@chapter@marker[1][4cm]{%
  \sbox\feline@chapter{%
    \resizebox{!}{#1}{\fboxsep=1pt%
      \colorbox{nicered}{\color{white} \bfseries\sffamily\thechapter}%
    }}%
  \rotatebox{90}{%
    \resizebox{%
      \heightof{\usebox{\feline@chapter}}+\depthof{\usebox{\feline@chapter}}}%
    {!}{\scshape\so\@chapapp}}\quad%
  \raisebox{\depthof{\usebox{\feline@chapter}}}{\usebox{\feline@chapter}}%
}
\newcommand\feline@chm[1][4cm]{%
  \sbox\feline@chapter{\feline@chapter@marker[#1]}%
  \makebox[0pt][l]{% aka \rlap
    \makebox[1cm][r]{\usebox\feline@chapter}%
  }}
\makechapterstyle{d402e13}{
  % Inspried by daleif1 (http://goo.gl/YKLm7)

  % Chapters
  \renewcommand\chapnamefont{\normalfont\Large\scshape\raggedleft\so}
  \renewcommand\chaptitlefont{\normalfont\huge\bfseries\scshape\color{nicered}\fontspec{OptimusPrinceps}}
  \renewcommand\chapternamenum{\fontspec{OptimusPrinceps}}
  \renewcommand\printchaptername{}
  \renewcommand\printchapternum{\null\hfill\feline@chm[2.5cm]\par}
  \renewcommand\afterchapternum{\par\vskip\midchapskip}
  \renewcommand\printchaptertitle[1]{\chaptitlefont\raggedleft ##1\par}
  
  % Sections
  \setsecheadstyle{\normalfont\Large\scshape\color{nicered}\fontspec{OptimusPrinceps}}

  % Subsections
  \setsubsecheadstyle{\normalfont\large\scshape\color{nicered}\fontspec{OptimusPrinceps}}
  
  % Parts
  \renewcommand*{\partnamefont}{\color{white}}
  \renewcommand*{\partnumfont}{\color{white}}
  \renewcommand*{\parttitlefont}{
    \hspace{100pt}
    \resizebox{!}{80pt}{\fboxsep=1pt%
      \colorbox{nicered}{\color{white} \bfseries\sffamily
      \fontspec{OptimusPrinceps} Appendix}%
    }%
    \HUGE\color{nicered}
  }
}
\makeatother


\chapterstyle{d402e13}


% Pygments syntex highlightning
\usepackage{fancyvrb}

\makeatletter
\def\PY@reset{\let\PY@it=\relax \let\PY@bf=\relax%
    \let\PY@ul=\relax \let\PY@tc=\relax%
    \let\PY@bc=\relax \let\PY@ff=\relax}
\def\PY@tok#1{\csname PY@tok@#1\endcsname}
\def\PY@toks#1+{\ifx\relax#1\empty\else%
    \PY@tok{#1}\expandafter\PY@toks\fi}
\def\PY@do#1{\PY@bc{\PY@tc{\PY@ul{%
    \PY@it{\PY@bf{\PY@ff{#1}}}}}}}
\def\PY#1#2{\PY@reset\PY@toks#1+\relax+\PY@do{#2}}

\expandafter\def\csname PY@tok@gd\endcsname{\def\PY@tc##1{\textcolor[rgb]{0.63,0.00,0.00}{##1}}}
\expandafter\def\csname PY@tok@gu\endcsname{\let\PY@bf=\textbf\def\PY@tc##1{\textcolor[rgb]{0.50,0.00,0.50}{##1}}}
\expandafter\def\csname PY@tok@gt\endcsname{\def\PY@tc##1{\textcolor[rgb]{0.00,0.27,0.87}{##1}}}
\expandafter\def\csname PY@tok@gs\endcsname{\let\PY@bf=\textbf}
\expandafter\def\csname PY@tok@gr\endcsname{\def\PY@tc##1{\textcolor[rgb]{1.00,0.00,0.00}{##1}}}
\expandafter\def\csname PY@tok@cm\endcsname{\def\PY@tc##1{\textcolor[rgb]{0.53,0.53,0.53}{##1}}}
\expandafter\def\csname PY@tok@vg\endcsname{\let\PY@bf=\textbf\def\PY@tc##1{\textcolor[rgb]{0.87,0.47,0.00}{##1}}}
\expandafter\def\csname PY@tok@m\endcsname{\let\PY@bf=\textbf\def\PY@tc##1{\textcolor[rgb]{0.40,0.00,0.93}{##1}}}
\expandafter\def\csname PY@tok@mh\endcsname{\let\PY@bf=\textbf\def\PY@tc##1{\textcolor[rgb]{0.00,0.33,0.53}{##1}}}
\expandafter\def\csname PY@tok@cs\endcsname{\let\PY@bf=\textbf\def\PY@tc##1{\textcolor[rgb]{0.80,0.00,0.00}{##1}}}
\expandafter\def\csname PY@tok@ge\endcsname{\let\PY@it=\textit}
\expandafter\def\csname PY@tok@vc\endcsname{\def\PY@tc##1{\textcolor[rgb]{0.20,0.40,0.60}{##1}}}
\expandafter\def\csname PY@tok@il\endcsname{\let\PY@bf=\textbf\def\PY@tc##1{\textcolor[rgb]{0.00,0.00,0.87}{##1}}}
\expandafter\def\csname PY@tok@go\endcsname{\def\PY@tc##1{\textcolor[rgb]{0.53,0.53,0.53}{##1}}}
\expandafter\def\csname PY@tok@cp\endcsname{\def\PY@tc##1{\textcolor[rgb]{0.33,0.47,0.60}{##1}}}
\expandafter\def\csname PY@tok@gi\endcsname{\def\PY@tc##1{\textcolor[rgb]{0.00,0.63,0.00}{##1}}}
\expandafter\def\csname PY@tok@gh\endcsname{\let\PY@bf=\textbf\def\PY@tc##1{\textcolor[rgb]{0.00,0.00,0.50}{##1}}}
\expandafter\def\csname PY@tok@ni\endcsname{\let\PY@bf=\textbf\def\PY@tc##1{\textcolor[rgb]{0.53,0.00,0.00}{##1}}}
\expandafter\def\csname PY@tok@nl\endcsname{\let\PY@bf=\textbf\def\PY@tc##1{\textcolor[rgb]{0.60,0.47,0.00}{##1}}}
\expandafter\def\csname PY@tok@nn\endcsname{\let\PY@bf=\textbf\def\PY@tc##1{\textcolor[rgb]{0.05,0.52,0.71}{##1}}}
\expandafter\def\csname PY@tok@no\endcsname{\let\PY@bf=\textbf\def\PY@tc##1{\textcolor[rgb]{0.00,0.20,0.40}{##1}}}
\expandafter\def\csname PY@tok@na\endcsname{\def\PY@tc##1{\textcolor[rgb]{0.00,0.00,0.80}{##1}}}
\expandafter\def\csname PY@tok@nb\endcsname{\def\PY@tc##1{\textcolor[rgb]{0.00,0.44,0.13}{##1}}}
\expandafter\def\csname PY@tok@nc\endcsname{\let\PY@bf=\textbf\def\PY@tc##1{\textcolor[rgb]{0.73,0.00,0.40}{##1}}}
\expandafter\def\csname PY@tok@nd\endcsname{\let\PY@bf=\textbf\def\PY@tc##1{\textcolor[rgb]{0.33,0.33,0.33}{##1}}}
\expandafter\def\csname PY@tok@ne\endcsname{\let\PY@bf=\textbf\def\PY@tc##1{\textcolor[rgb]{1.00,0.00,0.00}{##1}}}
\expandafter\def\csname PY@tok@nf\endcsname{\let\PY@bf=\textbf\def\PY@tc##1{\textcolor[rgb]{0.00,0.40,0.73}{##1}}}
\expandafter\def\csname PY@tok@si\endcsname{\def\PY@bc##1{\setlength{\fboxsep}{0pt}\colorbox[rgb]{0.93,0.93,0.93}{\strut ##1}}}
\expandafter\def\csname PY@tok@s2\endcsname{\def\PY@bc##1{\setlength{\fboxsep}{0pt}\colorbox[rgb]{1.00,0.94,0.94}{\strut ##1}}}
\expandafter\def\csname PY@tok@vi\endcsname{\def\PY@tc##1{\textcolor[rgb]{0.20,0.20,0.73}{##1}}}
\expandafter\def\csname PY@tok@nt\endcsname{\def\PY@tc##1{\textcolor[rgb]{0.00,0.47,0.00}{##1}}}
\expandafter\def\csname PY@tok@nv\endcsname{\def\PY@tc##1{\textcolor[rgb]{0.60,0.40,0.20}{##1}}}
\expandafter\def\csname PY@tok@s1\endcsname{\def\PY@bc##1{\setlength{\fboxsep}{0pt}\colorbox[rgb]{1.00,0.94,0.94}{\strut ##1}}}
\expandafter\def\csname PY@tok@gp\endcsname{\let\PY@bf=\textbf\def\PY@tc##1{\textcolor[rgb]{0.78,0.36,0.04}{##1}}}
\expandafter\def\csname PY@tok@sh\endcsname{\def\PY@bc##1{\setlength{\fboxsep}{0pt}\colorbox[rgb]{1.00,0.94,0.94}{\strut ##1}}}
\expandafter\def\csname PY@tok@ow\endcsname{\let\PY@bf=\textbf\def\PY@tc##1{\textcolor[rgb]{0.00,0.00,0.00}{##1}}}
\expandafter\def\csname PY@tok@sx\endcsname{\def\PY@tc##1{\textcolor[rgb]{0.87,0.13,0.00}{##1}}\def\PY@bc##1{\setlength{\fboxsep}{0pt}\colorbox[rgb]{1.00,0.94,0.94}{\strut ##1}}}
\expandafter\def\csname PY@tok@bp\endcsname{\def\PY@tc##1{\textcolor[rgb]{0.00,0.44,0.13}{##1}}}
\expandafter\def\csname PY@tok@c1\endcsname{\def\PY@tc##1{\textcolor[rgb]{0.53,0.53,0.53}{##1}}}
\expandafter\def\csname PY@tok@kc\endcsname{\let\PY@bf=\textbf\def\PY@tc##1{\textcolor[rgb]{0.00,0.53,0.00}{##1}}}
\expandafter\def\csname PY@tok@c\endcsname{\def\PY@tc##1{\textcolor[rgb]{0.53,0.53,0.53}{##1}}}
\expandafter\def\csname PY@tok@mf\endcsname{\let\PY@bf=\textbf\def\PY@tc##1{\textcolor[rgb]{0.40,0.00,0.93}{##1}}}
\expandafter\def\csname PY@tok@err\endcsname{\def\PY@tc##1{\textcolor[rgb]{1.00,0.00,0.00}{##1}}\def\PY@bc##1{\setlength{\fboxsep}{0pt}\colorbox[rgb]{1.00,0.67,0.67}{\strut ##1}}}
\expandafter\def\csname PY@tok@kd\endcsname{\let\PY@bf=\textbf\def\PY@tc##1{\textcolor[rgb]{0.00,0.53,0.00}{##1}}}
\expandafter\def\csname PY@tok@ss\endcsname{\def\PY@tc##1{\textcolor[rgb]{0.67,0.40,0.00}{##1}}}
\expandafter\def\csname PY@tok@sr\endcsname{\def\PY@tc##1{\textcolor[rgb]{0.00,0.00,0.00}{##1}}\def\PY@bc##1{\setlength{\fboxsep}{0pt}\colorbox[rgb]{1.00,0.94,1.00}{\strut ##1}}}
\expandafter\def\csname PY@tok@mo\endcsname{\let\PY@bf=\textbf\def\PY@tc##1{\textcolor[rgb]{0.27,0.00,0.93}{##1}}}
\expandafter\def\csname PY@tok@mi\endcsname{\let\PY@bf=\textbf\def\PY@tc##1{\textcolor[rgb]{0.00,0.00,0.87}{##1}}}
\expandafter\def\csname PY@tok@kn\endcsname{\let\PY@bf=\textbf\def\PY@tc##1{\textcolor[rgb]{0.00,0.53,0.00}{##1}}}
\expandafter\def\csname PY@tok@o\endcsname{\def\PY@tc##1{\textcolor[rgb]{0.20,0.20,0.20}{##1}}}
\expandafter\def\csname PY@tok@kr\endcsname{\let\PY@bf=\textbf\def\PY@tc##1{\textcolor[rgb]{0.00,0.53,0.00}{##1}}}
\expandafter\def\csname PY@tok@s\endcsname{\def\PY@bc##1{\setlength{\fboxsep}{0pt}\colorbox[rgb]{1.00,0.94,0.94}{\strut ##1}}}
\expandafter\def\csname PY@tok@kp\endcsname{\let\PY@bf=\textbf\def\PY@tc##1{\textcolor[rgb]{0.00,0.20,0.53}{##1}}}
\expandafter\def\csname PY@tok@w\endcsname{\def\PY@tc##1{\textcolor[rgb]{0.73,0.73,0.73}{##1}}}
\expandafter\def\csname PY@tok@kt\endcsname{\let\PY@bf=\textbf\def\PY@tc##1{\textcolor[rgb]{0.20,0.20,0.60}{##1}}}
\expandafter\def\csname PY@tok@sc\endcsname{\def\PY@tc##1{\textcolor[rgb]{0.00,0.27,0.87}{##1}}}
\expandafter\def\csname PY@tok@sb\endcsname{\def\PY@bc##1{\setlength{\fboxsep}{0pt}\colorbox[rgb]{1.00,0.94,0.94}{\strut ##1}}}
\expandafter\def\csname PY@tok@k\endcsname{\let\PY@bf=\textbf\def\PY@tc##1{\textcolor[rgb]{0.00,0.53,0.00}{##1}}}
\expandafter\def\csname PY@tok@se\endcsname{\let\PY@bf=\textbf\def\PY@tc##1{\textcolor[rgb]{0.40,0.40,0.40}{##1}}\def\PY@bc##1{\setlength{\fboxsep}{0pt}\colorbox[rgb]{1.00,0.94,0.94}{\strut ##1}}}
\expandafter\def\csname PY@tok@sd\endcsname{\def\PY@tc##1{\textcolor[rgb]{0.87,0.27,0.13}{##1}}}

\def\PYZbs{\char`\\}
\def\PYZus{\char`\_}
\def\PYZob{\char`\{}
\def\PYZcb{\char`\}}
\def\PYZca{\char`\^}
\def\PYZam{\char`\&}
\def\PYZlt{\char`\<}
\def\PYZgt{\char`\>}
\def\PYZsh{\char`\#}
\def\PYZpc{\char`\%}
\def\PYZdl{\char`\$}
\def\PYZhy{\char`\-}
\def\PYZsq{\char`\'}
\def\PYZdq{\char`\"}
\def\PYZti{\char`\~}
% for compatibility with earlier versions
\def\PYZat{@}
\def\PYZlb{[}
\def\PYZrb{]}
\makeatother


